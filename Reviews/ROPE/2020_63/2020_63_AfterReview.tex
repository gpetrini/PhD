% Created 2020-11-20 sex 11:14
% Intended LaTeX compiler: pdflatex
\documentclass[11pt]{article}
\usepackage[utf8]{inputenc}
\usepackage{lmodern}
\usepackage[T1]{fontenc}
\usepackage[top=3cm, bottom=2cm, left=3cm, right=2cm]{geometry}
\usepackage{graphicx}
\usepackage{longtable}
\usepackage{float}
\usepackage{wrapfig}
\usepackage{rotating}
\usepackage[normalem]{ulem}
\usepackage{amsmath}
\usepackage{textcomp}
\usepackage{marvosym}
\usepackage{wasysym}
\usepackage{amssymb}
\usepackage{amsmath}
\usepackage[theorems, skins]{tcolorbox}
\usepackage[style=abnt,noslsn,extrayear,uniquename=init,giveninits,justify,sccite,
scbib,repeattitles,doi=false,isbn=false,url=false,maxcitenames=2,
natbib=true,backend=biber]{biblatex}
\usepackage{url}
\usepackage[cache=false]{minted}
\usepackage[linktocpage,pdfstartview=FitH,colorlinks,
linkcolor=blue,anchorcolor=blue,
citecolor=blue,filecolor=blue,menucolor=blue,urlcolor=blue]{hyperref}
\usepackage{attachfile}
\usepackage{setspace}
\usepackage{tikz}
\renewcommand{\abstractname}{Overview and Recommendation}
\author{Journal Manuscript No. 2020/63}
\date{\today}
\title{Recommendation letter after authors' review}
\begin{document}

\maketitle
\noindent \textbf{Report on} ``Housing is NOT ONLY the Business Cycle: A Luxemburg-Kalecki External Market Empirical Investigation for the United States'' 


\begin{abstract}
As mentioned before, the manuscript contributes to the empirical housing-related literature and reports interesting results based on quite  new methodology.
The authors properly adapted the manuscript as suggested.
However, there are some structure flaws and redundancies therefore my recommendation is for \textbf{approve} of the manuscript with minor \textbf{revisions}.
The following topics are on this direction.
\end{abstract}

\begin{enumerate}
\item The new abstract is clearer and more accurate than the previous version.
\item Some paragraphs in the Introduction seems to be more adequate in Section 3 and presents some redundancies.
\begin{itemize}
\item This is the case in paragraph 1, page 2 which is quite similar with paragraph 1 page 8.
\item Since the theoretical model will be presented in section 3, I suggest to discuss Luxemburg-Kalecki External Market without going into too much detail in the Sraffian supermultiplier model.
\end{itemize}
\item Empirical literature review (Section 2) is better organized
\begin{itemize}
\item A suggestion is to present the review in a table. In doing this, it is possible to highlight the empirical, methodological and theoretical gap
\item X-axis in Figure 1 (page 2) is not in standard datetime (month/day/year) format.
\item There is different citation entries for the same paper (Huang et al 2018 and Huang et al 2020).
\begin{itemize}
\item Time span selection and consumer credit exclusion are more explicit now
\begin{itemize}
\item Additionally, I would suggest to explicitly indicates FRED data series code instead of pointing that all the data was retrieved from FRED database.
\end{itemize}
\end{itemize}
\item Table 1 does not have a proper caption. It is not clear that variables are in log. May be changing ``(see Table 1 for the summary statistics of them)'' to the end of the paragraph is enough.
\end{itemize}
\item Methodology presentation (Section 4, mainly 4.4) is clearer than previous version
\begin{itemize}
\item Section 4.2 is unnecessary. The reference to Table 2 in Section 5 (page 16) seems to be enough.
\begin{itemize}
\item It is worth noting that moving part of methodology discussion to the appendix have improved the reading
\end{itemize}
\end{itemize}
\item In Section 5, after point out the differences with existing empirical literature, it would be interesting to compare with Huang et al. (2020) conclusion in more detail in order to highlight the theoretical divergences since it is argued that they report that (p. 5) ``housing factors [\ldots{}]   autonomously drive the business cycle in the US''
\end{enumerate}

Considering the items discussed above, I recommend the \textbf{approve} of the manuscript \textbf{with minor revisions}.
\end{document}

\section{Concluding Remarks}\label{sec:Conclusion}

In this article, we present a residential investment growth rate specification compatible with the demand-led growth agenda.
To do so, we estimate a bi-dimensional VEC to evaluate \textcite{teixeira_crescimento_2015}
proposal. 
We report: 
	(i) Houses' own interest rate ($own$) and residential investment growth rate ($g_{I_h}$) share a common long-run trend;
	(ii) $g_{I_h}$ effects over $own$ are negligible and; 
	(iii) own interest rate has a negative and persistent effect on $g_{I_h}$ and is its main determinant (see Figure \ref{fevd}).
Besides the parsimony, our estimations does not show residuals serial autocorrelation and heteroscedasticity. 


In conclusion,  we report lack of work analyzing residential investment in a Sraffian supermultiplier-friendly framework in the macroeconometric literature.
Our estimation supports houses' own interest rate relevance in describing residential investment growth rate for the US as depicted by \textcite{teixeira_crescimento_2015}.
Thus, our  proposal differs from the usual empirical literature by:
	(i) considering housing as a non-capacity creating autonomous expenditure;
	(ii) reporting that mortgage interest rates are relevant to describe long-run residential investment dynamics; and notably 
	(iii) including asset bubble through houses own interest rate.


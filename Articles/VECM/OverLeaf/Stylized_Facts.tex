\section{Housing Dynamics and Business Cycle in the US Economy}\label{sec:Stylized_Facts}

After the Great Recession, the literature has analyzed the relevance of housing at macroeconomic level (\cites{leamer_housing_2015}{teixeira_crescimento_2015}{fiebiger_semi-autonomous_2018}).
Despite some authors had highlighted the empirical importance of this expenditure in determining economic cycles since the post-war period (\cites{grebler_capital_1956}{green_follow_1997}{leamer_housing_2007}), only a few macroeconomists have given due attention to this regularity.
\textcite{duesenberry_investment_1958} was an exception and reported the relevance of residential investment and real estate inflation in
determining the economic cycle long before the Great Recession.
\textcite{keynes_collected_1978} is another example and --- despite dedicating himself to the firms' investment --- wrote to President Roosevelt about the relevance of real estate for economic recovery in the context of the Great Depression:


\begin{quotation}
    ``[...] Housing is by far the best aid to recovery because of the large and continuing scale
of potential demand; because of the wide geographical distribution of this demand; and
because the sources of its finance are largely independent of the stock exchanges. I should
advise putting most of your eggs in this basket, caring about this more than about anything,
and making absolutely sure that they are being hatched without delay. In this country we
partly depended for many years on direct subsidies. There are few more proper objects for
such than working-class houses. If a direct subsidy is required to get a move on (we gave
our subsidies through the local authorities), it should be given without delay or hesitation.''
(\cite[p.~436]{keynes_collected_1978})
\end{quotation}
As the above excerpt suggests, the relevance of housing is not restricted to the Great Recession nor the USA case.
Moreover, however small its share on GDP is (see Figure \ref{FigAutonomos}), it does not imply that it has negligible effects on the business cycle or low volatility (see Figure \ref{FigVolatilidade}).

\begin{figure}[H]
	\caption{Housing's Particular Stylized Facts}
	\label{fig:figs}
	\begin{subfigure}[t]{.5\textwidth}
		\centering
		\caption{Selected growth rate distribution (1947-2019)}
		\label{FigVolatilidade}
		\includegraphics[width=.8\linewidth]{./figs/Volatilidade.eps}
	\end{subfigure}
	\begin{subfigure}[t]{.5\textwidth}
		\centering
		\caption{Autonomous expenditures share on GDP (US, 1979-2019)}
		\label{FigAutonomos}
		\includegraphics[width=.8\linewidth]{./figs/Gastos_autonomos.eps}  
	\end{subfigure}
	\caption*{\textbf{Source:} U.S. Bureau of Economic Analysis, Authors' Elaboration}
\end{figure}

In order to depict the relation between housing and business cycle, we present Figure \ref{FigIh_u}  in which each cycle is represented in a different panel\footnote{This similar reasoning can be found in \textcites{fiebiger_trend_2017}. Unlike them, we plot only residential investment without including other households expenses financed by credit.}. 
The vertical axis represents residential investment-GDP ratio and the horizontal
axis represents the rate of capacity utilization as a proxy for business cycle.
Economic recovery is generally characterized residential investment growing faster than GDP --- with the 1991-2000 period being a particular case. 
As a consequence of this higher growth rate, is the increase of both residential investment share on GDP and capacity utilization. 
Following the Sraffian supermultiplier growth model, we conclude that increase of non-residential investment is the result of capital stock adjustment principle.
This increase implies GDP to grow faster than residential investment, therefore reducing both its share on GDP and capacity utilization ratio. 
Finally, as a result of economic burst, capacity utilization ratio falls and the cycle.


\begin{figure}[H]
	\centering
	\caption{Residential investment share on GDP VS. capacity utilization during recessions}
	\label{FigIh_u}
	\includegraphics[width=\textwidth]{./figs/Ciclo_Ih_u.eps}
	\caption*{\textbf{Source:} Authors' Elaboration}
\end{figure}

We also report an indirect relation between housing and aggregate demand. 
Real estate constitutes a significant portion of household wealth so houses serves as collateral to borrowing (\cite{teixeira_uma_2011}). 
As a consequence of US institutional arrangement, households --- especially the poorest ones --- could increase their indebtedness as houses prices went up (see Figure \ref{FigDividaPreco}) as a way to ``make'' capital gains without % Realize é um falso cognato, não?
selling their homes during house bubble of the 2000s (\cite{teixeira_crescimento_2015}).
Therefore, real estate inflation and durable goods consumption are connected and has relevant consequences for business cycle.
\textcite{zezza_u.s._2008} and \textcite{barba_rising_2009}, for example, report that credit-financed consumption was one of the main drivers of economic growth before the Great Recession.


In this paper, we argue that this relation between households indebtedness and real estate inflation has other relevant implications.
The first one is the increasing gap between assets and liabilities in the course of the Great Recession.
This dynamic is due both to the housing prices burst (post-2005) and to the insensitivity of households' financial commitments.
In other words, real estate (assets) has a market value while debt (liabilities) has a contractual one, thus, households net worth decreases onset of the subprime crisis.
Therefore, the second implication is the sharp reduction in the net worth of the poorest households in absolute and relative terms (see Figure \ref{FigDistPassivos}).


\begin{figure}[H]
	\centering
	\caption{Household indebtedness and house prices dynamics (jan/2000=100)}
	\label{FigDividaPreco}
	\includegraphics[width=\textwidth]{./figs/Divida_PrecoImoveis.eps}
	\caption*{\textbf{Source:} U.S. Bureau of Economic Analysis, Authors' Elaboration}
\end{figure}

\begin{figure}[H]
	\centering
	\caption{Liabilities evolution by wealth percentile (1989/07=1)}
	\label{FigDistPassivos}
	\includegraphics[width=.8\textwidth]{./figs/Distribuicao_Passivos.eps}
	\caption*{\textbf{Source:} \textcite{us_census_bureau_characteristics_2017}, Authors' Elaboration}
\end{figure}

Before we move forward, it is worth mentioning that the relevance of housing is not restricted to its growth effects. 
For example, \textcite{jorda_great_2016} report that credit and financial sector growth has been led mainly by mortgages. 
As a consequence, banking activities were redirected towards granting credit majorly to households and not towards productive investment (\cites{erturk_banks_2007}{kohl_more_2018}).
Other studies have shown that real estate inflation is the main determinant of household indebtedness, distribution of wealth and that it has implications for macroeconomic stability (\cites{ryoo_household_2015}{stockhammer_debt-driven_2016}{barnes_private_2016}{johnston_global_2017}{mian_household_2017}{anderson_politics_2020}{fuller_housing_2020} ). 
In summary, what we intended to show is that one cannot analyze the US business cycle without considering housing dynamics.
%we conclude that housing is relevant to understand the specificity of US business cycle.
On the following section, we analyze how econometric literature has dealt with this topic.

\begin{comment}
It is worth mentioning the novelty of \textcite{green_follow_1997} and \textcite{leamer_housing_2007} --- revisited in \textcite{leamer_housing_2015} and by \textcite{fiebiger_trend_2017} --- when shedding light on the relevance of residential investment even before of the Great Recession. CITAÇÃO KEYNES DO PROJETO + DUESEMBERRY

\end{comment}
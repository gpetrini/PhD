% Created 2021-07-12 seg 11:47
% Intended LaTeX compiler: pdflatex
\documentclass[10pt]{beamer}
  \usepackage{caption, csquotes, appendixnumberbeamer, graphicx}
\usepackage{caption, multirow, booktabs, tabularx, lscape, tablefootnote, threeparttable, makecell}
\newcolumntype{b}{>{\hsize=2.3\hsize}X}
\newcolumntype{s}{>{\hsize=.45\hsize}X}
\newcolumntype{m}{>{\hsize=.9\hsize}X}
\usepackage[backend=biber,style=authoryear-comp, uniquename=init, giveninits, doi=false, isbn=false, maxcitenames= 2, natbib=true, sorting=ynt]{biblatex}
\usepackage[english]{babel}
\addbibresource{./ref.bib}
\setbeamertemplate{navigation symbols}{} %remove navigation symbols
\institute{Institute of Economics - University of Campinas/Brazil}
\DeclareLabeldate{%
\field{date}
\field{year}
\field{eventdate}
\field{urldate}
\literal{nodate}
}
\renewbibmacro*{date+extradate}{%
\iffieldundef{labelyear}
{}
{\printtext[parens]{%
\iffieldundef{origyear}
{}
{\printtext[brackets]{\printorigdate}%
\setunit{\addspace}}%
\iflabeldateisdate
{\printdateextra}
{\printlabeldateextra}}}}
\renewbibmacro*{cite:labeldate+extradate}{%
\iffieldundef{labelyear}
{}
{\printtext[bibhyperref]{%
\iffieldundef{origyear}
{}
{\printtext[brackets]{\printorigdate}%
\setunit{\addspace}}%
\printlabeldateextra}}}
\usetheme{metropolis}
\author{Gabriel Petrini and Lucas Teixeira}
\date{July 20th, 2021}
\title{Determinants of residential investment growth rate in the US economy (1992-2019)}
\begin{document}

\maketitle





\begin{frame}{Table of contents}
  \setbeamertemplate{section in toc}[sections numbered]
  \tableofcontents[hideallsubsections]
\end{frame}

\section{Introduction}
\label{sec:org1d8202a}

\begin{frame}[label={sec:org96d5086}]{Introduction}
After the subprime crisis (2008) and the Great Recession (2008-2009), policymakers have payed more attention on housing market.
However, residential investment is only \alert{indirectly} considered.

The growing theoretical and empirical literature based on the Sraffian supermultiplier model (SSM, \cites{serrano_long_1995}{bortis_institutions_1997}) establishes a prominent role for non-capacity creating autonomous expenditures such as \alert{residential investment}.
Few works have examined its \alert{macroeconomic} determinants and still there is not a consensus about it.


\metroset{block=fill}
\begin{block}{Objective}
Present and evaluate a parsimonious way of explaining the behavior of the residential investment growth rate, focusing on the US case (1992-2019).
\end{block}
\end{frame}

\section{Empirical motivation}
\label{sec:org8872f58}

\begin{frame}[label={sec:orgfe75453}]{Residential investment, business cycle and economic growth in the US economy}
\begin{itemize}
\item Macroeconomic relevance of residential investment systematically neglected
\begin{itemize}
\item Small share on aggregate demand
\item Some exceptions (\cite{keynes_collected_1978}, \cite{duesenberry_investment_1958})
\end{itemize}
\end{itemize}

From the review of empirical literature, residential investment:

\begin{itemize}
\item leads the business cycle \citep{green_follow_1997,leamer_housing_2007,leamer_housing_2015,kydland_2016_housing}
\item is a source of autonomous demand \citep{fiebiger_semi-autonomous_2018,teixeira_crescimento_2015,fiebiger_trend_2017}
\item is relevant for long-run economic growth \citep{perez_Montiel_2021,girardi_autonomous_2018}
\end{itemize}
\end{frame}



\begin{frame}[label={sec:orgd16731c}]{Residential investment and the business cycle \citep{green_follow_1997,leamer_housing_2007}}
\metroset{block=transparent}
\begin{block}{Selected expenditures growth rates two years before and two years after recession starts\footnote{\tiny{Vertical lines indicate the begining of the recession (NBER recession dating procedure)}}}
\begin{figure}[H]
	\centering
	\includegraphics[height=.57\textheight, width = \textwidth]{./figs/Centered_Begin_pct1.png}
	\caption*{\scriptsize{\textbf{Source:} U.S. Bureau of Economic Analysis, Authors' Elaboration}}
\end{figure}
\end{block}
\end{frame}




\begin{frame}[label={sec:org0d23be6}]{Residential investment and long-run economic growth\footnote{\tiny{HP filter, $\lambda = 1600$. We are aware of the econometric problems regarding this type of filter. For details see \textcite{NBER_HP}. This is just a visualization procedure.}} \citep{perez_Montiel_2021}}
\begin{figure}[H]
	\centering
	\includegraphics[height=.68\textheight, width = \textwidth]{./figs/Trend.png}
	\caption*{\scriptsize{\textbf{Source:} U.S. Bureau of Economic Analysis, Authors' Elaboration}}
\end{figure}
\end{frame}





\section{Econometric literature review}
\label{sec:orgd2499b8}

\begin{frame}[label={sec:org865c519}]{Determinants of residential investment}
Few econometric papers analyze housing in macroeconomic terms and its determinants.

\begin{description}
\item[{\textcite{poterba_tax_1984}}] houses as an asset and not only as a durable consumption good
\item[{House price}] positively related
\item[{Interest rate}] negatively related, interest rate choice varies
\item[{80s-90s Institutional changes}] residential investment responsiveness to interest rate is still valid
\end{description}

\metroset{block=fill}
\begin{block}{Main conclusion}
House prices (nominal or deflated by some general price index) and some interest rate (mortgage or long term, as proxy) may be the only consensus about residential investment determinants.
\end{block}
\end{frame}



\section{Estimation proposition and economic discussion}
\label{sec:org7a790da}

\begin{frame}[label={sec:orgdbf343b}]{Houses own-rate of interest: definition}
\textcite{teixeira_crescimento_2015}  proposed the index named houses own-rate of interest (\(own\)) to analyze the 2000s US housing boom episode.

\begin{latex}
\begin{equation}
\label{txpropria}
own =  \left(\frac{1+r_{mo}}{1+\pi} - 1\right)
\end{equation}
\end{latex}

\begin{description}
\item[{House price inflation (\(\pi\))}] relation between the costs of acquiring the new house market price and its potential sale price
\item[{Mortgage interest rate (\(r_{mo}\))}] financial cost of this operation
\end{description}

\metroset{block=fill}
\begin{block}{Houses own-rate of interest in a nutshell}
It is the combination of the relevant nominal rate of interest with the relevant price index for the one acquiring a new house, representing the \alert{real cost in houses from buying houses} \parencite[p.~53]{teixeira_crescimento_2015}.
\end{block}
\end{frame}

\begin{frame}[label={sec:org0201f94}]{Houses own-rate of interest: clarifications}
\label{OwnClarification}
Based on \citeauthor*{sraffaDrHayekMoney1932}'s (\citeyear{sraffaDrHayekMoney1932}) commodity rate, houses own-rate is:

\begin{itemize}
\item A \alert{real cost} measured in terms of houses (that's why \alert{houses} own-rate of interest)
\item Not restricted to speculation episodes (such as housing bubble)
\end{itemize}

Additionally, only \alert{persistent} changes in houses own-rate of interest will influence residential investment growth rate
\begin{itemize}
\item What is relevant is the \alert{future} price in relation to the buying price
\item Construction takes time (from 7 to 12 months depending on housing unit type) \hyperlink{constructionPlot}{\beamerbutton{Figure}}
\end{itemize}

\begin{block}{Technical note}
Data series of expected house inflation do not exist.
So, we use the lags of this variable as a proxy for the expected one.
\end{block}
\end{frame}

\begin{frame}[label={sec:org3e444bd}]{Houses own-rate of interest and residential investment growth rate}
\begin{figure}[htb]
	\centering
	\includegraphics[width=\textwidth]{./figs/TxPropria_Investo.png}
	\caption*{\textbf{Source:} U.S. Bureau of Economic Analysis, Authors' elaboration}
\end{figure}
\end{frame}



\begin{frame}[label={sec:orgfbc40fe}]{Data and inspection}
\label{DataInspection}
We rely on the following quarterly seasonally adjusted data:
\begin{description}
\item[{Mortgage interest rate}] 30-year fixed (MORTGAGE30US)
\item[{Residential investiment}] Private residential investment (PRFI)
\item[{House price inflation}] Case-Shiller house price index (CSUSHPISA)
\end{description}

The time range was selected because it captures the effects of changes in depository institutions (1980s and 1990s), the rise of house price inflation, the bubble of the 2000s, and the aftermath of the 2008 crisis.

\begin{itemize}
\item \textcite{yeo_new_2000} transformation \hyperlink{YeoTransformation}{\beamerbutton{Figure}}
\item Unit root and Structural break tests \(\Rightarrow\) non-stationary \hyperlink{UnitTest}{\beamerbutton{Tests results}}
\item \textcite{johansen_estimation_1991} procedure \(\Rightarrow\) Series are cointegrated \hyperlink{CointTest}{\beamerbutton{Tests results}}
\end{itemize}
\end{frame}

\begin{frame}[label={sec:orge30213b}]{Estimation strategy}
Previous results allows us to estimate a error correction model.
We assume the following long-run relationship:


\begin{latex}
\begin{equation}
\label{gihLR}
g_{I_{h_{t}}} = \phi_{0} + \phi_{1}\cdot own_{t}
\end{equation}
\end{latex}

and short run adjustment process:

{\scriptsize
\begin{equation}
\label{matrix}
\begin{bmatrix}
\Delta own_{t}\\
\Delta g_{I_{h_{t}}}
\end{bmatrix} = \begin{bmatrix}\delta_{1}\\ \delta_{2}\end{bmatrix} + \begin{bmatrix}\alpha_{1}\\ \alpha_{2}\end{bmatrix} \begin{bmatrix}g_{I_{h_{t-1}}} - \phi_{0} - \phi_{1,1}\cdot own_{t-1}\\g_{I_{h_{t-1}}} - \phi_{0} - \phi_{1,2}\cdot own_{t-1}\end{bmatrix}^{\prime} + \sum^N_{i=1} \begin{bmatrix}\beta_{1,i} & \gamma_{1,i} \\\beta_{2,i} & \gamma_{2,i} \end{bmatrix} \begin{bmatrix}\Delta g_{I_{h_{t-i}}} \\\Delta own_{t-i}\end{bmatrix} + \begin{bmatrix}\varepsilon_{1,t}\\\varepsilon_{2,t}\end{bmatrix}
\end{equation}
}

where \(\delta_{is}\) indicate linear trend (level);
\(\alpha_{is}\) are adjustment parameters;
\(\beta_{is}\) and \(\gamma_{is}\) are coefficients associated with lagged \(g_{I_h}\) and \(own\) respectively and; \(\varepsilon_{is}\) are the residuals.
We estimated a VECM with four lags.
\end{frame}


\begin{frame}[label={sec:orgd723291}]{Expected results and economic meaning}

% Please add the following required packages to your document preamble:
% \usepackage{graphicx}
\begin{table}[H]
	\centering
	\caption{Summary of expected results of the macroeconometric model}
	\label{resultados_esperados}
	\resizebox{\textwidth}{!}{%
		\begin{tabular}{l|l|l}
			\hline\hline
			\textbf{\begin{tabular}[c]{@{}l@{}}Expected\\ Result\end{tabular}} &
			\textbf{Econometric Meaning} &
			\textbf{Economic Meaning} \\ \hline\hline
			\textbf{1. $\varepsilon \sim I(0)$} &
			\begin{tabular}[c]{@{}l@{}} Stationary residuals indicates cointegration relationship\end{tabular} &
			\begin{tabular}[c]{@{}l@{}} Series share a common\\long-run trend\end{tabular} \\ \hline
			\textbf{2. $\alpha_1 = 0$} &
			\begin{tabular}[c]{@{}l@{}} $own$ is weakly exogenous\\ compered to $g_{I_h}$\end{tabular} & \begin{tabular}[c]{@{}l@{}} 
				$own$ dynamics is not affected\\by previous equilibrium deviation\end{tabular}
			\\ \hline
			\textbf{3. $\alpha_2 < 0$} &
			\begin{tabular}[c]{@{}l@{}}Own interest rate Granger-causes\\
				residential investment growth rate\end{tabular} & \begin{tabular}[c]{@{}l@{}} $g_{I_h}$ dynamics is not affected\\ by previous equilibrium deviation\end{tabular}
			\\ \hline
			\textbf{4. $\phi_1 > 0$} &
			\begin{tabular}[c]{@{}l@{}}Series share a common\\negative long-run relationship\end{tabular} &
			\begin{tabular}[c]{@{}l@{}}Own interest rate affects\\residential investment growth rate negatively\end{tabular} \\ \hline
			\textbf{5. $\phi_0 < 0$} &
			\begin{tabular}[c]{@{}l@{}}
			Real estate demand for non-speculation\\reasons is statistically significant
			\end{tabular} &
			\begin{tabular}[c]{@{}l@{}}
				Real estate demand associated with\\institutional particularities and demographic\\ changes affects residential investment\\growth rate positively\end{tabular} \\ \hline
			\textbf{6. $\gamma_{2,is} < 0$} &
			\begin{tabular}[c]{@{}l@{}}Residential investment growth rate\\coefficient is statistically significant\end{tabular} &
			\begin{tabular}[c]{@{}l@{}}Own interest rate affects\\$g_{I_h}$ in the short-run\end{tabular} \\ \hline
			\textbf{7. $\beta_{1,is} = 0$} &
			\begin{tabular}[c]{@{}l@{}}
				$g_{I_h}$ effects over own interest\\ rate is not statistically significant\end{tabular} &
			\begin{tabular}[c]{@{}l@{}}
				$g_{I_h}$ effects over own interest\\ rate is negligible since dwellings stock is much\\bigger than residential investment (flow)\end{tabular} \\ \hline\hline
		\end{tabular}%
	}
\caption*{\textbf{Source:} Authors' elaboration}
\end{table}

\end{frame}


\section{Results}
\label{sec:org3774fd4}

\begin{frame}[label={sec:org63487ba}]{Estimation parameters \hyperlink{robust_frame}{\beamerbutton{Robustness check}}}
\label{back}
\begin{table}[H]
	\centering
    \caption{VECM parameters - four lags}
    \label{Estimacao}
    \resizebox*{!}{\dimexpr\textheight-45\lineskip\relax}{%
	     \begin{tabular}{lrrrrr}
\toprule
{} &  Base scenario &  $\Delta \phi_0$ &  $\Delta \omega$ &  $\Delta rm$ &  $\pi$ \\
\midrule
$\alpha$      &         0.5000 &           0.5000 &           0.5000 &       0.5000 & 0.5000 \\
$\gamma_F$    &         0.0800 &           0.0800 &           0.0800 &       0.0800 & 0.0800 \\
$\gamma_u$    &         0.0900 &           0.0900 &           0.0900 &       0.0900 & 0.0900 \\
$\omega$      &         0.5000 &           0.5000 &           0.4900 &       0.5000 & 0.5000 \\
$rm$          &         0.0100 &           0.0100 &           0.0100 &       0.0200 & 0.0100 \\
$\sigma_{l}$  &         0.0000 &           0.0000 &           0.0000 &       0.0000 & 0.0000 \\
$\sigma_{mo}$ &         0.0000 &           0.0000 &           0.0000 &       0.0000 & 0.0000 \\
$u_N$         &         0.8000 &           0.8000 &           0.8000 &       0.8000 & 0.8000 \\
$v$           &         1.2000 &           1.2000 &           1.2000 &       1.2000 & 1.2000 \\
$\phi_0$      &         0.0250 &           0.0300 &           0.0250 &       0.0250 & 0.0250 \\
$\phi_1$      &         0.1000 &           0.1000 &           0.1000 &       0.1000 & 0.1000 \\
$R$           &         0.7000 &           0.7000 &           0.7000 &       0.7000 & 0.7000 \\
$\pi$         &         0.0000 &           0.0000 &           0.0000 &       0.0000 & 0.0500 \\
\bottomrule
\end{tabular}

               }
\end{table}
\end{frame}


\begin{frame}[label={sec:org6ed18ea}]{Orthogonalized Impulse Response Function}
\begin{figure}[H]
	\centering
	\includegraphics[height=.85\textheight]{./figs/Impulse_VECMOrth_grey.png}
\end{figure}
\end{frame}
\begin{frame}[label={sec:org1275123}]{Forecast error variance decomposition}
\begin{figure}[H]
	\centering
	\includegraphics[width=\linewidth,height=\textheight,keepaspectratio]{./figs/FEVD_VECMpython_TxPropria.png}
\end{figure}
\end{frame}

\section{Conclusions}
\label{sec:org5663cf1}
\begin{frame}[label={sec:org5b5b03d}]{Concluding remarks}
We presented a simple specification for residential investment growth rate based on houses own-rate of interest proposed by \textcite{teixeira_crescimento_2015}.
\begin{itemize}
\item houses own-rate of interest (\(own\)) and residential investment growth rate (\(g_{I_h}\)) share a common long-run trend;
\item residential investment growth rate effects over \(own\) are negligible; and
\item houses own-rate of interest has a negative effect on residential investment growth rate and it explains more than a half of its variance after the second quarter.
\end{itemize}

\metroset{block=fill}
\begin{block}{5 Second synthesis}
Houses own-rate of interest determines — but is not determined by — residential investment growth rate and these variables
present a long-term relationship.
All expected results hold for lags 1 to 7 and most of them hold for the other lags.
Thus, houses own-rate of interest has a prominent role in explaining residential investment growth rate.
\end{block}
\end{frame}



\appendix

\begin{frame}[allowframebreaks]{References}
\printbibliography[heading=none, sorting=nyt]

\end{frame}



\section{Additional plots}


\begin{frame}{Time-series with \textcite{yeo_new_2000} transformation \hyperlink{DataInspection}{\beamerbutton{Back to presentation}}}
\label{YeoTransformation}
\begin{figure}[htb]
	\centering
	\label{YeoJhonson}
	\includegraphics[width=\textwidth]{./figs/YeoJohnson_All.png}
	\caption*{\textbf{Source:} U.S. Bureau of Economic Analysis, Authors' elaboration}
\end{figure}
\end{frame}

\begin{frame}{Average construction time (approval to completion) of properties for a family unit by construction purposes except manufactured houses (1976-2018) \hyperlink{OwnClarification}{\beamerbutton{Back to presentation}}}
\label{constructionPlot}
\begin{figure}[H]
	\centering
	\includegraphics[width=\textwidth]{./figs/Meses_construcao.png}
	\caption*{\textbf{Source:} Survey of Construction (SOC), Authors' elaboration}
\end{figure}

\end{frame}

\section{Estimation}

\begin{frame}{Selection model order}
\scriptsize{\begin{center}
\begin{tabular}{lcccc}
\toprule
            & \textbf{AIC} & \textbf{BIC} & \textbf{FPE} & \textbf{HQIC}  \\
\midrule
\textbf{0}  &      -16.27  &     -16.00*  &   8.622e-08  &       -16.16   \\
\textbf{1}  &      -16.24  &      -15.86  &   8.864e-08  &       -16.09   \\
\textbf{2}  &      -16.36  &      -15.87  &   7.878e-08  &       -16.16   \\
\textbf{3}  &      -16.40  &      -15.80  &   7.558e-08  &       -16.16   \\
\textbf{4}  &     -16.50*  &      -15.80  &  6.827e-08*  &      -16.22*   \\
\textbf{5}  &      -16.44  &      -15.63  &   7.305e-08  &       -16.11   \\
\textbf{6}  &      -16.39  &      -15.47  &   7.682e-08  &       -16.02   \\
\textbf{7}  &      -16.33  &      -15.30  &   8.192e-08  &       -15.91   \\
\textbf{8}  &      -16.33  &      -15.20  &   8.193e-08  &       -15.87   \\
\textbf{9}  &      -16.27  &      -15.02  &   8.777e-08  &       -15.77   \\
\textbf{10} &      -16.26  &      -14.90  &   8.947e-08  &       -15.71   \\
\textbf{11} &      -16.49  &      -15.03  &   7.113e-08  &       -15.90   \\
\textbf{12} &      -16.43  &      -14.86  &   7.637e-08  &       -15.80   \\
\textbf{13} &      -16.41  &      -14.73  &   7.847e-08  &       -15.73   \\
\textbf{14} &      -16.37  &      -14.58  &   8.312e-08  &       -15.64   \\
\textbf{15} &      -16.32  &      -14.42  &   8.854e-08  &       -15.55   \\
\bottomrule
\end{tabular}
%\caption{VECM Order Selection (* highlights the minimums)}
\end{center}}
\end{frame}

\section{Statistical tests}


\begin{frame}{Unit root tests \hyperlink{DataInspection}{\beamerbutton{Back to presentation}}}
\label{UnitTest}
\scriptsize{% Please add the following required packages to your document preamble:
% \usepackage{multirow}
% \usepackage{graphicx}
\begin{table}[H]
	\centering
	\caption{Unit root tests}
	\label{unitroot}
	\resizebox{\textwidth}{!}{%
	\begin{threeparttable}
		\begin{tabular}{l|l|cccccccc}
\hline\hline
\multicolumn{2}{l|}{\multirow{2}{*}{\textbf{Variable}}} & \multicolumn{2}{c}{\textbf{ADF}\tnote{a}} & \multicolumn{2}{c}{\textbf{Zivot Andrews}\tnote{b}} & \multicolumn{2}{c}{\textbf{Phillips Perron}\tnote{a}} & \multicolumn{2}{c}{\textbf{KPSS}\tnote{c}} \\ \cline{3-10} 
\multicolumn{2}{l|}{} & \multicolumn{1}{l}{Statistic} & \multicolumn{1}{l}{p-value} & \multicolumn{1}{l}{Statistic} & \multicolumn{1}{l}{p-value} & \multicolumn{1}{l}{Statistic} & \multicolumn{1}{l}{p-value} & \multicolumn{1}{l}{Statistic} & \multicolumn{1}{l}{p-value} \\ \hline
\textbf{Residential} & level &-3.333&0.013&-4.439&0.139&-6.165&0.000&0.181&0.309\\
\textbf{investment ($g_{I_h}$)} & first difference &-7.155&0.000&-7.739&0.000&-20.346&0.000&0.106&0.558\\ \hline
% \textbf{Real estate} & level &-2.671&0.079&-4.871&0.043&-2.704&0.073&0.148&0.395 \\
% \textbf{inflation} & first difference &-4.680&0.000&-6.122&0.001&-11.340&0.000&0.059&0.819 \\ \hline
\textbf{Houses own-rate of interest} & level &-2.330&0.162&-4.203&0.237&-2.425&0.135&0.690&0.014 \\
\textbf{rate of interest}& first difference &-5.087&0.000&-6.340&0.000&-10.408&0.000&0.062&0.804\\ \hline
% \textbf{Mortgage} & level &-3.638&0.027&-4.494&0.215&-3.604&0.030&0.081&0.264 \\
% \textbf{interest rate}& first difference &-8.050&0.000&-8.144&0.000&-11.127&0.000 &0.034&0.962 \\
\hline\hline
\end{tabular}%
\begin{tablenotes}\footnotesize
	\item [a] H0: has a unit root.
	\item [b] H0: has a unit root and a structural break.
	\item [c] H0: series is weakly stationary.
\end{tablenotes}
\end{threeparttable}
	}
\caption*{\textbf{Source:} Authors' elaboration}
\end{table}
}
\end{frame}


\begin{frame}{Structural break tests \hyperlink{DataInspection}{\beamerbutton{Back to presentation}}}
\label{StructTest}
\scriptsize{% Please add the following required packages to your document preamble:
% \usepackage{multirow}
% \usepackage{graphicx}
\begin{table}[H]
	\centering
	\caption{Structural break test}
	\label{structbreak}
	\begin{threeparttable}
	%\resizebox{\textwidth}{!}{%
		\begin{tabular}{l|l|cc}
			\hline \hline
			\multirow{2}{*}{\textbf{Variable}} & \multirow{2}{*}{\textbf{Break}} & \multicolumn{2}{c}{\textbf{Chow test}\tnote{a}} \\ \cline{3-4} 
			&& Statistic & p-value \\ \hline
			\multirow{3}{*}{\textbf{Residential investment ($g_{I_h}$)}} & 1991/Q3 & 5.1147 & 0.0254 \\
			& 2005/Q4 & 7.286 & 0.007881 \\
			& 2010/Q3 & 6.1013 & 0.01481 \\ \hline
			\multirow{5}{*}{\textbf{Houses own-rate of interest}} & 1991/Q3 & 63.453 & 7.487e-13 \\
			& 1996/Q3 & 107.47 & \textless 2.2e-16 \\
			& 2001/Q2 & 78.378 & 5.662e-15 \\
			& 2006/Q1 & 20.68 & 1.236e-05 \\
			& 2011/Q1 & 78.969 & 4.663e-15 \\ \hline
			% \multirow{4}{*}{\textbf{Mortgage interest rate}} & 1991/Q3 & 124.35 & \textless 2.2e-16 \\
			% & 1997/Q1 & 199.25 & \textless 2.2e-16 \\
			% & 2002/Q1 & 301.18 & \textless 2.2e-16 \\
			% & 2009/Q4 & 172.97 & \textless 2.2e-16 \\ \hline
			% \multirow{3}{*}{\textbf{Real estate inflation}} & 1997/Q3 & 1.5508 & 0.2153 \\
			% & 2005/Q4 & 23.49 & 3.569e-06 \\
			% & 2011/Q3 & 4.4981 & 0.03586 \\
			\hline \hline
		\end{tabular}%
	%}
	\begin{tablenotes}\footnotesize
		\item [a] H0: There is no structural break.
	\end{tablenotes}
\end{threeparttable}
	\caption*{\textbf{Source:} Authors' elaboration}
\end{table}
}
\end{frame}


\begin{frame}{Cointegration test \hyperlink{DataInspection}{\beamerbutton{Back to presentation}}}
\label{CointTest}
\scriptsize{% Please add the following required packages to your document preamble:
% \usepackage{multirow}
% \usepackage{graphicx}
\begin{table}[h]
\centering
\caption{Cointegration test}
\label{Johansen}
\begin{threeparttable}
%\resizebox{\textwidth}{!}{%
\begin{tabular}{l|l|c|c}
\hline
 \hline
\multirow{2}{*}{\textbf{Model specification}} & \multirow{2}{*}{\textbf{Hypothesis}} & \multicolumn{2}{c}{\textbf{Johansen Procedure\tnote{a}}} \\ \cline{3-4} 
 &  & \multicolumn{1}{c|}{Statistic} & critical value (5\%) \\ \hline
\multirow{3}{*}{\textbf{$g_{I_h}$, Own interest rate}} & $r = 0$ &22.51&19.96\\
 & $r = 1^*$ &2.91&9.24\\\hline	
\multirow{4}{*}{\textbf{$g_{I_h}$, Inflation and Mortgage interest rate}} & $r = 0$ &46.05&34.91\\
 & $r = 1^*$ &15.08&19.96\\
 & $r = 2$ &6.44&9.24\\\hline
\multirow{3}{*}{\textbf{$g_{I_h}$, Inflation and exogenous  mortgages interest rate}} & $r = 0$ &36.88& 19.96\\ 
 & $r = 1^*$ &7.87&9.24\\ 
  \hline
\end{tabular}%
%}
\footnotesize{(a) Using trace test with constant for the 5th lag (according to AIC criteria). (*) Indicates the selected rank that implies cointegration.}
\end{threeparttable}
\caption*{\textbf{Source:} Authors' elaboration}
\end{table}}
\end{frame}




\section{Robustness check}
\begin{frame}{Robustness check \hyperlink{back}{\beamerbutton{Back}}}
\label{robust_frame}
	     % Please add the following required packages to your document preamble:
% \usepackage{multirow}
% \usepackage{graphicx}
\begin{table}[]
\centering
\caption{Robustness check}
\label{tab:robust}
\resizebox{\textwidth}{!}{%
\begin{tabular}{c|c|c|c|c|c|c|c|c}
\hline\hline
\multirow{2}{*}{\textbf{Lags}} &
  \multirow{2}{*}{\textbf{\begin{tabular}[c]{@{}c@{}}Information\\ Criteria\end{tabular}}} &
  \multirow{2}{*}{\textbf{\begin{tabular}[c]{@{}c@{}}Reject $H_0$ for Adjusted \\ Portmanteau test?\\ (zero residual auto corr. )\end{tabular}}} &
  \multicolumn{2}{c|}{\textbf{\begin{tabular}[c]{@{}c@{}}Reject $H_0$ for\\ ARCH-LM test?\\ (Homoscedasticity)\end{tabular}}} &
  \multirow{2}{*}{\textbf{\begin{tabular}[c]{@{}c@{}}Reject $H_0$ for\\ Jarque-Bera test?\\ (Normality)\end{tabular}}} &
  \multicolumn{2}{c|}{\textbf{\begin{tabular}[c]{@{}c@{}}Reject $H_0$\\ for KPSS \\ Stationarity test?\end{tabular}}} &
  \multirow{2}{*}{\textbf{\begin{tabular}[c]{@{}c@{}}Confirmed\\hypothesis\end{tabular}}} \\ \cline{4-5} \cline{7-8}
   &                &     & \textbf{$g_{I_h}$} & \textbf{$own$} &     & \textbf{$g_{I_h}$} & \textbf{$own$} &                       \\ \hline\hline
0  & BIC            & yes & yes                & yes            & yes & no                 & no             & 1,2, 3 and 5          \\
1  & $-$            & yes & no                 & yes            & yes & no                 & no             & all                   \\
2  & $-$            & yes & no                 & yes            & yes & no                 & no             & all                   \\
3  & $-$            & yes & no                 & yes            & yes & no                 & no             & all                   \\
4  & AIC, FPE, HQIC & no  & no                 & no             & yes & no                 & no             & all                   \\
5  & $-$            & no  & no                 & no             & yes & no                 & no             & all                   \\
6  & $-$            & no  & no                 & no             & yes & no                 & no             & all                   \\
7  & $-$            & yes & no                 & no             & yes & no                 & no             & all                   \\
8  & $-$            & yes & no                 & no             & yes & no                 & no             & all except 7          \\
9  & $-$            & yes & no                 & no             & yes & no                 & no             & all except 2 and 7    \\
10 & $-$            & yes & no                 & no             & yes & no                 & no             & all except 2, 3 and 7 \\
11 & $-$            & no  & yes                & no             & yes & no                 & no             & all except 7          \\
12 & $-$            & yes & no                 & no             & yes & no                 & no             & all except 2 and 7    \\ \hline\hline
\end{tabular}%
}
\caption*{\textbf{Source:} Authors' elaboration}
\end{table}

\end{frame}
\end{document}

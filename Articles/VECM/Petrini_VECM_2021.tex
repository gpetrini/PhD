% Created 2021-03-10 qua 18:12
% Intended LaTeX compiler: pdflatex
\documentclass[12pt, a4paper]{article}
\usepackage[utf8]{inputenc}
\usepackage{lmodern}
\usepackage[T1]{fontenc}
\usepackage[top=3cm, bottom=2cm, left=3cm, right=2cm]{geometry}
\usepackage{graphicx}
\usepackage{longtable}
\usepackage{float}
\usepackage{wrapfig}
\usepackage{rotating}
\usepackage[normalem]{ulem}
\usepackage{amsmath}
\usepackage{textcomp}
\usepackage{marvosym}
\usepackage{wasysym}
\usepackage{amssymb}
\usepackage{amsmath}
\usepackage[theorems, skins]{tcolorbox}
\usepackage[style=abnt,noslsn,extrayear,uniquename=init,giveninits,justify,sccite,
scbib,repeattitles,doi=false,isbn=false,url=false,maxcitenames=2,
natbib=true,backend=biber]{biblatex}
\usepackage{url}
\usepackage[cache=false]{minted}
\usepackage[linktocpage,pdfstartview=FitH,colorlinks,
linkcolor=blue,anchorcolor=blue,
citecolor=blue,filecolor=blue,menucolor=blue,urlcolor=blue]{hyperref}
\usepackage{attachfile}
\usepackage{setspace}
\usepackage{tikz}
\addbibresource{ref.bib}
\usepackage{svg, caption, multirow, booktabs, tabularx, subfigure, subcaption, lscape, tablefootnote, threeparttable, makecell}
\newcolumntype{b}{>{\hsize=2.3\hsize}X}
\newcolumntype{s}{>{\hsize=.45\hsize}X}
\newcolumntype{m}{>{\hsize=.9\hsize}X}
\usepackage[english]{babel}
%\usepackage{endfloat}
\usepackage{authblk}
\author[1]{Gabriel Petrini}
\affil[1]{PhD Student at University of Campinas (Brazil), Email: \url{gpetrinidasilveira@gmail.com}} % Author affiliation
\author[2]{Lucas Teixeira}
\affil[2]{Assistant Professor at University of Campinas (Brazil), Email: \url{lucastei@unicamp.br}} % Author affiliation
\author[3]{Frankin Serrano}
\affil[3]{Associate Professor at Federal University of Rio de Janeiro (Brazil), Email: \url{}} % Author affiliation
\date{\today}
\title{Untitled}
\begin{document}

\maketitle



\begin{abstract}
This article investigates the relationship between residential investment, asset inflation, and macroeconomic dynamics based on
the US post-deregulation case (1992-2019). To do so, we estimate a vector error correction model (VECM) to asses the relevance of
real interest rate of real estate. We find residential investment growth rate and houses’ own interest rate to be cointegrated. We
also find that long-run causality goes unidirectionally from houses’ own interest rate to residential investment growth rate,
as expected. In summary, our findings support the relevance of houses’ own interest rate in determining residential investment
while the other way round does not occur.

\noindent \textbf{Keywords:} Residential Investment; Asset price inflation; Mortgage Interest rate; Vector-Error correction model
\noindent \textbf{JEL cods:} E39; E44; R39
\end{abstract}


\section{Introduction}
\label{sec:org524c5e2}
\label{sec:Introduction}
Among aggregate demand expenditures, non-residential investment is the most examined  one between (at least) heterodox macroeconomists.
As a consequence, the relevance of others (autonomous) expenditures on macroeconomic dynamics has been underestimated \cite{brochier_macroeconomics_2017}.
The Sraffian supermultiplier (SSM) model presented by \textcite{serrano_long_1995} establishes a prominent role for non-capacity creating autonomous expenditures in the theoretical ground.
Despite the late interest in those expenditures \cites{freitas_pattern_2013}{girardi_long-run_2016}{girardi_autonomous_2018}{braga_investment_2018}, there still is a lack of studies on the role of residential investment in particular\footnote{Except for \textcite{green_follow_1997} and \textcite{leamer_housing_2007} --- which shows the relevance of this expenditure to US business cycle at least since the post-war period ---, most of those studies were published after the Great Recession (2008-2009).}. 

Our main objective is to assess the determinants of residential investment growth rate.
We argue that the scarce attention that residential investment receives is not compatible with its relevance for the US and its significance is not restricted to the Great Recession.
In Section \ref{sec:Stylized_Facts}, we present some stylized facts for the US economy highlighting the relevance of residential investment.
Next, in Section \ref{sec:empirical_review}, we will present and compare different macroeconometric models that explicitly incorporate residential investment.
In Section \ref{sec:VECM}, we estimate a bi-dimensional vector error-correction model (VECM) using time-series data for the US economy from 1992 onward to test the houses' own interest rate presented by \textcite{teixeira_crescimento_2015}. 
Section \ref{sec:Conclusion} offers some concluding remarks.
The results for all statistical tests are provided in Appendix \ref{appen:A}.



\section{Residential investment, economic growth and business cycle in the US economy}
\label{sec:org3770e2f}
\label{sec:Stylized_Facts}
Before the subprime crisis and the subsequent Great Recession (2008-9), economists usually did not pay due attention to the role of residential investment to macroeconomic dynamics.
Perhaps, this lack of attention can be explained by the small share of residential investment on aggregate demand (see figure \ref{fig:share}).
The subprime crisis and the Great Recession of the US economy had changed the landscape, making clear the macroeconomic role of housing prices and finance, and residential investment.
Specifically, we argue however small its share on GDP is, it does have a fundamental role in explaining business cycles and long-run growth.

There are, nevertheless, two notable exceptions. In a letter to President Franklin D. Roosevelt, in the mid of the Great Depression, Keynes present the case of housing as the best aid to fight the crisis and to foster economic recovery \cite[p.~436]{keynes_collected_1978}.
The other exception is \textcite{duesenberry_investment_1958} who presents a critique of ordinary theories of aggregate investment because they put together firms’ investment in capital and structures and investment in residential housing.
In this sense, he builds a model dedicated for the latter type of investment, emphasizing the role of house prices.


\begin{figure}[H]
    \centering
	\caption{Expenditures share on GDP}
	\label{fig:share}
\begin{figure}[htb]
    \includegraphics[width = \textwidth]{./figs/Share_AD.png}
    \end{figure}
	\caption*{\textbf{Source:} U.S. Bureau of Economic Analysis, Authors' Elaboration}
\end{figure}

Recently, more and more macroeconomists are paying attention to the relevance of residential investment for the business cycles.
\textcite{green_follow_1997} analyzed this connection even before the Great Recession.
Based on Granger casualty tests for the US from 1952 to 1992, he concludes that residential investment leads the cycle, while non-residential investment (i.e., firms business investment) lags the cycle.
\textcite{green_follow_1997} relates this result to the forward-looking behavior of households.
Thus, he restricts the relevance of residential investment as a temporal precedence indicator.

\textcite{leamer_housing_2007} is more incisive. He states that housing is the business cycle.
In analyzing the US economy for the post-war period, he concludes that residential investment is the best predictor of economic crises.
The economic cycle of US economy would be a consumer cycle: ``first homes, then cars, last business investment'' \cite[p.~8]{leamer_housing_2007}.
In a follow up work, the author declare that nine out of eleven US economic recessions were preceded by a severe downturn of residential investment, in the post-war period \cite{leamer_housing_2015}.

\textcite{kydland_2016_housing} extends \citeauthor*{leamer_housing_2007}'s \citeyear{leamer_housing_2007} proposition to other countries to evaluate how unique the US case is.
Similarly to \textcite{green_follow_1997}, they also conclude that residential investment leads the cycle -- while non-residential investment lags it -- for the US and Canada.
The results, however, are more robust when they include housing starts instead of residential investment.
Briefly, they find that housing starts precedes GDP the shorter the time to build is\footnote{For the other countries in the sample, \textcite{kydland_2016_housing} find that residential investment is coincident (and not lagging) with GDP. Additionally, they conclude that long-term nominal mortgage rate is a relevant transmission channel from interest rates to housing costs.}.
More recently, \textcite{huang_is_2018} assess  prediction and causality relations proposed by \textcite{leamer_housing_2007} for 17 OECD countries.
For the US both statements are valid\footnote{For the other countries, they found inconclusive results regarding fluctuations due to institutional heterogeneity. However, \textcite{huang_is_2018} claim that for most G7 countries, residential investment at least amplify the business cycle.} and housing related variables (house prices, real mortgage rate --- deflated by a consumer price index --- and bank spread) lead the business cycle.


In order to depict the relation between residential investment and business cycle, we present Figure \ref{fig:cycle}.
In each panel, we present the growth rate of selected expenditures two years before and after recession begin\footnote{This similar reasoning can be found in \textcite{leamer_housing_2007}.}\textsuperscript{,}\,\footnote{We adopt usual NBER recession dating procedure.}.
In most of them, residential investment growth rate is negative before the recession start.
The Great Recession is a remarkable example of this pattern while dot-com crisis (2000-1) is an exception as already pointed out by \textcite{leamer_housing_2007}.


\begin{figure}[H]
	\centering
	\caption{Selected expecditure growth rates 2 years before and after recession start}
	\label{fig:cycle}
	\includegraphics[width=\textwidth]{./figs/Centered_Begin_pct1.png}
	\caption*{\textbf{Source:} Authors' Elaboration}
\end{figure}


In a heterodox perspective, \textcites{fiebiger_semi-autonomous_2018}{fiebiger_trend_2017} present residential investment as a Luxemburg-type external market providing a source of autonomous demand to absorb economic surplus.
Both works state its role (together with debt-financed consumption) in inducing non-residential investment and driving output growth, during the business cycle as well in the long-run.
Figure \ref{fig:trend} presents the trend of GDP and residential investment growth rate by estimating a HP filter.
In summary, both series share a correlation relation in which residential investment growth rate is more volatile than GDP growth rate.
Recently, \textcite{perez_Montiel_2021} test econometrically the assertion that this kind of expenditure leads not only the businesses cycle but also the long-run economic growth for the US economy. Their research finds evidence of this claim, without finding any relevant feedback from output to residential investment.

\begin{figure}[H]
	\centering
	\caption{GDP and Residential investment growth rate tred (HP filter, $\lambda = 1600$)}
	\label{fig:trend}
	\includegraphics[width=\textwidth]{./figs/Trend.png}
	\caption*{\textbf{Source:} Authors' Elaboration}
\end{figure}


Before we move forward, it is worth mentioning that the relevance of housing is not restricted to its implications to growth and business cycles.
For example, \textcite{jorda_great_2016} report that credit and financial sector growth has been led mainly by mortgages. 
As a consequence, banking activities were redirected towards granting credit majorly to households and not towards productive investment \cites{erturk_banks_2007}{kohl_more_2018}.
Other studies have shown that real estate inflation is the main determinant of household indebtedness, distribution of wealth and that it has implications for macroeconomic stability \cites{ryoo_household_2015}{stockhammer_debt-driven_2016}{barnes_private_2016}{johnston_global_2017}{mian_household_2017}{anderson_politics_2020}{fuller_housing_2020}. 
In summary, what we intended to show was the macroeconomic relevance of residential investment.
On the following section, we analyze how econometric literature has dealt with determinants of the latter.
\section{Determinants of residential investment: review of empirical literature}
\label{sec:orge13e10e}
\label{sec:empirical_review}
As we have seen in the previous session, there has been a growing attention to the macroeconomic implications of residential investment. It is worth noting, there is a vast empirical literature on housing. However, this literature is mainly focused on urban and regional economics or microeconomic issues. As a consequence, few econometric papers analyze housing in macroeconomic terms. Additionally, if housing macroeconomic literature is scarce, even fewer scholars examine residential investment and its determinants specifically. In this sense, we review papers that analyse the determinants of residential investment (or some proxy) focusing on the US case, as well on some panel analysis of multiple countries.

In this sense, \citeauthor*{poterba_tax_1984}'s \citeyear{poterba_tax_1984} contribution stands out once it considers houses as an asset and not only as a durable consumption good.
Based on asset-market model for the construction sector, he estimates a residential investment supply function for the US from 1974 to 1982.
In order to estimate the determinants of residential investment, he includes real house prices, a non-residential constructor deflator index, wages of the housing sector and alternates between two measures  of  credit  rationing.
In summary, real house prices is the main driver of constructors behavior.
Therefore, \textcite{poterba_tax_1984} concludes that residential investment is positively induced house prices.
\textcite{topel_1988_Housing} estimate the determinants of housing starts (as a proxy for residential investment) for the US from 1963 to 183.
Their functional form for residential investment includes hedonic prices and some cost shifters for the construction sector such as interest rate, expected inflation, wages and median market time since the construction begins.
Similar to \textcite{poterba_tax_1984}, they conclude that residential investment is highly sensible to house prices movements while interest rate is a relevant cost shifter.

Recently,  \textcite{arestis_residential_2015} update \citeauthor*{poterba_tax_1984}'s \citeyear{poterba_tax_1984} framework by estimating an autoregressive distributed lag (ARDL) model for 17 OECD countries.
In their model, they specify residential investment as a positive function of house prices, real disposable income and bank credit while mortgage interest and unemployment rate are expected to have negative effects.
They report that disposable income is the major determinant of residential investment for most OECD countries while banking credit and house prices are more relevant for the US.

Some scholars estimated residential investment based on a Tobin's Q adapted for the housing sector\footnote{This adapted Tobin's Q is defined as the ratio between the market price of new houses and construction cost to build a new one.}.
\textcite{barot_2002_House}, for instance, investigate the differences and similarities for the Sweden and UK housing market based on a Stock-flow framework from 1970 to 1998.
Their residential investment functional form includes both adapted Tobin's Q and real interest rate.
They report some opposite results regarding Granger causality tests for house prices, financial wealth, household debt and interest rate.
The only result that is equally valid for both Swedwn and UK is that housing Tobin's Q granger causes housing investment.
More recently, \textcite{kohlscheen_2018_Residential}, also estimate residential investment as a function of housing Tobin's Q with an unbalanced panel data for 15 OECD countries from 1970 ownwards\footnote{More precisely, \textcite{kohlscheen_2018_Residential} include Tobin's Q numerator and denominator as separate variables.}.
Beyond Tobin's Q they also include real house prices, GDP level, population density, net migration rates and house stock share on GDP.
In summary, they conclude that real house prices growth, nominal interest rate, net migration rates and existing house stock are the main determinants of residential investment.

Other common ground in macroeconomic housing is the relevance of interest rate for residential investment.
In a OLS model for 7 OECD, \textcite{egebo_1990_MODEL} concludes that real disposable income and real interest rate are relevant for residential investment.
For the US in particular, they report that mortgage interest rate coefficients are quite robust.
\textcite{mccarthyMonetaryPolicyTransmission2002}, for instance, evaluate the response of housing market to a monetary policy shock.
In order to isolate some institutional change effects, they estimate a GMM model for the US in two sub-samples (1975-1985 and 1986-2000).
They conclude that the magnitude of the response to shock is similar to before the institutional break of mid-1980s.
\textcite{gauger_residential_2003} also evaluate the consequences of deregulation of depository institutions throughout the 1980s.
In order to do so, they estimate a VECM between monetary aggregates (M2), GDP, residential investment and alternate between short-term government bonds and long-term mortgage interest rates.
They report an increasing contribution of long-term mortgages interest rate over resident investment variance after those institutional changes mentioned above.


Table \ref{tab:summary_model} presents an overview of the discussed topics and its main conclusions.
From this brief review of the literature, we conclude that may be the only consensus about the determinants of residential investment is on house prices (nominal or deflated by consumer price index) and some rate of interest (mortgage or long term, as proxy)\footnote{For a broader review of econometric literature, see \textcite{egebo_1990_MODEL}.}.
Some authors deflate the rate of interest by the CPI, others use nominal interest rate to capture the effect of monetary illusion.
In the next section, we will present a particular way of combining house prices and mortgage rate of interest in a single index that we will use on our econometric estimation.



% \begin{landscape}
\begin{table}[htb]
    \caption{Residential investment determinants in macroeconometric models}
    \label{tab:summary_models}
    \begin{threeparttable}
        % \resizebox{\textheight}{!}{%
% \begin{tabular}{p{0.1\textwidth}p{0.1\textwidth}p{0.1\textwidth}p{0.7\textwidth}}
      \begin{tabularx}{\textwidth}{s|s|s|m}
    \hline\hline
    \textbf{Authors} & \textbf{Sample} & \textbf{Estimation method} & \textbf{Covariates}\\\hline
    \textcite{poterba_tax_1984} & US (1974-1982) & & \makecell{$RHP(+), NCD(-), W(+),$\\$Cr(+)$} \\\hline
    \textcite{topel_1988_Housing} & US (1963-1983) & Instrumental variable & \makecell{$HeP(+), INT(-), INFLA(-),$\\$TIME(-), W(-)$} \\\hline
    \textcite{egebo_1990_MODEL} & 7 OECD countries (1960-1987) & OLS & \makecell{$RDY(+), RINT(-), PRIr(-),$\\$PRSr(+)$} \\\hline
    \textcite{mccarthyMonetaryPolicyTransmission2002} & US (1975-1985 and 1986-2000) & EC-GMM & \makecell{$HP(+), CC(-), SINT(-),$\\$LAND(-), H(-)$} \\\hline
    \textcite{barot_2002_House} & UK and Sweden (1970-1998) & EC model & $Q(+), RINT(-)$ \\\hline
    \textcite{gauger_residential_2003} & US (1959-79 and 1982-99) & VECM and Granger test & $M2, GDP(+), SINT(-), MR(-),$\\\hline
    \textcite{arestis_residential_2015} & 17 OECD countries (1970-2013) & ARDL Model & \makecell{$RHP(+), RDY(-), MR(-),$\\$ Cr(+), UN(-)$}\\\hline
    \textcite{kohlscheen_2018_Residential} & 15 OECD countries (1970-2017) & OLS & \makecell{$Q(+), RHP(+), GDPc(+),$\\$ POP(+), NM(+), H(-)$} \\\hline
    \hline
    \end{tabularx}
    % \end{tabular}
    % } %end resize

    \footnotesize{\textbf{Notes:} Signs inside parenthesis indicates expected effects. \textcite{topel_1988_Housing} use housing starts as the dependend variable; all the other use residential investment ($RI$). \textbf{Covariates dictionary:} $RHP$ Real House Prices; $NCD$ Non-residential construction deflator; $W$ Real Construction Wage; $Cr$ Banking Credit; $HS$ Housing Starts; $HeP$ Hedonic Prices; $INT$ Interest Rates (non-specified); $INFLA$ Expected inflation; $TIME$ Median market time since the begin of construction; $RDY$ Real Disposable Income; $RINT$ Real Interest Rate; $PRIr$ Residential investment relative price; $PRSr$ Residential services relative prices; $HP$ House Prices; $CC$ Construction Costs; $SINT$ Short-term Interest Rate; $LAND$ Land Price; $H$ House Stock; $Q$ Adapted Tobin's Q for the housing sector; $RINT$ Real interest rate; $M2$ Monetary aggregates; $GDP$ Gross Domestic Product; $MR$ Mortgage Interest rate; $UN$ Unemployment rate; $GDPc$ GDP per capita; $POP$ Population Density; $NM$ Net Migration rates (per 1000 population).}
  \end{threeparttable}
    \caption*{\textbf{Source:} Authors' elaboration}
\end{table}
% \end{landscape}


\section{Macroeconometric analysis}
\label{sec:org9f87bd5}
\label{sec:VECM}
\subsection{Houses' own interest rate and residential investment growth rate in the US	Economy}
\label{sec:org183e022}
\label{sc:own}


\textcite{arestis_residential_2015} argue that the last house bubble episode (during the 2000s to the US economy) changed the traditional view on the motivations to acquire housing.
Accordingly to previous view, house was a durable good that renders a service through it lifetime to its owner.
As well, it is a way of investing wealth more accessible to most households than others options, like stock shares.
During the bubble, households would acquire housings only to sell it later with capital gains.
Or, would benefit from renting their properties.

The speculative behavior was disseminated through households (see figure \ref{fig:mort_houses}).
Bottom 50\% of households with wealth increased their acquisition of housing and also their mortgage indebtedness.
This movement supports the ordinary view of the Great Recession as originated by a subprime crisis \cite{mian_consequences_2009}.
The top 1\%, however, also increased their wealth in form of housing, as well as their mortgage debts, corroborating to a ``New narrative'' of the crisis \cite{albanesi_2017_Credit}.
As discussed before, this housing boom was possible because of some changes of housing market regulation and housing finance \cite{federal_deposit_insurance_corporation_savings_1997,mccarthyMonetaryPolicyTransmission2002,wall_too_2010}.

\begin{figure}[H]
	\centering
	\caption{Houses and Mortgage shares evolution by wealth percentile (1989/07=1)}
	\label{fig:mort_houses}
	\includegraphics[width=.9\textwidth]{./figs/Houses_Mortgages.png}
	\caption*{\textbf{Source:} Board of Governors of the Federal Reserve System (US), Authors' Elaboration}
\end{figure}
Next, we describe the relationship between residential investment growth rate (\(g_{I_h}\)) and houses' own interest rate (\(own\)) as proposed by \textcite{teixeira_crescimento_2015}.
Next, we will present the hypothesis to be tested on the Section \ref{sec:estimation}. To obtain this relationship, we deflate mortgage interest rate (\(r_{mo}\)) by real estate inflation (\(\pi\)) as follows:

$$
g_{I_h} = \phi_0 - \phi_1\cdot \overbrace{\left(\frac{1+r_{mo}}{1+\pi} - 1\right)}^{own}
$$

\begin{equation}
g_{I_h} = \phi_0 - \phi_1\cdot own
\end{equation}

where \(\phi_0\) stands for long-term determinants (\emph{e.g.} demographic factors, housing and credit policies, etc.) while \(\phi_1\) captures the demand for real estate arising from expectations of capital gains resulting from speculation with the existing dwellings stock.
This particular real interest rate is the most relevant for households since it is the real cost in real estate from buying real estate  \cite[p.~53]{teixeira_crescimento_2015}.

Figure \ref{propria_investo} shows how this deflation procedure is more adequate than a general price index --- as \textcite[p.~143--6]{fair_macroeconometric_2013} does --- to describe the housing dynamics. It worth noting that during a houses' bubble period, it is real estate inflation that governs own's interest rate dynamics.
Therefore, the lower this rate is, the greater the capital gains (in real estate) for speculating with real estate will be. This negative relation between houses' own interest rate and residential investment is shown in Figure \ref{propria_investo} in which this particular real interest rate has been gradually decreased over the real estate boom (2002-5).

Despite shedding light on some relevant relationships, \citeauthor*{teixeira_crescimento_2015}'s \citeyear{teixeira_crescimento_2015} proposition was not evaluated econometrically and this will be done in Section \ref{sec:estimation}. To do so, we assume the following long-run relationship:

\begin{equation}
g_{I_{h_{t}}} = \phi_0 - \phi_1\cdot own_t
\end{equation}

therefore, if these time-series are co-integrated, we specify the short-run adjustment process through the following VECM:
\begin{equation}
\begin{cases}
\Delta own_t = \delta_{1} + \alpha_1\left(g_{I_{h_{t-1}}} - \phi_0 + \phi_1\cdot own_{t-1}\right) + {\sum^{N}_{i=1}}\beta_{1,i}\cdot \Delta g_{I_{h_{t-i}}} +
\sum^{N}_{i=1}\gamma_{1,i}\cdot \Delta own_{t-i} +\varepsilon_{t,1}
\\
\Delta g_{I_{h_{t}}} = \delta_{2} + \alpha_2\left(g_{I_{h_{t-1}}} - \phi_0 + \phi_1\cdot own_{t-1}\right) + \sum^{N}_{i=1}\beta_{2,i}\cdot \Delta g_{I_{h_{t-i}}} +
\sum^{N}_{i=1}\gamma_{2,i}\cdot \Delta own_{t-i} +\varepsilon_{t,2}
\end{cases}
\end{equation}

where \(\delta_s\) indicate linear trend (level);
\(\alpha_{is}\) are the error correction coefficients;
\(\beta_s\) and \(\gamma_s\) are coefficients associated with lagged \(g_{I_h}\) and \(own\) respectively and; \(\varepsilon_s\) are the residuals.
Based on \textcite{teixeira_crescimento_2015}, we depict the expected results in Table \ref{resultados_esperados} below:

Figure \ref{Fig:CreditFDICIA} illustrates the relevance of institutional reforms due to the savings and loans crisis throughout the 80's and early 90's.
This institutional changes --- notably Financial Institutions Reform, Recovery, and Enforcement Act (FIRREA) in 1989 and Federal Deposit Insurance Corporation Improvement Act  (FDICIA) in 1991 --- increased the credit volume to households.
As a consequence, real estate finance has increased considerably in the following periods.


\begin{figure}[htb]
	\centering
	\caption{Mortgage and Consumer credit growth rate (1979-2019)}
	\label{Fig:CreditFDICIA}
	\includegraphics[width=\textwidth]{./figs/FDICIA.png}
	\caption*{\textbf{Source:} U.S. Bureau of Economic Analysis, Authors' elaboration}
\end{figure}
% Please add the following required packages to your document preamble:
% \usepackage{graphicx}
\begin{table}[H]
	\centering
	\caption{Summary of expected results of the macroeconometric model}
	\label{resultados_esperados}
	\resizebox{\textwidth}{!}{%
		\begin{tabular}{l|l|l}
			\hline\hline
			\textbf{\begin{tabular}[c]{@{}l@{}}Expected\\ Result\end{tabular}} &
			\textbf{Econometric Meaning} &
			\textbf{Economic Meaning} \\ \hline\hline
			\textbf{1. $\varepsilon \sim I(0)$} &
			\begin{tabular}[c]{@{}l@{}} Stationary residuals indicates cointegration relationship\end{tabular} &
			\begin{tabular}[c]{@{}l@{}} Series share a common\\long-run trend\end{tabular} \\ \hline
			\textbf{2. $\alpha_1 = 0$} &
			\begin{tabular}[c]{@{}l@{}} $own$ is weakly exogenous\\ compered to $g_{I_h}$\end{tabular} & \begin{tabular}[c]{@{}l@{}} 
				$own$ dynamics is not affected\\by previous equilibrium deviation\end{tabular}
			\\ \hline
			\textbf{3. $\alpha_2 < 0$} &
			\begin{tabular}[c]{@{}l@{}}Own interest rate Granger-causes\\
				residential investment growth rate\end{tabular} & \begin{tabular}[c]{@{}l@{}} $g_{I_h}$ dynamics is not affected\\ by previous equilibrium deviation\end{tabular}
			\\ \hline
			\textbf{4. $\phi_1 > 0$} &
			\begin{tabular}[c]{@{}l@{}}Series share a common\\negative long-run relationship\end{tabular} &
			\begin{tabular}[c]{@{}l@{}}Own interest rate affects\\residential investment growth rate negatively\end{tabular} \\ \hline
			\textbf{5. $\phi_0 < 0$} &
			\begin{tabular}[c]{@{}l@{}}
			Real estate demand for non-speculation\\reasons is statistically significant
			\end{tabular} &
			\begin{tabular}[c]{@{}l@{}}
				Real estate demand associated with\\institutional particularities and demographic\\ changes affects residential investment\\growth rate positively\end{tabular} \\ \hline
			\textbf{6. $\gamma_{2,is} < 0$} &
			\begin{tabular}[c]{@{}l@{}}Residential investment growth rate\\coefficient is statistically significant\end{tabular} &
			\begin{tabular}[c]{@{}l@{}}Own interest rate affects\\$g_{I_h}$ in the short-run\end{tabular} \\ \hline
			\textbf{7. $\beta_{1,is} = 0$} &
			\begin{tabular}[c]{@{}l@{}}
				$g_{I_h}$ effects over own interest\\ rate is not statistically significant\end{tabular} &
			\begin{tabular}[c]{@{}l@{}}
				$g_{I_h}$ effects over own interest\\ rate is negligible since dwellings stock is much\\bigger than residential investment (flow)\end{tabular} \\ \hline\hline
		\end{tabular}%
	}
\caption*{\textbf{Source:} Authors' elaboration}
\end{table}


\subsection{Data and estimation strategy}
\label{sec:org9162229}
\label{sec:estimation}


In this section, we employ a model to test whether or not houses own rate of interest describes residential investment growth rate dynamics\footnote{Scripts are available under request.}.
Our sample period (1992:Q1 to 2019:Q1) starts after institutional changes (FDIC e
FIRREA) due to the Savings and Loans crisis (see Table \ref{structbreak} in appendix \ref{appen:A} for structural break tests).
We rely on the following  quarterly seasonally adjusted data: (i) 30-Year fixed mortgage interest rate (MORTGAGE30US, resampled by end of period), private residential investment (PRFI, growth rate as percent change from the previous quarter) and Case-Shiller home price index
(CSUSHPISA, resampled by end of period). Figure \ref{propria_investo}  shows the original series.

\begin{figure}[htb]
	\centering
	\caption{Residential investment growth rate vs. Houses Own interest rate}
	\label{propria_investo}
	\includegraphics[width=\textwidth]{./figs/TxPropria_Investo.png}
	\caption*{\textbf{Source:} U.S. Bureau of Economic Analysis, Authors' elaboration}
\end{figure}


Next, we applied \textcite{yeo_new_2000} transformation since these series are volatile. We use this procedure instead of a standard \textcite{box_analysis_1964} transformation  because it can be applied to non-positive values.
Then, we employ standard unit root tests (see Table \ref{unitroot} in appendix \ref{appen:A}) as well as \textcite{johansen_estimation_1991} procedure to assess whether houses' own interest rate and residential investment growth rate share a common long-run trend (see Table \ref{Johansen} in appendix \ref{appen:A}).
Our series are co-integrated at 5\% significance level which allows us to estimate a error correction model and evaluate the previous hypothesis \cite{enders_applied_2014}.

\begin{figure}[htb]
	\centering
	\caption{Time-series with \textcite{yeo_new_2000} transformation}
	\label{YeoJhonson}
	\includegraphics[width=\textwidth]{./figs/YeoJohnson_All.png}
	\caption*{\textbf{Source:} U.S. Bureau of Economic Analysis, Authors' elaboration}
\end{figure}




The next step is to define the model order. According to usual information criteria, both first and forth lags are eligible (see Table \ref{criterios} in appendix \ref{appen:A}).
Although parsimonious, we argue that the first lag has no empirical support.
Considering the average construction time (from approval to completion), we should include at least the second lag in order to incorporate homes built for capital gains purposes which only take place once the construction is completed (see Figure \ref{meses}).

\begin{figure}[H]
	\centering
	\caption{Average construction time (approval to completion) of properties for a family unit by construction purposes except manufactured houses (1976-2018)}
    \label{meses}
	\includegraphics[width=\textwidth]{./figs/Meses_construcao.png}
	\caption*{\textbf{Source:} Survey of Construction (SOC), Authors' elaboration}
\end{figure}

This procedure, however, it is not enough to determine the model lag order selection.
Since residential investment (newly constructed homes) is significantly smaller than existing housing stock, we verify an price-effect even if the construction is unfinished\footnote{\textcite{poterba_tax_1984}, for instance, suggests that the time that houses takes to be sold should be include.}.
We argue that this price effect is a result of future real estate inflation.
Such dynamic could be captured by the \textbf{expected} houses' own interest rate.
However, such series does not exist.
So, we use lagged houses' own interest rate as a first approximation to the expected one\footnote{This procedure is similar to \textcite{keynes_general_1937} ``practical theory of the future'' in which decision-making process for buying a new property depends on expectations/conventions based on past observations.
In summary, in the absence of a series for the expected own interest rate, the lag of this variable will be used as a proxy for the future one.}.

In order to display the relation between lagged own interest rate and current residential investment growth rate, Figure \ref{defasagens} depicts one variable of interest against the other variable lagged according to lags that minimize the information criteria (1 and 4 respectively)\footnote{Similar data plot can be seen in \textcite[p.~16]{girardi_autonomous_2015}.}.
This simple procedure allows checking if there is any relationship between the expected own interest rate (in this case, lagged effective rate) and residential investment growth rate\footnote{In order to consider non-linearities, we presented quadratic regression between variables of interest.}.
In the same Figure, we verify the non-occurrence of the inverse relationship from residential investment to own interest rate.
Since residential investment (flow) is much lower than the existing stock of dwellings, it is expected that such relationship does not exist.
In summary, speculation with the dwellings stock generates inflation of these assets, which affects the construction of new houses (flow) and not the other way round\footnote{It is worth noting a particular aspect of house price formation: land scarcity. As a consequence, speculation with residences is, in the end, speculation with land (the only scarce resource involved in its production) and, therefore, it is relevant for speculation with the dwellings stock. 
	\textcite[p.~349, emphasis added]{leamer_housing_2007} points out this particularity as follows:
	\begin{quotation}
		It’s not the structure that has a volatile price; \textbf{it's the land}. Where there is plenty of buildable land, the response to an increase is demand for homes is mostly to build more, not to increase prices. Where there is little buildable land, the response to an increase in demand for homes is mostly a price increase, sufficient to discourage buyers enough to reequilibrate the supply and demand.
	\end{quotation}}.

\begin{figure}
	\centering
	\caption{Dispersion between houses' own interest rate and residential investment growth: lags selected based on information criteria}
	\label{defasagens}
	\includegraphics[height=.4\textheight]{./figs/VEC_Defasagens.png}
	\caption*{\textbf{Source:} Authors' elaboration}
\end{figure}

Considering this theoretical and econometric discussion of model order specification, we estimate a four lag VEC  (see Table \ref{Estimacao})\footnote{In addition to being theoretically based, this lag also generates homoscedastic residuals without serial auto-correlation (see Table \ref{testes_resduos} in Appendix \ref{appen:A}).}. 
% Figure ref:residuos displays an inspection of the residuals while
Table \ref{testes_resduos} in Appendix \ref{appen:A} presents a few residual tests to check the model's specification while Table \ref{tab:robust} presents some robustness check.
On the following subsection, we analyze the results and compare with the theoretical expected ones presented in Table \ref{resultados_esperados} above.  




\subsection{Estimation results}
\label{sec:org55c0725}
\label{sec:results}

According to parameters presented in Table \ref{Estimacao}, we find statistically significant co-integration  coefficients for both equations. 
Therefore, both variables share a (negative) long-run trend (validating hypotheses 1 and 4).
The short-term relationship between \(own\) and \(g_{Ih}\) (\(\beta_{1, is}\) coefficients) are not statistically significant at 5\%\footnote{The expected result (7) can also be validated from the inspection of Table \ref{Estimacao} in which only the fourth lag of own interest rate equation is statistically significant.}.
In addition, coefficients \(\gamma_{2,s}\) are negative and statistically significant at 5\%, supporting hypothesis 6 (see Table  \ref{Estimacao}).
We also find statistically significant coefficients related to demand for houses for non-speculative reasons (\(\phi_0\)), validating proposition 5.
On the other hand, the error correction parameter is statistically significant only for the residential investment growth rate equation.
In this sense, \(own\) is weakly exogenous compared to \(g_{I_h}\) while houses' own interest rate Granger-causes \(g_{I_h}\), supporting the hypothesis (2) and (3).
In conclusion, our estimation results are in line with the hypothesis presented above and can be summarized as follows: houses' own interest rate determines --- but is not determined by --- residential investment growth rate and these variables present a negative long-term relationship (are co-integrated).

\begin{table}[h!]
	\caption{Estimation parameters}
	\centering
	\begin{tabular}{lrrrrr}
\toprule
{} &  Base scenario &  $\Delta \phi_0$ &  $\Delta \omega$ &  $\Delta rm$ &  $\pi$ \\
\midrule
$\alpha$      &         0.5000 &           0.5000 &           0.5000 &       0.5000 & 0.5000 \\
$\gamma_F$    &         0.0800 &           0.0800 &           0.0800 &       0.0800 & 0.0800 \\
$\gamma_u$    &         0.0900 &           0.0900 &           0.0900 &       0.0900 & 0.0900 \\
$\omega$      &         0.5000 &           0.5000 &           0.4900 &       0.5000 & 0.5000 \\
$rm$          &         0.0100 &           0.0100 &           0.0100 &       0.0200 & 0.0100 \\
$\sigma_{l}$  &         0.0000 &           0.0000 &           0.0000 &       0.0000 & 0.0000 \\
$\sigma_{mo}$ &         0.0000 &           0.0000 &           0.0000 &       0.0000 & 0.0000 \\
$u_N$         &         0.8000 &           0.8000 &           0.8000 &       0.8000 & 0.8000 \\
$v$           &         1.2000 &           1.2000 &           1.2000 &       1.2000 & 1.2000 \\
$\phi_0$      &         0.0250 &           0.0300 &           0.0250 &       0.0250 & 0.0250 \\
$\phi_1$      &         0.1000 &           0.1000 &           0.1000 &       0.1000 & 0.1000 \\
$R$           &         0.7000 &           0.7000 &           0.7000 &       0.7000 & 0.7000 \\
$\pi$         &         0.0000 &           0.0000 &           0.0000 &       0.0000 & 0.0500 \\
\bottomrule
\end{tabular}

	\caption*{\textbf{Source:} Authors' elaboration}
\end{table}
Figure \ref{fevd} display the forecast error variance decomposition (FEVD) which reports houses own interest rate in describing residential investment growth rate dynamics\footnote{It is important to note that the number of variables (two) used generates similar  of a Structural VEC, which means that Choleski's decomposition is sufficient to analyze the (orthogonalized) impulse response function.}.
We report that own interest rate has  depicted  \(g_{Ih}\) --- while the reverse is not valid --- after the first quarter.
In addition, we find that such contribution is greater than 50\% beyond the third quarter.
Therefore, houses' own interest rate is explained mainly by itself and explains \(g_{I_h}\) considerably.

\begin{figure}[H]
	\centering
	\caption{Forecast error variance decomposition (FEVD)}
	\label{fevd}
	\includegraphics[width=.9\textwidth]{./figs/FEVD_VECMpython_TxPropria.png}
	\caption*{\textbf{Source:} Authors' elaboration}
\end{figure}

Next, we analyze the orthoganilized impulse response function (Figure \ref{irf}).
In summary, we report a stable system since the increase in \(g_{I_h}\) on itself are dampened over time while equivalent shock on own interest rate has a non-explosive permanent effect.
On the other hand, an increase in \(g_{I_h}\) has a null effect over \(own\).
The most relevant result reported in Figure \ref{irf} is the considerable and lasting negative effect due to an increase in own interest rate over \(g_{I_h}\), validating \citeauthor*{teixeira_crescimento_2015}'s \citeyear{teixeira_crescimento_2015} proposition.
In short, our results shows that an increase in mortgage interest rate (equivalent to an increase in houses' own interest rate) has a negative and persistent effect on residential investment growth rate while an increase in real estate inflation  has an opposite effect.
\begin{figure}[H]
	\centering
	\caption{Orthogonalized Impulse Response Function}
	\label{irf}
	\includegraphics[height=.4\textheight]{./figs/Impulse_VECM.png}
	\caption*{\textbf{Source:} Authors' elaboration}
\end{figure}

\subsection{Comparison with previous studies}
\label{sec:orgf502b0a}

In this section,  we contrast our findings with those obtained by the literature.
At this stage, we restrict the comparison with  \textcite{gauger_residential_2003} and \textcite{arestis_residential_2015} since we share the same topic.
Similar to \textcite{gauger_residential_2003}, we report that mortgage interest rate is relevant for residential investment.
Despite some theoretical differences, our estimations are in line with \textcite{arestis_residential_2015} (at least for the US): house prices are relevant to describe residential investment growth rate.
However, they report insignificant coefficients for mortgages nominal interest rate which is at odds with our conclusions.
In summary, our estimation reports that houses' own interest rate has a prominent role in describing residential investment growth rate movements. 
It is worth noting that despite the amplitude of VEC order, our model is parsimonious considering the number of variables used.
Thus, we conclude that our estimation depicts residential investment growth rate satisfactorily.
On the following section we present some concluding remarks.

\section{Concluding Remarks}
\label{sec:org3d696b6}
\label{sec:Conclusion}
In this article, we present a residential investment growth rate specification compatible with the Srrafian supermultiplier model.
To do so, we estimate a bi-dimension VEC evaluate \textcite{teixeira_crescimento_2015}
proposal. 
We report: 
	(i) Houses' own interest rate (\(own\)) and residential investment growth rate (\(g_{I_h}\)) share a common long-run trend;
	(ii) \(g_{I_h}\) effects over \(own\) are negligible and; 
	(iii) own interest rate has a negative effect on \(g_{I_h}\) and is its main determinant (see Figure \ref{fevd}).
Besides being parsimonious, our estimations does not show residuals serial autocorrelation and heteroscedasticity. Thus, our results are quite satisfactory.

It remains to contrast our findings with those obtained by \textcite{arestis_residential_2015}.
It worth remembering that one of the authors' hypotheses is that residential investment depends on disposable income (is induced expenditure).
However, the authors themselves find that such results are not statistically significant for the US. Therefore, we can compare this result with our model.
Despite the differences, some results of the model are in line with those of \textcite{arestis_residential_2015}.
Among them, house prices relevance in determining residential investment dynamics for the US.
However, they report insignificant coefficients for mortgages nominal interest rate, that is, the opposite conclusion of our model.

In conclusion,  we report lack of work analyzing residential investment in a Sraffian supermultiplier-friendly framework in the macroeconometric literature.
Our estimation supports houses' own interest rate relevance in describing residential investment growth rate for the US as depicted by \textcite{teixeira_crescimento_2015}.
Thus, our  proposal differs from the usual empirical literature by:
	(i) considering housing as a non-capacity creating autonomous expenditure;
	(ii) reporting that mortgage interest rates are relevant to describe long-run residential investment dynamics; and notably 
	(iii) including asset bubble through houses own interest rate.


\section*{Acknowledgments}
\label{sec:org0ca90a3}
\noindent The authors wish to acknowledge the financial support from the Brazilian National Research Council (CNPq; grant 130777/2018-8). We are grateful to Rosângela Ballini, Carolina Baltar, Júlia Braga, Ítalo Pedrosa, Cecon/Unicamp and UFRJ Macroeconomic discussion groups for useful comments and suggestions on earlier drafts of this article. All remaining errors are, of course, our own.


\section*{Disclosure statement}
\label{sec:org4373a36}
No potential conflict of interest was reported by the authors.

\section*{References}
\label{sec:org6a4a0f7}
\printbibliography[heading=none]


\appendix
\section{Statistical Appendix}
\label{sec:orgaa61a4d}
\label{appen:A}

In this appendix we report several tests: unit root tests on our variables of interest, Structural break test, Johansen procedure and hypothesis tests on residuals. 
The time-series of residential investment growth rate, mortgage interest rate and real estate inflation are all taken using \textcite{yeo_new_2000} transformation.  
As shown in Table \ref{unitroot}, the null hypothesis of a unit root in the first differences of the series is overwhelmingly rejected.
As mentioned, we found structural breaks related to institutional changes and our series are co-integrated (see Tables \ref{structbreak} and \ref{Johansen}).
Our estimation order selection was based both on statistical and theoretical reasoning with homoscedasticity residuals (see Tables \ref{criterios} and \ref{testes_resduos}).
All expected results are statistically significant and most of them are not restrict to model specification (see Table \ref{tab:robust}).

% Please add the following required packages to your document preamble:
% \usepackage{multirow}
% \usepackage{graphicx}
\begin{table}[H]
	\centering
	\caption{Unit root tests}
	\label{unitroot}
	\resizebox{\textwidth}{!}{%
	\begin{threeparttable}
		\begin{tabular}{l|l|cccccccc}
\hline\hline
\multicolumn{2}{l|}{\multirow{2}{*}{\textbf{Variable}}} & \multicolumn{2}{c}{\textbf{ADF}\tnote{a}} & \multicolumn{2}{c}{\textbf{Zivot Andrews}\tnote{b}} & \multicolumn{2}{c}{\textbf{Phillips Perron}\tnote{a}} & \multicolumn{2}{c}{\textbf{KPSS}\tnote{c}} \\ \cline{3-10} 
\multicolumn{2}{l|}{} & \multicolumn{1}{l}{Statistic} & \multicolumn{1}{l}{p-value} & \multicolumn{1}{l}{Statistic} & \multicolumn{1}{l}{p-value} & \multicolumn{1}{l}{Statistic} & \multicolumn{1}{l}{p-value} & \multicolumn{1}{l}{Statistic} & \multicolumn{1}{l}{p-value} \\ \hline
\textbf{Residential} & level &-3.333&0.013&-4.439&0.139&-6.165&0.000&0.181&0.309\\
\textbf{investment ($g_{I_h}$)} & first difference &-7.155&0.000&-7.739&0.000&-20.346&0.000&0.106&0.558\\ \hline
% \textbf{Real estate} & level &-2.671&0.079&-4.871&0.043&-2.704&0.073&0.148&0.395 \\
% \textbf{inflation} & first difference &-4.680&0.000&-6.122&0.001&-11.340&0.000&0.059&0.819 \\ \hline
\textbf{Houses own-rate of interest} & level &-2.330&0.162&-4.203&0.237&-2.425&0.135&0.690&0.014 \\
\textbf{rate of interest}& first difference &-5.087&0.000&-6.340&0.000&-10.408&0.000&0.062&0.804\\ \hline
% \textbf{Mortgage} & level &-3.638&0.027&-4.494&0.215&-3.604&0.030&0.081&0.264 \\
% \textbf{interest rate}& first difference &-8.050&0.000&-8.144&0.000&-11.127&0.000 &0.034&0.962 \\
\hline\hline
\end{tabular}%
\begin{tablenotes}\footnotesize
	\item [a] H0: has a unit root.
	\item [b] H0: has a unit root and a structural break.
	\item [c] H0: series is weakly stationary.
\end{tablenotes}
\end{threeparttable}
	}
\caption*{\textbf{Source:} Authors' elaboration}
\end{table}

% Please add the following required packages to your document preamble:
% \usepackage{multirow}
% \usepackage{graphicx}
\begin{table}[H]
	\centering
	\caption{Structural break test}
	\label{structbreak}
	\begin{threeparttable}
	%\resizebox{\textwidth}{!}{%
		\begin{tabular}{l|l|cc}
			\hline \hline
			\multirow{2}{*}{\textbf{Variable}} & \multirow{2}{*}{\textbf{Break}} & \multicolumn{2}{c}{\textbf{Chow test}\tnote{a}} \\ \cline{3-4} 
			&& Statistic & p-value \\ \hline
			\multirow{3}{*}{\textbf{Residential investment ($g_{I_h}$)}} & 1991/Q3 & 5.1147 & 0.0254 \\
			& 2005/Q4 & 7.286 & 0.007881 \\
			& 2010/Q3 & 6.1013 & 0.01481 \\ \hline
			\multirow{5}{*}{\textbf{Houses own-rate of interest}} & 1991/Q3 & 63.453 & 7.487e-13 \\
			& 1996/Q3 & 107.47 & \textless 2.2e-16 \\
			& 2001/Q2 & 78.378 & 5.662e-15 \\
			& 2006/Q1 & 20.68 & 1.236e-05 \\
			& 2011/Q1 & 78.969 & 4.663e-15 \\ \hline
			% \multirow{4}{*}{\textbf{Mortgage interest rate}} & 1991/Q3 & 124.35 & \textless 2.2e-16 \\
			% & 1997/Q1 & 199.25 & \textless 2.2e-16 \\
			% & 2002/Q1 & 301.18 & \textless 2.2e-16 \\
			% & 2009/Q4 & 172.97 & \textless 2.2e-16 \\ \hline
			% \multirow{3}{*}{\textbf{Real estate inflation}} & 1997/Q3 & 1.5508 & 0.2153 \\
			% & 2005/Q4 & 23.49 & 3.569e-06 \\
			% & 2011/Q3 & 4.4981 & 0.03586 \\
			\hline \hline
		\end{tabular}%
	%}
	\begin{tablenotes}\footnotesize
		\item [a] H0: There is no structural break.
	\end{tablenotes}
\end{threeparttable}
	\caption*{\textbf{Source:} Authors' elaboration}
\end{table}

% Please add the following required packages to your document preamble:
% \usepackage{multirow}
% \usepackage{graphicx}
\begin{table}[h]
\centering
\caption{Cointegration test}
\label{Johansen}
\begin{threeparttable}
%\resizebox{\textwidth}{!}{%
\begin{tabular}{l|l|c|c}
\hline
 \hline
\multirow{2}{*}{\textbf{Model specification}} & \multirow{2}{*}{\textbf{Hypothesis}} & \multicolumn{2}{c}{\textbf{Johansen Procedure\tnote{a}}} \\ \cline{3-4} 
 &  & \multicolumn{1}{c|}{Statistic} & critical value (5\%) \\ \hline
\multirow{3}{*}{\textbf{$g_{I_h}$, Own interest rate}} & $r = 0$ &22.51&19.96\\
 & $r = 1^*$ &2.91&9.24\\\hline	
\multirow{4}{*}{\textbf{$g_{I_h}$, Inflation and Mortgage interest rate}} & $r = 0$ &46.05&34.91\\
 & $r = 1^*$ &15.08&19.96\\
 & $r = 2$ &6.44&9.24\\\hline
\multirow{3}{*}{\textbf{$g_{I_h}$, Inflation and exogenous  mortgages interest rate}} & $r = 0$ &36.88& 19.96\\ 
 & $r = 1^*$ &7.87&9.24\\ 
  \hline
\end{tabular}%
%}
\footnotesize{(a) Using trace test with constant for the 5th lag (according to AIC criteria). (*) Indicates the selected rank that implies cointegration.}
\end{threeparttable}
\caption*{\textbf{Source:} Authors' elaboration}
\end{table}
\begin{table}
\caption{Selection model order (* indicates the minimum)}
\label{criterios}
\centering
\begin{tabular}{lcccc}
\toprule
            & \textbf{AIC} & \textbf{BIC} & \textbf{FPE} & \textbf{HQIC}  \\
\midrule
\textbf{0}  &      -13.86  &      -13.71  &   9.553e-07  &       -13.80   \\
\textbf{1}  &      -14.71  &     -14.46*  &   4.102e-07  &       -14.61   \\
\textbf{2}  &      -14.72  &      -14.38  &   4.038e-07  &       -14.58   \\
\textbf{3}  &      -14.74  &      -14.29  &   3.983e-07  &       -14.56   \\
\textbf{4}  &      -14.88  &      -14.34  &   3.442e-07  &      -14.66*   \\
\textbf{5}  &     -14.89*  &      -14.24  &  3.425e-07*  &       -14.63   \\
\textbf{6}  &      -14.84  &      -14.09  &   3.612e-07  &       -14.54   \\
\textbf{7}  &      -14.82  &      -13.97  &   3.696e-07  &       -14.47   \\
\textbf{8}  &      -14.77  &      -13.82  &   3.891e-07  &       -14.38   \\
\textbf{9}  &      -14.75  &      -13.71  &   3.961e-07  &       -14.33   \\
\textbf{10} &      -14.70  &      -13.56  &   4.189e-07  &       -14.24   \\
\textbf{11} &      -14.67  &      -13.43  &   4.308e-07  &       -14.17   \\
\textbf{12} &      -14.63  &      -13.29  &   4.535e-07  &       -14.08   \\
\textbf{13} &      -14.66  &      -13.22  &   4.406e-07  &       -14.08   \\
\textbf{14} &      -14.63  &      -13.09  &   4.582e-07  &       -14.00   \\
\textbf{15} &      -14.59  &      -12.95  &   4.806e-07  &       -13.92   \\
\bottomrule
\end{tabular}
\caption*{\textbf{Source:} Authors' elaboration}
\end{table}
% Please add the following required packages to your document preamble:
% \usepackage{multirow}
% \usepackage{graphicx}
\begin{table}[H]
\centering
\caption{Hypothesis tests on residuals}
\label{testes_resduos}
	\begin{threeparttable}
\begin{tabular}{l|c|c|c}
\hline
\multicolumn{2}{l|}{} & \textbf{Statistic} & \textbf{p-value} \\ \hline
\textbf{Serial autocorrelation}\tnote{a} & System & 54.51 & 0.093 \\ \hline
\multirow{2}{*}{\textbf{Homoscedasticity}\tnote{b}} & $own$ & 1.863 & 0.175 \\ \cline{2-4} 
 & $g_{I_h}$ & 3.080 & 0.082 \\ \hline
\textbf{Normality}\tnote{c} & System & 46.64 & 0.000 \\ \hline
\end{tabular}%
\begin{tablenotes}\footnotesize
	\item [a] Adjusted Portmanteau tested until up to 15th \textit{lag}. H0: autocorrelations up to the selected lag equal zero.
	\item [b] ARCH-LM test. H0: Residuals are homoscedastic.
	\item [c] Jarque-Bera test. H0: data generated by normally-distributed process.
\end{tablenotes}
\end{threeparttable}
\caption*{\textbf{Source:} Authors' elaboration}
\end{table}
% Please add the following required packages to your document preamble:
% \usepackage{multirow}
% \usepackage{graphicx}
\begin{table}[]
\centering
\caption{Robustness check}
\label{tab:robust}
\resizebox{\textwidth}{!}{%
\begin{tabular}{c|c|c|c|c|c|c|c|c}
\hline\hline
\multirow{2}{*}{\textbf{Lags}} &
  \multirow{2}{*}{\textbf{\begin{tabular}[c]{@{}c@{}}Information\\ Criteria\end{tabular}}} &
  \multirow{2}{*}{\textbf{\begin{tabular}[c]{@{}c@{}}Reject $H_0$ for Adjusted \\ Portmanteau test?\\ (zero residual auto corr. )\end{tabular}}} &
  \multicolumn{2}{c|}{\textbf{\begin{tabular}[c]{@{}c@{}}Reject $H_0$ for\\ ARCH-LM test?\\ (Homoscedasticity)\end{tabular}}} &
  \multirow{2}{*}{\textbf{\begin{tabular}[c]{@{}c@{}}Reject $H_0$ for\\ Jarque-Bera test?\\ (Normality)\end{tabular}}} &
  \multicolumn{2}{c|}{\textbf{\begin{tabular}[c]{@{}c@{}}Reject $H_0$\\ for KPSS \\ Stationarity test?\end{tabular}}} &
  \multirow{2}{*}{\textbf{\begin{tabular}[c]{@{}c@{}}Confirmed\\hypothesis\end{tabular}}} \\ \cline{4-5} \cline{7-8}
   &                &     & \textbf{$g_{I_h}$} & \textbf{$own$} &     & \textbf{$g_{I_h}$} & \textbf{$own$} &                       \\ \hline\hline
0  & BIC            & yes & yes                & yes            & yes & no                 & no             & 1,2, 3 and 5          \\
1  & $-$            & yes & no                 & yes            & yes & no                 & no             & all                   \\
2  & $-$            & yes & no                 & yes            & yes & no                 & no             & all                   \\
3  & $-$            & yes & no                 & yes            & yes & no                 & no             & all                   \\
4  & AIC, FPE, HQIC & no  & no                 & no             & yes & no                 & no             & all                   \\
5  & $-$            & no  & no                 & no             & yes & no                 & no             & all                   \\
6  & $-$            & no  & no                 & no             & yes & no                 & no             & all                   \\
7  & $-$            & yes & no                 & no             & yes & no                 & no             & all                   \\
8  & $-$            & yes & no                 & no             & yes & no                 & no             & all except 7          \\
9  & $-$            & yes & no                 & no             & yes & no                 & no             & all except 2 and 7    \\
10 & $-$            & yes & no                 & no             & yes & no                 & no             & all except 2, 3 and 7 \\
11 & $-$            & no  & yes                & no             & yes & no                 & no             & all except 7          \\
12 & $-$            & yes & no                 & no             & yes & no                 & no             & all except 2 and 7    \\ \hline\hline
\end{tabular}%
}
\caption*{\textbf{Source:} Authors' elaboration}
\end{table}

\end{document}

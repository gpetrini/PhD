% Created 2021-02-01 seg 12:44
% Intended LaTeX compiler: pdflatex
\documentclass[12pt, a4]{article}
\usepackage[utf8]{inputenc}
\usepackage{lmodern}
\usepackage[T1]{fontenc}
\usepackage[top=3cm, bottom=2cm, left=3cm, right=2cm]{geometry}
\usepackage{graphicx}
\usepackage{longtable}
\usepackage{float}
\usepackage{wrapfig}
\usepackage{rotating}
\usepackage[normalem]{ulem}
\usepackage{amsmath}
\usepackage{textcomp}
\usepackage{marvosym}
\usepackage{wasysym}
\usepackage{amssymb}
\usepackage{amsmath}
\usepackage[theorems, skins]{tcolorbox}
\usepackage[style=abnt,noslsn,extrayear,uniquename=init,giveninits,justify,sccite,
scbib,repeattitles,doi=false,isbn=false,url=false,maxcitenames=2,
natbib=true,backend=biber]{biblatex}
\usepackage{url}
\usepackage[cache=false]{minted}
\usepackage[linktocpage,pdfstartview=FitH,colorlinks,
linkcolor=blue,anchorcolor=blue,
citecolor=blue,filecolor=blue,menucolor=blue,urlcolor=blue]{hyperref}
\usepackage{attachfile}
\usepackage{setspace}
\usepackage{tikz}
\addbibresource{ref.bib}
\usepackage{svg, caption, multirow, booktabs, tabularx, subfigure, subcaption, lscape, tablefootnote, threeparttable}
\newcolumntype{b}{>{\hsize=2.3\hsize}X}
\newcolumntype{s}{>{\hsize=.45\hsize}X}
\newcolumntype{m}{>{\hsize=.9\hsize}X}
\usepackage{authblk}
\usepackage[english]{babel}
\author[1]{Gabriel Petrini}
\affil[1]{PhD Student at University of Campinas (Brazil), Email: \url{gpetrinidasilveira@gmail.com}} % Author affiliation
\author[2]{Lucas Teixeira}
\affil[2]{Assistent Professor at University of Campinas (Brazil), Email: \url{lucastei@unicamp.br}} % Author affiliation
\author[3]{Frankin Serrano}
\affil[3]{Professor at Federal University of Rio de Janeiro (Brazil), Email: \url{}} % Author affiliation
\date{\today}
\title{Untitled}
\begin{document}

\maketitle
\begin{abstract}
This article investigates the relationship between residential investment, asset inflation, and macroeconomic dynamics based on
the US post-deregulation case (1992-2019). To do so, we estimate a vector error correction model (VECM) to asses the relevance of
real interest rate of real estate. We find residential investment growth rate and houses’ own interest rate to be cointegrated. We
also find that that long-run causality goes unidirectionally from houses’ own interest rate to residential investment growth rate,
as expected. In summary, our findings support the relevance of houses’ own interest rate in determining residential investment
while the other way round does not occur.

\noindent \textbf{Keywords:}
\end{abstract}


\section{Introduction}
\label{sec:org5207b1c}
\label{sec:Introduction}
Among aggregate demand expenditures, non-residential investment is the most examined  one between (at least) heterodox macroeconomists.
As a consequence, the relevance of others (autonomous) expenditures on macroeconomic dynamics has been underestimated \cite{brochier_macroeconomics_2017}.
The Sraffian supermultiplier (SSM) model presented by \textcite{serrano_long_1995} establishes a prominent role for non-capacity creating autonomous expenditures in the theoretical ground.
Despite the late interest in those expenditures \cites{freitas_pattern_2013}{girardi_long-run_2016}{girardi_autonomous_2018}{braga_investment_2018}, there still is a lack of studies on the role of residential investment in particular\footnote{Except for \textcite{green_follow_1997} and \textcite{leamer_housing_2007} --- which shows the relevance of this expenditure to US business cycle at least since the post-war period ---, most of those studies were published after the Great Recession (2008-2009).}. 

Our main objective is to assess the determinants of residential investment growth rate.
We argue that the scarce attention that residential investment receives is not compatible with its relevance for the US and its significance is not restricted to the Great Recession.
In Section \ref{sec:Stylized_Facts}, we present some stylized facts for the US economy highlighting the relevance of residential investment.
Next, in Section \ref{sec:empirical_review}, we will present and compare different macroeconometric models that explicitly incorporate residential investment.
In Section \ref{sec:VECM}, we estimate a bi-dimensional vector error-correction model (VECM) using time-series data for the US economy from 1992 onward to test the houses' own interest rate presented by \textcite{teixeira_crescimento_2015}. 
Section \ref{sec:Conclusion} offers some concluding remarks.
The results for all statistical tests are provided in Appendix \ref{appen:A}.



\section{Housing Dynamics and Business Cycle in the US Economy}
\label{sec:orge6e2d42}
\label{sec:Stylized_Facts}
After the Great Recession, the literature has analyzed the relevance of housing at macroeconomic level \cites{leamer_housing_2015}{teixeira_crescimento_2015}{fiebiger_semi-autonomous_2018}.
Despite some authors had highlighted the empirical importance of this expenditure in determining economic cycles since the post-war period \cites{grebler_capital_1956}{green_follow_1997}{leamer_housing_2007}, only a few macroeconomists have given due attention to this regularity.
\textcite{duesenberry_investment_1958} was an exception and reported the relevance of residential investment and real estate inflation in
determining the economic cycle long before the Great Recession.
\textcite{keynes_collected_1978} is another example and --- despite dedicating himself to the firms' investment --- wrote to President Roosevelt about the relevance of real estate for economic recovery in the context of the Great Depression:

\begin{quote}
    ``[\ldots{}] Housing is by far the best aid to recovery because of the large and continuing scale
of potential demand; because of the wide geographical distribution of this demand; and
because the sources of its finance are largely independent of the stock exchanges. I should
advise putting most of your eggs in this basket, caring about this more than about anything,
and making absolutely sure that they are being hatched without delay. In this country we
partly depended for many years on direct subsidies. There are few more proper objects for
such than working-class houses. If a direct subsidy is required to get a move on (we gave
our subsidies through the local authorities), it should be given without delay or hesitation.''
\cite[p.~436]{keynes_collected_1978}
\end{quote}

As the above excerpt suggests, the relevance of housing is not restricted to the Great Recession nor the USA case.
Moreover, however small its share on GDP is (see Figure \ref{vol_share} A), it does not imply that it has negligible effects on the business cycle or low volatility (see Figure \ref{vol_share} B).

\begin{figure}[H]
    \centering
	\caption{Housing's Particular Stylized Facts}
	\label{fig:vol_share}
\begin{figure}[htb]
    \includegraphics[width = \textwidth]{./figs/Volatility_share.png}
    \end{figure}
	\caption*{\textbf{Source:} U.S. Bureau of Economic Analysis, Authors' Elaboration}
\end{figure}



In order to depict the relation between housing and business cycle, we present Figure \ref{FigIh_u}  in which each cycle is represented in a different panel\footnote{This similar reasoning can be found in \textcites{fiebiger_trend_2017}. Unlike them, we plot only residential investment without including other households expenses financed by credit.}. 
The vertical axis represents residential investment-GDP ratio and the horizontal
axis represents the rate of capacity utilization as a proxy for business cycle.
Economic recovery is generally characterized residential investment growing faster than GDP --- with the 1991-2000 period being a particular case. 
As a consequence of this higher growth rate, is the increase of both residential investment share on GDP and capacity utilization. 
Following the Sraffian supermultiplier growth model, we conclude that increase of non-residential investment is the result of capital stock adjustment principle.
This increase implies GDP to grow faster than residential investment, therefore reducing both its share on GDP and capacity utilization ratio. 
Finally, as a result of economic burst, capacity utilization ratio falls and the cycle.

\begin{figure}[H]
	\centering
	\caption{Residential investment share on GDP VS. capacity utilization during recessions}
	\label{FigIh_u}
	\includegraphics[width=\textwidth]{./figs/Ciclo_Ih_u.png}
	\caption*{\textbf{Source:} Authors' Elaboration}
\end{figure}

We also report an indirect relation between housing and aggregate demand. 
Real estate constitutes a significant portion of household wealth so houses serves as collateral to borrowing \cite{teixeira_uma_2011}. 
As a consequence of US institutional arrangement, households --- especially the poorest ones --- could increase their indebtedness as houses prices went up (see Figure \ref{FigDividaPreco}) as a way to ``make'' capital gains without 
selling their homes during house bubble of the 2000s \cite{teixeira_crescimento_2015}.
Therefore, real estate inflation and durable goods consumption are connected and has relevant consequences for business cycle.
\textcites{zezza_u.s._2008}{barba_rising_2009}, for example, report that credit-financed consumption was one of the main drivers of economic growth before the Great Recession.


In this paper, we argue that this relation between households indebtedness and real estate inflation has other relevant implications.
The first one is the increasing gap between assets and liabilities in the course of the Great Recession.
This dynamic is due both to the housing prices burst (post-2005) and to the insensitivity of households' financial commitments.
In other words, real estate (assets) has a market value while debt (liabilities) has a contractual one, thus, households net worth decreases onset of the subprime crisis.
Therefore, the second implication is the sharp reduction in the net worth of the poorest households in absolute and relative terms (see Figure \ref{FigDistPassivos}).

\begin{figure}[H]
	\centering
	\caption{Household indebtedness and house prices dynamics (jan/2000=100)}
	\label{FigDividaPreco}
	\includegraphics[width=\textwidth]{./figs/Divida_PrecoImoveis.png}
	\caption*{\textbf{Source:} U.S. Bureau of Economic Analysis, Authors' Elaboration}
\end{figure}

\begin{figure}[H]
	\centering
	\caption{Liabilities evolution by wealth percentile (1989/07=1)}
	\label{FigDistPassivos}
	\includegraphics[width=.8\textwidth]{./figs/Distribuicao_Passivos.png}
	\caption*{\textbf{Source:} \textcite{us_census_bureau_characteristics_2017}, Authors' Elaboration}
\end{figure}



Before we move forward, it is worth mentioning that the relevance of housing is not restricted to its growth effects. 
For example, \textcite{jorda_great_2016} report that credit and financial sector growth has been led mainly by mortgages. 
As a consequence, banking activities were redirected towards granting credit majorly to households and not towards productive investment \cites{erturk_banks_2007}{kohl_more_2018}.
Other studies have shown that real estate inflation is the main determinant of household indebtedness, distribution of wealth and that it has implications for macroeconomic stability \cites{ryoo_household_2015}{stockhammer_debt-driven_2016}{barnes_private_2016}{johnston_global_2017}{mian_household_2017}{anderson_politics_2020}{fuller_housing_2020}. 
In summary, what we intended to show is that one cannot analyze the US business cycle without considering housing dynamics.
On the following section, we analyze how econometric literature has dealt with this topic.
\section{Review of the Macroeconometric Literature}
\label{sec:org860dd0d}
\label{sec:empirical_review}
After the US housing bubble burst, there have been a growing attention in the macroeconomic implications of residential investment.
It worth noting that most econometric papers that includes residential investment has failed to treat it macroeconomically, restricting it to microeconomic and regional issues \cite{arestis_u.s._2008}.
Although the macroeconometric relevance of residential investment is not restricted to the US, most scholars have examined this specific case.
In this context, we analyze the macroeconometric literature that explicitly includes housing to evaluate the determinants of its growth rate.

In this sense, \citeauthor{poterba_tax_1984}'s \citeyear{poterba_tax_1984} contribution stands out once it does not assume instantly convergence of real estate supply to the desired level.
Furthermore, this theoretical prcedure considers residential investment as induced positively by house prices.
Despite the novelty, this work does not include asset bubbles.
In this context,  \textcite{arestis_residential_2015} update \citeauthor{poterba_tax_1984}'s \citeyear{poterba_tax_1984} framework by estimating an autoregressive distributed lag (ARDL) model for 17 OECD countries.
In summary, they conclude that residential investment depends mainly on disposable income.
This result would  question the possibility of treating housing as an autonomous expenditure and jeopardize the analysis from the Sraffian supermultiplier perspective.
However, with regard to the US they report that real house prices and the volume of banking credit are the main determinant of residential investment.
Therefore, this result allows considering housing as a non-capacity creating autonomous expenditure.

Others scholars analyzed the investment (residential and non-residential) to depict the determinants of the business cycle.
\textcite{green_follow_1997}, for example, estimates which investment Granger-causes GDP tests for the US from 1952 to 1992 and reports that residential investment leads --- more than firms' investment --- the business cycle.
However, he argues that this result does not imply a causal relationship: 

\begin{quote}
[P]erhaps residential investment, like stock prices and interest rates, is a good predictor of GDP because it is a series that reflects \textbf{forward-looking behavior}. Presumably households will not increase their expenditures on housing unless they expect to prosper in the future. Building a house is a natural mechanism for doing this. Thus, the series can do a good job of predicting GDP \textbf{without necessarily causing GDP} (\cite[p.~267, ephasis added]{green_follow_1997}).
\end{quote}


Despite paying attention to a non-capacity creating autonomous expenditure, \textcite{green_follow_1997}, restricts its relevance as temporal precedence indicator.
\textcite{leamer_housing_2007}, on the other hand, reports a causal relationship between housing and GDP.
In summary, states that residential investment implies a higher durable goods consumption, that is, the US business cycle is a ``consumer cycle''.

Alternatively, \textcite{huang_is_2018} assess both \citeauthor{leamer_housing_2007}'s (\citeyear{leamer_housing_2007}) hypotheses related to residential investment (prediction and causality). 
To do so, they estimate a Structural Vector Autoregressive (SVEC) model with wavelets transformation for the US and G7 countries.
They find residential investment is not only a monetary policy transmission channel, but it also has temporally distinct effects on business cycle.
In the short-run, housing is more predictive while house prices have a bigger influence in the long-run\footnote{More precisely, \textcite{huang_is_2018} also conclude that residential investment prediction increases with its share on GDP.}. 
These distinct temporal influence of housing occurs due to the large wealth effect in the long-run while credit and collateral effects are more relevant in the short-run.
Regarding the causal relationship described by \textcite{leamer_housing_2007}, 
\textcite{huang_is_2018} report inconclusive results for all countries due to their institutional heterogeneity\footnote{However, \textcite{huang_is_2018} claim that for most G7 countries, residential investment at least amplify the business cycle.}, but remains valid for the US.
Despite the inconclusive results on fluctuations, they find that housing related variables (house prices, real mortgage rate --- deflated by a general price index --- and bank spread) lead the business cycle.

Despite clarifying some macroeconomics  implications of housing on the business cycle, the results reported above are centered on supply side variables.
\textcite{gauger_residential_2003}, on the other hand, evaluate the consequences of deregulation of depository institutions throughout the 1980s.
To do so, they estimate a VECM between monetary aggregates (M2), GDP, residential investment and alternate between short-term government bonds and long-term mortgage interest rates. 
They report an increasing contribution of long-term mortgages interest rate over resident investment variance after those institutional chances mentioned above:

\begin{quote}
The findings for the two interest rates give valuable information to evaluate results in other studies. Results here suggest that use of a short-term FFR and post-deregulation data may lead to conclusions that `interest rate shocks are much less important after deregulation.' The fuller state of evidence here indicates that interest rate shocks remain important post-deregulation; however, now it is the long-term rate shocks that carry more information for housing sector movements (\cite[p.~346]{gauger_residential_2003})
\end{quote}
It worth noting that \citeauthor{gauger_residential_2003}'s (\citeyear{gauger_residential_2003}) work reports other two interesting results:
	(i) GDP level is determined by residential investment and both expenditures share a common long-term trend;
	(ii) show some relevant institutional changes in real estate market.

Figure \ref{Fig:CreditFDICIA} illustrates item (ii) mentioned above in which we mark some reforms that occurred due to the savings and loans crisis throughout the 80's and early 90's.
This institutional changes --- notably Financial Institutions Reform, Recovery, and Enforcement Act (FIRREA) in 1989 and Federal Deposit Insurance Corporation Improvement Act  (FDICIA) in 1991 --- increased the credit volume to households\footnote{\textcite{federal_deposit_insurance_corporation_savings_1997} argues that this consequence stems from the different regulation of S\&L compared to commercial banks. The financial deregulation of the 1980s encouraged speculation in other sectors, especially real estate. As a consequence, engendered a banking run, increasing overall credit volume, which, however, was followed by the S\&L crisis:
\begin{quotation}
Clearly, competition from savings and loans did not cause the various crises experienced by the commercial banking industry during the 1980s; these crises would have occurred regardless of the thrift situation. But the channeling of large volumes of deposits into high-risk institutions that speculated in real estate development did create marketplace distortions \cite[p.~168]{federal_deposit_insurance_corporation_savings_1997}
\end{quotation}
Therefore, the increase in credit volume cannot be dissociated from speculation with real estate.}\textsuperscript{,}\,\footnote{According to \textcite{federal_deposit_insurance_corporation_savings_1997}, had two main objectives:
		(i) Recapitalize the bank insurance fund and;
		(ii) Reform the deposit guarantee system and bank regulation to minimize  taxpayer in the event of bank collapse \cite{mishkin_evaluating_1997}.
		\textcite[p.~170]{federal_deposit_insurance_corporation_savings_1997} describe banking operation before FDICIA as follows:
\begin{quotation}
Legislation for S\&Ls was driven by the public policy goal of encouraging home ownership. It began with the Federal Home Loan Bank Act of 1932, which established the Federal Home Loan Bank System as a source of liquidity and low-cost financing for S\&Ls.
\end{quotation}
and the implications after its implementation is depicted as:
\begin{quotation}
Prior to the act’s passage, the FDIC and the Federal Savings and Loan Insurance Corporation provided 100 percent \textit{de facto} deposit insurance at almost all failed banks. The FDIC did so by comparing bids to acquire the entire bank (including all its deposits) with the cost of liquidating the bank, which generally produced the result that covering all deposits was less expensive (FDIC 2003, chap. 2). FDICIA sought to change this process by mandating least-cost resolution, which required consideration of all possible resolution methods (FDIC 2003, chap. 2) \cite[p.~iii]{wall_too_2010}
\end{quotation}}. 
As a consequence, real estate finance has increased considerably in the following periods.


\begin{figure}[htb]
	\centering
	\caption{Mortgage and Consumer credit growth rate (1979-2019)}
	\label{Fig:CreditFDICIA}
	\includegraphics[width=\textwidth]{./figs/FDICIA.eps}
	\caption*{\textbf{Source:} U.S. Bureau of Economic Analysis, Authors' elaboration}
\end{figure}

Although \textcite{gauger_residential_2003} emphasize the relevance of long-term mortgages interest rate in residential investment dynamics, this procedure is not appropriate once policy rate is determined by monetary aggregates.
Thus, such a proposal is incompatible with modern macroeconomic theory in which policy rate is an exogenous variable determined through a decision-making process \cite[p.~230--256]{lavoie_post-keynesian_2015}.

In a recent paper, \textcite{wood_house_2020} evaluate the relationship between economic growth, household indebtedness and house prices.
To do so, they estimate a ARDL model for 18 OECD countries from 1980 to 2017 and report that house prices determine household indebtedness which is central to describe recent economic growth rate.
Despite shedding light on the macroeconomic relevance of real estate, their model does not include both residential investment nor mortgage interest rate.
As discussed before, other scholars have found statistical significance for those variables to determine housing \cite{gauger_residential_2003}.



MELHORAR DEFINIÇÃO DE TAXA PRÓPRIA

From this literature review, we conclude that the econometric literature is more concerned with the implication of housing instead of focussing on its determinants.
One way to describe housing growth rate is the houses' own interest rate proposed by \textcite{teixeira_crescimento_2015} following SRAFFA's contribution (?).
In summary, this particular real interest rate depicts debt service and capital gains effects altogether.
On the following section, we discuss this proposal in further details and evaluate its econometric significance.

% \begin{landscape}
\begin{table}[htb]
    \caption{Residential investment determinants in macroeconometric models}
    \label{tab:summary_models}
    \begin{threeparttable}
        % \resizebox{\textheight}{!}{%
% \begin{tabular}{p{0.1\textwidth}p{0.1\textwidth}p{0.1\textwidth}p{0.7\textwidth}}
      \begin{tabularx}{\textwidth}{s|s|s|m}
    \hline\hline
    \textbf{Authors} & \textbf{Sample} & \textbf{Estimation method} & \textbf{Covariates}\\\hline
    \textcite{poterba_tax_1984} & US (1974-1982) & & \makecell{$RHP(+), NCD(-), W(+),$\\$Cr(+)$} \\\hline
    \textcite{topel_1988_Housing} & US (1963-1983) & Instrumental variable & \makecell{$HeP(+), INT(-), INFLA(-),$\\$TIME(-), W(-)$} \\\hline
    \textcite{egebo_1990_MODEL} & 7 OECD countries (1960-1987) & OLS & \makecell{$RDY(+), RINT(-), PRIr(-),$\\$PRSr(+)$} \\\hline
    \textcite{mccarthyMonetaryPolicyTransmission2002} & US (1975-1985 and 1986-2000) & EC-GMM & \makecell{$HP(+), CC(-), SINT(-),$\\$LAND(-), H(-)$} \\\hline
    \textcite{barot_2002_House} & UK and Sweden (1970-1998) & EC model & $Q(+), RINT(-)$ \\\hline
    \textcite{gauger_residential_2003} & US (1959-79 and 1982-99) & VECM and Granger test & $M2, GDP(+), SINT(-), MR(-),$\\\hline
    \textcite{arestis_residential_2015} & 17 OECD countries (1970-2013) & ARDL Model & \makecell{$RHP(+), RDY(-), MR(-),$\\$ Cr(+), UN(-)$}\\\hline
    \textcite{kohlscheen_2018_Residential} & 15 OECD countries (1970-2017) & OLS & \makecell{$Q(+), RHP(+), GDPc(+),$\\$ POP(+), NM(+), H(-)$} \\\hline
    \hline
    \end{tabularx}
    % \end{tabular}
    % } %end resize

    \footnotesize{\textbf{Notes:} Signs inside parenthesis indicates expected effects. \textcite{topel_1988_Housing} use housing starts as the dependend variable; all the other use residential investment ($RI$). \textbf{Covariates dictionary:} $RHP$ Real House Prices; $NCD$ Non-residential construction deflator; $W$ Real Construction Wage; $Cr$ Banking Credit; $HS$ Housing Starts; $HeP$ Hedonic Prices; $INT$ Interest Rates (non-specified); $INFLA$ Expected inflation; $TIME$ Median market time since the begin of construction; $RDY$ Real Disposable Income; $RINT$ Real Interest Rate; $PRIr$ Residential investment relative price; $PRSr$ Residential services relative prices; $HP$ House Prices; $CC$ Construction Costs; $SINT$ Short-term Interest Rate; $LAND$ Land Price; $H$ House Stock; $Q$ Adapted Tobin's Q for the housing sector; $RINT$ Real interest rate; $M2$ Monetary aggregates; $GDP$ Gross Domestic Product; $MR$ Mortgage Interest rate; $UN$ Unemployment rate; $GDPc$ GDP per capita; $POP$ Population Density; $NM$ Net Migration rates (per 1000 population).}
  \end{threeparttable}
    \caption*{\textbf{Source:} Authors' elaboration}
\end{table}
% \end{landscape}


\section{Macroeconometric analysis}
\label{sec:org5234367}
\label{sec:VECM}
\subsection{Houses' own interest rate and residential investment growth rate in the US	Economy}
\label{sec:orga9e6453}
\label{sc:own}

In this subsection, we describe the relationship between residential investment growth rate (\(g_{I_h}\)) and houses' own interest rate (\(own\)) as proposed by \textcite{teixeira_crescimento_2015}. 
Next, we will present the hypothesis to be tested on the Section \ref{sec:estimation}. To obtain this relationship, we deflate mortgage interest rate (\(r_{mo}\)) by real estate inflation (\(\pi\)) as follows:

$$
g_{I_h} = \phi_0 - \phi_1\cdot \overbrace{\left(\frac{1+r_{mo}}{1+\pi} - 1\right)}^{own}
$$

\begin{equation}
g_{I_h} = \phi_0 - \phi_1\cdot own
\end{equation}

where \(\phi_0\) stands for long-term determinants (\emph{e.g.} demographic factors, housing and credit policies, etc.) while \(\phi_1\) captures the demand for real estate arising from expectations of capital gains resulting from speculation with the existing dwellings stock. 
This particular real interest rate is the most relevant for households since it is the real cost in real estate from buying real estate  (\cite[p.~53]{teixeira_crescimento_2015}).

Figure \ref{propria_investo} shows how this deflation procedure is more adequate than a general price index --- as \textcite[p.~143--6]{fair_macroeconometric_2013} does --- to describe the housing dynamics. It worth noting that during a houses' bubble period, it is real estate inflation that governs own's interest rate dynamics.
Therefore, the lower this rate is, the greater the capital gains (in real estate) for speculating with real estate will be. This negative relation between houses' own interest rate and residential investment is shown in Figure \ref{propria_investo} in which this particular real interest rate has been gradually decreased over the real estate boom (2002-5).

Despite shedding light on some relevant relationships, \citeauthor{teixeira_crescimento_2015}'s (\citeyear{teixeira_crescimento_2015}) proposition was not evaluated econometrically and this will be done in Section \ref{sec:estimation}. To do so, we assume the following long-run relationship:

\begin{equation}
g_{I_ht} = \phi_0 - \phi_1\cdot own_t
\end{equation}

therefore, if these time-series are co-integrated, we model the short-run adjustment process through the following VECM:
\begin{equation}
\begin{cases}
\Delta own_t = \delta_{1} + \alpha_1\left(g_{I_{h_{t-1}}} - \phi_0 + \phi_1\cdot own_{t-1}\right) + {\sum^{N=4}_{i=1}}\beta_{1,i}\cdot \Delta g_{I_{h_{t-i}}} +
\sum^{N=4}_{i=1}\gamma_{1,i}\cdot \Delta own_{t-i} +\varepsilon_{t,1}
\\
\Delta g_{Z_{t}} = \delta_{2} + \alpha_2\left(g_{I_{h_{t-1}}} - \phi_0 + \phi_1\cdot own_{t-1}\right) + \sum^{N=4}_{i=1}\beta_{2,i}\cdot \Delta g_{I_{h_{t-i}}} +
\sum^{N=4}_{i=1}\gamma_{2,i}\cdot \Delta own_{t-i} +\varepsilon_{t,2}
\end{cases}
\end{equation}

where \(\delta_s\) indicate linear trend (level);
\(\alpha_{is}\) are the error correction coefficients; 
\(\beta_s\) and \(\gamma_s\) are coefficients associated with lagged \(g_{I_h}\) and \(own\) respectively and; \(\varepsilon_s\) are the residuals.
According to \textcite{teixeira_crescimento_2015}, the expected results are depicted in Table \ref{resultados_esperados} below:

% Please add the following required packages to your document preamble:
% \usepackage{graphicx}
\begin{table}[H]
	\centering
	\caption{Summary of expected results of the macroeconometric model}
	\label{resultados_esperados}
	\resizebox{\textwidth}{!}{%
		\begin{tabular}{l|l|l}
			\hline\hline
			\textbf{\begin{tabular}[c]{@{}l@{}}Expected\\ Result\end{tabular}} &
			\textbf{Econometric Meaning} &
			\textbf{Economic Meaning} \\ \hline\hline
			\textbf{1. $\varepsilon \sim I(0)$} &
			\begin{tabular}[c]{@{}l@{}} Stationary residuals indicates cointegration relationship\end{tabular} &
			\begin{tabular}[c]{@{}l@{}} Series share a common\\long-run trend\end{tabular} \\ \hline
			\textbf{2. $\alpha_1 = 0$} &
			\begin{tabular}[c]{@{}l@{}} $own$ is weakly exogenous\\ compered to $g_{I_h}$\end{tabular} & \begin{tabular}[c]{@{}l@{}} 
				$own$ dynamics is not affected\\by previous equilibrium deviation\end{tabular}
			\\ \hline
			\textbf{3. $\alpha_2 < 0$} &
			\begin{tabular}[c]{@{}l@{}}Own interest rate Granger-causes\\
				residential investment growth rate\end{tabular} & \begin{tabular}[c]{@{}l@{}} $g_{I_h}$ dynamics is not affected\\ by previous equilibrium deviation\end{tabular}
			\\ \hline
			\textbf{4. $\phi_1 > 0$} &
			\begin{tabular}[c]{@{}l@{}}Series share a common\\negative long-run relationship\end{tabular} &
			\begin{tabular}[c]{@{}l@{}}Own interest rate affects\\residential investment growth rate negatively\end{tabular} \\ \hline
			\textbf{5. $\phi_0 < 0$} &
			\begin{tabular}[c]{@{}l@{}}
			Real estate demand for non-speculation\\reasons is statistically significant
			\end{tabular} &
			\begin{tabular}[c]{@{}l@{}}
				Real estate demand associated with\\institutional particularities and demographic\\ changes affects residential investment\\growth rate positively\end{tabular} \\ \hline
			\textbf{6. $\gamma_{2,is} < 0$} &
			\begin{tabular}[c]{@{}l@{}}Residential investment growth rate\\coefficient is statistically significant\end{tabular} &
			\begin{tabular}[c]{@{}l@{}}Own interest rate affects\\$g_{I_h}$ in the short-run\end{tabular} \\ \hline
			\textbf{7. $\beta_{1,is} = 0$} &
			\begin{tabular}[c]{@{}l@{}}
				$g_{I_h}$ effects over own interest\\ rate is not statistically significant\end{tabular} &
			\begin{tabular}[c]{@{}l@{}}
				$g_{I_h}$ effects over own interest\\ rate is negligible since dwellings stock is much\\bigger than residential investment (flow)\end{tabular} \\ \hline\hline
		\end{tabular}%
	}
\caption*{\textbf{Source:} Authors' elaboration}
\end{table}


\subsection{Data and estimation strategy}
\label{sec:orge3c455a}
\label{sec:estimation}


In this section, we employ a model to test whether or not real estate inflation describes residential investment growth rate dynamics\footnote{Scripts are available under request.}.
Our sample period (1992:Q1 to 2019:Q1) starts after institutional changes (FDIC e
FIRREA) due to the Savings and Loans crisis (see Table \ref{structbreak} in appendix \ref{appen:A} for some related structural breaks). 
We rely on the following  quarterly seasonally adjusted data: (i) 30-Year fixed mortgage interest rate (MORTGAGE30US, resampled by end of period), private residential investment (PRFI, growth rate as percent change from the previous quarter) and Case-Shiller home price index
(CSUSHPISA, resampled by end of period). Figure FIGURE shows the original series.

\begin{figure}[htb]
	\centering
	\caption{Residential investment growth rate vs. Houses Own interest rate}
	\label{propria_investo}
	\includegraphics[width=\textwidth]{./figs/TxPropria_Investo.png}
	\caption*{\textbf{Source:} U.S. Bureau of Economic Analysis, Authors' elaboration}
\end{figure}


Next, we applied \textcite{yeo_new_2000} transformation since these series are quite volatile. We use this procedure instead of a standard \textcite{box_analysis_1964} transformation  because it can be applied to non-positive values. 
Then, we employ standard unit root tests (see Table \ref{unitroot} in appendix \ref{appen:A}) as well as \textcite{johansen_estimation_1991} procedure to assess whether houses' own interest rate and residential investment growth rate share a common long-run trend (see Table \ref{Johansen} in appendix \ref{appen:A}).
Our series are co-integrated at 5\% significance level which allows us to estimate a error correction model and evaluate the previous hypothesis \cite{enders_applied_2014}.

\begin{figure}[htb]
	\centering
	\caption{Time-series with \textcite{yeo_new_2000} transformation}
	\label{YeoJhonson}
	\includegraphics[width=\textwidth]{./figs/YeoJohnson_All.png}
	\caption*{\textbf{Source:} U.S. Bureau of Economic Analysis, Authors' elaboration}
\end{figure}




The next step is to define the model order. According to usual information criteria, both first and forth lags are eligible (see Table \ref{criterios} in appendix \ref{appen:A}).
Although parsimonious, we argue that the first lag has no empirical support.
Considering the average construction time (from approval to completion), we should include at least the second lag in order to incorporate homes built for capital gains purposes which only take place once the construction is completed (see Figure \ref{meses}).

\begin{figure}[H]
	\centering
	\caption{Average construction time (approval to completion) of properties for a family unit by construction purposes except manufactured houses (1976-2018)}
    \label{meses}
	\includegraphics[width=\textwidth]{./figs/Meses_construcao.png}
	\caption*{\textbf{Source:} Survey of Construction (SOC), Authors' elaboration}
\end{figure}

This procedure, however, it is not enough to determine the model order. 
Since residential investment (flow) is significantly smaller than dwelling (existing stock), the price variation effect  is verified even when the construction is unfinished. 
We argue that this price effect is a result of future real estate inflation.
Such dynamic could be captured by the \textbf{expected} houses' own interest rate.
However, such series does not exist. So, we use lagged houses' own interest rate as a first approximation to the expected one\footnote{This procedure is similar to \textcite{keynes_general_1937} ``practical theory of the future'' in which decision-making process for buying a new property depends on expectations/conventions based on past observations. 
In summary, in the absence of a series for the expected own interest rate, the lag of this variable will be used as a proxy for the future one.}.

In order to display the relation between lagged own interest rate and current residential investment growth rate, Figure \ref{defasagens} depicts one variable of interest against the other variable lagged according to lags that minimize the information criteria (1 and 4 respectively)\footnote{Similar plot can be seen in \textcite[p.~16]{girardi_autonomous_2015}.}.
This simple procedure allows checking if there is any relationship between the expected own interest rate (in this case, lagged effective rate) and residential investment growth rate\footnote{In order to consider non-linearities, we presented quadratic regression between variables of interest.}.
In the same Figure, we verify that the inverse relationship, that is, from residential investment to own interest rate, does not occur. 
The lack of relationship in the opposite direction reflects a dynamics already mentioned before.
Since residential investment (flow) is much lower than the existing stock of dwellings, it is expected that such relationship does not exist.
In summary, speculation with the dwellings stock generates inflation of these assets, which affects the construction of new houses (flow) and not the other way round\footnote{It is worth noting a particular aspect of house price formation: land scarcity. As a consequence, speculation with residences is, in the end, speculation with land (the only scarce resource involved in its production) and, therefore, it is relevant for speculation with the dwellings stock. 
	\textcite[p.~349, emphasis added]{leamer_housing_2007} points out this particularity as follows:
	\begin{quotation}
		It’s not the structure that has a volatile price; \textbf{it's the land}. Where there is plenty of buildable land, the response to an increase is demand for homes is mostly to build more, not to increase prices. Where there is little buildable land, the response to an increase in demand for homes is mostly a price increase, sufficient to discourage buyers enough to reequilibrate the supply and demand.
	\end{quotation}}.

\begin{figure}
	\centering
	\caption{Dispersion between houses' own interest rate and residential investment growth: lags selected based on information criteria}
	\label{defasagens}
	\includegraphics[height=.4\textheight]{./figs/VEC_Defasagens.png}
	\caption*{\textbf{Source:} Authors' elaboration}
\end{figure}

Considering this theoretical and econometric discussion of model order specification, we estimate a four lag VEC  (see Table \ref{Estimacao})\footnote{In addition to being theoretically based, this lag also generates homoscedastic residuals without serial auto-correlation (see Table \ref{testes_resduos} in Appendix \ref{appen:A}).}. 
Figure \ref{residuos} displays an inspection of the residuals while Table \ref{testes_resduos} in Appendix \ref{appen:A} presents a few residual tests to check the model's specification while Table \ref{tab:robust} presents some robustness check.
On the following subsection, we analyze the results and compare with the theoretical expected ones presented in Table \ref{resultados_esperados} above.  




\subsection{Estimation results}
\label{sec:org597825f}
\label{sec:results}

According to parameters presented in Table \ref{Estimacao}, we find statistically significant co-integration  coefficients for both equations. 
Therefore, both variables share a (negative) long-run trend (validating hypotheses 1 and 4).
The short-term relationship between \(own\) and \(g_{Ih}\) (\(\beta_{1, is}\) coefficients) are not statistically significant at 5\%\footnote{The expected result (7) can also be validated from the inspection of Table \ref{Estimacao} in which only the fourth lag of own interest rate equation is statistically significant.}.
In addition, coefficients \(\gamma_{2,s}\) are negative and statistically significant at 5\%, supporting hypothesis 6 (see Table  \ref{Estimacao}).
We also find statistically significant coefficients related to demand for houses for non-speculative reasons (\(\phi_0\)), validating proposition 5.
On the other hand, the error correction parameter is statistically significant only for the residential investment growth rate equation.
In this sense, \(own\) is weakly exogenous compared to \(g_{I_h}\) while houses' own interest rate Granger-causes \(g_{I_h}\), supporting the hypothesis (2) and (3).
In conclusion, our estimation results are in line with the hypothesis presented above and can be summarized as follows: houses' own interest rate determines --- but is not determined by --- residential investment growth rate and these variables present a negative long-term relationship (are co-integrated).

\begin{table}[h!]
	\caption{Estimation parameters}
	\centering
	\begin{tabular}{lrrrrr}
\toprule
{} &  Base scenario &  $\Delta \phi_0$ &  $\Delta \omega$ &  $\Delta rm$ &  $\pi$ \\
\midrule
$\alpha$      &         0.5000 &           0.5000 &           0.5000 &       0.5000 & 0.5000 \\
$\gamma_F$    &         0.0800 &           0.0800 &           0.0800 &       0.0800 & 0.0800 \\
$\gamma_u$    &         0.0900 &           0.0900 &           0.0900 &       0.0900 & 0.0900 \\
$\omega$      &         0.5000 &           0.5000 &           0.4900 &       0.5000 & 0.5000 \\
$rm$          &         0.0100 &           0.0100 &           0.0100 &       0.0200 & 0.0100 \\
$\sigma_{l}$  &         0.0000 &           0.0000 &           0.0000 &       0.0000 & 0.0000 \\
$\sigma_{mo}$ &         0.0000 &           0.0000 &           0.0000 &       0.0000 & 0.0000 \\
$u_N$         &         0.8000 &           0.8000 &           0.8000 &       0.8000 & 0.8000 \\
$v$           &         1.2000 &           1.2000 &           1.2000 &       1.2000 & 1.2000 \\
$\phi_0$      &         0.0250 &           0.0300 &           0.0250 &       0.0250 & 0.0250 \\
$\phi_1$      &         0.1000 &           0.1000 &           0.1000 &       0.1000 & 0.1000 \\
$R$           &         0.7000 &           0.7000 &           0.7000 &       0.7000 & 0.7000 \\
$\pi$         &         0.0000 &           0.0000 &           0.0000 &       0.0000 & 0.0500 \\
\bottomrule
\end{tabular}

	\caption*{\textbf{Source:} Authors' elaboration}
\end{table}

\begin{figure}
	\centering
	\caption{Inspection of estimation residuals}
	\label{residuos}
	\includegraphics[height=.4\textheight]{./figs/Residuals_4VECM.png}
	\caption*{\textbf{Source:} Authors' elaboration}
\end{figure}

Figure \ref{fevd} display the forecast error variance decomposition (FEVD) which reports houses own interest rate in describing residential investment growth rate dynamics\footnote{It is important to note that the number of variables (two) used generates similar  of a Structural VEC, which means that Choleski's decomposition is sufficient to analyze the (orthogonalized) impulse response function.}.
We report that own interest rate has  depicted  \(g_{Ih}\) --- while the reverse is not valid --- after the first quarter.
In addition, we find that such contribution is greater than 50\% beyond the third quarter.
Therefore, houses' own interest rate is explained mainly by itself and explains \(g_{I_h}\) considerably.

\begin{figure}[H]
	\centering
	\caption{Forecast error variance decomposition (FEVD)}
	\label{fevd}
	\includegraphics[height=.4\textheight]{./figs/FEVD_VECMpython_TxPropria.png}
	\caption*{\textbf{Source:} Authors' elaboration}
\end{figure}

Next, we analyze the orthoganilized impulse response function (Figure \ref{irf}).
In summary, we report a stable system since the increase in \(g_{I_h}\) on itself are dampened over time while equivalent shock on own interest rate has a non-explosive permanent effect.
On the other hand, an increase in \(g_{I_h}\) has a null effect over \(own\).
The most relevant result reported in Figure \ref{irf} is the considerable and lasting negative effect due to an increase in own interest rate over \(g_{I_h}\), validating \citeauthor{teixeira_crescimento_2015}'s (\citeyear{teixeira_crescimento_2015}) proposition.
In short, our results shows that an increase in mortgage interest rate (equivalent to an increase in houses' own interest rate) has a negative and persistent effect on residential investment growth rate while an increase in real estate inflation  has an opposite effect.
\begin{figure}[H]
	\centering
	\caption{Orthogonalized Impulse Response Function}
	\label{irf}
	\includegraphics[height=.4\textheight]{./figs/Impulse_VECM.png}
	\caption*{\textbf{Source:} Authors' elaboration}
\end{figure}

In summary, our estimation reports that houses' own interest rate has a prominent role in describing residential investment growth rate movements. 
It is worth noting that despite the amplitude of VEC order, our model is quite parsimonious considering the number of variables used.
Thus, we conclude that our estimation depicts residential investment growth rate satisfactorily.
Finally,  we contrast our findings with those obtained by the literature. 
At this stage, we restrict the comparison with  \textcite{gauger_residential_2003} and \textcite{arestis_residential_2015} since we share the same topic.
Similar to \textcite{gauger_residential_2003}, we report that long-run mortgage interest rate determines residential investment.
Despite some theoretical differences, our estimations are in line with \textcite{arestis_residential_2015} (at least for the US): house prices are relevant to describe residential investment growth rate.
However, they report insignificant coefficients for mortgages nominal interest rate which is at odds with our conclusions.
On the following section we present some concluding remarks.

\section{Concluding Remarks}
\label{sec:orgfb6a1ba}
\label{sec:Conclusion}
In this article, we present a residential investment growth rate specification compatible with the Srrafian supermultiplier model.
To do so, we estimate a bi-dimension VEC evaluate \textcite{teixeira_crescimento_2015}
proposal. 
We report: 
	(i) Houses' own interest rate (\(own\)) and residential investment growth rate (\(g_{I_h}\)) share a common long-run trend;
	(ii) \(g_{I_h}\) effects over \(own\) are negligible and; 
	(iii) own interest rate has a negative effect on \(g_{I_h}\) and is its main determinant (see Figure \ref{fevd}).
Besides being parsimonious, our estimations does not show residuals serial autocorrelation and heteroscedasticity. Thus, our results are quite satisfactory.

It remains to contrast our findings with those obtained by \textcite{arestis_residential_2015}.
It worth remembering that one of the authors' hypotheses is that residential investment depends on disposable income (is induced expenditure).
However, the authors themselves find that such results are not statistically significant for the US. Therefore, we can compare this result with our model.
Despite the differences, some results of the model are in line with those of \textcite{arestis_residential_2015}.
Among them, house prices relevance in determining residential investment dynamics for the US.
However, they report insignificant coefficients for mortgages nominal interest rate, that is, the opposite conclusion of our model.

In conclusion,  we report lack of work analyzing residential investment in a Sraffian supermultiplier-friendly framework in the macroeconometric literature.
Our estimation supports houses' own interest rate relevance in describing residential investment growth rate for the US as depicted by \textcite{teixeira_crescimento_2015}.
Thus, our  proposal differs from the usual empirical literature by:
	(i) considering housing as a non-capacity creating autonomous expenditure;
	(ii) reporting that mortgage interest rates are relevant to describe long-run residential investment dynamics; and notably 
	(iii) including asset bubble through houses own interest rate.


\section*{Acknowledgments}
\label{sec:orgb2a3cb2}
\noindent The authors wish to acknowledge the financial support from the Brazilian National Research Council (CNPq; grant 130777/2018-8). We are grateful to Rosângela Ballini, Carolina Baltar, Júlia Braga, Ítalo Pedrosa, Cecon/Unicamp and UFRJ Macroeconomic discussion groups for useful comments and suggestions on earlier drafts of this article. All remaining errors are, of course, our own.


\section*{Disclosure statement}
\label{sec:orgd0fd5ff}
No potential conflict of interest was reported by the authors.

\section*{References}
\label{sec:orgb9968e6}
\printbibliography[heading=none]


\appendix
\section{Statistical Appendix}
\label{sec:orgadd36bb}
\label{appen:A}

In this appendix we report several tests: unit root tests on our variables of interest, Structural break test, Johansen procedure and hypothesis tests on residuals. 
The time-series of residential investment growth rate, mortgage interest rate and real estate inflation are all taken using \textcite{yeo_new_2000} transformation.  
As shown in Table \ref{unitroot}, the null hypothesis of a unit root in the first differences of the series is overwhelmingly rejected.
As mentioned, we found structural breaks related to institutional changes and our series are co-integrated (see Tables \ref{structbreak} and \ref{Johansen}).
Our estimation order selection was based both on statistical and theoretical reasoning with homoscedasticity residuals (see Tables \ref{criterios} and \ref{testes_resduos}).
All expected results are statistically significant and most of them are not restrict to model specification (see Table \ref{tab:robust}).

% Please add the following required packages to your document preamble:
% \usepackage{multirow}
% \usepackage{graphicx}
\begin{table}[H]
	\centering
	\caption{Unit root tests}
	\label{unitroot}
	\resizebox{\textwidth}{!}{%
	\begin{threeparttable}
		\begin{tabular}{l|l|cccccccc}
\hline\hline
\multicolumn{2}{l|}{\multirow{2}{*}{\textbf{Variable}}} & \multicolumn{2}{c}{\textbf{ADF}\tnote{a}} & \multicolumn{2}{c}{\textbf{Zivot Andrews}\tnote{b}} & \multicolumn{2}{c}{\textbf{Phillips Perron}\tnote{a}} & \multicolumn{2}{c}{\textbf{KPSS}\tnote{c}} \\ \cline{3-10} 
\multicolumn{2}{l|}{} & \multicolumn{1}{l}{Statistic} & \multicolumn{1}{l}{p-value} & \multicolumn{1}{l}{Statistic} & \multicolumn{1}{l}{p-value} & \multicolumn{1}{l}{Statistic} & \multicolumn{1}{l}{p-value} & \multicolumn{1}{l}{Statistic} & \multicolumn{1}{l}{p-value} \\ \hline
\textbf{Residential} & level &-3.333&0.013&-4.439&0.139&-6.165&0.000&0.181&0.309\\
\textbf{investment ($g_{I_h}$)} & first difference &-7.155&0.000&-7.739&0.000&-20.346&0.000&0.106&0.558\\ \hline
% \textbf{Real estate} & level &-2.671&0.079&-4.871&0.043&-2.704&0.073&0.148&0.395 \\
% \textbf{inflation} & first difference &-4.680&0.000&-6.122&0.001&-11.340&0.000&0.059&0.819 \\ \hline
\textbf{Houses own-rate of interest} & level &-2.330&0.162&-4.203&0.237&-2.425&0.135&0.690&0.014 \\
\textbf{rate of interest}& first difference &-5.087&0.000&-6.340&0.000&-10.408&0.000&0.062&0.804\\ \hline
% \textbf{Mortgage} & level &-3.638&0.027&-4.494&0.215&-3.604&0.030&0.081&0.264 \\
% \textbf{interest rate}& first difference &-8.050&0.000&-8.144&0.000&-11.127&0.000 &0.034&0.962 \\
\hline\hline
\end{tabular}%
\begin{tablenotes}\footnotesize
	\item [a] H0: has a unit root.
	\item [b] H0: has a unit root and a structural break.
	\item [c] H0: series is weakly stationary.
\end{tablenotes}
\end{threeparttable}
	}
\caption*{\textbf{Source:} Authors' elaboration}
\end{table}

% Please add the following required packages to your document preamble:
% \usepackage{multirow}
% \usepackage{graphicx}
\begin{table}[H]
	\centering
	\caption{Structural break test}
	\label{structbreak}
	\begin{threeparttable}
	%\resizebox{\textwidth}{!}{%
		\begin{tabular}{l|l|cc}
			\hline \hline
			\multirow{2}{*}{\textbf{Variable}} & \multirow{2}{*}{\textbf{Break}} & \multicolumn{2}{c}{\textbf{Chow test}\tnote{a}} \\ \cline{3-4} 
			&& Statistic & p-value \\ \hline
			\multirow{3}{*}{\textbf{Residential investment ($g_{I_h}$)}} & 1991/Q3 & 5.1147 & 0.0254 \\
			& 2005/Q4 & 7.286 & 0.007881 \\
			& 2010/Q3 & 6.1013 & 0.01481 \\ \hline
			\multirow{5}{*}{\textbf{Houses own-rate of interest}} & 1991/Q3 & 63.453 & 7.487e-13 \\
			& 1996/Q3 & 107.47 & \textless 2.2e-16 \\
			& 2001/Q2 & 78.378 & 5.662e-15 \\
			& 2006/Q1 & 20.68 & 1.236e-05 \\
			& 2011/Q1 & 78.969 & 4.663e-15 \\ \hline
			% \multirow{4}{*}{\textbf{Mortgage interest rate}} & 1991/Q3 & 124.35 & \textless 2.2e-16 \\
			% & 1997/Q1 & 199.25 & \textless 2.2e-16 \\
			% & 2002/Q1 & 301.18 & \textless 2.2e-16 \\
			% & 2009/Q4 & 172.97 & \textless 2.2e-16 \\ \hline
			% \multirow{3}{*}{\textbf{Real estate inflation}} & 1997/Q3 & 1.5508 & 0.2153 \\
			% & 2005/Q4 & 23.49 & 3.569e-06 \\
			% & 2011/Q3 & 4.4981 & 0.03586 \\
			\hline \hline
		\end{tabular}%
	%}
	\begin{tablenotes}\footnotesize
		\item [a] H0: There is no structural break.
	\end{tablenotes}
\end{threeparttable}
	\caption*{\textbf{Source:} Authors' elaboration}
\end{table}

% Please add the following required packages to your document preamble:
% \usepackage{multirow}
% \usepackage{graphicx}
\begin{table}[h]
\centering
\caption{Cointegration test}
\label{Johansen}
\begin{threeparttable}
%\resizebox{\textwidth}{!}{%
\begin{tabular}{l|l|c|c}
\hline
 \hline
\multirow{2}{*}{\textbf{Model specification}} & \multirow{2}{*}{\textbf{Hypothesis}} & \multicolumn{2}{c}{\textbf{Johansen Procedure\tnote{a}}} \\ \cline{3-4} 
 &  & \multicolumn{1}{c|}{Statistic} & critical value (5\%) \\ \hline
\multirow{3}{*}{\textbf{$g_{I_h}$, Own interest rate}} & $r = 0$ &22.51&19.96\\
 & $r = 1^*$ &2.91&9.24\\\hline	
\multirow{4}{*}{\textbf{$g_{I_h}$, Inflation and Mortgage interest rate}} & $r = 0$ &46.05&34.91\\
 & $r = 1^*$ &15.08&19.96\\
 & $r = 2$ &6.44&9.24\\\hline
\multirow{3}{*}{\textbf{$g_{I_h}$, Inflation and exogenous  mortgages interest rate}} & $r = 0$ &36.88& 19.96\\ 
 & $r = 1^*$ &7.87&9.24\\ 
  \hline
\end{tabular}%
%}
\footnotesize{(a) Using trace test with constant for the 5th lag (according to AIC criteria). (*) Indicates the selected rank that implies cointegration.}
\end{threeparttable}
\caption*{\textbf{Source:} Authors' elaboration}
\end{table}
\begin{table}
\caption{Selection model order (* indicates the minimum)}
\label{criterios}
\centering
\begin{tabular}{lcccc}
\toprule
            & \textbf{AIC} & \textbf{BIC} & \textbf{FPE} & \textbf{HQIC}  \\
\midrule
\textbf{0}  &      -13.86  &      -13.71  &   9.553e-07  &       -13.80   \\
\textbf{1}  &      -14.71  &     -14.46*  &   4.102e-07  &       -14.61   \\
\textbf{2}  &      -14.72  &      -14.38  &   4.038e-07  &       -14.58   \\
\textbf{3}  &      -14.74  &      -14.29  &   3.983e-07  &       -14.56   \\
\textbf{4}  &      -14.88  &      -14.34  &   3.442e-07  &      -14.66*   \\
\textbf{5}  &     -14.89*  &      -14.24  &  3.425e-07*  &       -14.63   \\
\textbf{6}  &      -14.84  &      -14.09  &   3.612e-07  &       -14.54   \\
\textbf{7}  &      -14.82  &      -13.97  &   3.696e-07  &       -14.47   \\
\textbf{8}  &      -14.77  &      -13.82  &   3.891e-07  &       -14.38   \\
\textbf{9}  &      -14.75  &      -13.71  &   3.961e-07  &       -14.33   \\
\textbf{10} &      -14.70  &      -13.56  &   4.189e-07  &       -14.24   \\
\textbf{11} &      -14.67  &      -13.43  &   4.308e-07  &       -14.17   \\
\textbf{12} &      -14.63  &      -13.29  &   4.535e-07  &       -14.08   \\
\textbf{13} &      -14.66  &      -13.22  &   4.406e-07  &       -14.08   \\
\textbf{14} &      -14.63  &      -13.09  &   4.582e-07  &       -14.00   \\
\textbf{15} &      -14.59  &      -12.95  &   4.806e-07  &       -13.92   \\
\bottomrule
\end{tabular}
\caption*{\textbf{Source:} Authors' elaboration}
\end{table}
% Please add the following required packages to your document preamble:
% \usepackage{multirow}
% \usepackage{graphicx}
\begin{table}[H]
\centering
\caption{Hypothesis tests on residuals}
\label{testes_resduos}
	\begin{threeparttable}
\begin{tabular}{l|c|c|c}
\hline
\multicolumn{2}{l|}{} & \textbf{Statistic} & \textbf{p-value} \\ \hline
\textbf{Serial autocorrelation}\tnote{a} & System & 54.51 & 0.093 \\ \hline
\multirow{2}{*}{\textbf{Homoscedasticity}\tnote{b}} & $own$ & 1.863 & 0.175 \\ \cline{2-4} 
 & $g_{I_h}$ & 3.080 & 0.082 \\ \hline
\textbf{Normality}\tnote{c} & System & 46.64 & 0.000 \\ \hline
\end{tabular}%
\begin{tablenotes}\footnotesize
	\item [a] Adjusted Portmanteau tested until up to 15th \textit{lag}. H0: autocorrelations up to the selected lag equal zero.
	\item [b] ARCH-LM test. H0: Residuals are homoscedastic.
	\item [c] Jarque-Bera test. H0: data generated by normally-distributed process.
\end{tablenotes}
\end{threeparttable}
\caption*{\textbf{Source:} Authors' elaboration}
\end{table}
% Please add the following required packages to your document preamble:
% \usepackage{multirow}
% \usepackage{graphicx}
\begin{table}[]
\centering
\caption{Robustness check}
\label{tab:robust}
\resizebox{\textwidth}{!}{%
\begin{tabular}{c|c|c|c|c|c|c|c|c}
\hline\hline
\multirow{2}{*}{\textbf{Lags}} &
  \multirow{2}{*}{\textbf{\begin{tabular}[c]{@{}c@{}}Information\\ Criteria\end{tabular}}} &
  \multirow{2}{*}{\textbf{\begin{tabular}[c]{@{}c@{}}Reject $H_0$ for Adjusted \\ Portmanteau test?\\ (zero residual auto corr. )\end{tabular}}} &
  \multicolumn{2}{c|}{\textbf{\begin{tabular}[c]{@{}c@{}}Reject $H_0$ for\\ ARCH-LM test?\\ (Homoscedasticity)\end{tabular}}} &
  \multirow{2}{*}{\textbf{\begin{tabular}[c]{@{}c@{}}Reject $H_0$ for\\ Jarque-Bera test?\\ (Normality)\end{tabular}}} &
  \multicolumn{2}{c|}{\textbf{\begin{tabular}[c]{@{}c@{}}Reject $H_0$\\ for KPSS \\ Stationarity test?\end{tabular}}} &
  \multirow{2}{*}{\textbf{\begin{tabular}[c]{@{}c@{}}Confirmed\\hypothesis\end{tabular}}} \\ \cline{4-5} \cline{7-8}
   &                &     & \textbf{$g_{I_h}$} & \textbf{$own$} &     & \textbf{$g_{I_h}$} & \textbf{$own$} &                       \\ \hline\hline
0  & BIC            & yes & yes                & yes            & yes & no                 & no             & 1,2, 3 and 5          \\
1  & $-$            & yes & no                 & yes            & yes & no                 & no             & all                   \\
2  & $-$            & yes & no                 & yes            & yes & no                 & no             & all                   \\
3  & $-$            & yes & no                 & yes            & yes & no                 & no             & all                   \\
4  & AIC, FPE, HQIC & no  & no                 & no             & yes & no                 & no             & all                   \\
5  & $-$            & no  & no                 & no             & yes & no                 & no             & all                   \\
6  & $-$            & no  & no                 & no             & yes & no                 & no             & all                   \\
7  & $-$            & yes & no                 & no             & yes & no                 & no             & all                   \\
8  & $-$            & yes & no                 & no             & yes & no                 & no             & all except 7          \\
9  & $-$            & yes & no                 & no             & yes & no                 & no             & all except 2 and 7    \\
10 & $-$            & yes & no                 & no             & yes & no                 & no             & all except 2, 3 and 7 \\
11 & $-$            & no  & yes                & no             & yes & no                 & no             & all except 7          \\
12 & $-$            & yes & no                 & no             & yes & no                 & no             & all except 2 and 7    \\ \hline\hline
\end{tabular}%
}
\caption*{\textbf{Source:} Authors' elaboration}
\end{table}

\end{document}

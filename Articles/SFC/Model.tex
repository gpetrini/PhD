% Created 2020-10-12 seg 16:25
% Intended LaTeX compiler: pdflatex
\documentclass[11pt]{article}
\usepackage[utf8]{inputenc}
\usepackage{lmodern}
\usepackage[T1]{fontenc}
\usepackage[top=3cm, bottom=2cm, left=3cm, right=2cm]{geometry}
\usepackage{graphicx}
\usepackage{longtable}
\usepackage{float}
\usepackage{wrapfig}
\usepackage{rotating}
\usepackage[normalem]{ulem}
\usepackage{amsmath}
\usepackage{textcomp}
\usepackage{marvosym}
\usepackage{wasysym}
\usepackage{amssymb}
\usepackage{amsmath}
\usepackage[theorems, skins]{tcolorbox}
\usepackage[style=abnt,noslsn,extrayear,uniquename=init,giveninits,justify,sccite,
scbib,repeattitles,doi=false,isbn=false,url=false,maxcitenames=2,
natbib=true,backend=biber]{biblatex}
\usepackage{url}
\usepackage[cache=false]{minted}
\usepackage[linktocpage,pdfstartview=FitH,colorlinks,
linkcolor=blue,anchorcolor=blue,
citecolor=blue,filecolor=blue,menucolor=blue,urlcolor=blue]{hyperref}
\usepackage{attachfile}
\usepackage{setspace}
\usepackage{tikz}
\author{Gabriel Petrini da Silveira}
\date{\today}
\title{}
\begin{document}

\tableofcontents

\section{General equations}
\label{sec:org44cf9d2}

Our model is the most parsimonious as possible: a closed capitalist economy without government sector. Output (\(Y\)) is determined by  a fixed combination of a homogeneous labor (\(L\)) input with homogeneous fixed business capital (\(K_f\)). 
For simplicity, we put technological progress, depreciation and goods inflation aside so investment is presented in net terms and all variables --- except for houses --- are measured in real terms.
Assuming a Leontief production function and that growth is not constrained by labor scarcity, full capacity output (\(Y_{FC}\)) is
determined by firms' capital stock:
\begin{equation}
\label{_Leontieff}
    Y_{FC} = \min (Y_L, Y_K)
\end{equation}
\begin{equation}
\label{_YFC}
    Y_{FC} = \frac{K_{f_{-1}}}{v}
\end{equation}
\begin{equation}
\label{_u}
    u = \frac{Y}{Y_{FC}}
\end{equation}
where \(Y_L\) and \(Y_K\) stands for full employment and full capacity output respectively, \(v\) is exogenous capital-output ratio and \(u\) is utilization rate.

We further assume an economic structure composed by both workers (denoted by \(w\)) and capitalists (denoted by \(k\)) households\footnote{In accordance with \textcite{albanesi_credit_2017}, we consider that only the latter invest in real estate and are not credit constrained.}.
Thus, demand-determined output level (\(Y\))  is the sum of workers and capitalists consumption (\(C_w\) and \(C_k\) respectively) and both households and firms investment (\(I_h\) and \(I_f\) respectively) and only the latter creates capacity to the business sector of the economy:
\begin{equation}
\label{_Ct}
    C = C_w + C_k
\end{equation}
\begin{equation}
\label{_It}
    I_t = I_f + I_h
\end{equation}
\begin{equation}
\label{_Y}
    Y = \overbrace{[C_w + \underbrace{C_k + I_h}_{\text{Capitalists}}]}^{\text{Households}} + \overbrace{[I_f]}^{\text{Firms}}
\end{equation}

In other words, from institutional sectors perspective, household expenditures have two components (consumption and residential investment) and firms just one (non-residential investment). 
Only non-residential investment creates productive capacity. 
So, the novelty of this model is the inclusion of a second investment component all made by household sector and held by capitalists households for simplicity. 
Therefore, this economy produces two types of real assets: firms productive capital (\(K_f\)) and households housing (\(K_h\)):
\begin{equation}
    \label{_K}
    K = K_f + K_h
\end{equation}

Denoting the houses share in total real assets as \(k\), we can rewrite equation \ref{_K} as:
\begin{equation}
\label{_k}
    k = \frac{K_h}{K}
\end{equation}
$$
K = (1-k)\cdot K + k\cdot K
$$

We further assume an exogenous functional income distribution. 
Following Sraffian strands, profit-share is determined both by historical-institutional factors and class struggle.
As a consequence, we define total wages (\(W\), Eq. \ref{_W}) as a function of wage-share (\(\omega\)):

\begin{equation}
\label{_W}
    W = \omega\cdot Y
\end{equation}

Table \ref{Matriz_Estoques} presents the balance sheet matrix for all institutional sectors. 
Capitalists households hold financial wealth as bank deposits (\(M\)) and residential investment is financed by mortgages (\(MO\)).
Capitalists' total net wealth (\(NW_{k}\)) is the sum of their net financial wealth (\(V_{k}\)) and real assets (\textit{i.e.} housing, \(K_h\)). 
Table  \ref{Matriz_Fluxos} presents both transactions flows and the flow of funds matrix. 
This table shows all economic relations between institutional sectors ensuring that there is no  ``black holes''
so all financial and real transaction are explicitly defined \cite{macedo_e_silva_peering_2011}.

In this model, capitalist consumption (\(C_k\)) is fully autonomous and financed by loans (\(L_{k}\)) while workers consumption (\(C_w\)) is fully induced by their wages.
We assume that workers expend what they earn while capitalists earn what they expend, so workers financial and real wealth are both null.
Firms finance their investment primarily by undistributed profits (\(FU\)) and the residual by bank loans (\(L_f\)) --- thus they do not hold deposits. 
Banks create credit \textit{ex nihilo} and then collect the deposits, paying the same interest rate that they charge.
On the following subsections, we will present the equations of each of these institutional sectors.



\begin{table}[H]
\centering
\caption{Balance Sheet matrix}
\label{Matriz_Estoques}
\begin{tabular}{lccccc}
\hline
\hline
                          & Workers & Capitalists      & Firms        & Banks  &    $\sum$ \\ \hline

Deposits & & $+M$ & & $-M$ & 0\\
Loans& &$-L_{k}$ &$-L_f$& $+L$ & 0\\
Mortages & &$-MO$&  & $+MO$ & 0\\\hline
$\sum$ Net Financial Wealth &--- &$V_{k}$&$V_f$&$V_b$& $0$\\\hline
Capital & & &$+K_f$&  & $+K_f$\\
Houses & &$+K_{hd}$& &   & $+K_h$\\\hline
$\sum$ Net Wealth &---&$NW_{k}$&$NW_f$&$NW_b$& $+K$\\
\hline
\hline
\end{tabular}%
\caption*{\textbf{Source:} Authors' Elaboration}
\end{table}


\begin{table}[H]
\centering
\caption{Transactions flow matrix and flow of funds
}
\label{Matriz_Fluxos}
\resizebox{\textwidth}{!}{%
\begin{tabular}{lccccccc}
\hline
\hline
& Workers
& \multicolumn{2}{c}{Capitalists}
& \multicolumn{2}{c}{Firms}                        
& Banks       & Total    \\ \cline{3-4}\cline{5-6}
& &
Current & Capital & 
Current & Capital     & 
&       $\sum$ \\ 
Consumption                       &$-Cw$&$-C_k$& & $+C$& & & 0\\
Non-residential Investment                   & & & &$+I_f$&$-I_f$ & & 0\\
Residential Investment       &  & &$-I_h$&$+I_h$& & & 0\\
\textbf{{[}Output{]}}   & & & &{[}$Y${]}& & & {[}$Y${]}\\
Wages                        &$+W$&& &$-W$& & & 0\\
Profits                      & &$+FD$& &$-FT$&$+FU$& & 0\\
Deposits interest rate         & &$+r_m\cdot M_{-1}$& && &$-r_m\cdot M_{-1}$& 0\\
Loans interest rate         & &$-r_l\cdot L_{k_{-1}}$& &$-r_l\cdot L_{f_{-1}}$& &$+r_l\cdot L_{-1}$& 0\\

Mortages interest rates         & &$-r_{mo}\cdot MO_{-1}$& && &$+r_{mo}\cdot MO_{-1}$& 0\\\hline
\textbf{Subtotal}           &---&$+S_h$&$-I_h$& &$+NFW_f$&$+NFW_b$& 0\\\hline
Change in deposits     & &$-\Delta M$& & & &$+\Delta M$& 0\\
Change in mortgages     & & &$+ \Delta MO$& & &$-\Delta MO$& 0\\
Change in loans     & &$+\Delta L_{k}$&&$+\Delta L_f$& &$-\Delta L$& 0\\
\textbf{Total} & & 0 & 0 & 0  & 0  & 0  & 0\\
\hline
\hline
\end{tabular}%
}
\caption*{\textbf{Source:} Authors' Elaboration}
\end{table}


\section{Firms}
\label{sec:org72e52d6}

In order to produce, firms purchase capital goods (\(-I_f\) in capital account) and hire workers, whom total remuneration is the economy wage bill. 
Their total profits (\(FT\)) are a residual between sales (\(Y\)) and total wages (\(W\)). 
Firms retain part (\(\gamma_F\)) of profits net of interest payments (\(FU\)) --- to reinvest --- and distribute the rest to capitalists (\(FD\)):

\begin{equation}
\label{_FT}
    FT = Y - W = FD + FU
\end{equation}
\begin{equation}
    FU = \gamma_F\cdot (FT - r_l\cdot L_{f_{-1}})
\end{equation}
\begin{equation}
    FD = (1-\gamma_F)\cdot (FT - r_l\cdot L_{f_{-1}})
\end{equation}

Firms (non-residential) investment is fully induced by the level of effective demand (Eq. \ref{_If}), and its growth rate changes accordingly to the capital stock adjustment principle \cite{freitas_growth_2015}.
Equation \ref{_h} in one simple way to describe this mechanism.
According to it, the marginal propensity to invest (\(h\)) endogenously adjust the discrepancies between actual and normal utilization rates (\(u\) and \(u_N\), respectively). To do so, the adjustment parameter (\$\(\gamma\)\_u) must sufficiently small and non-negative\footnote{The size of this parameter guards a fundamental relation to the stability of the model, as shown by \textcite{freitas_growth_2015}.}. 
As a consequence, productive capacity gradually reacts to effective demand movements.

\begin{equation}
\label{_If}
    I_f = h\cdot Y
\end{equation}
\begin{equation}
\label{_h}
    \Delta h = h_{t-1}\cdot \gamma_u\cdot (u - u_N)
\end{equation}
\begin{equation}
    \Delta K_f = I_f
\end{equation}


Firms finance part of investment that exceeds undistributed profits by bank loans, paying an interest rate on it (\(r_l\)) charged by the banks. 
We assume an elastic supply of credit for investment. 
Moreover, tables \ref{Matriz_Estoques} and \ref{Matriz_Fluxos} show firms net wealth (\(NW_f\)) and net financial balance (\(NFW_f\)) explicitly:

\begin{equation}
\label{_Lf}
    \Delta L_f = I_f - FU
\end{equation}
$$
r_g = \frac{1-\omega\cdot u}{v}
$$
$$
r_n = r_g - r_l\cdot\frac{L_{f_{-1}}}{K_f}
$$
\begin{equation}
    NFW_f = FU - I_f
\end{equation}
\begin{equation}
    NW_f = K_f - L_f
\end{equation}
where \(r_g\) and \(r_n\) denotes gross and net profit rate respectively.


\textbf{TODO:} Substituir letra da taxa de lucro por outra.

\section{Banks}
\label{sec:orge2c8486}

As in most part of SFC literature, banks do not have an active role in this model.
They create money as credit is demanded and just after they collect deposits \cite{le_bourva_money_1992}. 
Firms finance part of their investment with credit (\(L_f\)) and capitalists households finance all their residential investment by mortgages (\(MO\)) and consumption by loans (\(L_{k}\)), as already mentioned. 
Each operation has its own interest rate defined by a spread (\(\sigma_l\) and \(\sigma_{mo}\)) over deposits interest rate (\(r_m\)) exogenously determined by banks.
For simplicity, we assume null bank spreads so interest rate on mortgages and on loans
are the same as on deposits.
Banks net balances (\(NFW_b\)) are defined by interests received net of interests payments. 
As those interests are the same, banks net wealth is necessarily zero (see table \ref{Matriz_Estoques}) and deposits are residuum:

\begin{equation}
L = L_f + L_{k}
\end{equation}
\begin{equation}
    r_l = (1+\sigma_l)\cdot r_m
\end{equation}
\begin{equation}
    r_{mo} = (1+\sigma_{mo})\cdot r_m
\end{equation}
\begin{equation}
    r_m = \overline r_m
\end{equation}
\begin{equation}
    NFW_b = r_{mo}\cdot MO_{-1} + r_l\cdot L_{-1} - r_m\cdot M_{-1}
\end{equation}
$$
NFW_b = \Delta MO + \Delta L - \Delta M
$$
\begin{equation}
    NW_b = V_b \equiv 0
\end{equation}
\begin{equation}
\label{_M}
    \Delta M = \Delta L + \Delta MO
\end{equation}

\section{Households}
\label{sec:org0bf7f34}

\subsection*{Workers}
\label{sec:orgbb6b9b0}
As mentioned before, we assume that workers expend (\(C_w\)) what they earn (\(W\)). 
For simplicity, we consider that wages are the only source of income workers' disposable income (\(YD_{w}\)) and do not have access to consumption loans, so worker' saving (\(S_{hw}\)) are null.
Therefore, accordingly to our hypothesis, workers' do not hold both net financial and total wealth.

\begin{equation}
C_w = W
\end{equation}
\begin{equation}
YD_w = W
\end{equation}
\begin{equation}
S_{w} = YD_w - C_w
\end{equation}
$$
S_{w} = 0
$$
\begin{equation}
NFW_{w} = S_{w} = 0
\end{equation}
\begin{equation}
V_{w} = 0
\end{equation}

\subsection*{Capitalists}
\label{sec:org57a8d35}
This is the most complex institutional sector of our model. 
We assume consumption (\(C_k\)) is fully-autonomous and financed by loans (\(L_{k}\)). 
Disposable income (\(YD_k\)) is the sum of distributed profits and received interests on deposits, net of interests payments
on both mortgages and loans.
Capitalists savings (\(S_{k}\)) are disposable income net of consumption.
At odds with SFC literature, savings are not equal to net balance (\(NFW_{k}\)) since we have included residential investment as in Equation \ref{NFWh}.

\begin{equation}
\Delta L_{k} = C_k
\end{equation}
\begin{equation}
    \label{EqYD}
    YD_k = FD + \overline r_m\cdot M_{-1} - r_{mo}\cdot MO_{-1} - r_{l}\cdot L_{k_{-1}}
\end{equation}
\begin{equation}
    \label{EqSh}
    S_{k} = YD_k - C_k
\end{equation}
\begin{equation}
\label{NFWh}
    NFW_{k} = S_{k} - I_h
\end{equation}


As mentioned before, capitalist households are the only institutional section investing in real estate which is financed by mortgages as in equation (\ref{EqMO}). Next, we present residential investment growth rate (\(g_{I_h}\)) as determined by houses own interest rate (\(own\), equation \ref{_own}) as introduced by \textcite{teixeira_crescimento_2015} and discussed in section \ref{sec:empirical}.


\begin{equation}
    \label{EqMO}
    \Delta MO = I_h
\end{equation}





\begin{equation}
    I_h = (1 + g_{I_h})\cdot Ih_{-1}
\end{equation}
\begin{equation}
\label{g_Z_own}
g_{I_h} = \phi_0 - \phi_1\cdot own
\end{equation}

\begin{equation}
\label{_own}
own = \left(\frac{1+r_{mo}}{1+\pi}\right) -1
\end{equation}
$$
\pi = \frac{\Delta p_h}{p_{h_{t-1}}}
$$
where \(\pi\) stands for real estate inflation, \(\phi_0\) represents long-term determinants (\emph{e.g.} demographic factors, housing and credit policies, etc.) while \(\phi_1\) captures the demand for real estate arising from expectations of capital gains resulting from speculation with the existing dwellings stock. 

Accordingly to our hypothesis, nominal (\(V_{nk}\)) and real net wealth (\(V_{k}\)) are distinguished only by the inclusion of real estate price (\(p_h\)) and are defined as follows:
\begin{equation}
V_{k} = K_{hd} + M - L_{k} - MO
\end{equation}
\begin{equation}
V_{nk} = K_{hd}\cdot p_h + M - L_{k} - MO
\end{equation}


In order to fulfill our goals, we employ \citeauthor*{freitas_baseline_2020}'s \citeyear{freitas_baseline_2020} procedure in which NCC autonomous expenditure (\(Z\)) composition (\(R\)) remains unchanged. Thus, we can express capitalists total consumption as follows:

\begin{equation}
\label{_Z}
Z = C_k + I_h
\end{equation}
$$
\frac{C_k}{Z} + \frac{I_h}{Z} = R + (1-R)
$$
\begin{equation}
\label{_Ck}
    C_k = R\cdot Z
\end{equation}
\begin{equation}
\label{ConsumoTotal}
C = C_w + C_k
\end{equation}
$$
C = C_w + R\cdot Z
$$
which allows us to rewrite both NCC and autonomous consumption in terms of residential investment (Eq. \ref{Z_Ih}):
\begin{equation}
\label{Z_Ih}
Z = \frac{I_h}{(1-R)}
\end{equation}

\begin{equation}
\label{C_kZ}
C\textsubscript{k} = I\_h\(\cdot\) \frac{R}{(1-R)}
\end{equation}

Finally, replacing Equation \ref{Z_Ih} in \ref{_Ck}, we can describe NCC autonomous expenditure growth rate as follows:
\begin{equation}
\label{g_Z}
g\textsubscript{C\textsubscript{k}} = g\textsubscript{Z} = g\textsubscript{I\textsubscript{h}} = \(\phi\)\textsubscript{0} - \(\phi\)\textsubscript{1}\(\cdot\) own
\end{equation}

In this section, we presented a fully-specified parsimonious model to connect asset bubbles with aggregate demand. It worth mentioning that, although simplified, our hypotheses are supported data as discussed in Section. 
%Following textcites:teixeira_crescimento_2015,petrini_demanda_2019, we specify a econometrically significant residential investment growth rate function which allows us to include housing bubbles in the SMM model.
On the next Section, we present the short-run and fully-adjusted position dynamics in order to show the particularities of a model with two types of capital stock in the presence of asset bubble.
\end{document}
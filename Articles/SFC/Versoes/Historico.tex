% Created 2020-10-15 qui 18:33
% Intended LaTeX compiler: pdflatex
\documentclass[11pt]{article}
\usepackage[utf8]{inputenc}
\usepackage{lmodern}
\usepackage[T1]{fontenc}
\usepackage[top=3cm, bottom=2cm, left=3cm, right=2cm]{geometry}
\usepackage{graphicx}
\usepackage{longtable}
\usepackage{float}
\usepackage{wrapfig}
\usepackage{rotating}
\usepackage[normalem]{ulem}
\usepackage{amsmath}
\usepackage{textcomp}
\usepackage{marvosym}
\usepackage{wasysym}
\usepackage{amssymb}
\usepackage{amsmath}
\usepackage[theorems, skins]{tcolorbox}
\usepackage[style=abnt,noslsn,extrayear,uniquename=init,giveninits,justify,sccite,
scbib,repeattitles,doi=false,isbn=false,url=false,maxcitenames=2,
natbib=true,backend=biber]{biblatex}
\usepackage{url}
\usepackage[cache=false]{minted}
\usepackage[linktocpage,pdfstartview=FitH,colorlinks,
linkcolor=blue,anchorcolor=blue,
citecolor=blue,filecolor=blue,menucolor=blue,urlcolor=blue]{hyperref}
\usepackage{attachfile}
\usepackage{setspace}
\usepackage{tikz}
\author{Gabriel Petrini}
\date{2020}
\title{Histórico de versões artigo SFC}
\begin{document}

\maketitle
\tableofcontents



\section{Versão \textit{<2020-10-16 sex>}}
\label{sec:org47be807}

\subsection{Introdução}
\label{sec:orgf1e2461}

Não alterei nada do texto que estava na versão anterior. Apenas adicionei dois parágrafos (preliminares) em que tenho fazer uma mini revisão da literatura. Acho que não é muito interessante aumentar mais ainda a introdução. Digo isso porque a maioria dos textos citados na antiga revisão da literatura não eram sobre investimento residencial, mas sim sobre gastos autônomos. A discussão sobre investimento residencial ali era minúscula.

\subsection{Seção dos fatos estilizados}
\label{sec:org8eb5a30}

Não alterei nada do texto. Apenas substitui o gráfico do preço dos imóveis x endividamento (em número índice) pela curva de concentração. Estou pensando em como deixar esse gráfico mais \emph{printer-friendly} (talvez deixar os as curvas como linhas cheias e tracejadas).

\subsection{Revisão da literatura}
\label{sec:org43fd6a2}

Não alterei nada dessa seção. Apenas pedi para não compilar (dado que optamos por tirá-la).

\subsection{Modelo}
\label{sec:org5155526}

Não houve nenhuma mudança radical. Fiz as correções que sugeriu. Mais no fim dessa seção reorganizei a ordem das equações de um jeito que fizesse mais sentido. Para isso fiz pequenas adaptações no texto. Também removi (comentei) a parte que igualava a oferta à demanda por imóveis. Não parecem ser equações essenciais. Se achar necessário, desfaço isso. De todo modo, achei que estavam jogadas ali. Seria o caso de repensar onde incluí-las.

Dentre as perfumarias (estão nas minhas respostas aos seus comentários), ainda estou pensando em qual letra usar no lugar de taxa de lucro bruta e líquida.

\subsection{Solução analítica}
\label{sec:orge62f0c0}

Fiz um meio termo entre as suas equações e as minhas anteriores. Em um primeiro momento, priorizei aquelas equações que me enviou no PDF. Apenas rearranjei a posição de algumas variáveis. Se achar melhor, deixo da forma que me enviou. Corrigi e atualizei a equação do \(k\). Estava faltando o termo \((1-R)\).

Adicionei a discussão da \emph{traverse} partindo relação entre \(g\) e \(g_K\) expressa na sua equação do grau de utilização. Inclui uma nota de rodapé pontuando a diferença entre a traverse e o \emph{medium-run} do Freitas e Serrano. Poderia dar uma olhada nesta nota de rodapé? não estou contente com o resultado.

Na subseção da \emph{fully-adjusted position}, apenas retirei as equações desnecessárias e adaptei para ficar de uma forma mais elegante que o \emph{output} do \emph{python}. Deixei registrado as etapas até chegar nos \emph{stock-flow ratio} porque pensamos em mudar isso para o apêndice. Nesta versão, as variáveis ainda estão normalizadas pelo estoque de capital das firmas ou de casas. Ainda não alterei para a normalização pela renda porque não sei se aquelas notas que te envie estavam certas.

\subsection{Simulações}
\label{sec:org8497d01}

Essa foi a seção que teve mais mudança no texto. Acho que a principal coisa para nos preocuparmos aqui é se faz sentido aquele "novo paradoxo" sumir a partir do valor dos parâmetros. Ainda acho que esse resultado pode ter sumido antes e eu não vi por conta do efeito de escala no gráfico anterior. O mais estranho é que os parâmetros que estão diferentes em relação à dissertação não aparecem nas derivadas parciais do documento que te enviei. Outro possível problema é que esse paradoxo era bem pequeno na dissertação. Ainda assim, acho estranho mudar de direção.

Algumas observações em relação aos gráficos: 
\begin{itemize}
\item Na figura 6 (página 19), desconsidere o segundo gráfico da primeira linha e o primeiro da segunda (\(u\) e \(Z/Y\)). Estão com os títulos e a linha tracejada trocados. Consigo arrumar isso rápido, mas quero resolver outras questões antes (essa simulação leva mais tempo para rodar também)
\item Pretendo incluir identificadores (A, B, C e D) nos gráficos para mencionar no texto com mais facilidade.
\item Dadas as dicas do Rochon, estou pensando em formas de deixar os gráficos mais \emph{printer-friendly}, mas ainda sim legíveis.
\begin{itemize}
\item Na maioria dos gráficos, consigo diferenciar cada linha em termos de estilo do tracejado. No entanto, isso é mais difícil no gráfico de ciclos (primeiro gráfico figura 5).
\begin{itemize}
\item No gráfico com os dados reais, isso já não é um problema já que é uma única série
\end{itemize}
\end{itemize}
\end{itemize}
\subsection{Conclusões}
\label{sec:org81a9bbe}
Não alterei nada
\subsection{Apêndice A}
\label{sec:org3a83492}

Vi nos seus comentários que a nossa diferença com Freitas e Serrano não é tão importante. Dito isso, optei por comentar toda aquela discussão da estabilidade (é a mesma que a deles com a diferença de que \(own < \frac{\phi_0}{\phi_1}\)). Comecei esse apêndice a partir do trecho em que você indicou ser outro apêndice. Nesta nova versão, existem várias equações tal como saídas do \emph{python}. Não comecei a melhorá-las porque está em aberto se vamos priorizar a normalização pelo capital ou pela renda. De todo modo, imagino que as derivadas do \(k\) não devem mudar muito (já está com o \(k\) corrigido). Também achei melhor tirar as derivadas do estoque de moeda (uma vez que é resíduo, não tem muita importância para nós).

\subsection{Apêndice B}
\label{sec:orgea96fff}

Talvez essa tabela mude na medida que forem usados outros valores dos parâmetros. Uma vez encerradas as alterações no modelo, rodo o comando para gerar essa tabela. Ainda não descobri como formatar melhor pelo python. Essa é uma das poucas coisas que acho melhor manter o mais próximo do \emph{output} do computador. Se um parecerista pedir o \emph{script}, ele deve chegar nesse mesmo lugar. Além disso, imagino que esse processo de formatação seja função do journal e não nossa.

\subsection{Apêndice C (em construção)}
\label{sec:org172efcf}

Eu tinha iniciado (faz um tempo) um código para retornar uma tabela de análise de sensibilidade similar à do Fazzari. Esse é um código que leva um certo tempo para rodar porque faço vários \emph{nested for loops} (a.k.a simulo um porrilhão de cenários rs). No final das contas acabei deixando de lado. Havendo tempo, penso em retornar a isso. Imagino que se desse certo, seria bastante interessante mas também acho que não esta entre as nossas prioridades. Talvez pudesse trabalhar nisso enquanto o artigo estivesse sendo revisado na expectativa de que algum parecerista peça. Por fim, esse apêndice inexistente é um dos motivos de ter tantos \emph{hashs} ("lixos") no fim do apêndice B.
\end{document}
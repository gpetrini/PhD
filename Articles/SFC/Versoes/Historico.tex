% Created 2020-11-11 qua 11:32
% Intended LaTeX compiler: pdflatex
\documentclass[11pt]{article}
\usepackage[utf8]{inputenc}
\usepackage{lmodern}
\usepackage[T1]{fontenc}
\usepackage[top=3cm, bottom=2cm, left=3cm, right=2cm]{geometry}
\usepackage{graphicx}
\usepackage{longtable}
\usepackage{float}
\usepackage{wrapfig}
\usepackage{rotating}
\usepackage[normalem]{ulem}
\usepackage{amsmath}
\usepackage{textcomp}
\usepackage{marvosym}
\usepackage{wasysym}
\usepackage{amssymb}
\usepackage{amsmath}
\usepackage[theorems, skins]{tcolorbox}
\usepackage[style=abnt,noslsn,extrayear,uniquename=init,giveninits,justify,sccite,
scbib,repeattitles,doi=false,isbn=false,url=false,maxcitenames=2,
natbib=true,backend=biber]{biblatex}
\usepackage{url}
\usepackage[cache=false]{minted}
\usepackage[linktocpage,pdfstartview=FitH,colorlinks,
linkcolor=blue,anchorcolor=blue,
citecolor=blue,filecolor=blue,menucolor=blue,urlcolor=blue]{hyperref}
\usepackage{attachfile}
\usepackage{setspace}
\usepackage{tikz}
\addbibresource{../refs.bib}
\author{Gabriel Petrini}
\date{2020}
\title{Histórico de versões artigo SFC}
\begin{document}

\maketitle

\section*{Versão \textit{<2020-10-16 sex>}}
\label{sec:org362d4fc}

\subsection*{Introdução}
\label{sec:org803ba43}

Não alterei nada do texto que estava na versão anterior. Apenas adicionei dois parágrafos (preliminares) em que tenho fazer uma mini revisão da literatura. Acho que não é muito interessante aumentar mais ainda a introdução. Digo isso porque a maioria dos textos citados na antiga revisão da literatura não eram sobre investimento residencial, mas sim sobre gastos autônomos. A discussão sobre investimento residencial ali era minúscula.

\subsection*{Seção dos fatos estilizados}
\label{sec:org274d0dc}

Não alterei nada do texto. Apenas substitui o gráfico do preço dos imóveis x endividamento (em número índice) pela curva de concentração. Estou pensando em como deixar esse gráfico mais \emph{printer-friendly} (talvez deixar os as curvas como linhas cheias e tracejadas).

\subsection*{Revisão da literatura}
\label{sec:orge9f0ce7}

Não alterei nada dessa seção. Apenas pedi para não compilar (dado que optamos por tirá-la).

\subsection*{Modelo}
\label{sec:org624838f}

Não houve nenhuma mudança radical. Fiz as correções que sugeriu. Mais no fim dessa seção reorganizei a ordem das equações de um jeito que fizesse mais sentido. Para isso fiz pequenas adaptações no texto. Também removi (comentei) a parte que igualava a oferta à demanda por imóveis. Não parecem ser equações essenciais. Se achar necessário, desfaço isso. De todo modo, achei que estavam jogadas ali. Seria o caso de repensar onde incluí-las.

Dentre as perfumarias (estão nas minhas respostas aos seus comentários), ainda estou pensando em qual letra usar no lugar de taxa de lucro bruta e líquida.

\subsection*{Solução analítica}
\label{sec:org7e4a99e}

Fiz um meio termo entre as suas equações e as minhas anteriores. Em um primeiro momento, priorizei aquelas equações que me enviou no PDF. Apenas rearranjei a posição de algumas variáveis. Se achar melhor, deixo da forma que me enviou. Corrigi e atualizei a equação do \(k\). Estava faltando o termo \((1-R)\).

Adicionei a discussão da \emph{traverse} partindo relação entre \(g\) e \(g_K\) expressa na sua equação do grau de utilização. Inclui uma nota de rodapé pontuando a diferença entre a traverse e o \emph{medium-run} do Freitas e Serrano. Poderia dar uma olhada nesta nota de rodapé? não estou contente com o resultado.

Na subseção da \emph{fully-adjusted position}, apenas retirei as equações desnecessárias e adaptei para ficar de uma forma mais elegante que o \emph{output} do \emph{python}. Deixei registrado as etapas até chegar nos \emph{stock-flow ratio} porque pensamos em mudar isso para o apêndice. Nesta versão, as variáveis ainda estão normalizadas pelo estoque de capital das firmas ou de casas. Ainda não alterei para a normalização pela renda porque não sei se aquelas notas que te envie estavam certas.

\subsection*{Simulações}
\label{sec:orge812c17}

Essa foi a seção que teve mais mudança no texto. Acho que a principal coisa para nos preocuparmos aqui é se faz sentido aquele ``novo paradoxo'' sumir a partir do valor dos parâmetros. Ainda acho que esse resultado pode ter sumido antes e eu não vi por conta do efeito de escala no gráfico anterior. O mais estranho é que os parâmetros que estão diferentes em relação à dissertação não aparecem nas derivadas parciais do documento que te enviei. Outro possível problema é que esse paradoxo era bem pequeno na dissertação. Ainda assim, acho estranho mudar de direção.

Algumas observações em relação aos gráficos: 
\begin{itemize}
\item Na figura 6 (página 19), desconsidere o segundo gráfico da primeira linha e o primeiro da segunda (\(u\) e \(Z/Y\)). Estão com os títulos e a linha tracejada trocados. Consigo arrumar isso rápido, mas quero resolver outras questões antes (essa simulação leva mais tempo para rodar também)
\item Pretendo incluir identificadores (A, B, C e D) nos gráficos para mencionar no texto com mais facilidade.
\item Dadas as dicas do Rochon, estou pensando em formas de deixar os gráficos mais \emph{printer-friendly}, mas ainda sim legíveis.
\begin{itemize}
\item Na maioria dos gráficos, consigo diferenciar cada linha em termos de estilo do tracejado. No entanto, isso é mais difícil no gráfico de ciclos (primeiro gráfico figura 5).
\begin{itemize}
\item No gráfico com os dados reais, isso já não é um problema já que é uma única série
\end{itemize}
\end{itemize}
\end{itemize}
\subsection*{Conclusões}
\label{sec:org0e33752}
Não alterei nada
\subsection*{Apêndice A}
\label{sec:orgcc9b78a}

Vi nos seus comentários que a nossa diferença com Freitas e Serrano não é tão importante. Dito isso, optei por comentar toda aquela discussão da estabilidade (é a mesma que a deles com a diferença de que \(own < \frac{\phi_0}{\phi_1}\)). Comecei esse apêndice a partir do trecho em que você indicou ser outro apêndice. Nesta nova versão, existem várias equações tal como saídas do \emph{python}. Não comecei a melhorá-las porque está em aberto se vamos priorizar a normalização pelo capital ou pela renda. De todo modo, imagino que as derivadas do \(k\) não devem mudar muito (já está com o \(k\) corrigido). Também achei melhor tirar as derivadas do estoque de moeda (uma vez que é resíduo, não tem muita importância para nós).

\subsection*{Apêndice B}
\label{sec:orge776442}

Talvez essa tabela mude na medida que forem usados outros valores dos parâmetros. Uma vez encerradas as alterações no modelo, rodo o comando para gerar essa tabela. Ainda não descobri como formatar melhor pelo python. Essa é uma das poucas coisas que acho melhor manter o mais próximo do \emph{output} do computador. Se um parecerista pedir o \emph{script}, ele deve chegar nesse mesmo lugar. Além disso, imagino que esse processo de formatação seja função do journal e não nossa.

\subsection*{Apêndice C (em construção)}
\label{sec:orgbac1a93}

Eu tinha iniciado (faz um tempo) um código para retornar uma tabela de análise de sensibilidade similar à do Fazzari. Esse é um código que leva um certo tempo para rodar porque faço vários \emph{nested for loops} (a.k.a simulo um porrilhão de cenários rs). No final das contas acabei deixando de lado. Havendo tempo, penso em retornar a isso. Imagino que se desse certo, seria bastante interessante mas também acho que não esta entre as nossas prioridades. Talvez pudesse trabalhar nisso enquanto o artigo estivesse sendo revisado na expectativa de que algum parecerista peça. Por fim, esse apêndice inexistente é um dos motivos de ter tantos \emph{hashs} (``lixos'') no fim do apêndice B.

\section*{Versão \textit{<2020-11-03 ter>}}
\label{sec:org1870ee9}

\subsection*{Seção empírica}
\label{sec:org68973cc}

\begin{itemize}
\item Feitas algumas pequenas correções no inglês
\item Incluídas discussões empíricas (Green, Leamer e Fiebiger)
\begin{itemize}
\item O mesmo foi feito para artigos econometricos
\end{itemize}
\item Dividida em duas subseções
\item Ajustadas proporções e qualidade dos gráficos
\item Gráficos foram adaptados para escala de cinza
\item Foram incluídas algumas referências empíricas
\item Foram deixadas marcas de edição para inclusão posterior da taxa própria (sugestão)
\item Incluída explicação da curva de concentração
\item Testado gráfico das curvas de concentração em diferentes axis
\item Algumas padronizações dos gráficos (tamanho da legenda e dos eixos)
\end{itemize}


\subsection*{Modelo}
\label{sec:org98c4940}

\begin{itemize}
\item[{$\square$}] Corrigidas referências quebradas (estavam com ??) desta seção
\end{itemize}

\subsection*{Analítica}
\label{sec:orga256dbf}

\begin{itemize}
\item[{$\square$}] Corrigidas referências quebradas (estavam com ??) desta seção
\end{itemize}

\subsection*{Simulações}
\label{sec:orgbef91b1}

\begin{itemize}
\item Gráficos foram adaptados para escala de cinza
\item Foram adicionados identificadores aos gráficos (A, B, C e D)
\item Correções de legenda nos gráficos
\item Gráfico das simulações com dados reais corrigido
\item Corrigida exportação para latex no Apêndice B
\item Retomado apêndice C
\item Corrigidas referências quebradas (estavam com ??) desta seção
\item[{$\square$}] Reduzido espaçamento entre título e o gráfico
\item Valor de \(\alpha\) alterado para 1.0 e \(\omega = 0.25\) (igual a \(\alpha\cdot \omega\) do Fazzari)
\end{itemize}

\section*{Versão \textit{<2020-11-06 sex>}}
\label{sec:org3f702af}

\subsection*{Introdução}
\label{sec:org3c1b4ae}

Inalterada em relação à versão anterior

\subsection*{Seção Empírica}
\label{sec:org4a5c13e}

Inalterada em relação à versão anterior

\subsection*{Modelo}
\label{sec:orga5ca41b}

\subsubsection*{Equações gerais}
\label{sec:org81eaf7e}

\begin{itemize}
\item Feitas as correções sugeridas. Cabe pontuar
\begin{itemize}
\item Sraffian Strands \(\Rightarrow\) Sraffian literature
\item \emph{while capitalists earn what they expend} mantido
\end{itemize}
\end{itemize}

\subsubsection*{Firmas}
\label{sec:orgbe27bc1}
\begin{itemize}
\item Feitas as correções sugeridas. Vale pontuar
\begin{itemize}
\item Não foram incluídas as equações da taxa de crescimento do investimento das firmas
\begin{itemize}
\item Conforme discutido via Telegram
\end{itemize}
\item \emph{To do so} (p. 9) \(\Rightarrow\) \emph{For this mechanism to take place}
\end{itemize}
\end{itemize}
\subsubsection*{Bancos}
\label{sec:org1ae6434}
\begin{itemize}
\item Feitas as correções sugeridas. Vale pontuar
\begin{itemize}
\item Não foi alterada a letra que representa taxa de juros
\begin{itemize}
\item \textbf{Motivo:} Falta de ideias/pragmatismo (rs)
\end{itemize}
\item Concordo que as duas frases sobre ter taxas de juros específicas estavam contraditórias (p. 10 do pdf comentado). Alterei para
\end{itemize}
\end{itemize}

\begin{quote}
For simplicity, we assume null bank spreads (\(\sigma_{mo} = \sigma_l = 0\)) so interest rate on mortgages (\(r_{mo}\)) and on loans (\(r_{l}\))
are the same as on deposits (\(r_{m}\)) which is  exogenously determined by banks.
\end{quote}


\subsubsection*{Trabalhadores}
\label{sec:org3263e62}
\begin{itemize}
\item Feitas as correções sugeridas. Nada a comentar
\end{itemize}
\subsubsection*{Capitalistas}
\label{sec:org4ec7c6b}
\begin{itemize}
\item Feitas as correções sugeridas. Vale pontuar
\begin{itemize}
\item Dívida total dos capitalistas (\(D\)) definida
\item Funções da taxa própria movidas para a seção empírica. Função da taxa de crescimento to investimento residencial mantida
\item Adicionadas referências sobre tendência do mercado imobiliário e institucionalidade das hipotecas (ao descrever \(\phi_0\))
\begin{itemize}
\item Citar orientando do Lavoie? \cite{gowans_introducing_2014}
\begin{itemize}
\item Discussão sobre determinantes demográficos da taxa de crescimento do investimento residencial. Resumidamente, argumenta por ai do porquê investimento residencial é NCC.
\item \href{https://ruor.uottawa.ca/bitstream/10393/32013/1/Gowans\_Dylan\_2014\_researchpaper.pdf}{Link}
\end{itemize}
\end{itemize}
\item Trecho sobre \(\phi_1\) na equação da taxa de crescimento do investimento residencial alterado para
\end{itemize}
\end{itemize}

\begin{quote}
\(\phi_1\) captures the demand for housing arising from expectations of capital gains resulting from speculation with flow of new houses. 
\end{quote}

\subsection*{Solução analítica}
\label{sec:orgcf83b1c}
\subsubsection*{Short-run}
\label{sec:orge06ba4a}

\begin{itemize}
\item Feitas correções sugeridas. Vale pontuar
\begin{itemize}
\item Não foi adicionada discussão sobre os efeitos de mudanças na taxa de crescimento do investimento residencial em \(k\)
\begin{itemize}
\item Anteriormente estava no fim da seção seguinte
\end{itemize}
\end{itemize}
\end{itemize}

\subsubsection*{Analitical Traverse}
\label{sec:org2c92291}

\begin{itemize}
\item Unida com a subseção seguinte e renomeada para \emph{Traverse and long-run Equilibrium}
\begin{itemize}
\item Nota de rodapé pontuando a diferença com \textcite{freitas_growth_2015} esta ok?
\end{itemize}
\item Mantidas etapas para obter os \emph{stock ratios}
\begin{itemize}
\item \textbf{Motivo:} Caso fossem movidas para um apêndice, faria sentido apresentar as etapas em maiores detalhes. Acho que não temos tanto espaço sobrando assim
\end{itemize}
\item Conforme mencionado, parágrafo final que tratada dos impactos de \(g_{I_h}\) em \(k\) foi removido
\end{itemize}
\subsection*{Simulações}
\label{sec:orgec8e83a}
Inalterada em relação à versão anterior
\subsection*{Conclusão}
\label{sec:orgfefc751}
Inalterada em relação à versão anterior
\subsection*{Apêndice A}
\label{sec:org53b3937}

\begin{itemize}
\item Resultados do python para \(k^\star\) e \(\ell_f^\star\) foram rearranjados
\begin{itemize}
\item Mantidas versões antigas para comparação
\end{itemize}
\item Documento Endividamentos.pdf corrigido em \textit{<2020-11-04 qua>}
\begin{itemize}
\item Já enviado
\end{itemize}
\end{itemize}

\section*{Versão \textit{<2020-11-11 qua>}}
\label{sec:orgdbd008e}

\subsection*{Introdução}
\label{sec:org02ba475}

Não alterado.

\subsection*{Fatos estilizados}
\label{sec:org2b4f7f8}

Não alterado.

\subsection*{Modelo}
\label{sec:orgd2b3564}

Não alterado.

\subsection*{Solução analítica}
\label{sec:orgf50ae67}

\begin{itemize}
\item Algumas correções pequenas
\begin{itemize}
\item \(l_f^\star \Rightarrow \ell_f^\star\)
\item \(l^\star \Rightarrow \ell_k^\star\)
\end{itemize}
\item Razão \(\ell_f^\star\) rearranjada. Passos
\end{itemize}

\begin{equation}
  \tag{Original}
  \ell_f^\star = \frac{g^\star\cdot v - \gamma_F\cdot u^\star (1-\omega)}{v\cdot (g^\star - \gamma_F\cdot r_m)}
\end{equation}

Lembrando a definição de taxa de lucro bruta e simplificando por \(v\):

\begin{equation}
  \ell_f^\star = \frac{g^\star - \gamma_Fr_g^\star}{g^\star - \gamma_F\cdot r_m}
\end{equation}


Somando e subtraindo \(\gamma_F\cdot r_m\) do numerador e simplificando

\begin{equation}
  \ell_f^\star = 1 - \gamma_F\frac{r_g^\star - r_m}{g^\star - \gamma_F\cdot r_m}
\end{equation}


Relembrando a equação da taxa de lucro líquida e fazendo \(r_l = r_m\),

\begin{equation}
  r_n^\star = r_g - r_m\cdot \ell_f^\star
\end{equation}


Desse modo, se \(\ell_f^\star\) reduz, taxa de lucro bruta e líquida se aproximam.

Deixei a equação original enquanto etapa para chegar nessa nova equação e adicionei uma para a taxa de lucro líquida em função de \(\ell_f^\star\) para referenciar nas simulações.

\subsection*{Simulações}
\label{sec:org7f248cd}

\begin{itemize}
\item Para manter consistência entre as equações e simulações, será plotada a razão dívida dos capitalistas/imóveis. O que acha?
\item Tabela com resumo dos resultados corrigida
\begin{itemize}
\item \(k \Rightarrow K_k\)
\item Ajustada equação endividamento
\end{itemize}
\end{itemize}



\subsubsection*{Introdução}
\label{sec:orgfcf00e5}

Feitas as correções sugeridas.

\subsubsection*{Choque na taxa de salários}
\label{sec:orgc207e2f}

\begin{itemize}
\item Primeira frase alterada para
\end{itemize}

\begin{quote}
A wage-share decrease has permanent negative impact on output level --- due to changes on the supermultiplier --- and temporary negative effects on degree of capacity utilization (see Figure 5 A).
\end{quote}

\begin{itemize}
\item Lower \(\Rightarrow\) smaller
\item Expenditure -> Expenditures
\item This result stem from the \(\Rightarrow\) This result is explained by
\item Figuras das simulações (e a primeira com dados reais) voltaram a ser coloridas
\begin{itemize}
\item Tanto da seção dos fatos estilizados quanto os demais graficos da simulação com dados reais continuam em escala de cinza
\end{itemize}
\item Conforme falei por menssagem, não temos mais o paradoxo nessa seção. Em função disso, comentei as frases que descrevem esse resultado.
\item Corrigi \emph{oan-to-profit ratio} para \emph{loan-to-capital ratio}
\end{itemize}
\subsubsection*{Aumento da inflação de imóveis}
\label{sec:orgce68f86}

Feitas as correções sugeridas. Vale destacar que o último parágrafo foi alterado para

\begin{quote}
The most distinct result is real houses share decreases on total capital stock as a result of residential investment growth rate increase.
Although counterintuitive, this result is similar to the conventional paradox of debt.
This is the case since houses are always equivalent to  mortgage debt.
Additionally, this result is in line with SSM literature.
Firms investment follows capital stock adjustment principle, so a higher firms investment growth rate implies that
GDP grows faster than residential investment.
Thus, both residential investment share on GDP and degree of capacity  utilization decrease.
In other words, both autonomous expenditures share on GDP and houses share on real assets (see Equation X)\footnote{Não consigo referenciar as equações do paper nesse documento.} decline as a result of the already described non-residential investment positive reaction.
\end{quote}


A seguinte frase foi removida: Besides the usual SSM results, we report some particularities regarding real assets composition.

\begin{itemize}
\item Pareceu muito genérica e pouco descritiva.
\end{itemize}


\subsubsection*{Aumento da taxa de juros}
\label{sec:org74e0867}
\begin{itemize}
\item This result stem from the \(\Rightarrow\) This effect is a result of the
\end{itemize}

\subsubsection*{Dados reais}
\label{sec:org8dffb25}

\begin{itemize}
\item Mencionei a seção dos fatos estilizados
\item Deixei explícito que a taxa de crescimento do consumo dos capitalistas passou a ser
\end{itemize}

\begin{equation}
  \label{gck_real}
  g_{C_k} = \frac{1}{N} \sum_{t=1992}^{2019} g_{I_{h_{t}}}
\end{equation}

\begin{itemize}
\item Não alterei os fatos estilizados itemizados porque a seção de motivação empírica ainda está em aberto
\item Deixei a marca de edição que não replicamos a maior volatilidade do investimento residencial
\item Primeiro gráfico desta simulação passou a ser colorido
\begin{itemize}
\item Note que o investimento das firmas é o mais volátil
\end{itemize}
\item Removi a seguinte frase: \emph{Since the analyzed period does not correspond to the fully-adjusted position, discrepancies between actual and normal capacity utilization rate are adjusted through changes in marginal propensity to invest}
\begin{itemize}
\item Além de estar confusa (como apontado), o texto pareceu mais fluido depois que tirei.
\end{itemize}
\end{itemize}

\section*{Referências}
\label{sec:org9d58dd1}
\printbibliography[heading=none]
\end{document}
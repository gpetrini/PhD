% Created 2020-11-04 qua 10:29
% Intended LaTeX compiler: pdflatex
\documentclass[11pt]{article}
\usepackage[utf8]{inputenc}
\usepackage{lmodern}
\usepackage[T1]{fontenc}
\usepackage[top=3cm, bottom=2cm, left=3cm, right=2cm]{geometry}
\usepackage{graphicx}
\usepackage{longtable}
\usepackage{float}
\usepackage{wrapfig}
\usepackage{rotating}
\usepackage[normalem]{ulem}
\usepackage{amsmath}
\usepackage{textcomp}
\usepackage{marvosym}
\usepackage{wasysym}
\usepackage{amssymb}
\usepackage{amsmath}
\usepackage[theorems, skins]{tcolorbox}
\usepackage[style=abnt,noslsn,extrayear,uniquename=init,giveninits,justify,sccite,
scbib,repeattitles,doi=false,isbn=false,url=false,maxcitenames=2,
natbib=true,backend=biber]{biblatex}
\usepackage{url}
\usepackage[cache=false]{minted}
\usepackage[linktocpage,pdfstartview=FitH,colorlinks,
linkcolor=blue,anchorcolor=blue,
citecolor=blue,filecolor=blue,menucolor=blue,urlcolor=blue]{hyperref}
\usepackage{attachfile}
\usepackage{setspace}
\usepackage{tikz}
\author{Gabriel Petrini}
\date{14 de Outubro de 2020}
\title{Desenvolvimento de equações para endividamento sobre a renda}
\begin{document}

\maketitle
O objetivo dessa nota é apresentar o desenvolvimento algébrico para obter a relação endividamento das firmas e das famílias capitalistas em termos da renda (respectivamente \(\ell_{fy}\) e \(\ell_{ky}\))\footnote{Depois pensamos em uma notação melhor}. 

\section*{Endividamento das firmas}
\label{sec:orgb5d8f80}

Na versão atual do artigo, temos o endividamento das firmas em relação ao seu capital e será denotado por \(\ell_f\)

$$
\ell_f = \frac{L_f}{K_f}
$$
podemos reescrever da seguinte maneira
$$
\ell_f = \frac{L_f}{Y}\frac{Y}{Y_{fc}}\frac{Y_fc}{K_f}
$$
$$
\ell_f = \ell_{fy}\frac{u}{v}
$$
rearranjando
$$
\ell_{fy} = \ell_f\frac{v}{u} \Rightarrow \ell_{fy}^\star = \ell_f^\star\frac{v}{u}
$$

Sendo assim, podemos dividir a norma de \emph{steady state} entre o endividamento das firmas e seu estoque de capital (\(\ell_f^\star\)) apresentada na equação 57 pelo grau de utilização e multiplicar pela relação técnica capital produto para obter \(\ell_{fy}^\star\). Vou repetir a equação 57 abaixo para facilitar\footnote{Lembrando que estou pensando ainda por qual letra irei substituir as taxas de juros/lucro.}:

\begin{equation}
\tag{57}
\ell_f^\star = \frac{g^\star\cdot v + \gamma_F\cdot u^\star\cdot (1-\omega)}{v\cdot (g^\star - \gamma_F \cdot r_m)}
\end{equation}
Multiplicando por \(v\)

\begin{equation}
\ell_f^\star\cdot v = \frac{g^\star\cdot v + \gamma_F\cdot u^\star\cdot (1-\omega)}{g^\star - \gamma_F \cdot r_m}
\end{equation}
Dividindo por \(u\star\)
\begin{equation}
\ell_f^\star\cdot \frac{v}{u\star} = \ell_{fy} = \frac{\frac{g^\star\cdot v}{u^\star} + \gamma_F\cdot (1-\omega)}{g^\star - \gamma_F\cdot r_m}
\end{equation}
lembrando da definição de \(h^\star\):

\begin{equation}
\label{endiv_firm}
\ell_{fy} = \frac{h^\star + \gamma_F\cdot (1-\omega)}{g^\star - \gamma_F\cdot r_m}
\end{equation}

\section*{Endividamento das famílias capitalistas}
\label{sec:org1d7da5e}

Nesta seção, irei indicar as etapas para obter o endividamento capitalista em termos da renda disponível. Para isso, seguem algumas definições e hipóteses\footnote{Ainda estou pensando em quais letrar utilizar na versão final}:

\begin{itemize}
\item \(DT\): Dívida total \(= L_k + MO\)
\item \(DL\): Dívida líquida dos depósitos \(= L_k + MO - M\)
\item Não há spread sobre a taxa básica de juros: \(r_m = r_{mo} = r_l\)
\end{itemize}

Relembrando algumas identidades contábeis, temos

$$
M = L_f + L_k + MO
$$
então
$$
M - L_k - MO = L_f
$$
logo, 
\begin{equation}
\label{DL}
DL = -L_f
\end{equation}

Dito isso, vou reescrever a equação da renda disponível das famílias capitalistas abaixo tal como a equação 32 do artigo

\begin{equation}
\tag{32}
YD_k = FD + r_m\cdot M_{-1} - r_{mo}\cdot MO_{-1}  - r_{L}\cdot L_{k_{-1}}
\end{equation}
fazendo \(r_m = r_{mo} = r_l\)
\begin{equation}
YD_k = FD + r_m\cdot (M_{-1} - MO_{-1}  - L_{k_{-1}})
\end{equation}
que é equivalente à 
\begin{equation}
YD_k = FD - r_m\cdot DL_{-1}
\end{equation}
ou ainda
\begin{equation}
YD_k = FD + r_m\cdot L_{f_{-1}}
\end{equation}

Substituindo a equação dos lucros distribuídos (Eq 12) na equação anterior, temos

\begin{equation}
YD_k = (1-\gamma_F)\cdot ((1-\omega)Y - r_m L_{f_{-1}}) + r_m\cdot L_{f_{-1}}
\end{equation}

\begin{equation}
\label{New_YD}
YD_k = (1-\gamma_F)\cdot (1-\omega)Y - r_m \cdot \gamma_F \cdot L_{f_{-1}}
\end{equation}

Com isso, podemos encontrar a relação entre endividamento (líquido) capitalista e renda disponível (\(\ell_k^\star\))

$$
\ell_{ky}^\star = \frac{DL}{YD_k}
$$

Substituindo \ref{DL} e \ref{New_YD} na relação acima, temos
\begin{equation}
\ell_{ky}^\star = -\frac{L_{f_{-1}}}{Y}\frac{1}{(1-\gamma_F)\cdot (1-\omega)} - \frac{-L_{f_{-1}}}{r_m\cdot \gamma_F\cdot L_{f_{-1}}}
\end{equation}

\begin{equation}
\ell_k^\star = -\frac{L_{f_{-1}}}{Y}\frac{1}{(1-\gamma_F)\cdot (1-\omega)} + \frac{1}{r_m\cdot \gamma_F}
\end{equation}

\begin{equation}
\label{endiv_k_parcial}
\ell_{ky}^\star = \frac{1}{r_m\cdot \gamma_F} - \frac{\ell_{fy}\star}{(1-\gamma_F)\cdot (1-\omega)}
\end{equation}

Finalmente, substituindo \ref{endiv_firm} em \ref{endiv_k_parcial}, temos

\begin{equation}
\ell_{ky}^\star = \frac{1}{r_m\cdot \gamma_F} - \frac{h^\star + \gamma_F\cdot (1-\omega)}{(1-\gamma_F)\cdot (1-\omega)\cdot (g^\star - \gamma_F)}
\end{equation}

Rearranjando e lembrando a definição de \(h^\star\)

\begin{equation}
\ell_{ky}^\star = \frac{1}{r_m\cdot \gamma_F} - \frac{g^\star\cdot v}{u^\star \cdot (1-\gamma_F)\cdot (1-\omega)\cdot (g^\star - \gamma_F)} - \frac{\gamma_F}{(1-\gamma_F)\cdot (g^\star - \gamma_F)}
\end{equation}

\textbf{Nota:} Diferenciei as equações seguintes a mão e notei que não me lembro tanto assim de derivada parcial. Então considere como uma primeira tentativa.

Dado que
$$
\frac{\partial g^\star}{\partial \omega} = 0
$$

$$
\frac{\partial u^\star}{\partial \omega} = 0
$$


Podemos avaliar como se dá o endividamento das famílias capitalistas dada uma variação positiva no \emph{wage-share}:
$$
\frac{\partial \ell_{ky}^\star}{\partial \omega} = -\frac{g^\star\cdot v}{(1-\omega)^2\cdot u^\star \cdot (1-\gamma_F)\cdot (g^\star - \gamma_F)}
$$

Dado \(\gamma_F > g^\star > 0\), a redução no \emph{wage-share} (simulação 1) faz com que o endividamento capitalista \textbf{diminua}

$$
\frac{\partial \ell_{ky}^\star}{\partial \omega} < 0
$$

Isso contradiz a nova versão do modelo enviada em \textit{<2020-11-03 ter> } em que o valor de \(\omega\) foi de \(0.5\) para \(0. 25 (= \omega\cdot \alpha)\)   

\textbf{Nota:} Removi a derivada parcial da taxa de juros porque

$$
\frac{\partial g^\star}{\partial r_m} \neq 0
$$


\section*{Considerações finais}
\label{sec:org5faa777}

\begin{itemize}
\item Talvez precise desenvolver melhor esta última equação
\begin{itemize}
\item Pode ser que seja interessante calcular as derivadas parciais em função dos parâmetros que fizemos os choques na simulação
\item Acho que essa última equação pode ajudar a explicar um resultado que você havia perguntado se temos algumas equação (p. 14 do PDF comentado)
\end{itemize}
\item Não sei dizer em que medida faz sentido tratar de endividamento líquido do pagamento de juros dos depósitos, mas acho que não deve ser muito complicado tratar de endividamento total a partir da equação que desenvolvi
\item No primeiro gráfico da segunda linha da figura 6, o endividamento capitalista plotado \textbf{não é} líquido do pagamento do juros dos depósitos
\end{itemize}
\end{document}
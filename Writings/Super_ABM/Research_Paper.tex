% Created 2021-07-12 seg 18:04
% Intended LaTeX compiler: pdflatex
\documentclass{SelfArx}
  \usepackage[T1]{fontenc}
\usepackage[utf8]{inputenc}
\usepackage{booktabs}
\renewcommand{\arraystretch}{1.1} % Unclear
\usepackage{graphicx}
\usepackage{float}
\usepackage{amsmath}
\usepackage{csquotes}
\setlength{\columnsep}{0.55cm} % Distance between the two columns of text
\setlength{\fboxrule}{0.75pt} % Width of the border around the abstract
\definecolor{color1}{RGB}{0,0,90} % Color of the article title and sections
\definecolor{color2}{RGB}{0,20,20} % Color of the boxes behind the abstract and headings
\usepackage[english]{babel} % Specify a different language here - english by default
\usepackage{lipsum} % Required to insert dummy text. To be removed otherwise
\setlength{\columnsep}{0.55cm} % Distance between the two columns of text
\setlength{\fboxrule}{0.75pt} % Width of the border around the abstract
\usepackage[backend=biber,%
style = abnt,%
noslsn, %
isbn = false,
url = false,
extrayear, %
uniquename=init,%
giveninits, %
justify, %
sccite,%
scbib, %
sorting=nyt,
% mergedate=compact,
natbib=true,
repeattitles, %
maxcitenames=3]{biblatex}
\AtEveryBibitem{%
\clearfield{urlyear}
\clearfield{urlmonth}
\clearfield{note}
\clearfield{issn} % Remove issn
\clearfield{doi} % Remove doi
\ifentrytype{online}{}{% Remove url except for @online
\clearfield{url}
}
}
\date{}
\title{Sraffian submultiplier model: an Agent Based Sraffian supermultiplier Model Introduction Model}
\begin{document}

\JournalInfo{Draft - \today} % Journal information
\Archive{Please do not quote} % Additional notes (e.g. copyright, DOI, review/research article)

\Authors{Gabriel Petrini\textsuperscript{1}*, Lucas Teixeira\textsuperscript{2}, Ivette Luna\textsuperscript{2}, Ítalo Pedrosa\textsuperscript{3}} % Authors
\affiliation{\textsuperscript{1}\textit{PhD Student at Unicamp/Brazil}} % Author affiliation
\affiliation{\textsuperscript{2}\textit{Professor at Unicamp/Brazil}} % Author affiliation
\affiliation{\textsuperscript{3}\textit{Professor at UFRJ/Brazil}} % Author affiliation
\affiliation{*\textbf{Corresponding author}: gpetrinidasilveira@gmail.com} % Corresponding author

\Keywords{Keyword1 --- Keyword2 --- Keyword3} % Keywords - if you don't want any simply remove all the text between the curly brackets
\newcommand{\keywordname}{Keywords} % Defines the keywords heading name

\Abstract{This is an example of an Abstract}

\flushbottom % Makes all text pages the same height
\maketitle % Print the title and abstract box
\thispagestyle{empty} % Removes page numbering from the first page

\section*{Introduction}
\label{sec:org21382be}


\section*{Empirical motivation}
\label{sec:org7563e47}
\section*{Conceptual model}
\label{sec:org72b8958}
In this section, we present the conceptual model that will be simulated later in section SEÇÃO.
This model represent an economy as simple as possible with heterogeneous firms while all the other institutional sectors are aggregated.
At odds with conventional AB literature, we assume no thecnological progress to keep it simple\footnote{For an detailed description of innovations and thecnological change in AB models, see DOSI whom we refer to as a benchmark model.}.
All other auxiliary assumption will follow AB literature as long as does not generate incompatibility with SSM closure.

Before moving further to the description of every institutional sectors, we will present the hypothesis of this simplified economy.
For simplicity, there is only one productive sector, no labor supply restriction, government nor external sector.
As a consequence of the previous assumptions, the economy is composed by \(N\) homogeneous households and \(F\) heterogeneous firms\footnote{Aggregates variables will be represented by subscript \(N\) for households and \(F\) for firms.}.

\subsection*{Sequence of events}
\label{sec:org8af3eca}



\subsection*{Production and distribution}
\label{sec:orge496b6f}


\subsubsection*{Firms production plans}
\label{sec:org456bbf7}

Each firm (\(f\)) investment plan (\(Y_{f,t}\), Equation \ref{EQ_Yf}) is defined accordingly to its expected real sales level (\(S^{e}_{f,t}\)) and a desired inventories to sales ratio \((\iota^{T} \in [0,1])\) -- exogenous defined and equal across firms -- to attend unexpected demand boosts as usual in \(K+S\) models.

\begin{latex}
\begin{equation}
\label{EQ_Yf}
Y_{f,t} = (1+\iota^{T})\cdot S^{e}_{f,t} - INV_{f,t-1}
\end{equation}
\end{latex}
in which \(INV_{f,t-1}\) is the firm's inventories.
In baseline scenario, we assume na\{$\backslash$``i\}ve expectational procedure in which expected sales level is equal to its lagged value.
Alternatives expectational procedures will be discussed in section \ref{sec:switching}.
After defining an production plan, firms calculates its desired capacitiy utilization rate (\(u^{d}_{f}\), Equation) as follows:

\begin{latex}
\begin{equation}
u^{d}_{f} = \max\left[ 0, \min\left[ \frac{Y_{f,t}}{Y_{f,t}^{FC}}, 1 \right] \right]
\end{equation}
\end{latex}
in which \(Y_{f,t}^{FC}\) is the firm's current output at the full capacity.
Since we assume no labor restriction, full capacity output is define by its existing capital stock (\(K_{f,t-1}\)) given a maximum capital to output ratio (\(\nu\)) constant, exogenous and equal across firms:

\begin{latex}
\begin{equation}
Y_{f,t}^{FC} = \frac{K_{f,t-1}}{\nu}
\end{equation}
\end{latex}
Consequently, current firm's capacity utilization ratio is constrained do be equal or less than one since production is limited by physical capital.



\subsubsection*{Capital stock}
\label{sec:orgdc32565}


Since there is no thecnological progress nor innovation in the current version of this model, firms' capital stock only differs according to it lifetime (\(\kappa\)).
Consequently, total capital stock is defined as the sum of different capital good vintage (\(k_{f}\)) as follows:

\begin{latex}
\begin{equation}
K_{f} = \sum_{j=1}^{\kappa < \infty} k_{f,t-j}
\end{equation}
\end{latex}
For simplicity, we assume no use depreciation, so each capital good will be replaced only if achieve its maximum lifetime.

\subsubsection*{Labor demand and wage bill}
\label{sec:orge901aab}

In order to produce, firms use a Leontieff-type technology with a fixed combination between labor and physical capital.
Given no a fixed labor productivity equally defined across workers, labor demand is a function (\(N^{d}_{f,t}\)) of each firm production plan at current capacity utilization level:

\begin{latex}
\begin{equation}
N^{d}_{f,t} = u_{f,t}\cdot Y_{f,t}
\end{equation}
\end{latex}

Since our main purpose with this model is to evaluate emergence of firm's behavior, we assume single nominal wage for all workers (\(w_{t}\)) exogenously defined.
Thus, each firm wage bill (\(W_{f}\)) is:
\begin{latex}
\begin{equation}
W_{f,t} = N^{d}\cdot w_{f,t}
\end{equation}
\end{latex}
In order to prevent null consumption levels, we assume that at each time nominal wage is updated accordingly to previous inflation rate (\(\pi_{t-t}\)) which is the market-share weighted market price mean (Equation \ref{EQ_Infla})

\begin{latex}
\begin{equation}
w_{t} = w_{t-1}\cdot(1+\pi_{t-1})
\end{equation}
\end{latex}
\begin{latex}
\begin{equation}
\label{EQ_Infla}
\pi_{t} = \sum_{f=1}^{F} ms_{f,t}\cdot p_{f,t}
\end{equation}
\end{latex}
in which \(ms_{f}\) and \(p_{f,t}\) stands for firm's \(f\) market share and price respectively.



\subsubsection*{Pricing}
\label{sec:orgda3c0b5}

We assume a simple pricing mechanism as possible which is defined as a mark-up (\(\theta_{f}\)) over direct unit labor costs (in this case, \(w_{t}\)):
\begin{latex}
\begin{equation}
p_{f,t} = (1+\theta_{f})\cdot w_{t}
\end{equation}
\end{latex}
At odds with \(K+S\) model, and a a result of our simplifying assumptions, there is no need to specify a mark-up law of motion equation\footnote{This is the case for \(K+S\) models because unit labor costs are not under firms strict control. Once we assume no technological progress nor innovation, uncertainty regarding labor productivity level is vanished.}.
As we will further explore in section \ref{sec:switching}, expected market-share levels play a prominent role in investment decision and not in pricing as usual.



\subsubsection*{Household demand}
\label{sec:org2133aac}


In order to investigate the consequences of heterogeneous investment decisions in a Sraffian supermultiplier friendly framework, we a assume that household's consumption (\(C_{N}\)) is composed both by an induced (\(W_{N,t}\)) and by an autonomous component (\(Z_{t}\))\footnote{As discussed before, there is a multitude of non-capacitiy creating autonomous expenditures. Autonomous households consumption component was selected only to reduce the complexity of this model.}\textsuperscript{,}\,\footnote{Following \textcite{serrano_1995_Long}, we consider \(Z_{t}\) as an non-capacitiy creating autonomous expenditure because it does not depends on firms' production decisions. Additionally, since banking credit is endogenous, consumption loans does not affect credit availability for other sectors.}:

\begin{latex}
\begin{equation}
C_{N} = c_{w}\cdot W_{N,t} + Z_{t}
\end{equation}
\end{latex}
in which \(c_{w}\) is households marginal propensity to consume out of wages while autonomous expenditure is given by its exogenously defined growth rate (\(g_{Z}\)):
\begin{latex}
\begin{equation}
Z_{t} = (1+g_{Z})\cdot Z_{t-1}
\end{equation}
\end{latex}
in which is financed either by financial wealth and by banking credit.

Since we assume no households heterogeinity, consumption loans restriction has no economic meaning at the aggregate level.
Differently from firms, we impose that households are not credit constrained\footnote{We are aware of the simplifications of these assumption, but our main propose here is to elaborate a simple model to discuss heterogeneity in the Sraffian Supermultiplier macroeconomic model.}.
Implicitly, we assume that banking evaluation of households default probability (\(pr^{D}_{N,t}\)) is null.

\subsubsection*{Distribution of demand}
\label{sec:orge49c46f}


Considering previous assumptions, total demand of this economy (\(Y\)) is the sum of aggregate household consumption (\(C_{N}\)) and total firms' investment (\(I_{F}\)) which is the sum of individual firms investment decisions (\(I_{f}\)).
\begin{latex}
\begin{equation}
\label{EQ_GDP_D}
Y = C_{N} + \sum_{f}^{F} I_{f}
\end{equation}
\end{latex}
As usual in AB models, distribution of total demand depends on relative competitiveness.
Since there is no price diversity, we assume that firms relative competitiveness depends only on the level of unfilled demand (\(l_{f,t}\)), normalised to the whole sector’s weighted averages (\(\overline{l}_{f,t-1}\)):

\begin{latex}
\begin{equation}
E_{f,t} = -\beta \frac{l_{f}}{\overline{l}_{f,t-1}}
\end{equation}
\end{latex}
Following SILVERBERG E DOSI, effective market-share is defined accordingly to a quasi-replicator mechanism (Equation ) in which firms that were not able to fulfill its demand level will have a lower market-share:
\begin{latex}
\begin{equation}
\label{EQ_Replicator}
ms_{f,t} = ms_{f,t-1}\cdot (1+\chi \frac{E_{f,t} - \overline{E}_{t}}{\overline{E}_{t}})
\end{equation}
\end{latex}
in which \(\chi\) is a positive exogenous parameter and \(\overline{E}_{t}\) is the average competitiveness of the whole sector:
\begin{latex}
\begin{equation}
\overline{E}_{t} = \sum_{f=1}^{F}E_{f,t}\cdot ms_{f,t-1}
\end{equation}
\end{latex}

As will be discussed in section \ref{sec:switching}, effective market-share depends both on expected mark-share and effective investment decision, each one explained separately.
Thus, investment emergence patterns depends on how firms adapts its expectations and changes (or not) its investment functions.

\subsection*{Banking sector, credit and financial implications}
\label{sec:org1ad42ab}

\subsubsection*{Firms' credit}
\label{sec:orgbf7a8ca}

\subsubsection*{Household credit}
\label{sec:org88a4f98}

\subsubsection*{Profits and dividends}
\label{sec:orgd5d482e}

\subsubsection*{Interest rate and profit}
\label{sec:org9d06929}

\subsection*{Switching mechanism}
\label{sec:switching}
\subsubsection*{Expectations}
\label{sec:orgd336c6d}

\subsubsection*{Investment decisions}
\label{sec:orgb92e62e}



\subsection*{Aggregating and closing the model}
\label{sec:org62252b2}

\subsubsection*{Entry and exit of firms}
\label{sec:orga61a073}
\subsubsection*{Closing the accounting}
\label{sec:orgbf000ae}

\section*{Validation}
\label{sec:orge6e9ec5}
\section*{Experiments}
\label{sec:org5dfe25f}
\section*{Concluding remarks}
\label{sec:org1e3d8cc}

\section*{Acknowledgements}
\label{sec:orga36283c}

Agradecimentos


\printbibliography
\end{document}

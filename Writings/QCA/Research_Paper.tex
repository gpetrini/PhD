% Created 2021-07-12 seg 12:02
% Intended LaTeX compiler: pdflatex
\documentclass{SelfArx}
  \usepackage[T1]{fontenc}
\usepackage[utf8]{inputenc}
\usepackage{booktabs}
\renewcommand{\arraystretch}{1.1} % Unclear
\usepackage{graphicx}
\usepackage{float}
\usepackage{amsmath}
\usepackage{csquotes}
\setlength{\columnsep}{0.55cm} % Distance between the two columns of text
\setlength{\fboxrule}{0.75pt} % Width of the border around the abstract
\definecolor{color1}{RGB}{0,0,90} % Color of the article title and sections
\definecolor{color2}{RGB}{0,20,20} % Color of the boxes behind the abstract and headings
\usepackage[english]{babel} % Specify a different language here - english by default
\usepackage{lipsum} % Required to insert dummy text. To be removed otherwise
\setlength{\columnsep}{0.55cm} % Distance between the two columns of text
\setlength{\fboxrule}{0.75pt} % Width of the border around the abstract
\usepackage[backend=biber,%
style = abnt,%
noslsn, %
isbn = false,
url = false,
extrayear, %
uniquename=init,%
giveninits, %
justify, %
sccite,%
scbib, %
sorting=nyt,
% mergedate=compact,
% natbib=true,
repeattitles, %
maxcitenames=3]{biblatex}
\AtEveryBibitem{%
\clearfield{urlyear}
\clearfield{urlmonth}
\clearfield{note}
\clearfield{issn} % Remove issn
\clearfield{doi} % Remove doi
\ifentrytype{online}{}{% Remove url except for @online
\clearfield{url}
}
}
\date{}
\title{The great mortgaging and the housing financing revolution: enabling each other? An institutional fuzzy-set Qualitative Comparative Analysis (fsQCA) for 17 OECD countries (1989-2016)}
\begin{document}

\JournalInfo{Draft} % Journal information
\Archive{Please do not quote} % Additional notes (e.g. copyright, DOI, review/research article)

\Authors{Gabriel Petrini\textsuperscript{1}*, Lucas Teixeira\textsuperscript{2}, Ivette Luna\textsuperscript{2}} % Authors
\affiliation{\textsuperscript{1}\textit{PhD Student at Unicamp/Brazil}} % Author affiliation
\affiliation{\textsuperscript{2}\textit{Professor at Unicamp/Brazil}} % Author affiliation
\affiliation{*\textbf{Corresponding author}: gpetrinidasilveira@gmail.com} % Corresponding author

\Keywords{Keyword1 --- Keyword2 --- Keyword3} % Keywords - if you don't want any simply remove all the text between the curly brackets
\newcommand{\keywordname}{Keywords} % Defines the keywords heading name

\Abstract{This is an example of an Abstract}

\flushbottom % Makes all text pages the same height
\maketitle % Print the title and abstract box
\thispagestyle{empty} % Removes page numbering from the first page

\section*{Títulos}
\label{sec:org372e3b6}

\begin{itemize}
\item The great mortgaging and the housing financing revolution: enabling each other? An institutional fuzzy-set Qualitative Comparative Analysis (fsQCA) for 17 OECD countries (1989-2016)
\item Institutional diversity still matters? Comparing housing financing milestones using fsQCA (1989-2014)
\item Reassessing the dual convergence debate: a methodological proposal based on the housing finance Great Revolution
\end{itemize}
\section*{Introduction}
\label{sec:org13525b0}
\subsection*{Importância do mercado imobiliário para compreender características da GFC}
\label{sec:orgf5394ae}
\begin{itemize}
\item Reforçar como o interesse aumentou pós GFC
\end{itemize}
\subsection*{Relação entre endividamento das famílias, preço dos imóveis e estabilidade financeira}
\label{sec:org508faea}
\begin{itemize}
\item Misleading?
\end{itemize}
\subsection*{Apresentação do conceito de hipotecarização e mudanças de longo prazo}
\label{sec:org31b7fe9}
\subsection*{Apresentação da revolução}
\label{sec:org8790cf9}
\subsection*{Estrutura do artigo}
\label{sec:orga0fff0c}

\section*{Empirical motivation}
\label{sec:org45fb646}
\subsection*{Background: desregulamentação dos anos 80}
\label{sec:org5e483de}
\subsection*{Apresentar conceito da hipotecarização}
\label{sec:org4ece29e}
\subsection*{Analisar os dados}
\label{sec:org5e6f22c}
\subsection*{Agrupar países e explicitar possíveis hipóteses direcionais e trajetórias}
\label{sec:org393888c}
\subsection*{Apresentar a revolução em maiores detalhes}
\label{sec:orgdd22a94}
\subsection*{Reagrupar informações institucionais de forma qualitativa}
\label{sec:org85d9044}
\subsection*{Gancho para a seção seguinte: relevância de particularidades institucionais para compreender suas implicações}
\label{sec:orgbfeed74}
\section*{Review of literature}
\label{sec:org9213916}
\subsection*{Crescente atenção ao mercado imobiliário, mas pouca atenção em sua institucionalidade em uma perspectiva macro-financeira}
\label{sec:org1aa3133}
\subsection*{Revisão da literatura de financeirização e pouca anteção para o setor imobiliário}
\label{sec:org8ab0273}
\begin{itemize}
\item Por quê? Dá atenção a elementos financeiros, mais pouco à instituição e menos ainda a housing
\item Misleading?
\end{itemize}
\subsection*{Revisão de VoC}
\label{sec:org57609da}
\subsubsection*{Discutir convergência e divergência institucional}
\label{sec:org32d78d2}
Pontuar também a importância de se levar em consideração se a institucionalidade é formal (legislatória) ou informal (convenção, prática bancária)
\subsubsection*{Críticas a VoC}
\label{sec:org2ef062e}
\begin{itemize}
\item Firm-centric
\item Supply-side
\item Complementarities
\end{itemize}
\subsubsection*{Revisão da literature recente VoC macroeconômica}
\label{sec:org199392c}
\begin{itemize}
\item Maior destaque para distribuição
\end{itemize}
\subsubsection*{Retomar importância de housing e introduzir a ideia de como instituições tratadas de forma diferente podem esclarecer as questões de convergência, divergência e consequência}
\label{sec:orgeb73e8f}
\subsubsection*{Divergência na semelhança e Convergência no diferente}
\label{sec:org665b150}
\begin{itemize}
\item Hipótese da divergência aparente: diferentes trajetórias e configurações institucionais podem estar associadas a um mesmo resultado. Assim, os países podem ser divergentes no sentido de quais características os descrevem, mas geram resultados similares (uma meta-convergência)
\end{itemize}
\subsection*{Conclusão da seção: Revisão da literatura empírica e incapacidade de analisar variáveis qualitativas adequadamente}
\label{sec:org9af346c}
\section*{Data and Methods}
\label{sec:org0c99d55}
\subsection*{Apresentar origens, aplicações e vantagens do QCA}
\label{sec:org5179d81}
\subsection*{Explicar etapas do QCA}
\label{sec:org0b05d98}
\subsection*{Calibrar dados}
\label{sec:org0281a0b}
\begin{itemize}
\item Mortgaging (Outcome)
\begin{itemize}
\item Post 1990 (after deregulation/Germany reunification) to 2006 (prior to the GFC)?
\begin{itemize}
\item Não incluir 2014 para evitar mudanças institutionais na europa (ver EMCB, EMF video)
\item Desvio em relação à média da OCDE
\end{itemize}
\end{itemize}
\item Recortes:
\begin{itemize}
\item 1990(?)-2001: Pós Basileia I e pré bolha imobiliária e bolha-ponto-com
\item 2002(?)-2006: Bolha imobiliária e pré-crise
\item 2007-2014: Crise, pós-crise e pré EMCB
\begin{itemize}
\item Isolar efeito da crise?
\end{itemize}
\end{itemize}
\end{itemize}


\subsection*{Elaborar Truth table}
\label{sec:org319796a}
\section*{Results and parameters of fit}
\label{sec:orgd9578b0}
\subsection*{Minimização da presença do resultado}
\label{sec:org5239b39}
\subsection*{Minimização da ausência do resultado}
\label{sec:orgf68f65d}
\subsection*{Apresentar parâmetros dos resultados}
\label{sec:org170d93b}
\subsubsection*{Testes de reobustez}
\label{sec:orgd2ea7d2}

\section*{Concluding remarks}
\label{sec:org642db22}

\section*{Acknowledgements}
\label{sec:org4eb6e38}

Agradecimentos


\printbibliography
\end{document}

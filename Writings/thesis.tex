% Created 2021-02-04 qui 12:09
% Intended LaTeX compiler: pdflatex
\documentclass[12pt,openright,oneside,a4paper,english,sumario=tradicional]{gpsabntex}
  \usepackage{lipsum}             % Pacote que gera texto dummy
\usepackage{blindtext}          % Pacote que gera texto dummy
\usepackage{array}
\usepackage{epstopdf}           % Pacote que converte as figuras em eps para pdf
\usepackage{venndiagram}
\usepackage{chngcntr}
\usepackage{makeidx}
\makeindex
\usepackage[top=3cm,bottom=2cm,right=3cm,left=2cm]{geometry}
\usepackage{bmpsize}
\usepackage{pdfpages}
\usepackage{graphicx}
\usepackage{svg}
\usepackage{tikz}
\usetikzlibrary{calc,trees,positioning,arrows,chains,%
decorations.pathreplacing,decorations.pathmorphing,%
matrix,shapes.symbols,through}
\usetikzlibrary{fit}
\usepackage{wrapfig}
\usepackage{etoolbox}
\usepackage{url}
%\usepackage{breakurl} %Pacote incompatível
\usepackage{hyperref}
\makeatletter
\hypersetup{
%pagebackref=true,
pdftitle={\@title},
pdfauthor={\@author},
pdfsubject={\imprimirpreambulo},
pdfcreator={LaTeX with abnTeX2},
pdfkeywords={abnt}{latex}{abntex}{abntex2}{trabalho acadêmico},
hidelinks,					% desabilita as bordas dos links
colorlinks=false,       	% false: boxed links; true: colored links
linkcolor=blue,          	% color of internal links
citecolor=blue,        		% color of links to bibliography
filecolor=magenta,      	% color of file links
urlcolor=blue	% set to white
}
\makeatother
\usepackage{hyphenat}
\hyphenation{super-multiplier}
\usepackage[backend=biber,%
style = abnt,%
noslsn, %
isbn = false,
url = false,
extrayear, %
uniquename=init,%
giveninits, %
justify, %
sccite,%
scbib, %
sorting=nyt,
%	mergedate=compact,
repeattitles, %
maxcitenames=3]{biblatex}
\AtEveryBibitem{%
\clearfield{urlyear}
\clearfield{urlmonth}
\clearfield{note}
\clearfield{issn} % Remove issn
\clearfield{doi} % Remove doi
\ifentrytype{online}{}{% Remove url except for @online
\clearfield{url}
}
}
\usepackage{lmodern} \normalfont
\usepackage{mathrsfs}
\usepackage{cmap}
\usepackage[T1]{fontenc}
\usepackage[utf8]{inputenc}
\usepackage{indentfirst}
\usepackage{titlesec}
\usepackage{csquotes}
\usepackage[protrusion = true, final]{microtype}   % To avoid overfull hbox
\emergencystretch=1.5em   % To avoid overfull hbox: https://tex.stackexchange.com/questions/20585/fixing-an-overfull-box
\setlength{\parskip}{\onelineskip}
\usepackage{mathptmx}
\renewcommand{\ABNTEXchapterfont}{\rmfamily\bfseries}
\usepackage{threeparttable, tablefootnote} % Nota de rodapé em pagina
\counterwithout*{footnote}{chapter} % Impede reset das notas de rodapé
\usepackage[multiple, bottom]{footmisc} % Para nota de rodapé no fim da página
\newcommand{\annexname}{Annex} % TODO: Change here if language change
\makeatletter
\newcommand\annex{\par\setcounter{chapter}{0}%
\setcounter{section}{0}%
\gdef\@chapapp{\annexname}%
\gdef\thechapter{\@Roman\c@chapter}%
}
\makeatother
\linespread{1.3}
\setlength{\parskip}{0.2cm}
\setlength{\parindent}{2.0cm}
\usepackage{appendix}
\usepackage{tocloft}
\newcommand{\dropcap}[1]{#1}
\newcommand{\matmethods}[1]{#1}
\newcommand{\showmatmethods}[1]{}
\newcommand{\acknow}[1]{#1}
\newcommand{\showacknow}[1]{}
\newcommand{\thesisnoop}[1]{}
\usepackage{multicol}
\usepackage{multirow}
\usepackage{array}
\usepackage{booktabs}
\usepackage{caption}
\usepackage{subcaption}
\usepackage{longtable}
\usepackage{lscape}
\usepackage{amsmath}
\usepackage{amsfonts}
\usepackage{amssymb}
\usepackage{amsthm}
\usepackage{breqn}
\newcommand{\mb}[1]{\mathbf{#1}}
\newcommand{\abs}[1]{\left|#1\right|}
\newcommand{\norm}[1]{\left\|#1\right\|}
\newcommand{\partialorder}{\cdot \geq}
\newcommand{\h}{\mathrm{h}}
\newcommand{\x}{\mathrm{x}}
\newcommand{\z}{\mathrm{z}}
\newcommand{\entropia}[1]{H{(#1)}}
\usepackage{listings}
\usepackage{minted}
\setminted{autogobble=true,fontsize=\small,baselinestretch=0.8}
\setminted[python]{python3=true,tabsize=4}
\usemintedstyle{trac}
\lstset{abovecaptionskip=0}
\numberwithin{listing}{chapter}
\renewcommand{\listingscaption}{Code Snippet}
\AtEndEnvironment{listing}{\vspace{-16pt}}
\usepackage{color}				      % Controle das cores
\definecolor{blue}{RGB}{41,5,195} % Alterando o aspecto do azul
\usepackage{algorithmic}
\usepackage[chapter]{algorithm}
\floatname{algorithm}{Algoritmo}
\renewcommand{\listalgorithmname}{Lista de Algoritmos}
\usepackage{./abnt/unicamp}
\usepackage{lastpage}			      % Usado pela Ficha catalogr\'{a}fica
\appto\frontmatter{\pagestyle{plain}}  % Adiciona o estilo plano de página.
\instituicao{%
UNIVERSIDADE ESTADUAL DE CAMPINAS
\par
Instituto de Economia
}
\tipotrabalho{Tese (Doutorado)}
%% O preambulo deve conter o tipo do trabalho, o objetivo, o nome da institui\c{c}\~{a}o e a \'{a}rea de concentra\c{c}\~{a}o
\preambulo{Tese apresentada ao Instituto de Economia da Universidade Estadual de Campinas como parte dos requisitos exigidos para a obtenção do t\'{i}tulo de Doutor em Ci\^encias Econ\^omicas.} % Acentos devem ser escritos dessa forma no preambulo
\titulo{Teste}
\orientador{Lucas Azeredo da Silva Teixeira}
\coorientador{Ivette Raymunda Luna Huamani}
\newcommand{\ThesisTitle}{{Three Essays in Housing Macroeconomics}}
\author{Gabriel Petrini}
\date{2024}
\title{}
\begin{document}

\pretextual
\frenchspacing
	
\frontmatter
\pagenumbering{roman}
\imprimircapa 
\stepcounter{page} % Adicionando a numeração em um. Motivo: folha de rosto possui numeração repetida em relação à capa. https://tex.stackexchange.com/questions/416589/how-to-increment-page-number-by-one
\imprimirfolhaderosto
\nopagebreak

% 2020-09-09: Criando minipage para evitar que uma página em branco seja criada antes e depois da ficha catalográfica. Deve ser incluido um \newpage após minipage para evitar sobreposição.
% Atenção: Ficha catalográfica não aparece no Okular.
\begin{minipage}{\textwidth}
\begin{fichacatalografica}
	\begin{center}
		\includepdf[pagecommand={\thispagestyle{empty}},pages=1]{./figs/Fake_Ficha_Catalografica.pdf}
	\end{center}
\end{fichacatalografica}
\end{minipage}
\newpage

% 2020-09-09: Criando minipage para evitar que uma página em branco seja criada antes e depois da folha de aprovação. Deve ser incluido um \newpage após minipage para evitar sobreposição.
\begin{minipage}{\textwidth}
\begin{folhadeaprovacao}
	\begin{center}
		\includepdf[pagecommand={\thispagestyle{empty}},pages=1]{./figs/Fake_Folha_Aprovacao.pdf}
	\end{center}
\end{folhadeaprovacao}
\end{minipage}
\newpage

\begin{dedicatoria}
\thispagestyle{empty} % 2020-09-09: Tal como dissertação para homologação
	\vspace*{\fill}
	\centering
	\noindent
	\textit{Pela comunidade, para a comunidade}
	\vspace*{\fill}
\end{dedicatoria}

\begin{agradecimentos}
\thispagestyle{empty} % 2020-09-09: Tal como dissertação para homologação
Agradeço a todo mundo
\end{agradecimentos}

\begin{resumo}
\thispagestyle{empty} % 2020-09-09: Tal como dissertação para homologação
\begin{otherlanguage*}{portuguese}

Esta dissertação investiga a relação entre investimento residencial, inflação de ativos e dinâmica macroeconômica no médio prazo com base no caso americano no pós-desregulamentação financeira (1992-2019).
No primeiro capítulo, é feita uma revisão da literatura dos modelos de crescimento liderados pela demanda, elencando o supermultiplicador sraffiano (SSM) como o mais pertinente para atender os objetivos desta pesquisa.
No capítulo seguinte, avança-se em direção da discussão empírica e é estimado um modelo vetor de correção de erros (VECM) para testar a capacidade explicativa da taxa própria de juros dos imóveis.
No terceiro capítulo, é simulado um modelo \textit{Stock-Flow Consistent} com supermultiplicador sraffiano (SSM-SFC) com inflação de ativos em que se
prioriza a parcimônia de modo a representar uma economia fechada e sem governo com famílias trabalhadoras e capitalistas em que somente estas últimas têm acesso a crédito para financiar tanto o consumo quanto o investimento residencial.
A especificidade deste modelo é a existência do estoque de capital das firmas (criador de capacidade produtiva) e das famílias
cuja participação deste último se reduz dado um aumento na taxa de crescimento do investimento residencial.
Adicionalmente, são introduzidos alguns dados observados para simular ciclos econômicos.
Conclui-se que a taxa própria de juros dos imóveis explica a taxa de crescimento residencial empiricamente e que o modelo SSM-SFC reproduz alguns fatos estilizados da economia norte-americana.



\vspace{\onelineskip}

\noindent\textbf{Palavras-chave}: 
	Supermultiplicador Sraffiano; 
	Investimento residencial; 
	Taxa própria de juros; 
	Modelo Vetor Correção de Erro;
	Consistência entre fluxos e estoques.
\end{otherlanguage*}
\newpage

\begin{otherlanguage*}{english}
	\thispagestyle{empty}
	\begin{center}{\ABNTEXchapterfont\huge Abstract}\end{center}
	
	This thesis investigates the relationship between residential investment, asset inflation, and medium-term macroeconomic dynamics based on the US post-deregulation case  (1992-2019). 
	The first chapter presents a review of demand-led growth models, choosing the Sraffian supermultiplier (SSM) as the best one to achieve the objectives of this research. 
	In the following chapter, we move towards the empirical discussion and estimate a vector error correction model (VECM) to test the explanatory capacity of real interest rate of real estate. 
	In the third chapter, a Sraffian Supermultiplier Stock-Flow Consistent model (SSM-SFC) with asset inflation is simulated prioritizing parsimony to represent a closed and without government economy with working and capitalist households in which only the latter have access to credit to finance both consumption and dwellings. 
	The specificity of this model is the existence of firms' (capacity creating) and households' capital stock  whose participation of the latter is reduced given an increase in the growth rate of residential investment. 
	Additionally, some observed data are introduced to simulate economic cycles. 
 	We conclude that housing own interest rate explains residential investment growth rate empirically and that the SSM-SFC model reproduces some stylized facts of the US economy.
	
	\vspace{\onelineskip}
	
	\noindent\textbf{Keywords}: Sraffian Supermultiplier; real state, own interest rate, Vector Error Correction Model, Stock-Flow Consistent Approach. 
	
\end{otherlanguage*}
\end{resumo}

\begin{epigrafe}
\thispagestyle{empty} % 2020-09-09: Tal como dissertação para homologação
	\vspace*{\fill}
	\begin{flushright}
		\textit{``Não está ao meu alcance criar uma sociedade ideal. Contudo, está ao meu alcance descrever o que, na sociedade existente, não é ideal para nenhuma espécie de existência humana em sociedade.''\\
			(Florestan Fernandes)}
	\end{flushright}
\end{epigrafe}

\newpage

\pdfbookmark[0]{\listfigurename}{lof}
\listoffigures*
\addcontentsline{toc}{chapter}{List of Figures}
\thispagestyle{empty}
\newpage
\pdfbookmark[0]{\listtablename}{lot}
\listoftables*
\addcontentsline{toc}{chapter}{List of Tables}
\thispagestyle{empty}
\newpage
\addcontentsline{toc}{chapter}{List of Variables}
\chapter*{List of Variables\markboth{List of Variables}{}}  % \markboth{}{} é utilizado para corrigir o cabeçalho.
\pagestyle{empty}

\thispagestyle{empty}
\newpage

\tableofcontents*
\thispagestyle{empty}

\mainmatter
\cleardoublepage
\pagestyle{plain} % restore the plain style
\renewcommand{\thepage}{\arabic{page}}  % Setting the page output to arabic   -- not necessary, unless `book` class and `\frontmatter` is used.


\chapter{QCA}
\label{sec:org7d6e719}

\epigraph{The Fed, the European Central Bank and the Bank of Japan together set monetary policy for a zone that accounts for 80 per cent of the world’s
industrialized economic activity ... Rarely, if ever, can so much power have been wielded by such a small number of institutions sitting outside the
direct democratic process.}{Goldman Sachs, 2000}


\section{Introduction}
\label{sec:org102f394}
\subsection{Importância do mercado imobiliário para compreender características da GFC}
\label{sec:orgda04412}
\subsection{Relação entre endividamento das famílias, preço dos imóveis e estabilidade financeira}
\label{sec:org05cbb8f}
\subsection{Apresentação do conceito de hipotecarização e mudanças de longo prazo}
\label{sec:org84fdfb0}
\subsection{Apresentação da revolução}
\label{sec:org9ff92df}
\subsection{Estrutura do artigo}
\label{sec:orgb6563cf}

\section{Empirical motivation}
\label{sec:org315cf4a}
\subsection{Apresentar conceito da hipotecarização}
\label{sec:org75b1661}
\subsection{Analisar os dados}
\label{sec:org937b7b1}
\subsection{Agrupar países e explicitar possíveis hipóteses direcionais e trajetórias}
\label{sec:org5908c61}
\section{Review of literature}
\label{sec:org13efb36}
\subsection{Revisão da literatura de financeirização e pouca anteção para o setor imobiliário}
\label{sec:orgdf73de0}
\subsection{Apresentar a revolução em maiores detalhes}
\label{sec:org14c520f}
\subsection{Apresentar comparação internacional e elementos institucionais}
\label{sec:org942298e}
\subsection{Revisão da literatura empírica e incapacidade de analisar variáveis qualitativas adequadamente}
\label{sec:orge438122}
\section{Data and Methods}
\label{sec:org8e0fdd9}
\subsection{Reagrupar informações institucionais de forma qualitativa}
\label{sec:org0fb21ac}
\subsection{Apresentar origens, aplicações e vantagens do QCA}
\label{sec:org9378e9c}
\subsection{Explicar etapas do QCA}
\label{sec:org60502bf}
\subsection{Calibrar dados}
\label{sec:orgf8e9d86}
\subsection{Elaborar Truth table}
\label{sec:orgd66eb9d}
\section{Results and parameters of fit}
\label{sec:orgd91ce8c}
\subsection{Minimização da presença do resultado}
\label{sec:org5a23eb3}
\subsection{Minimização da ausência do resultado}
\label{sec:org8a1207d}
\subsection{Apresentar parâmetros dos resultados}
\label{sec:orgbe8884b}
\subsubsection{Testes de reobustez}
\label{sec:org8bb4ded}

\section{Concluding remarks}
\label{sec:org6eeafd0}

\appendix
\end{document}
\documentclass[11pt,a4paper]{article}
\usepackage[utf8]{inputenc}
\usepackage[top=2cm, right=2cm, left=2.5cm, bottom=2cm]{geometry}
\usepackage[T1]{fontenc}
\usepackage[portuguese]{babel}
\usepackage{amsfonts}
\usepackage{bibentry}
\usepackage[style=abnt,	natbib=true]{biblatex}
\usepackage{tabularx}

\usepackage{filecontents}
\begin{filecontents}{\jobname.bib}
@inproceedings{petrini_investimento_2019,
		address = {Campinas},
		title = {Investimento residencial em um modelo {Stock}-{Flow} {Consistent} com supermultiplicador sraffiano},
		booktitle = {{XII} {Encontro} {Internacional} da {AKB}},
		author = {Petrini, Gabriel and Teixeira, Lucas},
		year = {2019}
	}
	
@inproceedings{petrini_long_2020,
		address = {Boston},
		title = {Long {Run} {Effective} {Demand}: {Introducing} {Residential} {Investment} in a {Sraffian} {Supermultiplier} {Stock}-{Flow} {Consistent} {Model}},
		language = {en},
		booktitle = {46th {Eastern} {Economic} {Association} {Conference}},
		author = {Petrini, Gabriel and Teixeira, Lucas},
		year = {2020}
	}

@inproceedings{petrini_long_2020b,
	address = {Bilbao},
	title = {Long-run effective demand and residential investment: a Sraffian supermultiplier based analysis},
	language = {en},
	booktitle = {32nd Annual EAEPE Conference},
	author = {Petrini, Gabriel and Teixeira, Lucas},
	year = {2020}
}
	
@inproceedings{petrini_long_2019,
		address = {Berlin},
		title = {Long {Run} {Effective} {Demand}: {Introducing} {Residential} {Investment} in a {Sraffian} {Supermultiplier} {Stock}-{Flow} {Consistent} {Model}},
		language = {en},
		booktitle = {23th {Forum} for {Macroeconics} and {Macroeconomic} {Policies}},
		author = {Petrini, Gabriel and Teixeira, Lucas},
		year = {2019}
}
\end{filecontents}

\nobibliography{\jobname}

\bibliography{\jobname}

\author{Gabriel Petrini da Silveira}
\title{Súmula Curricular}
\date{Julho de 2020}

\begin{document}
\begin{center}
	\rule{\textwidth}{1.2pt}
	\textbf{Súmula Curricular}\\
	\Large\textbf{Gabriel Petrini da Silveira}\\
	Julho de 2020
	\rule{\textwidth}{1.2pt}
\end{center}

\section{Formação}

\begin{center}
\begin{tabularx}{\textwidth}{c|c|X}
	\hline \hline 
	\textbf{Ano} & \textbf{Título} & \textbf{Instituição} \\ 
	\hline 
	2014 - 2017 & Graduação & Unicamp \\ \hline 
	2018 - 2019 & Mestrado - CNPq & Unicamp \\ \hline 
	2019 - 2019 & Aperfeiçoamento & Metodologia e Análise Qualitativa Comparada (QCA). (Carga horária: 30h). Universidade Estadual de Campinas\\ \hline
	2019 - 2019 & Aperfeiçoamento &  7th International Summer School on `Keynesian Macroeconomics and European Economic Policies'. (Carga Horária: 40h). Macroeconomic Policy Institute
	(IMK), Berlim \\ \hline
	2020 - (Atual) & Doutorado em andamento - CNPq & Unicamp \\ 
	\hline \hline 
\end{tabularx} 
\end{center}

\section{Histórico profissional}

NADA A DECLARAR


\section{Publicações recentes selecionadas}


\begin{itemize}
	\item \fullcite{petrini_investimento_2019}
	\item \fullcite{petrini_long_2019}
	\item \fullcite{petrini_long_2020}
	\item \fullcite{petrini_long_2020b}
\end{itemize}
	


\section{Lista de financiamentos à pesquisa vigentes}

\begin{itemize}
	\item \textbf{Tese de doutorado:} Três Ensaios em Macroeconomia imobiliária: instituições, inflação de ativos e instabilidade financeira. Duração: 2020 a 2023. Universidade Estadual de Campinas, Conselho Nacional de Desenvolvimento Científico e Tecnológico (CNPq). Número do processo: 140721/2020-7.
\end{itemize}

\section{Lista de orientações em andamento (com bolsas)}

NADA A DECLARAR

\section{Indicadores quantitativos}

\begin{itemize}
	\item Trabalhos apresentados em congressos internacionais: 4
\end{itemize}

\section{Links}

\begin{description}
	\item[Google Scholar:] \url{https://scholar.google.com.br/citations?user=0soJqxkAAAAJ&hl=pt-BR}
	\item[Orcid:] \url{https://orcid.org/0000-0002-3523-9826}
	\item[Publons:] \url{https://publons.com/researcher/3763984/gabriel-petrini-da-silveira/}
\end{description}

\section{Outras informações}

\begin{itemize}
	\item 2016: Programa de Apoio Didático (PAD) - Microeconomia III
	\item 2017: Programa de Apoio Didático (PAD) - Economia Matemática III
	\item 2017: Programa de Apoio Didático (PAD) - Macroeconomia II
	\item 2020: Programa de Estágio de Docência (PED) - Macroeconomia III
	\item Primeiro lugar na ANPEC 2018 na classificação de concorrência ampla para mestrado em teoria econômica do Instituto de Economia Unicamp.
	\item Primeiro lugar no processo seletivo de doutorado em teoria econômica do Instituto de Economia Unicamp (ampla concorrência)
	\item Manutenção e atualização de um pacote para solução e simulação de sistemas lineares escrito em \textit{python 3}. Disponível em \url{https://github.com/gpetrini/pysolve3}
	\item Contribuidor no Centro de Estudos de Conjuntura e Política Econômica (Cecon/Unicamp)
\end{itemize}

\end{document}
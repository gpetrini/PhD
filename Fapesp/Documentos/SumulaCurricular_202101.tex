\documentclass[11pt,a4paper]{article} \usepackage[utf8]{inputenc}
\usepackage[top=2cm, right=2cm, left=2.5cm, bottom=2cm]{geometry}
\usepackage[T1]{fontenc} \usepackage[portuguese]{babel} \usepackage{amsfonts}
\usepackage{bibentry} \usepackage[style=abnt, natbib=true]{biblatex}
\usepackage{tabularx}

\bibliography{Producoes.bib}

\author{Gabriel Petrini da Silveira} \title{Súmula Curricular} \date{Julho de
  2020}

\begin{document}
\begin{center}
  \rule{\textwidth}{1.2pt}
  \textbf{Súmula Curricular}\\
  \Large\textbf{Gabriel Petrini da Silveira}\\
  Janeiro de 2021 \rule{\textwidth}{1.2pt}
\end{center}

\section{Formação}

\begin{center}
  \begin{tabularx}{\textwidth}{c|c|X}
	\hline \hline 
	\textbf{Ano} & \textbf{Título} & \textbf{Instituição} \\ 
	\hline 
	2014 - 2017 & Graduação & Unicamp \\ \hline 
	2018 - 2019 & Mestrado - CNPq & Unicamp \\ \hline 
	2019 - 2019 & Aperfeiçoamento & Metodologia e Análise Qualitativa Comparada (QCA). (Carga horária: 30h). Universidade Estadual de Campinas\\ \hline
	2019 - 2019 & Aperfeiçoamento &  7th International Summer School on `Keynesian Macroeconomics and European Economic Policies'. (Carga Horária: 40h). Macroeconomic Policy Institute
									(IMK), Berlim \\ \hline
	2020 - (Atual) & Doutorado em andamento - CNPq & Unicamp \\\hline
	2020 - 2021 & Aperfeiçoamento & Econometrics III: Structural Modeling (Carga horária: 30h). University of Oklahoma\\
	\hline \hline 
  \end{tabularx}
\end{center}

\section{Histórico profissional}

NADA A DECLARAR


\section{Publicações recentes selecionadas}


\begin{itemize}
  \item \fullcite{petrini_investimento_2019}
  \item \fullcite{petrini_long_2019}
  \item \fullcite{cecon_atividade1}
  \item \fullcite{RPPS}
  \item \fullcite{cecon_corona}
  \item \fullcite{petrini_long_2020}
  \item \fullcite{petrini_long_2020b}
  \item \fullcite{petrini_TD}
\end{itemize}
	


\section{Lista de financiamentos à pesquisa vigentes}

\begin{itemize}
  \item \textbf{Tese de doutorado:} Três Ensaios em Macroeconomia imobiliária:
		instituições, inflação de ativos e instabilidade financeira. Duração:
		2020 a 2023. Universidade Estadual de Campinas, Conselho Nacional de
		Desenvolvimento Científico e Tecnológico (CNPq). Número do processo:
		140721/2020-7.
\end{itemize}

\section{Lista de orientações em andamento}

\begin{itemize}
  \item Lorena Salces Dourado. Desigualdade Sob o Red Scare: Um Estudo Dos Condicionantes Geopolíticos e Institucionais da Distribuição de Renda nos Países Centrais Durante a Guerra Fria. Início: 2021. Trabalho de Conclusão de Curso (Graduação em
		Ciências Econômicas) - Universidade Estadual de Campinas. (Orientador). Bolsa: Sem bolsa.
  \item Felipe Messias Barros. Padrões de Financeirização Brasileiros: Particularidades, Momentos e Ruptura. Início: 2021. Trabalho de Conclusão de Curso (Graduação em Ciências Econômicas) - Universidade Estadual de Campinas. (Orientador). Bolsa: Sem bolsa.
\end{itemize}

\section{Indicadores quantitativos}

\begin{itemize}
  \item Trabalhos apresentados em congressos internacionais: 4
  \item Parecer para periódicos internacionais: 1
  \item Participação em bancas de monografia: 1
\end{itemize}

\section{Links}

\begin{description}
  \item[Google Scholar:] \url{https://scholar.google.com.br/citations?user=0soJqxkAAAAJ&hl=pt-BR}
  \item[Orcid:] \url{https://orcid.org/0000-0002-3523-9826}
  \item[Publons:] \url{https://publons.com/researcher/3763984/gabriel-petrini-da-silveira/}
\end{description}

\section{Outras informações}
\subsection{Grupos de Pesquisa}
\begin{itemize}
 \item Membro do Centro de Estudos em Conjuntura e Política Econômica (CECON-Unicamp)
\end{itemize}


\subsection{Apoio Didático e Estágio de Docência}
\begin{itemize}
  \item 2016: Programa de Apoio Didático (PAD) - Microeconomia III
  \item 2017: Programa de Apoio Didático (PAD) - Economia Matemática III
  \item 2017: Programa de Apoio Didático (PAD) - Macroeconomia II
  \item 2020: Programa de Estágio de Docência (PED) - Macroeconomia III
  \item 2020: Programa de Estágio de Docência (PED) - Métodos de Análise
		Econômica III
\end{itemize}

\subsection{Processos Seletivos}
\begin{itemize}
  \item Primeiro lugar na ANPEC 2018 na classificação de concorrência ampla para
		mestrado em teoria econômica do Instituto de Economia Unicamp.
  \item Primeiro lugar no processo seletivo de doutorado em teoria econômica do
		Instituto de Economia Unicamp (ampla concorrência)
\end{itemize}

\subsection{Projetos de Pesquisa}
\begin{itemize}
  \item \textbf{Projeto de Pesquisa (2019):} Parecer acerca das inconsistências na justificativa oficial da emanda constitucional N. 103/2019 para a reforma do Regime Próprio de Previdência Social (RPPS). Coordenador: Pedro Paulo Zaluth Bastos. Pesquisadores contratados: Arthur Welle; Gabriel Petrini da Silveira (Beneficiário). Requerente: UNAFISCO Nacional (Ofício No 254/2019 - JUR).
  \item \textbf{Projeto de Pesquisa (2020):} Os impactos econômicos da pandemia do Covid-19. Coordenadores: Pedro Paulo Zaluth Bastos; Luiz Celso Gomes Jr. Pesquisadores: Lorena Salces Dourado; Gabriel Petrini (Beneficiário); Paulo Robilotti; Antônio Ibarra. Pesquisa em parceria com Ciência de Dados por uma Causa (CDC, UFPR).
\end{itemize}

\subsection{Softwares e pacotes}
\begin{itemize}
  \item Manutenção e atualização de um pacote para solução e simulação de sistemas lineares escrito em \textit{python 3}. Disponível em \url{https://github.com/gpetrini/pysolve3}
\end{itemize}

\end{document}

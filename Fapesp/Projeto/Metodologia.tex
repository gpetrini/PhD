\section{Metodologia}\label{passos}


Para atender os objetivos, a pesquisa será dividida em três capítulos independentes.
O primeiro deles trata das relações entre o mercado imobiliário e de crédito a luz das especificidades institucionais por meio de uma análise qualitativa comparativa.
No capítulo seguinte, será estimado um painel macrodinâmico para analisar os determinantes do investimento residencial.
Em seguida, CAPÍTULO SFC
 
 
A hipótese de trabalho do primeiro capítulo é que o arranjo institucional macroeconômico é relevante para explicar o grau de hipotecarização de um país.
Para tanto, será realizada uma  análise comparativa qualitativa (QCA)\footnote{A metodologia que pretendemos usar para dar conta desse objetivo é semelhante a utilizada em outro trabalho \cite{petrini_comparacao_2019} aplicada a outro objeto.}. 


Desenvolvida originalmente por \textcite{ragin_comparative_1989} ---  aprimorada e ampliada por \textcite{ragin_set_2006}, \textcite{box-steffensmeier_measurement_2009} e \textcite{smithson_fuzzy_2006} ---, esta metodologia associa todas as configurações possíveis a um resultado específico por meio de álgebra booleana e teoria dos conjuntos.
Por se tratar de uma metodologia pouco utilizada em economia, serão apresentados seus procedimentos em maiores detalhes tal como descrito por CITAR.
Estabelecido o fenômeno (resultado) de interesse, os casos a serem analisados e quais seus possíveis determinantes, a primeira etapa  consiste na adequação dos dados à variante QCA a ser utilizada.
Na etapa seguinte, constrói-se uma tabela verdade em que são apresentadas as configurações comuns de cada caso em relação ao resultado.
A partir desta tabela, é possível avaliar as condições necessárias e suficientes destas configurações, bem como as contradições.
Na ausência de contradições e determinadas as condições suficientes\footnote{COMO RESOLVER CONTRADIÇÕES}, são realizados procedimentos de minimização --- por meio do algoritmo de Quine–McCluskey \cite{ragin_comparative_1989} --- para agrupar os casos semelhantes e obter a solução parcimoniosa.
Com a solução parcimoniosa em mãos, resta interpretar os resultados obtidos.

Em resumo, a escolha desta metodologia se dá por: 
	(i) enfatizar as singularidades de cada unidade de investigação; 
	(ii) por tratar os casos holisticamente, ou seja, como unidades integradas por uma complexa combinação de propriedades e\footnote{Tal metodologia permite incluir elementos de complexidade uma vez que pode reportar distintas trajetórias que levam ao mesmo resultado. Em outras palavras, diferentemente dos métodos estatísticos usuais, a metodologia QCA não pressupõe uniformidade  e simetria causal CITAR.}; 
	(iii) ser possível 
Nas palavras de CITAR (p.~16):

\begin{quote}
	
	[...] \textit{QCA does not yield new theories. What it may do, once
		performed, is to help the researcher generate some new insights, which
		may then be taken as a basis for a further theoretical developmentor for
		reexamination of existing theories. Only by returning to empirical cases
		will it be possible to evaluate whether it makes senseto highlight a particular condition.} 
\end{quote}		
Portanto, a partir desta metodologia, é possível destacar quais elementos institucionais são necessários ou suficientes para determinar o grau de hipotecarização de um país sem que para isso seja necessário desconsiderar as especificidades de cada caso analisado.

No que diz respeito a essa pesquisa, o resultado a ser analisado é o grau de hipotecarização de um pais, ou seja, quanto maior a participação das hipotecas no balanço patrimonial dos bancos mais ``hipotecarizado''.
Por se tratar de uma variável contínua, a variante \textit{fuzzy} se mostra a melhor alternativa para abordar este objetivo, portanto trata-se de um \textit{fuzzy-set} QCA (fsQCA).
Para tanto, serão utilizados tanto o método direto (teórico) quanto
indireto (estatístico) para a determinação da função de pertencimento \textit{fuzzy} (\textit{fuzzy membership function}).
As variáveis serão selecionadas a partir de uma ampla revisão de literatura em que serão identificados os condicionantes institucionais da hipotecarização.
Os casos serão os países da base de dados de \textcite{jorda_great_2014} para os anos com quebras estruturais\footnote{Vale mencionar que por serem países membros da OCDE, estes países possuem um grau maior de comparação entre si e, portanto, destaca-se melhor as especificidades institucionais mencionadas anteriormente.}.
A análise dos resultados será baseada nos índices de consistência e abrangência propostos por \textcite{ragin_set_2006} e interpretados a luz da literatura pós-keynesiana.
Com isso, espera-se reportar quais são as características institucionais necessárias e suficientes para explicar o grau de hipotecarização de um país.

Uma versão preliminar e ilustrativa da primeira etapa da metodologia QCA pode ser vista na tabela TABELA em que são apresentadas as especificidades institucionais da tabela \ref{Institucional} em seu equivalente \textit{fuzzy} e associados ao grau de hipotecarização médio.
A transformação do grau de hipotecarização em seu equivalente \textit{fuzzy} é automática, ou seja, quanto mais próximo de um mais hipotecarizado. 
O mesmo raciocínio é estendido para o financiamento pelo mercado de capitais.
No caso do tipo de reembolso antecipado, considerou-se igual a unidade quando completamente legislado, igual a zero quando totalmente contratual e igual a meio na presença de ambos os tipos\footnote{Antes de prosseguir, vale pontuar que o valor associado a cada característica não interfere nos resultados uma vez que são avaliadas as configurações, ou seja, características conjuntas associadas ao resultado.}.
Em relação a permissividade de retirada do capital próprio, codificou-se como um quando permitido, zero caso contrário  e meio quando limitado.
No que diz respeito ao tipo de taxa de juros hipotecária, considerou-se igual a unidade quando flexível e zero caso contrário.
Para as variáveis restantes (maturidade e execução hipotecária), utilizou-se o procedimento proposto por CITAR RAGIN (método indireto) para obtenção do equivalente \textit{fuzzy}.
A partir desta tabela, observa-se que existe uma pluralidade de configurações institucionais associada a diferentes níveis de hipotecarização.
A etapas seguintes e o aprimoramento desta versão preliminar serão realizadas ao longo do desenvolvimento da pesquisa.

\begin{table}[htb]
	\centering
	\caption{Características institucionais fuzzyficadas e grau de hipotecarização
		 médio}
	\label{fuzzyfied}
		\resizebox{\textwidth}{!}{%
			\begin{tabular}{c|c|c|c|c|c|c|c}
				\hline\hline \\
				\multirow{2}{*}{\textbf{Países}} & \multicolumn{7}{c}{\textbf{Características institucionais}} \\\cline{2-8}
				&
\textbf{\begin{tabular}[c]{@{}c@{}}Maturidade\\ Hipotecária\end{tabular}} &
\textbf{\begin{tabular}[c]{@{}c@{}}Taxa de juros\\ Hipotecária\\(Flexível)\end{tabular}} &
\textbf{\begin{tabular}[c]{@{}c@{}}Reembolso antecipado:\\ Contratado (0)\\ Legislado (1)\end{tabular}} &
\textbf{\begin{tabular}[c]{@{}c@{}}Retirada de \\ Capital Próprio\\Permitido (1)\end{tabular}} &
\textbf{\begin{tabular}[c]{@{}c@{}}Financiamento pelo\\ Mercado de capitais\end{tabular}} &
\textbf{\begin{tabular}[c]{@{}c@{}}Execução\\ Hipotecária\end{tabular}} &
\textbf{\begin{tabular}[c]{@{}c@{}}Hipotecarização média\\(1870-2016)\end{tabular}} \\\hline
	\textbf{Alemanha} & 0,500 & 0,000 & 0,500 & 0,000 & 0,140 & 0,067 & 0,411 \\
	\textbf{Espanha}  & 0,500 & 1,000 & 0,500 & 0,500 & 0,450 & 0,042 & 0,236 \\
	\textbf{França}   & 0,004 & 0,000 & 0,500 & 0,000 & 0,120 & 0,874 & 0,319 \\
	\textbf{Holanda}  & 0,500 & 0,000 & 0,000 & 1,000 & 0,250 & 0,010 & 0,427 \\
	\textbf{Itália}   & 0,018 & 1,000 & 1,000 & 0,000 & 0,200 & 1,000 & 0,255 \\
	\textbf{Portugal} & 1,000 & 1,000 & 1,000 &     - & 0,270 & 0,966 & 0,208 \\\hline
	\hline
				
			\end{tabular}%
		}
	\caption*{\textbf{Fonte:}  Elaboração própria}
\end{table}

% DADOS EM PAINEL

Compreendidos os fatores institucionais, a segunda parte desta pesquisa irá analisar a dimensão quantitativa da macroeconomia imobiliária.
Em particular, serão analisados os determinantes do investimento residencial que, como visto na revisão de literatura, são fundamentais para a compreensão da dinâmica macroeconômica.
A hipótese de trabalho é que além de não criar capacidade produtiva, o investimento residencial é autônomo.

a ser testada nesse capítulo é a capacidade explicativa da já mencionada taxa própria de juros dos imóveis na determinação da taxa de crescimento do investimento residencial.
DADOS EM PAINEL
por meio de um modelo de dados em painel dinâmicos por permitir incorporar as defasagens de algumas variáveis e, assim, enriquecer a análise\footnote{Cabe aqui pontuar que \textcite{petrini_investimento_2019} encontrou defasagens estatisticamente significantes entre taxa real de juros dos imóveis e taxa de crescimento dos imóveis para o caso norte-americano por meio de um VEC. A realização de um modelo de dados em painel também é, portanto, uma extensão de \textcite{petrini_demanda_2019}.}.

É importante ressaltar que para manter a comparatibilidade entre esses dois capítulos, serão utilizados os países presentes na base de dados desenvolvida por \textcite{jorda_great_2014}. Vale pontuar que a grande contribuição desta base de dados é reunir os subcomponentes dos empréstimos bancários desde 1870 que abre uma extensa agenda de pesquisa ainda não suficientemente explorada.
Apesar da amplitude temporal desta base, o modelo macroeconométrico se restringirá ao pós-década de 70 para captar os efeitos da ``hipotecarização'' e contrastá-los com o modelo qualitativo desenvolvido no capítulo anterior.

No capítulo seguinte, será desenvolvido um modelo SFC representando uma economia capitalista fechada e sem governo\footnote{Para tanto, será utilizado o pacote \textit{pysolve3} escrito em python 3 e desenvolvido por \textcite{petrini_pysolve3_2019}.}. 
De acordo com \textcite{macedo_e_silva_peering_2011}, tal metodologia é composta de
três procedimentos: (i) determinação da estrutura contábil; (ii) construção das equações comportamentais
e; (iii) solução/simulação\footnote{
	Vale destacar a expansão dos trabalhos que seguem esta metodologia a partir das análises de \textcite{godley_money_1999} e da sistematização de \textcite{godley_monetary_2007}.
}.
As etapas contábeis da abordagem SFC constituem em: (i) seleção dos setores institucionais e dos ativos a serem incorporados; (ii) mapeamento das relações dos fluxos entre os mencionados setores por meio da construção da matriz de fluxos; (iii) construção da matriz dos estoques de riqueza (real e financeira) em que são contabilizadas os ativos e passivos  bem como a posição líquida de cada setor; (iv) identificação das formas que os fluxos são financiados e sua respectiva acumulação/alocação dos estoques. 
Como todo modelo macroeconômico, ao partir de um aparato analítico
baseado em identidades contábeis, surgem restrições que precisam ser seguidas mas o que distingue a metodologia SFC das demais é a conexão do lado real com o financeiro de forma integrada.
Tal procedimento garante que para que um setor acumule riqueza financeira, outro precisa necessariamente liquidá-la de modo que não existam ``buracos negros'' \cite{godley_money_1996}.

As relações de causalidade, por sua vez, decorrem das equações comportamentais que, respeitando a consistência, podem ser de qualquer linhagem teórica.
Dada a estrutura contábil e explicitadas as hipóteses e equações comportamentais, resta seguir para a resolução do modelo. Como pontuam \textcite{caverzasi_stock-flow_2013}, existem três vias: (i) simulação; (ii) analítica e; (iii) descritiva. A primeira delas permite expor as relações entre as variáveis de modelos mais complexos em que a solução analítica não é facilmente encontrada. No entanto, tal caminho fez com que o grau de complexidade dos modelos simulados fosse exponencializada de modo que a intuição econômica torna-se facilmente turva.  

Esta pesquisa priorizará a parcimônia de modo que serão incluídos apenas os elementos necessários dados os objetivos desta pesquisa.
Em outras palavras, modificações que dizem respeito às relações entre famílias, firmas e bancos seguirão os resultados dos capítulos anteriores.
Extensões do modelo básico --- como inclusão do governo e setor externo --- ocorrerão se resultados reportados anteriormente indicarem a relevância da inclusão destes setores institucionais.
A justificativa deste procedimento decorre da maior clareza
da modelagem frente a um menor ``realismo''. Além disso, tal postura permite explicitar os parâmetros mais relevantes para as
trajetórias de longo prazo\footnote{
	Adicionalmente, será realizada uma exploração do espaço paramétrico por meio de análises de sensibilidade global como proposto por \textcite{saltelli_variance_2010}.
}. Apesar da parcimônia do modelo, a simulação tem a vantagem de fornecer
informações que não se restringem às soluções de equilíbrio e esta forma também será selecionada
para resolver o modelo uma vez que permite também analisar o \textit{traverse}\footnote{Cabe aqui pontuar a realização de modelos de simulação em \textcite{da_silveira_investimento_2019} e \textcite{petrini_demanda_2019}.}. 
A hipótese de trabalho deste capítulo é que os condicionantes institucionais --- agrupados no capítulo primeiro --- e os determinantes da taxa de crescimento do investimento residencial --- reportados no capítulo segundo ---implicam dinâmicas macroeconômicas distintas.
Dessa forma,a partir do modelo SFC, serão reunidos os esforços da análise qualitativa, bem como os resultados do modelo empírico.


\begin{comment}

Mais especificamente, uma condição é considerada suficiente quando o números de um condicionantes de um resultado supera o número de casos que apresentam este resultado. Um condição necessária é caracterizada pelo oposto, ou seja, existem mais casos que apresentam este resultado do que condições.
\end{comment}

\subsection{Análise qualitativa comparativa}
 
A hipótese de trabalho do primeiro capítulo é que a institucionalidade presente nos diversos sistemas financeiros é relevante para explicar o grau de hipotecarização de um país.
Adicionalmente, partir-se-á da fundamentação teórica de  \textcite{chang_institutions_2011} em que diferentes configurações institucionais podem desempenhar as mesmas funções. 
Para tanto, será realizada uma  análise comparativa qualitativa (QCA, na sigla em inglês)\footnote{A metodologia que será utilizada para dar conta desse objetivo é semelhante à empregada em outro trabalho aplicada a outro objeto \cite{petrini_comparacao_2019}. 
}. 
Desenvolvida e aprimorada por \textcites{ragin_comparative_1989}{ragin_set_2006}{box-steffensmeier_measurement_2009}, esta metodologia associa todas as configurações possíveis das variáveis explicativas a um resultado específico por meio de operações lógicas e teoria dos conjuntos. 
A escolha desta metodologia se dá por: 
	(i) enfatizar as singularidades de cada unidade de investigação; 
	(ii) tratar os casos holisticamente e; %, ou seja, como unidades integradas por uma complexa combinação de propriedades e;
	(iii) incluir assimetria causal.
A partir desta metodologia, é possível destacar quais elementos institucionais são necessários ou suficientes para determinar o grau de hipotecarização de um país sem que para isso seja necessário desconsiderar as especificidades de cada caso analisado.

Por se tratar de uma metodologia pouco utilizada em economia, serão apresentados seus procedimentos em maiores detalhes como em \textcite{rihoux_configurational_2009}.
Estabelecido o fenômeno (resultado) de interesse, os casos a serem analisados e quais seus possíveis determinantes, a primeira etapa  consiste na adequação dos dados à variante QCA a ser utilizada (\textit{crisp-set}, \textit{multivalue} ou \textit{fuzzy-set}).
Na etapa seguinte, constrói-se uma tabela verdade em que são apresentadas as configurações comuns de cada caso em relação ao resultado.
A partir desta tabela, é possível avaliar as condições necessárias e suficientes destas configurações, bem como as contradições\footnote{\textcite[p.~48--56]{rihoux_configurational_2009} apresentam formas de resolver tais contradições.}.
Na ausência de contradições e determinadas as condições suficientes, são realizados procedimentos de minimização para agrupar os casos semelhantes e obter a solução parcimoniosa\footnote{Para uma discussão dos algoritmos de minimização ver \textcite{dusa_mathematical_2010}.}.
Com a solução parcimoniosa em mãos, resta interpretar os resultados obtidos.

Em relação a essa pesquisa, o resultado a ser analisado é o grau de hipotecarização do sistema bancário de um país, ou seja, quanto maior a participação das hipotecas no balanço patrimonial dos bancos mais ``hipotecarizado''.
Por se tratar de uma variável contínua com limites de pertencimento pouco estabelecidos, a variante \textit{fuzzy} se mostra a melhor alternativa para abordar este objetivo \cite{zadeh_fuzzy_1965}, portanto se trata de um \textit{fuzzy-set} QCA (fsQCA) tal como proposto e aprimorado por \textcites{ragin_fuzzy-set_2000}{ragin_redesigning_2008}.
Para a determinação da função de pertinência \textit{fuzzy} (calibragem), serão utilizados o método direto (teórico) e indireto (estatístico).
As variáveis serão selecionadas a partir de uma ampla revisão de literatura em que serão identificados os condicionantes institucionais da hipotecarização.
Os casos serão os países da base de dados de \textcite{jorda_rate_2019} para os anos com quebras estruturais ao longo do período caracterizado pelo aprofundamento do grau de hipotecarização (1980-2016)\footnote{Vale mencionar que por serem países-membros da OCDE, possuem bases estatísticas padronizadas. Sendo assim, tais economias possuem um grau maior de comparação entre si e, portanto, as especificidades institucionais podem ser melhor destacadas.}.
A validação do modelo será baseada nos índices de consistência e abrangência propostos por \textcite{ragin_set_2006}.
Com isso, espera-se averiguar quais são as características institucionais necessárias e suficientes para explicar o grau de hipotecarização de um país.


%%%%%%%%%%%%%%%%%%%%%%%%%%%%%%%%%%%%%%%%%%%%55 DADOS EM PAINEL %%%%%%%%%%%%%%%%%%%%%%%%%%%%%%%%%%%
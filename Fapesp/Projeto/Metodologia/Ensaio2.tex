\subsection{Modelo de séries temporais em painel}\label{metodologia2}

Compreendidos os fatores institucionais, a segunda parte desta pesquisa analisará a dimensão quantitativa da macroeconomia imobiliária.
Neste capítulo, serão estimados os determinantes do investimento residencial que, como visto na revisão de literatura, são fundamentais para a compreensão da dinâmica macroeconômica.
A hipótese de trabalho é que o investimento residencial depende tanto da concessão de crédito quanto da bolha de ativos.
Para tanto, será estimado um modelo de séries temporais em  painel para os países presentes na base de dados de \textcite{jorda_rate_2019}, com ênfase ao período de aumento do grau de hipotecarização (1980-2016)\footnote{Os países analisados são aqueles mencionados na nota de rodapé \ref{nota_paises}.}\footnote{Ao longo da pesquisa serão utilizados outros critérios para aprimorar o recorte temporal a ser utilizado, dentre eles, inclusão de \textit{dummies} referentes a quebra estrutural e a mudanças institucionais. Vale destacar que \textcite{petrini_demanda_2019} faz este procedimento para os EUA em que o recorte temporal do modelo econométrico considerou reformas regulatórias.}.
Serão incluídas variáveis destacadas pela literatura (\textit{e.g.} taxa de juros das hipotecas; \textit{dummies} para mudanças regulatórias) em diferentes especificações econométricas.
 Em particular, será testada a capacidade explicativa  da taxa própria de juros dos imóveis uma vez que \textcite{petrini_demanda_2019} encontrou que esta variável é relevante para o caso dos EUA.

%TODO: Marcar. Tinha alterado em outro varsão, não sei se chegou a ver.
Como alguns dos países analisados fazem parte de um bloco econômico (os países-membro da União Europeia), serão realizados testes de dependência \textit{cross-section}\footnote{Argumentação semelhante pode ser encontrada em \textcite{perez-montiel_autonomous_2020}.} \cites{breusch_lagrange_1980}{pesaran_bias-adjusted_2007}.
Além de integrados, espera-se que tais países sejam heterogêneos e, portanto, serão feitos testes de homogeneidade \textit{cross-section} antes de seguir para as estimações para evitar inconsistências \cites{pesaran_testing_2008}.
Dado que os testes de precedência temporal são baseados em séries estacionárias, serão realizados testes de raiz unitária e de cointegração específicos para dados em painel \cites{im_testing_2003}{pesaran_simple_2007}{canning_infrastructure_2008}. %{pedroni_panel_2004}
No que diz respeito à especificação do modelo, serão feitos diferentes ajustes (LSDV-FE, IV-FE, GLS-RE, G2SLS-RE, etc) e utilizados métodos (OLS, OL2S, FMOLS, etc) nos quais será priorizada a parcimônia e ausência de autocorrelação e heterocedasticidade residual. 
%No entanto, por se tratar de um macropainel em que extensão temporal é maior que o número de unidades (países) analisados, espera-se que que a estimação por efeitos fixos  e aleatórios sejam mais apropriadas que um sistema GMM \cite{judson_estimating_1999}\footnote{\textcite{andersen_gmm_1996}, \textcite{bowsher_testing_2002} e \textcite{roodman_note_2009} reportam que os testes de Hausman e de Sargan são comprometidos na medida que o número de instrumentos utilizados aumenta e este é o caso do modelo a ser estimado em que a dimensão temporal é significativamente grande ($T = 36 > 30$).}.
No pós-estimação, serão feitos testes para avaliar a qualidade do ajuste \cites{hausman_specification_1978}{maasoumi_testing_1988}.
Para avaliar a robustez dos resultados, os dados serão divididos em subperíodos (1980-2008 e 2008-2016) para isolar os efeitos da Grande Recessão e então estimar o modelo por um sistema GMM\footnote{ %TODO Karina
    Tal procedimento é recomendado uma vez que a dimensão temporal é reduzida, diminuindo o número de instrumentos necessários \cite{judson_estimating_1999}.
}. 



%%%%%%%%%%%%%%%%%%%% AB-SFC %%%%%%%%%%%%%%%
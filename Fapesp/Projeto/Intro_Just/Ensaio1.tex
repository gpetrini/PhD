\subsection{Especificidades institucionais da macroeconomia imobiliária}

Nos últimos anos, nota-se um aumento dos estudos que destacam o potencial explicativo das instituições em uma perspectiva comparada com especial ênfase no setor financeiro\footnote{
    \textcite{blackwell_origins_2018}, por exemplo, discutem as origens da institucionalidade bancária e financeira do mercado de hipotecas. \textcite{van_gunten_varieties_2018}, por sua vez, analisam mudanças institucionais na oferta de crédito em Portugal, Espanha, França e Alemanha e concluem que tais mudanças foram responsáveis pela maior intensificação financeira das famílias.
}.
Dentro deste paradigma, a literatura de Economia Política Comparada (CPE, no inglês) se destaca por conceituar o papel macroeconômico deste setor institucional.
No entanto, a vertente predominante da CPE (``Variedade de Capitalismos'', em inglês, VoC) tem redirecionado atenção a questões microeconômicas e relacionadas à oferta de crédito \cite{schwartz_thinking_2019}.
Recentemente, há um esforço de conectar tal abordagem comparativa aos modelos de crescimento liderados pela demanda em que o endividamento das famílias  é central \cite{baccaro_rethinking_2016}.
No entanto, 
estes modelos %\textcite{baccaro_rethinking_2016} 
não dão a devida atenção à composição e aos determinantes do crescimento do setor financeiro que, como visto, não podem ser desassociados do mercado imobiliário e das hipotecas \cite{wood_house_2020}.

Uma literatura que dá muita atenção aos fenômenos associados à aqui denominada macroeconomia imobiliária é a da financeirização que analisa, principalmente, as relações entre investimento produtivo, instabilidade financeira e crescimento \cites{stockhammer_financialisation_2004}{hein_demise_2015}{detzer_financialization_2019}. %\cites{stockhammer_financialisation_2004}{orhangazi_financialisation_2008}{hein_demise_2015}{detzer_financialization_2019}.
Além de não dar atenção às famílias e ao investimento residencial --- limitando-os à subprocessos da financeirização \cites{aalbers_financialization_2008}{schwartz_politics_2009}{bibow_financialization_2010} ---, esta literatura pouco tem avançado em uma análise institucional comparada. 
Todos os trabalhos que o fazem se restringem a um número pequeno de países nas suas amostras \cites{becker_peripheral_2010}{lapavitsas_financialisation_2013}. % em que analisam apenas quatro economias em ambos os trabalhos. %%% KARWARZASKI ANALISA VÁRIOS, MAS USA CORRELAÇÃO E NÃO É POSSÍVEL AVALIAR CONDIÇÕES NECESSÁRIAS E SUFICIENTES

Uma exceção é o trabalho de \textcite{karwowski_financialisation_2019} que avalia as diferentes consequências da financeirização para as famílias, firmas e bancos em 17 países da OCDE por meio do coeficiente de correlação de postos de Spearman, concluindo que o preço dos ativos (sobretudo imóveis) são relevantes para explicar o endividamento das famílias.
No entanto, este estudo não permite avaliar quais configurações institucionais são necessárias ou suficientes para explicar como a inflação de ativos está conectada com esse aumento dos passivos das famílias.
\textcite{mertzanis_financialisation_2019}, por sua vez, investiga os determinantes do acesso ao crédito para as firmas em 138 economias em desenvolvimento por meio de modelo \textit{probit} com \textit{dummies} institucionais e conclui que tais variáveis não só determinam o acesso ao financiamento como também amenizam as restrições financeiras das firmas.
Entretanto,  tal análise só permite investigar os efeitos marginais das instituições.
Sendo assim, conclui-se que estas análises comparativas para médias e grandes amostras carecem de um aparato metodológico mais adequado para tratar das instituições qualitativamente.


Desta discussão, conclui-se que os autores da financeirização, CPE e VoC que partem de uma perspectiva institucional comparada dão maior ênfase ao desempenho das firmas enquanto uma parcela menor analisa as famílias.
O aumento sem precedentes das hipotecas no balanço patrimonial dos bancos reportado por \textcite{jorda_great_2016} evidencia  que a pouca atenção dada às famílias, ao investimento residencial e às hipotecas é desproporcional à relevância que tais elementos possuem na dinâmica financeira recente. 
No entanto, mesmo os trabalhos que tocam os temas da aqui denominada macroeconomia imobiliária dão pouca ênfase às institucionalidades do mercado imobiliário em particular.
Por exemplo, \textcite{wijburg_alternative_2017} destacam que a especificidade institucional do mercado imobiliário alemão o configura como um contraponto ao norte-americano pela baixa liquidez do mercado hipotecário e que tal particularidade condiciona a relação entre preço dos imóveis, investimento residencial e crescimento econômico. %%%% EXEMPLOS
\textcite{johnston_global_2017}, por sua vez, reportam a importância das instituições na determinação dos preços dos imóveis enquanto \textcite{fuller_housing_2020} encontram evidências que a trajetória deste preço é relevante para explicar a concentração da riqueza em alguns países europeus.

Além de não tratar dos temas da macroeconomia imobiliária conjuntamente, a literatura não tem investigado o porquê de alguns sistemas bancários apresentarem um maior grau de hipotecarização que outros e esta é uma das lacunas a ser preenchida por esta tese.
%Ao longo desta pesquisa, 
Argumenta-se que a regularidade encontrada por \textcite{jorda_great_2016}
não significa que a especificidade de cada país deixa de ser relevante. 
%e, assim, parte das instituições formais de cada pais para avaliar 
Seguindo \textcite{chang_institutions_2011},  parte-se da hipótese de ausência de uniformidade causal do arranjo institucional.
Apropriando esta fundamentação teórica ao objeto desta tese,  assume-se que diferentes configurações institucionais podem desempenhar as mesmas funções e explicar o grau de hipotecarização. %dos sistemas bancários.

A tabela \ref{Institucional} reúne algumas características institucionais tratadas dispersamente pela literatura que dizem respeito às relações entre famílias, bancos e mercado imobiliário para alguns dos países presentes na base construída por \textcite{jorda_rate_2019}. São elas: 
    (i) maturidade das hipotecas \cite{green_american_2005}; 
    (ii) determinação  e tipo da taxa de juros das hipotecas (fixa ou flexível); 
    (iii) arranjo regulatório sobre quitação antecipada do crédito imobiliário (contrato ou legislação) e formas de refinanciamento;
    (iv) possibilidade de uma segunda hipoteca a partir da valorização do imóvel;
    (v) disponibilidade de crédito de longo-prazo para as famílias \cite{schwartz_politics_2009} e; 
    (vi) possibilidade de transferência de riscos (\textit{e.g.} securitização \cite{european_central_bank_housing_2010}).
    
    



\begin{table}[htb]
	\centering
	\caption{Características institucionais de alguns países europeus da OCDE}
	\label{Institucional}
		\resizebox{.7\textwidth}{!}{%
			\begin{tabular}{c|c|c|c|c|c|c}
				\hline\hline \\
				\multirow{2}{*}{\textbf{Países}} & \multicolumn{6}{c}{\textbf{Características institucionais}} \\\cline{2-7}
				&
				\textbf{\begin{tabular}[c]{@{}c@{}}Maturidade\\ Hipotecária\\(meses)\end{tabular}} &
				\textbf{\begin{tabular}[c]{@{}c@{}}Taxa de juros\\ Hipotecária\end{tabular}} &
				\textbf{\begin{tabular}[c]{@{}c@{}}Reembolso antecipado:\\ Contratado (C)/\\ Legislado (L)\end{tabular}} &
				\textbf{\begin{tabular}[c]{@{}c@{}}Possibilidade de segunda\\hipoteca a partir\\da valorização do imóvel\end{tabular}} &
				\textbf{\begin{tabular}[c]{@{}c@{}}Financiamento pelo\\ Mercado de capitais (\%)\end{tabular}} &
				\textbf{\begin{tabular}[c]{@{}c@{}}Execução\\ Hipotecária\\(meses)\end{tabular}} \\\hline
				\textbf{Alemanha}                & 30   & Fixa       & C/L   & Não permitido    & 14   & 9    \\\hline
				\textbf{Espanha}                 & 30   & Variável   & C/L   & Limitado         & 45   & 8    \\\hline
				\textbf{França}                  & 19   & Fixa       & C/L   & Não permitido    & 12   & 20   \\\hline
				\textbf{Holanda}                 & 30   & Fixa       & C     & Permitido        & 25   & 5    \\\hline
				\textbf{Itália}                  & 22   & Variável   & L     & Não permitido    & 20   & 56   \\\hline
				\textbf{Portugal}                & 40   & Variável   & L     & Sem informação   & 27   & 24  \\\hline
				\hline
				
			\end{tabular}%
		}
	\caption*{\textbf{Fonte:}  \textcite[p.~94, adaptado e traduzido]{van_gunten_varieties_2018}}
\end{table}
Uma breve inspeção da tabela \ref{Institucional} revela uma pluralidade de configurações institucionais no que dizem respeito ao mercado de crédito imobiliário. 
Apesar desta pluralidade institucional, destaca-se que a hipotecarização é um fenômeno comum a estes países.
Sendo assim, pontua-se a necessidade de se investigar quais desses arranjos institucionais propiciam um maior ou menor grau de hipotecarização.
Diferentemente de \textcites{karwowski_financialisation_2019}{mertzanis_financialisation_2019}, a presente pesquisa propõe uma análise institucional centrada em variáveis qualitativas e se difere da proposta de \textcites{becker_peripheral_2010}{lapavitsas_financialisation_2013} ao analisar um número maior de casos sem que, para isso, abandone a relevância das particularidades de cada país.
Sendo assim, ao partir de um aparato metodológico mais adequado para tratar de variáveis qualitativas, é possível investigar diferentes configurações institucionais e compreender em que medida estes elementos ajudam a compreender os determinantes necessários e suficientes da hipotecarização.

\begin{comment}
Por mais que estes trabalhos lancem luz sobre as implicações das instituições sobre a financeirização, não é possível 

tais análises só permite investigar os efeitos marginais uma vez que parte de uma especificação econométrica.

Como visto, \textcite{jorda_great_2016} reportam um aumento sem precedentes da participação das hipotecas no balanço patrimonial dos bancos. 


No entanto, ao partir de correlações, o estudo de \textcite{karwowski_financialisation_2019} não permite avaliar quais configurações são necessária ou suficientes para explicar o aumento das diferentes dimensões da financeirização.
No que diz respeito especificamente às configurações institucionais, o autor conclui que tais variáveis determinam o acesso ao financiamento e atenuam o impacto da financeirização.

Sendo assim, a pouca atenção dada às famílias, ao investimento residencial e às hipotecas é desproporcional à relevância que tais elementos possuem na dinâmica financeira recente.
\end{comment}


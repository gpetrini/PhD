\section{Plano de trabalho e cronograma de atividades}\label{cronograma}


O trabalho será orientado pelo Prof. Dr. Lucas Azeredo da Silva Teixeira (Unicamp) e coorientado pela Profa. Dra. Ivette Raymunda Luna Huamani (Unicamp). 
A tabela \ref{crono} apresenta um cronograma das atividades. Os capítulos estão destacados em vermelho, as etapas necessárias para concluir cada um deles está em laranja e em cinza as obrigações institucionais.
Cabe destacar que desde o ingresso no programa de doutorado até a
submissão deste projeto, o aluno concluiu as disciplinas necessárias ao cumprimento dos créditos exigidos pelo programa, realizou um estágio de docência, apresentou artigos em congressos
internacionais (\textit{EEA} e \textit{EAEPE}), contribuiu nas atividades do Centro de Estudos de Conjuntura e Política Econômica (Cecon-Unicamp), realizou cursos de R, Python, LSD e QCA (ferramentas a serem utilizadas na pesquisa) e, ao longo do período de avaliação do projeto, submeterá dois artigos referentes à dissertação.
 Como resultado da tese, planeja-se submeter ao menos três artigos para conferências internacionais e nacionais e ao menos três artigos para revistas de circulação internacional indexadas na área.
 %Vale destacar que serão produzidas rotinas e um pacote --- de forma livre e aberta --- para elaboração do modelo fsQCA como subproduto da tese\footnote{Cabe a menção do pacote \textit{fsQCA} em python2 desenvolvido por \textcite{reichert_kirq_2014}. Pretende-se adequar este pacote para python3 e, assim, compatibilizar com os avanços desta linguagem podendo ser estendido para análises de redes sócio-econômicas e redes neurais com \textit{machine learning}.}.

\begin{table}[H]
	\centering
	\caption{Cronograma de atividades}
	\tiny
	\label{crono}
	\resizebox{.7\textwidth}{!}{%
	\begin{tabular}{ll|l|l|l|l|ll}
	\hline\hline
\multicolumn{1}{c}{} & \multicolumn{6}{c}{\textbf{Período}} \\ \cline{2-7} 
\multicolumn{1}{c}{\multirow{-2}{*}{\textbf{Atividades}}} & \multicolumn{1}{c|}{\textbf{1º Semestre 2020}} & \multicolumn{1}{c|}{\textbf{2º Semestre  2020}} & \multicolumn{1}{c|}{\textbf{1º Semestre  2021}} & \multicolumn{1}{c|}{\textbf{2º Semestre  2021}} & \multicolumn{1}{c|}{\textbf{2022}} & \multicolumn{1}{c}{\textbf{2023}} \\ \hline

\textbf{1. Fundamentação teórica} & \cellcolor[HTML]{FF9933} &\cellcolor[HTML]{FF9933}&\cellcolor[HTML]{FF9933}&  & &  \\ \hline
1.1. Disciplinas & \cellcolor[HTML]{9B9B9B} &&&&&  \\ \hline
1.2. Revisão bibliográfica & \cellcolor[HTML]{FF9933} &\cellcolor[HTML]{FF9933}&\cellcolor[HTML]{FF9933}&  & &  \\ \hline

\textbf{2. Modelo QCA} &&\cellcolor[HTML]{FF0000}&\cellcolor[HTML]{FF0000}&\cellcolor[HTML]{FF0000}&& \\ \hline
2.1. Análise comparativa &&\cellcolor[HTML]{FF9933}&\cellcolor[HTML]{FF9933}&\cellcolor[HTML]{FF9933}&& \\ \hline
2.2. Construção e resultados &&&&\cellcolor[HTML]{FF9933}&& \\ \hline

\textbf{3. Qualificação} &&&&\cellcolor[HTML]{9B9B9B}&& \\ \hline

\textbf{4. Modelo Painel} &&\cellcolor[HTML]{FF0000}&\cellcolor[HTML]{FF0000}&\cellcolor[HTML]{FF0000}&\cellcolor[HTML]{FF0000}&\cellcolor[HTML]{FF0000} \\ \hline
4.1. Preparação dos dados &&\cellcolor[HTML]{FF9933}&\cellcolor[HTML]{FF9933}&\cellcolor[HTML]{FF9933}&& \\ \hline
4.2. Estimação e análise &&&&&\cellcolor[HTML]{FF9933}&\cellcolor[HTML]{FF9933} \\ \hline

\textbf{5. Bolsa de Estágio no Exterior (BEPE)}\footnotemark &&&&\cellcolor[HTML]{9B9B9B}&\cellcolor[HTML]{9B9B9B}&\cellcolor[HTML]{9B9B9B}\\ \hline
\textbf{6. Modelo AB-SFC} &&&\cellcolor[HTML]{FF0000}&\cellcolor[HTML]{FF0000}&\cellcolor[HTML]{FF0000}&\cellcolor[HTML]{FF0000} \\ \hline
6.1. Construção &&&\cellcolor[HTML]{FF9933}&\cellcolor[HTML]{FF9933}&& \\ \hline
6.2. Simulação e análise &&&&&\cellcolor[HTML]{FF9933}&\cellcolor[HTML]{FF9933}\\ \hline

\textbf{7. Conclusão e Defesa} & & &  &  & & \cellcolor[HTML]{9B9B9B} \\ \hline \hline
		
	

\end{tabular}%
	\renewcommand{\arraystretch}{0.4}
	}
\caption*{\textbf{Fonte:} Elaboração própria}
\end{table}
\footnotetext{A depender da disponibilidade de financiamento. Está indicado no segundo semestre de 2021 o tempo para reunir documentos e para se preparar para realizar este estágio.}



 
 
\begin{comment}

\end{comment}





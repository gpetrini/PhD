\section[QCA]{\textbf{Ensaio 1:} Características institucionais}

\begin{frame}{Literatura e suas Lacunas}
\framesubtitle{Análises institucionais comparadas}

\begin{description}
    \item[CPE:] Conceituam o papel macroeconômico do setor financeiro
        \begin{description}
            \item[VoC:] questões microeconômicas e relacionadas à oferta de crédito
        \end{description}
    \item[Financeirização:] pouca atenção às famílias e ao investimento residencial
\end{description}

\begin{alert}{Lacunas em comum}

Pouca ênfase às institucionalidades do mercado imobiliário; à composição e aos determinantes
do crescimento do setor financeiro. 

\textbf{Estudos de caso:} amostras pequenas (max. 4). 

\textbf{Análise Quantitativa:} Instituições $\to$ efeitos marg. (simétricos).
    
\end{alert}
    
\end{frame}

\begin{frame}{Características institucionais de alguns países da OCDE}
\framesubtitle{Dimensão qualitativa da Macroeconomia Imobiliária}
    \begin{table}[htb]
	\centering
	\caption{Características institucionais de alguns países europeus da OCDE}
	\label{Institucional}
		\resizebox{.7\textwidth}{!}{%
			\begin{tabular}{c|c|c|c|c|c|c}
				\hline\hline \\
				\multirow{2}{*}{\textbf{Países}} & \multicolumn{6}{c}{\textbf{Características institucionais}} \\\cline{2-7}
				&
				\textbf{\begin{tabular}[c]{@{}c@{}}Maturidade\\ Hipotecária\\(meses)\end{tabular}} &
				\textbf{\begin{tabular}[c]{@{}c@{}}Taxa de juros\\ Hipotecária\end{tabular}} &
				\textbf{\begin{tabular}[c]{@{}c@{}}Reembolso antecipado:\\ Contratado (C)/\\ Legislado (L)\end{tabular}} &
				\textbf{\begin{tabular}[c]{@{}c@{}}Possibilidade de segunda\\hipoteca a partir\\da valorização do imóvel\end{tabular}} &
				\textbf{\begin{tabular}[c]{@{}c@{}}Financiamento pelo\\ Mercado de capitais (\%)\end{tabular}} &
				\textbf{\begin{tabular}[c]{@{}c@{}}Execução\\ Hipotecária\\(meses)\end{tabular}} \\\hline
				\textbf{Alemanha}                & 30   & Fixa       & C/L   & Não permitido    & 14   & 9    \\\hline
				\textbf{Espanha}                 & 30   & Variável   & C/L   & Limitado         & 45   & 8    \\\hline
				\textbf{França}                  & 19   & Fixa       & C/L   & Não permitido    & 12   & 20   \\\hline
				\textbf{Holanda}                 & 30   & Fixa       & C     & Permitido        & 25   & 5    \\\hline
				\textbf{Itália}                  & 22   & Variável   & L     & Não permitido    & 20   & 56   \\\hline
				\textbf{Portugal}                & 40   & Variável   & L     & Sem informação   & 27   & 24  \\\hline
				\hline
				
			\end{tabular}%
		}
	\caption*{\textbf{Fonte:}  \textcite[p.~94, adaptado e traduzido]{van_gunten_varieties_2018}}
\end{table}
\end{frame}

\begin{frame}{Objetivo e Metodologia}
    
\begin{description}
    \item[Objetivo:]  examinar as configurações institucionais que determinam o grau de hipotecarização do sistema bancário de um país
    \item[Hipótese:] institucionalidade presente nos diversos sistemas financeiros é relevante para explicar o grau de hipotecarização
        \begin{itemize}
            \item ausência de uniformidade causal do arranjo institucional \cite{chang_institutions_2011}
        \end{itemize}
    \item[Metodologia:] \textit{fuzzy-set} QCA (fsQCA)
    \item[Recorte:] Países da base de dados de \textcite{jorda_rate_2019} para os anos de 1980 a 2016
\end{description}
\end{frame}

\begin{frame}{O que diabos é QCA?}
\framesubtitle{(Mas e o \textit{probit}?)}

\textbf{QCA:} Qualitative Comparative Analysis

 Desenvolvida originalmente por \textcite{ragin_comparative_1989}, esta metodologia associa todas as configurações possíveis das variáveis explicativas a um resultado específico por meio de operações lógicas e teoria dos conjuntos.

Permite identificar as condições necessárias e suficientes para ocorrência da hipotecarização

\begin{itemize}
    \item Equifinalidade
    \item Causação conjuntural
    \item Causação assimétrica
\end{itemize}
\end{frame}
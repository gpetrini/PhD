\section[Painel]{\textbf{Ensaio 2:} Investimento residencial e a dinâmica macroeconômica}


\begin{frame}{Literatura e suas Lacunas}
\framesubtitle{Dimensão ``quantitativa'' da Macroeconomia Imobiliária}

\begin{description}
    \item[Pré Grande Recessão] atenção ao investimento residencial foi esparsa e assimétrica
    \item[Pós Grande Recessão] $\Uparrow$ literatura econométrica sobre as \textbf{implicações} macroeconômicas do
investimento residencial
\end{description}

\begin{alert}{Lacunas em comum}

Poucos investigam os \textbf{determinantes} do investimento residencial. 

\begin{itemize}
    \item EUA em específico \cites{teixeira_crescimento_2015}{petrini_demanda_2019}
    \item \textcite{arestis_residential_2015} $\nRightarrow$ Inflação de ativos
\end{itemize}
\end{alert}
    
\end{frame}

\begin{frame}{FEVD}
\framesubtitle{EUA (1992 - 2019)}
    \begin{figure}
        \centering
        \includegraphics[width=.6\paperwidth, height=.58\paperheight]{figs/FEVD_VECMpython_TxPropria.png}
        \caption*{\textbf{Fonte:} \textcite{petrini_demanda_2019}}
    \end{figure}
\end{frame}

\begin{frame}{Objetivo e Metodologia}
    
\begin{description}
    \item[Objetivo:]  Estimar os determinantes do investimento residencial
    
        \begin{itemize}
            \item Investigar a capacidade explicativa da Tx. Própria para outros países
        \end{itemize}
    
    \item[Hipótese:] Investimento residencial depende tanto da concessão de crédito quanto da bolha de ativos
    \item[Metodologia:] Modelo de Séries Temporais em Painel
    \item[Recorte:] Países da base de dados de \textcite{jorda_rate_2019} para os anos de 1980 a 2016
\end{description}
\end{frame}
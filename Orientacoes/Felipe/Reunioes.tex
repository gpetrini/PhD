% Created 2020-10-08 qui 11:26
% Intended LaTeX compiler: pdflatex
\documentclass[11pt]{article}
\usepackage[utf8]{inputenc}
\usepackage{lmodern}
\usepackage[T1]{fontenc}
\usepackage[top=3cm, bottom=2cm, left=3cm, right=2cm]{geometry}
\usepackage{graphicx}
\usepackage{longtable}
\usepackage{float}
\usepackage{wrapfig}
\usepackage{rotating}
\usepackage[normalem]{ulem}
\usepackage{amsmath}
\usepackage{textcomp}
\usepackage{marvosym}
\usepackage{wasysym}
\usepackage{amssymb}
\usepackage{amsmath}
\usepackage[theorems, skins]{tcolorbox}
\usepackage[style=abnt,noslsn,extrayear,uniquename=init,giveninits,justify,sccite,
scbib,repeattitles,doi=false,isbn=false,url=false,maxcitenames=2,
natbib=true,backend=biber]{biblatex}
\usepackage{url}
\usepackage[cache=false]{minted}
\usepackage[linktocpage,pdfstartview=FitH,colorlinks,
linkcolor=blue,anchorcolor=blue,
citecolor=blue,filecolor=blue,menucolor=blue,urlcolor=blue]{hyperref}
\usepackage{attachfile}
\usepackage{setspace}
\usepackage{tikz}
\author{Gabriel Petrini}
\date{2020}
\title{Reuniões de Orientação - Felipe}
\begin{document}

\maketitle
\tableofcontents


\section*{Discussão do objeto da monografia}
\label{sec:org8e5cb8f}

\textbf{Texto Enviado pelo Aluno:}

\begin{quote}
Como já havíamos conversado anteriormente, eu gostaria de desenvolver a ideia de como a mudança na governança corporativa das firmas, que passou a ser voltada para gerar valor aos acionistas acaba por afetar os padrões de investimento macroeconômicos. Essa ideia tirei do texto do Karwowski “dimensions and determinants of financialisation: comparing oecd countries since 1997” que você havia me indicado.
\end{quote}



\subsection*{Textos indicados}
\label{sec:orgf663b19}


\subsection*{Proposta de estrutura de capítulos}
\label{sec:org601ab3b}

\begin{center}
\includegraphics[width=.9\linewidth]{capitulos.png}
\end{center}

\section*{Reunião \textit{<2020-09-30 qua>}}
\label{sec:org617a020}

\subsection*{Antes da próxima reunião \textit{<2020-10-07 qua>}}
\label{sec:orgd4bbcf6}

\subsubsection*{{\bfseries\sffamily TODO} Contatar Mariano}
\label{sec:orge9f02d6}
\subsubsection*{{\bfseries\sffamily TODO} Parágrafos sobre objetivo}
\label{sec:org03d42be}
\subsubsection*{{\bfseries\sffamily TODO} Kalecki e a financeirização hoje (congresso)}
\label{sec:orgc8106fc}
\subsubsection*{{\bfseries\sffamily TODO} Enviar projeto}
\label{sec:org4207253}
\subsubsection*{{\bfseries\sffamily TODO} Github para poetas}
\label{sec:org52c391d}
\subsubsection*{{\bfseries\sffamily TODO} Latex monografia}
\label{sec:org79126ec}
\subsubsection*{{\bfseries\sffamily TODO} Link Zotero}
\label{sec:org48fb42f}

\section*{Reunião \textit{<2020-10-07 qua 19:00>}}
\label{sec:org7ff123f}

Na nossa última reunião, combinamos discutir os objetivos da monografia a partir de um parágrafo preliminar. Segue abaixo (grifos meus):

\begin{quote}
Esse trabalho tem como objetivo estudar o \textbf{fenômeno da financeirização} para as economias periféricas, \textbf{em especial as da América Latina}. Partindo da \uline{noção de que a financeirização}  tem assumido um papel de crescente importância na determinação da \uline{realidade econômica desses países}, o \textbf{trabalho visa analisar a financeirização} sob a óptica da destinação de \uline{lucros e receitas} das firmas, partindo da hipótese de que a mudança de sua premissa, agora voltada a gerar valor para os acionistas, tem por consequência uma alteração dos níveis e \uline{padrões de investimento} macroeconômico. Por meio da análise e \uline{testagem} para algumas métricas dos \uline{dados} de investimento agregado e de balanço das firmas de capital societário, o trabalho pretende verificar se essa hipótese se confirma e em que medida.
\end{quote}


\subsection*{Destrinchando}
\label{sec:org2aaba06}

\begin{itemize}
\item Estudar o fenômeno da financeirização: de que forma?
\begin{itemize}
\item Empiricamente, conceitualmente, historicamente, etc
\end{itemize}
\item Ou é para as economias periféricas ou é para América Latina
\begin{itemize}
\item Os dois na mesma frase distancia o leitor do seu objeto
\end{itemize}
\item Alguns conceitos pouco rigorosos (excesso de adjetivos)
\begin{itemize}
\item Noção de que a financeirização \ldots{} \(\Rightarrow\) A financeirização tem um papel crescente \ldots{}
\item Determinação da realidade econômica \(\Rightarrow\) Crescimento? Distribuição? Estabilidade? Sustentatibilidade \ldots{} ser mais preciso
\end{itemize}
\item \textbf{Importante:} O trabalho visa analisar a financeirização sob a óptica \ldots{} Isso já é outro objetivo
\item \textbf{Hipótese:} Financeirização \(\Rightarrow\) Alteração dos níveis e padrão de investimento
\begin{itemize}
\item Sobre a hipótese: isso tem mais cara de justificativa\ldots{} (comentário/dúvida)
\item A hipótese é sobre a mudança da premissa (quebra) ou sobre a destinação do lucro ("pós-quebra", mudança da premissa como dada)
\begin{itemize}
\item Isso será importante para o recorte temporal
\end{itemize}
\item Por que nível e não crescimento?
\item O que quer dizer por "padrões de investimento"? Volatilidade? Composição? Especificar melhor
\item Mais adiante, você restringiu ainda mais (o que é bom) o objeto, mas está longe (na posição do texto) da definição dos objetivos
\begin{itemize}
\item Amostra: Economias periféricas? Países da América Latina? Firmas de capital societário de economias periféricas/América Latina?
\end{itemize}
\end{itemize}
\item Testagem não diz muito
\begin{itemize}
\item Adiante há outro objetivo \(\Rightarrow\) Verificar se essa hipótese se confirma\ldots{}
\end{itemize}
\end{itemize}

\subsection*{Dicas e sugestões}
\label{sec:org5cd63fc}

\begin{itemize}
\item Apresentar a hipótese o mais cedo possível
\item Um único objetivo geral e alguns objetivos específicos
\begin{itemize}
\item A estrutura de capítulos te ajuda nesse ponto
\begin{itemize}
\item Pontuar as diferentes dimensões da financeirização
\item Adequar alguns dados à amostra
\item Proposta (métricas)
\end{itemize}
\end{itemize}
\item Mais rigor e menos adjetivos
\item Incluir: perspectiva comparada
\begin{itemize}
\item Conforme conversamos em reunião (gravado), esse ponto é bem importante
\end{itemize}
\item As variáveis (ex Lucros e Receitas) e métricas a serem utilizadas precisam não precisam aparecer nos objetivos, mas sim na \textbf{metodologia}
\item Estrutura (sugestão, reforço)
\begin{itemize}
\item Objetivo
\item Recorte (países, tempo)
\item Hipótese
\item Relevância (Este item pode aparecer primeiro também)
\end{itemize}
\item Dar maior ênfase às outras dimensões (apenas uma foi abordada)
\begin{itemize}
\item Sobre demanda agregada
\item Preço dos ativos
\item Dica (para o futuro): Mudanças na demanda agregada \(\Rightarrow\) Efeitos econômicos (crescimento/nível de investimento) \(\Leftrightarrow\) Outras dimensões da financeirização
\end{itemize}
\end{itemize}


\subsection*{Para a próxima reunião \textit{<2020-10-14 qua 19:00>}}
\label{sec:org906296c}

\subsubsection*{{\bfseries\sffamily TODO} Novo parágrafo(s) sobre objetivos (geral e específico)}
\label{sec:orga5b3529}

\subsubsection*{{\bfseries\sffamily TODO} Pesquisar dados sobre bolsa de valores na América Latina (mais urgente)}
\label{sec:org3c164d0}

\begin{itemize}
\item Possível caminho alternativo: separação entre propriedade e gerência está na raiz da geração do valor ao acionista
\begin{itemize}
\item Bolsas na AL podem não ser significativas, mas as matrizes não são locais. Logo, a remessa de lucros para o exterior é geração de valor ao acionista não doméstico
\end{itemize}
\item Isso é importante para a viabilidade da monografia
\item Conforme sugerido em reunião, textos do Sarti, Laplane e Alex Whillians podem ajudar
\end{itemize}


\subsubsection*{{\bfseries\sffamily TODO} Contatar Mariano}
\label{sec:org1336633}

\subsubsection*{{\bfseries\sffamily TODO} Checar se é possível doutorando orientar dois alunos}
\label{sec:org188d3bf}

\subsubsection*{{\bfseries\sffamily TODO} Ver adequabilidade do texto passado pelo Lucas (Gabriel)}
\label{sec:orgee356e9}
\begin{itemize}
\item Levantamento bibliográfico sobre separação entre propriedade e gerência (Felipe)
\begin{itemize}
\item Foco menor na financeirização
\end{itemize}
\item Isso é importante na adequação dos dados, menos urgente mas pode ter mais literatura
\end{itemize}
\end{document}
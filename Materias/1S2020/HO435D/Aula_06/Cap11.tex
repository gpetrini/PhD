\section*{Engerman, Harber e Sokoloff: Inequality, institutions and differential paths of growth among New World Economies}

\subsection*{Introdução}

Os autores iniciam o capítulo pontuando que as explicações para o diferencial das taxas de crescimento e nível de renda entre os países centram muito em questões econômicas. Em seguida, destacam a importância de se entender as diferenças entre as instituições, como são introduzidas e como elas evoluem ao longo do tempo. Adiante, fazem uma distinção entre as formas institucionais que têm um impactos sobre o desenvolvimento econômico. O primeiro deles diz respeito aos regras e leis formais aos quais os indivíduos e firmas atuam. O segundo trata da esfera econômica em que destacam as escolhas voluntárias e cooperativas. O terceiro contempla valores culturais e religiosos a influência sobre o comportamento econômico. Os autores pontuam que o estudo destaca a importância da velocidade de adaptação das instituições.

 Em seguida, os autores destacam a importância das dotações iniciais de cada sociedade nos desdobramentos econômicos. Dentre os fatores, elencam a importância da qualidade do solo; concentração de áreas cultivadas e respectivos impactos sobre economias de escala. Em linhas gerais, a combinação de algumas dessas circunstâncias estimularam a evolução de sociedades relativamente mais homogêneas. Feito este panorama, os autores argumentam que é esperado que sociedades mais igualitárias tendem a ter uma trajetória de crescimento mais sustentada e a análise das colônias nas Américas suportam tal argumento.
 
 \subsection*{Uma breve descrição do crescimento das economias do Novo Mundo}
 
 Os autores chamam atenção que as colônias americanas tinham comumente um elevado produto marginal do trabalho dado o enorme montante de imigrantes trazidos para serem escravizados. Somado a isso, os autores pontuam que não existiam barreiras culturais para impedir o uso de escravos de forma generalizada. Dito isso, passam a discutir a dinâmica do fluxo migratória em que a participação de migrantes escravos cresceu continuamente, em especial nas colônias de produção especializada. Além disso, destacam o quão pequena é a participação populacional dos descendentes europeus e como foi rapidamente superada pelos descendentes africanos. Os autores usam o exemplo dos EUA e Canada para mostrar que a maior prevalência de proprietários brancos --- e, portanto, menos desigual --- pode ajudar a explicar o porquê que essas economias cresceram mais uma vez que encorajou o surgimento de instituições legais e políticas que conduziram uma maior participação no mercado.
 
 Em seguida, os autores ponderam que existem disparidades das taxas de crescimento dentre países colonizados pelas mesmas metrópoles de modo que a importância das condições iniciais devem reexaminadas. Além disso, reconhecem que EUA e Canada são diferentes das demais colônias uma vez que a dotação inicial dos fatores permitiu uma maior predisposição para padrões de desenvolvimento relativamente mais igualitários e que as respectivas instituições favoreceram a participação de uma boa parcela da sociedade nas atividades comerciais. Esta constatação, argumentam, é fundamental para entender o porquê da industrialização mais rápida dos EUA. As dotações das demais colônias, por sua vez, induziram uma distribuição mais desigual da riqueza, da renda, do capital humano e do poder político aliado às instituições que preservaram suas elites. Em conjunto, as configurações dessas colônias inibiram a difusão das atividades comerciais entre a população.
 
 Dito isso, os autores prosseguem para uma conceitualização dos tipos de colônias do Novo Mundo. A primeira categoria diz respeito às colônias que possuíram melhores condições mais favoráveis para a produção de \textit{commodities} com economias de escala e com o uso intensivo de escravos. Tais condicionantes são combinados com a desigualdade intrínseca da escravidão, preservação dos privilégios das elites e restrição de oportunidades para uma parcela significativa da população. O segundo tipo inclui as colônias espanholas que são caracterizadas pela convivência com um número considerável de nativos com menor capital humano em contato com colonizadores europeus que expropriavam recursos e mantinham seus privilégios. Argumentam que estas colônias foram as primeiras à desenvolver estruturas econômicas em que se predominavam empresas de larga escala. A terceira categoria diz respeito às colônias da América do Norte em que os nativos não foram a principal fonte de mão de obra e não possuíam condições naturais apropriadas e, assim, haviam poucos incentivos para a produção especializada intensiva em trabalho escravo. Os descendentes europeus, por sua vez, possuíam níveis elevados de capital humano e maior similaridade de distribuição de renda e riqueza.
 
 \subsection*{O papel das instituições na persistência da desigualdade}
 
 Os autores iniciam a seção pontuando que por mais que as instituições podem ser vistas como exógenas no início da colonização europeia, não é mais o caso quando se analisa ao longo do tempo. Em outras palavras, as condições iniciais e o grau de desigualdade podem influenciar as direções que as instituições evoluem enquanto essas instituições afetam a evolução das dotações e da distribuição de riqueza, capital humano e político. Para tanto, lançam mão da hipótese de que as elites em sociedades mais desiguais conseguem melhor estabelecer estruturas legais que asseguram seus privilégios e poder político, contribuindo para a persistência de um maior grau de desigualdade. Em seguida, usam o exemplo das políticas de terras das colônias e pontuam o insucesso da Argentina e do Brasil. 
 
 Dito isso, apresentam o modelo empírico em que analisam quão amplamente as franquias foram estendida e que frações das respectivas populações realmente votaram nas eleições. Concluem que EUA e Canada apresentam resultados melhores --- apesar de restringir o voto tal como nas demais colônias, mas em um grau menor --- por retirar restrições baseadas na renda e alfabetização. Adiante, os autores destacam o crescimento da proporção da população com direito a voto se comparado com as demais colônias. Em seguida, argumentam que o grau de desigualdade é relevante para explicar o padrão de sufrágio desses países e que existem diferentes níveis de interação entre distribuição e capital humano. Adiante, passam a discutir escolarização --- do nível básico --- e alfabetização e seus efeitos sobre a distribuição, pontuam que EUA e Canada foram as economias com maior nível de escolaridade em relação às demais colônias.
 
\subsection*{Extensão da desigualdade e o \textit{timming} da industrialização}
 
 Em linhas gerais, os autores argumentam que as colônias que apresentavam melhores padrões de vida e distribuição menos desigual foram as que se industrializaram mais cedo e tiveram um crescimento maior no longo prazo. Tal conclusão decorre da análise das fontes e características do aumento da produtividade. Nos EUA e Canada, por exemplo, houve um crescimento mais horizontal que abarcou mais setores e influenciou mudanças organizacionais, métodos e foram estimulados pela expansão dos mercados. Além disso, a desigualdade também afeta as trajetórias de desenvolvimento institucional. Mais adiante, discutem o caso do Sul dos EUA em que apesar das características em comuns com as demais colônias, possuía características compartilhadas dos estados do Norte e esta sob decisões a nível federal.
 
\subsection*{Conclusão}
 
 Segue o trecho que amarra as discussões deste capítulo:
 
 \begin{quotation}
 	we highlight the relevance of substantial differences in the
 	degree of inequality in wealth, human capital and political power, and sug-
 	gest that these disparities in the extent of inequality were rooted in differences
 	in the initial factor endowments of the respective colonies. 
 \end{quotation}
 
 
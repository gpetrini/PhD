\section*{Keefer e Shirley: Formal versus informal institutions in economic development}

\subsection*{Introdução}

A partir de uma comparação de casos e da revisão da literatura, os autores iniciam o capítulo pontuando que políticas macroeconômicas não são suficientes para estimular o crescimento. Além disso, argumentam que apesar das evidências de que a garantia dos direitos de propriedade são fundamentais para o desenvolvimento, as configurações institucionais que promovem tal garantia não são claras.

\subsection*{Em que medida a garantia dos direitos de propriedade são relevantes para o desenvolvimento?}

Os autores abrem a seção discutindo a importância dos custos de transação e das instituições para o sistema econômico. Pontuam também que na ausência de direitos de propriedade garantidos, os investimentos passam a ser mais vulneráveis à expropriação (do governo) e aos comportamentos oportunistas. Em seguida, ressaltam a dificuldade de se construir mais estudos de casos empíricos dada ausência de informações sobre a seguridade dos contratos e dos direitos de propriedades dos países. Apesar destas dificuldades, elencam o surgimento de \textit{proxies}, dentre elas, os indicadores de risco país (ICRG). Os autores seguem para a discussão de algumas economias em particular e para a relevância de instrumentos monetários no desenvolvimento de intermediações financeiras e enquanto \textit{proxies} da implementação de contratos. Adiante, apresentam trabalhos que encontram evidências da correção entre investimento e contratos intensivos em papel moeda e seguem para a discussão de \textit{proxies} mais objetivas.

\subsection*{Reformas políticas e instituições}

Nessa seção, os autores argumentam que reformas com o objetivo de incentivar o crescimento --- como o consenso de Washington ---  foram centradas em políticas econômicas e pouca atenção foi dada à necessidade de mudança institucional. Argumentam também que a diferença de reações do investimento às reformas pró-mercado podem ser explicadas pelas instituições de cada país, em especial àquelas que dizem respeito à garantia dos direitos de propriedade. Resumidamente, países com menos proteções dos direitos de propriedade e políticas macroeconômicas imprudentes tiveram crescimento negativo enquanto  os países com instituições com qualidade superior e políticas macroeconômicas ruins cresceram muito mais do que as economias com a combinação inversa.

\subsection*{Quais são as instituições que protegem os direitos de propriedade e de contratos?}

Os autores retomam a conclusão anterior de que reformas institucionais são uma prioridade e são necessárias para garantir políticas econômicas exitosas. Dito isso, discutem quais instituições são relevantes e as conceituam. Em seguida, afirmam que pelas instituições informais podem ser reformadas com mais facilidade, é importante compreender quais instituições informais e formais se complementam e se substituem. Adiante, pontuam que as instituições informais possuem limitações uma vez que não estão disponíveis para todos e que não protegem contra os crimes. Sendo assim, instituições formais podem promover o desenvolvimento de forma que as informais não são capazes além destas últimas poderem piorarem a distribuição de renda.

\subsection*{China, Gana e instituições formais e informais na execução dos contratos}

Os autores comparam o fluxo de IDE entre China e Gana e argumentam que as instituições informais são insuficientes para explicar as diferenças. Em linhas gerais, afirmam que existem arranjos institucionais formais na China que estão ausentes em Gana. Também chamam atenção para a decentralização da autoridade política entre os países em que o sistema a nível federal Chines complementa as instituições informais. Em resumo, afirma que por mais que China e Gana possuam arranjos institucionais informais com graus de semelhança, não explicam o porquê Gana receber menos investimento externo.

\subsection{Instituições e os desafios para o desenvolvimento de políticas e para economistas}

Os autores retomam a pouca atenção dada às reformas institucionais e que os resultados lançam luz sobre a importância das instituições formais para o desenvolvimento. Em seguida pontuam as dificuldades de se atingir melhores garantias de propriedade por meio de instituições formais e que não existem \textit{benchmarks} para reformar instituições informais para atingir tais objetivos. Por fim, explicitam algumas questões que podem ser melhor exploradas e aprimorar a agenda iniciado por Coase: identificas as consequências de instituições fracas sobre o comportamento da firma; como instituições fracas podem ser aprimoradas e reduzir os custos de transação. Argumentam que um melhor entendimento destas questões podem acelerar as transformações institucionais e promover crescimento.
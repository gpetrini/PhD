\section*{Organizações, aprendizado e mudança institucional}

Neste capítulo, \autor discute a importância das organizações na indução da mudança institucional. Esta é uma das contribuições de North em relação às demais abordagens institucionalistas em que as organizações recebem pouca atenção. Para North, organizações são entidades desenvolvidas para maximizar alguns objetivos (riqueza, renda, etc) dadas as oportunidades da matriz institucional e, ao mesmo tempo, a alteram incrementalmente. Em outras palavras, instituições definem as organizações e seus objetivos uma vez que definem as oportunidades de maximização.

\subsection*{I}


Nesta seção, o autor discute os diferentes tipos de aprendizado (tácito e comunicável) e sua importância na mudanças institucional. Destaca ainda que o tipo de aprendizado e de habilidade dos membros de uma organização refletem os \textbf{incentivos} das restrições institucionais. Apresentados alguns exemplos, enfatiza os seguintes pontos:
\begin{itemize}
	\item Na ausência de direitos de propriedade, o tamanho do mercado é o principal determinante da taxa de crescimento das inovações e da mudança tecnológica;
	\item O desenvolvimento de uma estrutura de incentivos (leis de patentes, por exemplo) aumentam a taxa de retorno das inovações e induzem o desenvolvimento da indústria da invenção
	\item A relação entre a ciência pura e aplicada não é simples, mas o conhecimento aplicado é uma das fontes de crescimento do conhecimento puro
	\item O desenvolvimento de tecnologias ilustram características de \textit{path-dependence} na forma que tais tecnologias mudam.
\end{itemize}
Dito isso, o autor segue para a discussão da relação entre conhecimento e ideologia em que argumenta uma relação dupla, ou seja, o conhecimento altera a percepção do mundo enquanto esta última  determinam a busca pelo conhecimento.

\subsection*{II}

Em seguida, \autor integra os objetivos de \textbf{maximização das organizações} com o desenvolvimento do conhecimento, ambos sujeitos ao arranjo institucional. Argumenta que descobertas não ocorrem no vácuo e requerem o desenvolvimento de conhecimento tácito. Além disso, o tipo de informação e de conhecimento são consequências do contexto institucional, ou seja, não determinam apenas a estrutura de governança das organizações, mas também as margens de lucro associadas aos objetivos de maximização. Dessa forma, afirma North, é necessário entender o contexto institucional para compreender o tipo de demanda associados a diferentes formas de conhecimento e habilidades, ou seja, o arranjo institucional altera os incentivos para aquisição de conhecimento. Como consequência, a direção da aquisição de conhecimento é um fator decisivo para o desenvolvimento de longo-prazo enquanto as instituições determinam as oportunidades para as organizações que, por sua vez, alteram (incrementalmente) o arranjo institucional por:
\begin{itemize}
	\item Alterar a demanda por investimento em conhecimento;
	\item Interagir com a atividade econômica, conhecimento e arranjo institucional e;
	\item Alterar (incrementalmente) as restrições informais como resultado da busca pela maximização
\end{itemize}

\subsection*{III}

Nesta seção, o autor pontua que a maximização dos ganhos pelas organizações pode ocorrer tanto pela tomada de decisões dadas as restrições impostas pelos arranjos institucional quanto também pela tentativa de alterá-los. Dito isso, discute a interação entre economia e política. Argumenta que instituições com poder de barganha elevado irão utilizá-lo para atingir objetivos na medida que o \textit{pay-off} de fazê-lo por meio de destinação de recursos à mudança institucional é maior do que por meio das restrições existentes. Destaca também que as organizações irão incentivar o investimento em certas habilidades e conhecimentos que contribuam para a sua rentabilidade que, por sua vez, irá determinar o crescimento econômico de longo-prazo.


\subsection*{IV}

Nesta última seção, \autor reforça as implicações da interação entre organizações e instituições no desempenho econômico por meio de investimento em conhecimento e habilidades. Em seguida, discute diferentes conceitos de eficiência em que destaca a eficiência adaptativa por dar destaque as regras que determinam a evolução da economia ao longo do tempo. Apesar dos determinantes deste tipo de eficiência não estarem claros, pontua que o arranjo institucional desempenha um papel central na forma que a economia incentiva tentativas, experimentos e inovações. Por fim, encerra ressaltando que eficiência alocativa e adaptativa nem sempre são \textbf{consistentes} entre si.
\section*{Estabilidade e mudança institucional}

\autor abre o capítulo pontuando que o ``empresário'' (econômico, político, etc) é o agente da mudança institucional uma vez que responde à \textbf{incetivos} do arranjo institucional. A fonte dessas mudanças são alterações nos \textbf{preços relativos e das preferências}. Estas mudanças, por fim, são \textbf{incrementais}. Além disso, pontua que a estabilidade do arranjo institucional não garante que seja eficiente, no entanto, sua estabilidade é uma \textbf{condição necessária} para as interações humanas, mas não é condição \textbf{suficiente} para sua eficiência.

\subsection*{I}

Nesta seção, o autor pontua que as mudanças nos preços relativos são a principal fonte de mudanças por alterar os incentivos e as preferências. Argumenta ainda que algumas das mudanças desses preços relativos são exógenas, mas são endógenas em sua maioria, refletindo as tentativas de maximização dos ``empresários''. O processo de aquisição de conhecimento também alteram os preços relativos enquanto mudanças do poder de barganha induzem uma reestruturação de contratos. Dito isso, pontua que mudanças institucionais (sobretudo o processo eleitoral) permitem que os agentes expressem suas preferências e ideologias a um custo menor para eles mesmos, ou seja, mudanças institucionais alteraram o preço relativos das convicções que, por sua vez, promovem mudanças institucionais.

\subsection*{II}

Na seção seguinte, \autor discute em que medida mudanças nos preços relativos induzem mudanças institucionais ou são reflexo de mudanças dentro de um mesmo arranjo institucional por meio do conceito de \textbf{equilíbrio institucional} definido como segue:

\begin{quotation}
	 Institutional equilibrium would be a siruation where
	given the bargaining strength of the players and the set of contractual
	bargains that made up total economic exchange, none of the players
	would find it advantageous to devote resources into restructuring the
	agreements. Note that such a situation does not imply that everyone is
	happy with the existing rules and contracts, but only that the relative
	costs and benefits of altering the game among the contracting parties does
	not make it worthwhile to do so.
\end{quotation}
com este conceito em mãos, descreve o processo de mudanças institucional da seguinte maneira:
\begin{quotation}
	A
	change in relative prices leads one or both parties to an exchange,
	whether it is political or economic, to perceive that either or both could
	do better with an altered agreement or contract. An attempt will be made
	to renegotiate the contract. However, because contracts are nested in a
	hierarchy of rules, the renegotiation may not be possible without restruc­
	turing a higher set of rules (or violating some norm of behavior). In that
	case, the party that stands to improve his or her bargaining position may
	very well attempt to devote resources to restructuring the rules at a higher
	level. In the case of a notm of behavior, a change in relative prices or a
	change in tastes will lead to its gradual erosion and to its replacement by
	a different norm.
\end{quotation}
Dito isso, os ``empresários'' irão responder às mudanças nos preços relativos diretamente ao direcionar esforços para as novas oportunidades que surgem ou, indiretamente, por estimar os custos e benefícios de alterar a execução e as regras vigentes. Em seguida, discute as mudanças das instituições informais em que afirma que se alteram a uma velocidade diferente (e talvez menor) que as formais.

\subsection*{III e IV}

Por fim, nestas seções, o autor discute as mudanças descontínuas, mas com a ressalva de que são incrementais em sua maioria. Por descontinuidades, \autor se refere à mudanças radicais e revoluções em que destaca:

\begin{enumerate}
	\item A chave para mudanças incrementais é o contexto institucional que torna possível novas configurações de barganha e compromissos entre as partes
	\item A incapacidade de atingir acordos pode refletir não apenas a ausência de instituições mediadoras, mas também graus de liberdade restritos de modo que o conjunto de escolhas possíveis pode não conter a intersecção do interesse das partes.
	\item O apoio de uma abordagem mais violenta requer a superação do problema do caronista
	\item O sucesso de revoluções tendem a ter vida curta. Além disso, o apoio ideológico e massivo de um revolução necessário não é sustentável.
\end{enumerate}

Por fim, o autor pontua que por mais que as instituições \textbf{formais} formais podem se alterar com resultado de uma revolução, o mesmo não pode ser dito das instituições \textbf{informais}. Como consequência, há uma tensão entre as instituições informais e as novas regras formais e podem até ser inconsistentes entre si. O resultado tende a ser um novo equilíbrio menos revolucionário.

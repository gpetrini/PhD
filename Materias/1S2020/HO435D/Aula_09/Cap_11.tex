\section*{Trajetória das mudanças institucionais}

Neste capítulo, \autor discute os determinantes da convergência e divergência das trajetórias institucionais das sociedades. Em especial, analisa o porquê da persistência de arranjos institucionais ineficientes (em termos de crescimento econômico). Argumenta que na ausência de custos de transação, a trajetória histórica não é relevante para explicar a trajetória institucional.

\subsection*{I}

Nesta seção, \autor discute o conceito de \textbf{path-dependence} a luz das contribuições de Arthur e P. David. Deste conceito, o autor destaca a importância de rendimentos crescentes; efeitos de aprendizado e de coordenação; custos fixos e expectativas adaptativas sob o prisma da análise institucional.

\subsection*{II}

Na seção seguinte, o autor pontua que apesar de Arthur e David destacarem a relevância dos rendimentos crescentes para explicar trajetórias tecnológicas, não dão muita atenção aos custos de transação. No entanto, volta a ênfase aos rendimentos crescentes que, argumenta, são fundamentais para entender as trajetórias de longo prazo das economias. Além disso, pontua os \textbf{modelos subjetivos} dos agentes econômicos e como a racionalidade limitada (dificultando como decifrar o ambiente complexo) e as convicções determinam suas reações às mudanças.

\subsection*{III}

\autor abre a seção discutindo \textbf{lei comum} e como esta instituição formal ajuda a compreender a mudança institucional. Apesar da adoção deste tipo de lei em um conjunto de sociedades, não é possível afirmar que são eficientes dado que a decisão judicial se dá em meio a informação incompleta. Adiante, o autor apresenta alguns exemplos cuja mensagem é que tais mudanças institucionais moldaram a trajetória dos EUA, evidenciando o caráter de \textit{path dependence}. Dito isso, o autor explicita que \textit{path dependence} não implica em determinismo histórico. Dito isso, \autor conecta este conceito com redes de externalidades; processos de aprendizagem das organizações e modelos subjetivos históricos e como estes determinam a trajetória. Em seguida, pontua que trajetórias ineficientes podem persistir. Além disso, destaca que tentativas de maximizar objetivos a curto prazo podem estar associados à manutenção de arranjos institucionais ineficientes e ter consequências indesejadas. Em resumo, destaca que \textit{path dependence} implica a relevância histórica.

\subsection*{IV}

Com estes novos conceitos em mãos, o autor discute como mudanças nos preços relativos afetam economias de formas diferentes ao longo do tempo. Em linhas gerais, tais mudanças induzem alterações incrementais no arranjo institucional que, por sua vez, estão sujeitas ao poder de barganha das partes que geram mudanças persistentes que não levam necessariamente à convergência destas economias.

\subsection*{VI}

Por fim, o autor fecha o capítulo comparando mudanças tecnológicas com as institucionais. Em linhas gerais, pontua que rendimentos crescentes é comum à ambas e que a percepção dos agentes desempenha um papel fundamental na mudanças institucional do que na mudanças tecnológica dado a relevância das convicções no primeiro. Além disso, destaca que as escolhas são mais multidimensionais na mudanças institucional dada a relação complexa entre instituições formais e informais. Como consequência, o \textit{lock-in} é maior na trajetória institucional do que na tecnológica.
\section{Hipóteses comportamentais em uma teoria das instituições}

\autor abre o capítulo discutindo os problemas associados às hipóteses de racionalidade substantiva em que os agentes econômicos têm e capacidade cognitiva e procedural de tomar as melhores decisões possíveis.

\subsection{I}

O autor inicia a seção se questionando quais são as hipóteses comportamentais compatíveis com um mundo em que as instituições são irrelevantes e conclui que a racionalidade substantiva só faz sentido nesses extremos. Adiante, apresenta algumas evidências empíricas que contestam tais hipóteses e discute outras hipóteses associadas à essa. Em particular, destaca a hipótese ``evolucionária'' em que a competição elimina os comportamentos não racionais de modo que os agentes que prevalecem são aqueles que adotaram as decisões de acordo com a teoria neoclássica.

\subsection{II}

Para explicitar as limitações da hipótese de racionalidade substantiva, \autor destaca as motivações, bem como a capacidade de ``decifrar o ambiente'', ou seja, o comportamento humano é mais complexo do que a otimização de uma função utilidade. Adiante, apresenta formas alternativas de se caracterizar o comportamento humano com destaque para reputação, confiança e altruísmo que, por sua vez, não são consistentes com a maximização da riqueza individual. Em seguida, argumenta que as instituições alteram os preços que os indivíduos pagam e suas escolhas.

\subsection{III}

Nesta seção, \autor destaca a importância das instituições para reduzir as incertezas e auxiliar os agentes a decifrarem o ambiente econômico que o cercam. Em especial, pontua que o sistema de trocas é um exemplo da internalização de uma instituição. Em seguida, explicita a relevância de Simon ao propor a racionalidade limitada como um contraponto à racionalidade substantiva. Em linhas gerais, a centralidade da racionalidade limitada para North se dá pelo relevância das informações subjetivas e incompletas no processo de decisão. Além disso, tal hipótese comportamental é consistente com a \textbf{formação das instituições}.

\subsection{IV}

Nesta seção, North contesta as hipóteses associadas à racionalidade substantiva ponto a ponto:
\begin{itemize}
	\item Não existem um único equilíbrio, mas sim equilíbrios múltiplos;
	\item Os agentes enfrentam situações que não são únicas ou repetitivas em que as informações são incompletas e os resultados incertos;
	\item Preferências mudam ao longo do tempo;
	\item Por mais que os agentes possam tentar melhor seus resultados, os \textit{feedbacks} de informação são insuficientes para corrigir seus erros;
	\item Competição não é suficiente para guiar os agentes evolucionariamente;
	\item Comportamentos podem ser mais complexos que uma postura não-cooperativa otimizadora
	\item \textbf{As hipóteses comportamentais são o principal impedimento para se compreender a existência, a formação e evolução das instituições}.
\end{itemize}

\subsection{V}

O autor retoma a noção de que as instituições reduzem as incertezas e que tal característica é central em um mundo com racionalidade limitada. No entanto, isso não implica que as instituições são \textbf{eficientes}. Além disso, destaca que as instituições afetam os \textbf{preços das convicções dos agentes econômicos} e, assim, são fundamentais para influenciar o processo decisório.
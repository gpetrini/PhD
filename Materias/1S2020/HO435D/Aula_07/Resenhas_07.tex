\documentclass[11pt,lineno]{../style}



\newcommand{\autor}{North (1990) }

\title{
\large{Nova Economia Institucional}\vspace{2pt}\\
\Huge{\autor - \textit{Institutions, Institutional change and Economic Performance}}
}
\date{22 de Maio de 2020}

\author[$\ast$]{Gabriel Petrini}

\affil[$\ast$]{PhD Student at Unicamp.}

\keywords{
	Keyword1\\ 
	Keyword2\\
	Keyword3
}

\runningtitle{Resenha} % For use in the footer 

%% For the footnote.
\runningauthor{Petrini}

\begin{abstract}

\end{abstract}

\begin{document}

\maketitle
\marginmark
\thispagestyle{firststyle}
% Please add here a significance statement to explain the relevance of your work

\section{Introdução às instituições e à mudança institucional}

\autor abre o capítulo definindo instituições como as regras do jogo, ou ainda, como restrições humanamente concebidas que moldam a interação humana. Em seguida afirma que mudanças institucionais determinam a forma que as sociedade evoluem ao longo do tempo e, portanto, são centrais para compreender a mudança histórica. Além disso, pontua que não existe um constructo teórico que integra uma análise institucional na teoria e na história econômica.

\subsection{I}

De modo geral, as instituições diminuem a incerteza ao prover uma estrutura a vida cotidiana e por guiar as interações humanas. Em seguida, categoria instituições em formais e informais em que regras são exemplos das primeira enquanto convenções e códigos de conduta são exemplos da segunda. Também destaca que as instituições podem ser criadas e evoluir ao longo do tempo. Pontua a dimensão negativa (restrição de ações) quanto positiva (permissividades) das instituições, ou seja, envolvem as interações humanas. Adiante, diferencia instituições de \textbf{organizações} em que as segundas surgem como consequência do --- e influenciam o --- arranjo institucional e incluem grupos políticos, econômicos, sociais e educacionais em que tais agrupamentos possuem objetivos em comum.
Outra distinção relevante é entre agentes e as regras/normas (jogadores das regras).

Adiante, \autor discute que as instituições são criadas e alteradas pelos humanos de modo a teoria institucional que propõe deve partir do \textbf{nível individual}.
Também pontua que as instituições determinam o desempenho econômico ao afetar os custos de produção.

\subsection{II}

Mais uma vez, o autor retoma que instituições reduzem as incertezas de uma sociedade ao promover uma estrutura estável às interações humanas, mas tal estabilidade não implica imutabilidade. Argumenta que as mudanças institucionais são complexas uma vez que podem ser consequências de mudanças formais, informais e nas formas de execução. Pontuam que as instituições informais estão menos sujeitas à ações políticas do que as formais. Dito isso, direciona a discussão para as condições que levam a uma maior convergência ou divergência das sociedades\footnote{Afirma que abandonou a explicação via incentivos de preços antes defendida em outro livro.}. Defende que a resposta a esse questão se dá pela interação entre instituições e organizações assim como na determinação de oportunidades associadas ao arranjo institucional em que as organizações são criadas para extrair vantagens dessas oportunidades e, na medida que se altera, modificam as instituições. 

Argumenta que a trajetória da mudança institucional é determinado pelo \textit{lock-in} derivado da relação entre instituições e organizações e pelos \textit{feedbacks} dos agentes. Destaca ainda que as mudanças institucionais (incrementais) decorrem da percepção dos agentes das organizações políticas e econômicas podem ser favorecidos ao alterar o arranjo institucional na margem. No entanto, esta percepção depende da obtenção e processamento dessas informações. Em seguida, destaca que custos de transação (políticos e econômicos) tornam os \textbf{direitos de propriedade} ineficientes, mas a racionalidade limitada dos agentes tornam tais direitos de propriedade persistentes.

\autor afirma também que as mudanças institucionais podem tanto melhorar quanto piorar o bem-estar econômico. À luz disso, defende que o sucesso da economia norte-americana decorre pelo arranjo institucional gerar --- em média --- efeitos positivos sobre a atividade produtiva (apesar de suas consequências adversas). Ao mesmo tempo, afirma que o insucesso dos países do ``Terceiro Mundo'' decorrem do favorecimento de atividades redistributivas invés de produtivas, restringindo oportunidades invés de ampliá-las uma vez que as organizações oriundas deste arranjo institucional são mais eficientes em tornar esta sociedade mais improdutiva. Argumenta que esta trajetória se torna persistente por conta dos custos tando do mercado econômico quanto do político que se soma aos modelos subjetivos dos agentes de modo que não mudam em direção mudanças (incrementais) mais eficientes.
\section*{Capítulo 02: Homem contratual}

Alguns dos temas resenhados deste capítulo já foram tratados na resenha entregue anteriormente. Por conta disso, será feita uma resenha mais direta, retomando a conexão entre os conceitos e pontuando as diferenças com o capítulo de Farina el all (1997). Dito isso, neste capítulo, Williamson analisa as diferentes abordagens dos contratos no que diz respeito à: (1) hipótese comportamental; (2) atributos das transações e; (3) o grau as disputas são resolvidas internamente ou esfera jurídica. Os dois primeiros temas são tratados neste capítulo enquanto o capítulo seguinte avança no terceiro.

\subsection*{Hipóteses comportamentais}

De modo geral, economistas partem de hipóteses comportamentais por conveniência e simplificação enquanto a ECT parte da racionalidade limitada e comportamento oportunista.

\subsubsection*{Racionalidade}

\begin{description}
	\item[Maximizadora:] Racionalidade forte em que todos os custos relevantes são identificados. Instituições dão lugar à funções-objetivo.
	\item[Limitada:] Racionalidade \textbf{intencional} e limitada em que a captura e processamento de informação são escassos. Como consequência, se faz necessário estudar organizações que vão além dos mercados.
	\item[Orgânica:] Presente na literatura austríaca e neo-Schumpeteriana. A primeira enfatiza processos gerais e instituições ``não planejadas''. A segunda vertente trata de processos evolucionários dentro e ente firmas
\end{description}


\subsubsection*{Autointeresse}

Os diferentes tipos de comportamento autocentrado são

\begin{description}
	\item[Oportunismo:] Refere-se à distorção das informações, especialmente para esforços calculados para enganar, distorcer disfarces, ofuscar ou confundir. Em linhas gerais, transações que estão sujeitas ao oportunismo \textit{ex post} se beneficiarão de salvaguardas elaboradas \textit{ex ante}. Além disso, oportunismo está associado com \textbf{incerteza} das transações econômicas
	\item[Oportunismo simples] Já descrito em outra resenha
	\item[Obediência] Já descrito em outra resenha
\end{description}


\subsubsection*{Comparações}

A tabela abaixo permite uma comparação entre as diferentes abordagens de acordo com as hipóteses comportamentais. Uma vez que é autoexplicativa, não será analisada em maiores detalhes e serão elencados pontos referentes à ECT e não às demais:

\begin{figure}[H]
	\centering
	\includegraphics[width=0.7\linewidth]{screenshot004}
	\caption{Hipóteses comportamentais de abordagens organizacionais alternativas}
	\label{fig:screenshot004}
\end{figure}

\subsection*{Dimensões}

\subsubsection*{Especificidade dos ativos}

Williamson destaca que apesar da especificidade dos ativos já ter sido apresentada em trabalhos anteriores, eram considerados insignificantes e, por conta disso, eram desconsiderados. Argumenta que a especificidade dos ativos surgem em um contexto \textbf{intertemporal}. Podem ser caracterizados como: são ativos que não podem ser empregados em funções as quais não foram designados sem perdas significativas de valor, ou ainda, seu valor se reduz se a \textbf{transação} a qual lhe é específico é interrompida. Ao considerar a especificidade dos ativos, surgem \textit{trade-offs} que precisam ser considerados que, por sua vez, são particulares da estrutural de governança. Para esclarecer algumas diferentes entre conceitos contratuais e contábeis, o autor apresenta o esquema abaixo em que diferencia custos associados à especificidade dos ativos ($v$) dos custos fixos ($F$) e variáveis ($V$):

\begin{figure}[H]
	\centering
	\caption{Distinção dos custos}
	\includegraphics[width=0.7\linewidth]{screenshot005}
	\label{fig:screenshot005}
\end{figure}

Em seguida, o autor pontua que a \textbf{transformação fundamenta} decorre da especificidade dos ativos que, por sua vez, podem ser distinguidos em:
\begin{itemize}
	\item especificidade física
	\item especificidade humana
	\item ativos dedicados
\end{itemize}
Além disso, destaca:
\begin{itemize}
	\item Especificidade dos ativos diz respeito à investimento duráveis associados à transações particulares, em que o custo de oportunidade do investimento é menor que o uso alternativo dos recursos
	\item Identidade das partes é relevante para a continuidade da transação
	\item Salvaguardas organizacionais e contratuais surgem para dar conta deste tipo de transação
	\item A importância da especificidade dos ativos deve ser considerada em conjunto com a racionalidade limitada e comportamento oportunista
	\item A análise acentuada de formas organizacionais convencionais se deve à desconsideração da especificidade  dos ativos
\end{itemize}

\subsubsection*{Incerteza}

Além das discussões sobre incerteza, o autor destaca que quanto maior o grau de incerteza, mais impositivo é a especificidade dos ativos.

\subsubsection*{Frequência}

Resumidamente, estruturas de governança especializadas são mais sensíveis à transações específicas do que estruturas de governança de propósito mais geral. No entanto, estruturas de governança específicas são custosas e o que deve ser avaliado é quando estes custos são justificados. Em linhas gerais, os benefícios de estruturas de governança específicas são maiores quão mais frequente a transação é quanto mais específicos são os ativos que ela diz respeito. Desse modo, o objetivo da estrutura de governança é reduzir os custos de transformação e de transação, não apenas um deles.

\subsection*{Transformação fundamental}

Como destacado anteriormente, a ECT reconhece a importância do estabelecimento contratual \textit{ex ante}, mas dá maior ênfase à execução (relação contratual\textit{ex post}). Além disso, pontua que relações contratuais mais numerosas não implicam necessariamente em maior eficiência, ou melhor, que tais relações contratuais irão prevalecer. Em resumo, a identidade contratual é relevante para a continuidade das relações entre as partes. Nesses termos, define a transformação fundamental nos seguintes termos:

\begin{description}
	\item[Definição:] É definido como a transformação de um conjunto numeroso de contratantes em uma relação bilateral
\end{description}
Argumenta que esta transformação tem implicações importantes como descrito abaixo (p.~63, grifos adicionados):

\begin{quotation}
	\textit{
	Joined as they are in a condition of \textbf{bilateral monopoly}, both buyerand seller are strategically situated to bargain over the disposition of any incremental gain whenever a proposal to adapt is made by the other party. [...]
	Efficient adaptations that would otherwise be made thus result in costly haggling or even go unmentioned, lest the gains be dissipated by costly	subgoal pursuit. Governancestructures that attentuate opportunism and otherwise infuse confidence are evidently needed.
	}
\end{quotation}
\section{Hipóteses comportamentais em uma teoria das instituições}

\autor abre o capítulo discutindo os problemas associados às hipóteses de racionalidade substantiva em que os agentes econômicos têm e capacidade cognitiva e procedural de tomar as melhores decisões possíveis.

\subsection{I}

O autor inicia a seção se questionando quais são as hipóteses comportamentais compatíveis com um mundo em que as instituições são irrelevantes e conclui que a racionalidade substantiva só faz sentido nesses extremos. Adiante, apresenta algumas evidências empíricas que contestam tais hipóteses e discute outras hipóteses associadas à essa. Em particular, destaca a hipótese ``evolucionária'' em que a competição elimina os comportamentos não racionais de modo que os agentes que prevalecem são aqueles que adotaram as decisões de acordo com a teoria neoclássica.

\subsection{II}

Para explicitar as limitações da hipótese de racionalidade substantiva, \autor destaca as motivações, bem como a capacidade de ``decifrar o ambiente'', ou seja, o comportamento humano é mais complexo do que a otimização de uma função utilidade. Adiante, apresenta formas alternativas de se caracterizar o comportamento humano com destaque para reputação, confiança e altruísmo que, por sua vez, não são consistentes com a maximização da riqueza individual. Em seguida, argumenta que as instituições alteram os preços que os indivíduos pagam e suas escolhas.

\subsection{III}

Nesta seção, \autor destaca a importância das instituições para reduzir as incertezas e auxiliar os agentes a decifrarem o ambiente econômico que o cercam. Em especial, pontua que o sistema de trocas é um exemplo da internalização de uma instituição. Em seguida, explicita a relevância de Simon ao propor a racionalidade limitada como um contraponto à racionalidade substantiva. Em linhas gerais, a centralidade da racionalidade limitada para North se dá pelo relevância das informações subjetivas e incompletas no processo de decisão. Além disso, tal hipótese comportamental é consistente com a \textbf{formação das instituições}.

\subsection{IV}

Nesta seção, North contesta as hipóteses associadas à racionalidade substantiva ponto a ponto:
\begin{itemize}
	\item Não existem um único equilíbrio, mas sim equilíbrios múltiplos;
	\item Os agentes enfrentam situações que não são únicas ou repetitivas em que as informações são incompletas e os resultados incertos;
	\item Preferências mudam ao longo do tempo;
	\item Por mais que os agentes possam tentar melhor seus resultados, os \textit{feedbacks} de informação são insuficientes para corrigir seus erros;
	\item Competição não é suficiente para guiar os agentes evolucionariamente;
	\item Comportamentos podem ser mais complexos que uma postura não-cooperativa otimizadora
	\item \textbf{As hipóteses comportamentais são o principal impedimento para se compreender a existência, a formação e evolução das instituições}.
\end{itemize}

\subsection{V}

O autor retoma a noção de que as instituições reduzem as incertezas e que tal característica é central em um mundo com racionalidade limitada. No entanto, isso não implica que as instituições são \textbf{eficientes}. Além disso, destaca que as instituições afetam os \textbf{preços das convicções dos agentes econômicos} e, assim, são fundamentais para influenciar o processo decisório.
\section{Teoria de troca com custos de transação}

Neste capítulo, \autor adiciona uma \textbf{teoria da produção} às hipóteses comportamentais discutidas anteriormente. Por custos de produção, considera tanto os custos de transformação quanto os de transação. O custo por se obter \textbf{informações} é o principal elemento dos custos de transação e consiste na \textbf{mensuração dos valores dos atributos} dos ativos e inclui o que está sendo transacionado; os custos de proteção dos direitos de propriedade e policiamento e; execução contratual. Em linhas gerais, argumenta que a mensuração e os custos de execução são os determinantes sociais, políticos e econômicos das instituições.

\subsection{I}

\autor pontua que os autores da teoria dos CT não têm se debruçado a entender o que tornais tais custos tão \textbf{elevados} e que esta é uma questão central em sua teoria. Resumidamente, uma troca gera custos associados à tentativa de mensurar o valor dos atributos desses ativos. Dentre os elementos que explicam o porquê que tais custos de mensuração são elevados, desta a assimetria de informação.

\subsection{II}

O autor inicia esta seção discutindo o modelo Walrasiano básico em que preços são mecanismos alocativos suficientes para determinar os valores de uso. A esse modelo, o autor soma os custos de informação e, assim, os ganhos líquidos devem considerar os custos de mensuração e de policiamento dos acordos. Uma vez que a mensuração de tais atributos é custosa, a renda capturada por meio da obtenção de informação se torna mais importante. Além disso, a maximização do valor de um ativo envolve uma estrutura de propriedade na qual as partes que podem influenciar a variabilidade de atributos específicos tornam-se requerentes residuais sobre esses atributos.

\subsection{III}

Uma vez que não se sabe os atributos de um bem ou serviço, assim como o desempenho de uma das partes e porque é necessário destinar recursos para obter e medir e monitorar tais características é que surgem questões de execução. Tal execução pode vir tanto da retaliação de uma das partes como também dos códigos de conduta internos ou de um terceiro ente de forma coercitiva (Estado). De todo modo, a execução não pode ser tomada como garantida e a dificuldade aumenta na medida que a divisão do trabalho também aumenta. Em seguida, afirma:

\begin{quotation}
	But without institutional con­straints, self-interested behavior will foreclose complex exchange, be­cause of the uncertainty that the other party will find it in his or her interest to live up to the agreement. The transaction cost will reflect the uncenainty by including a risk premium, the magnitude of which will turn on the likelihood of defection by the other party and the consequent cost to the first party. 
\end{quotation}

\subsection{IV}

\autor argumenta que a apropriação --- que é importante para a definição dos direitos de propriedade --- é função das regras legais, das formas organizacionais, execução e normas comportamentais. Em resumo, a apropriação depende do arranjo institucional. Uma vez que os custos de transação existem na presença de direitos de propriedade, tais direitos nunca são perfeitamente especificados ou executados. Além disso, pelos custos de transação terem mudado consideravelmente ao longo do tempo, as configurações de formas de proteção ou de captura dos direitos de propriedade também variam enormemente. Sendo assim, as instituições determinam a estrutura das trocas que, por sua vez, determinam os custos de produção (transformação e transação). As instituições são mais eficazes em resolver questões de coordenação e produção a depender da motivação dos agentes, a complexidade do ambiente e a capacidade dos agentes de decifrá-la. Dito isso, o autor pontua que os custos de transação aumentam com o grau de complexidade da economia, ou seja, são menores em uma economia de pequena escala e de transação local e conclui:

\begin{quotation}
	Thus, it should be readily apparent that to devcJop a model
	of institu­tions, we must explore in depth the structural characteristics of informal constraints, formal rules, and enforcement and the way in which they evolve. Then we shall be in a position to put them together to look at the overall institutional makeup of political/economic orders.
	
\end{quotation}

	
\begin{redbox}{Dúvidas e comentários}	
	Seria a teoria institucional de North uma ``Teoria Geral das Instituições'' em que a teoria neoclássica é um caso especial? Em outras palavras, North tenta estender/compatibilizar suas contribuições com a teoria \textit{mainstream} ou romper com ela? \textbf{Comentário:} Williamson me pareceu menos preocupado com essa compatibilização com a teoria-padrão.
\end{redbox}

\end{document}

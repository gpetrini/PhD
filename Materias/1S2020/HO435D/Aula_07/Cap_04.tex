\section{Teoria de troca com custos de transação}

Neste capítulo, \autor adiciona uma \textbf{teoria da produção} às hipóteses comportamentais discutidas anteriormente. Por custos de produção, considera tanto os custos de transformação quanto os de transação. O custo por se obter \textbf{informações} é o principal elemento dos custos de transação e consiste na \textbf{mensuração dos valores dos atributos} dos ativos e inclui o que está sendo transacionado; os custos de proteção dos direitos de propriedade e policiamento e; execução contratual. Em linhas gerais, argumenta que a mensuração e os custos de execução são os determinantes sociais, políticos e econômicos das instituições.

\subsection{I}

\autor pontua que os autores da teoria dos CT não têm se debruçado a entender o que tornais tais custos tão \textbf{elevados} e que esta é uma questão central em sua teoria. Resumidamente, uma troca gera custos associados à tentativa de mensurar o valor dos atributos desses ativos. Dentre os elementos que explicam o porquê que tais custos de mensuração são elevados, desta a assimetria de informação.

\subsection{II}

O autor inicia esta seção discutindo o modelo Walrasiano básico em que preços são mecanismos alocativos suficientes para determinar os valores de uso. A esse modelo, o autor soma os custos de informação e, assim, os ganhos líquidos devem considerar os custos de mensuração e de policiamento dos acordos. Uma vez que a mensuração de tais atributos é custosa, a renda capturada por meio da obtenção de informação se torna mais importante. Além disso, a maximização do valor de um ativo envolve uma estrutura de propriedade na qual as partes que podem influenciar a variabilidade de atributos específicos tornam-se requerentes residuais sobre esses atributos.

\subsection{III}

Uma vez que não se sabe os atributos de um bem ou serviço, assim como o desempenho de uma das partes e porque é necessário destinar recursos para obter e medir e monitorar tais características é que surgem questões de execução. Tal execução pode vir tanto da retaliação de uma das partes como também dos códigos de conduta internos ou de um terceiro ente de forma coercitiva (Estado). De todo modo, a execução não pode ser tomada como garantida e a dificuldade aumenta na medida que a divisão do trabalho também aumenta. Em seguida, afirma:

\begin{quotation}
	But without institutional con­straints, self-interested behavior will foreclose complex exchange, be­cause of the uncertainty that the other party will find it in his or her interest to live up to the agreement. The transaction cost will reflect the uncenainty by including a risk premium, the magnitude of which will turn on the likelihood of defection by the other party and the consequent cost to the first party. 
\end{quotation}

\subsection{IV}

\autor argumenta que a apropriação --- que é importante para a definição dos direitos de propriedade --- é função das regras legais, das formas organizacionais, execução e normas comportamentais. Em resumo, a apropriação depende do arranjo institucional. Uma vez que os custos de transação existem na presença de direitos de propriedade, tais direitos nunca são perfeitamente especificados ou executados. Além disso, pelos custos de transação terem mudado consideravelmente ao longo do tempo, as configurações de formas de proteção ou de captura dos direitos de propriedade também variam enormemente. Sendo assim, as instituições determinam a estrutura das trocas que, por sua vez, determinam os custos de produção (transformação e transação). As instituições são mais eficazes em resolver questões de coordenação e produção a depender da motivação dos agentes, a complexidade do ambiente e a capacidade dos agentes de decifrá-la. Dito isso, o autor pontua que os custos de transação aumentam com o grau de complexidade da economia, ou seja, são menores em uma economia de pequena escala e de transação local e conclui:

\begin{quotation}
	Thus, it should be readily apparent that to devcJop a model
	of institu­tions, we must explore in depth the structural characteristics of informal constraints, formal rules, and enforcement and the way in which they evolve. Then we shall be in a position to put them together to look at the overall institutional makeup of political/economic orders.
	
\end{quotation}
\section{Introdução às instituições e à mudança institucional}

\autor abre o capítulo definindo instituições como as regras do jogo, ou ainda, como restrições humanamente concebidas que moldam a interação humana. Em seguida afirma que mudanças institucionais determinam a forma que as sociedade evoluem ao longo do tempo e, portanto, são centrais para compreender a mudança histórica. Além disso, pontua que não existe um constructo teórico que integra uma análise institucional na teoria e na história econômica.

\subsection{I}

De modo geral, as instituições diminuem a incerteza ao prover uma estrutura a vida cotidiana e por guiar as interações humanas. Em seguida, categoria instituições em formais e informais em que regras são exemplos das primeira enquanto convenções e códigos de conduta são exemplos da segunda. Também destaca que as instituições podem ser criadas e evoluir ao longo do tempo. Pontua a dimensão negativa (restrição de ações) quanto positiva (permissividades) das instituições, ou seja, envolvem as interações humanas. Adiante, diferencia instituições de \textbf{organizações} em que as segundas surgem como consequência do --- e influenciam o --- arranjo institucional e incluem grupos políticos, econômicos, sociais e educacionais em que tais agrupamentos possuem objetivos em comum.
Outra distinção relevante é entre agentes e as regras/normas (jogadores das regras).

Adiante, \autor discute que as instituições são criadas e alteradas pelos humanos de modo a teoria institucional que propõe deve partir do \textbf{nível individual}.
Também pontua que as instituições determinam o desempenho econômico ao afetar os custos de produção.

\subsection{II}

Mais uma vez, o autor retoma que instituições reduzem as incertezas de uma sociedade ao promover uma estrutura estável às interações humanas, mas tal estabilidade não implica imutabilidade. Argumenta que as mudanças institucionais são complexas uma vez que podem ser consequências de mudanças formais, informais e nas formas de execução. Pontuam que as instituições informais estão menos sujeitas à ações políticas do que as formais. Dito isso, direciona a discussão para as condições que levam a uma maior convergência ou divergência das sociedades\footnote{Afirma que abandonou a explicação via incentivos de preços antes defendida em outro livro.}. Defende que a resposta a esse questão se dá pela interação entre instituições e organizações assim como na determinação de oportunidades associadas ao arranjo institucional em que as organizações são criadas para extrair vantagens dessas oportunidades e, na medida que se altera, modificam as instituições. 

Argumenta que a trajetória da mudança institucional é determinado pelo \textit{lock-in} derivado da relação entre instituições e organizações e pelos \textit{feedbacks} dos agentes. Destaca ainda que as mudanças institucionais (incrementais) decorrem da percepção dos agentes das organizações políticas e econômicas podem ser favorecidos ao alterar o arranjo institucional na margem. No entanto, esta percepção depende da obtenção e processamento dessas informações. Em seguida, destaca que custos de transação (políticos e econômicos) tornam os \textbf{direitos de propriedade} ineficientes, mas a racionalidade limitada dos agentes tornam tais direitos de propriedade persistentes.

\autor afirma também que as mudanças institucionais podem tanto melhorar quanto piorar o bem-estar econômico. À luz disso, defende que o sucesso da economia norte-americana decorre pelo arranjo institucional gerar --- em média --- efeitos positivos sobre a atividade produtiva (apesar de suas consequências adversas). Ao mesmo tempo, afirma que o insucesso dos países do ``Terceiro Mundo'' decorrem do favorecimento de atividades redistributivas invés de produtivas, restringindo oportunidades invés de ampliá-las uma vez que as organizações oriundas deste arranjo institucional são mais eficientes em tornar esta sociedade mais improdutiva. Argumenta que esta trajetória se torna persistente por conta dos custos tando do mercado econômico quanto do político que se soma aos modelos subjetivos dos agentes de modo que não mudam em direção mudanças (incrementais) mais eficientes.
\section{Cooperação: o problema teórico}

\autor abre o capítulo pontuando que a teoria neoclássica não apenas deixa de conceituar diferente organizações de troca (que não mercado) como também não explica a persistência de organizações ``ineficientes'' ao longo do tempo. No entanto, afirma que tal teoria não discute tais temas uma vez que partem da hipótese de que os direitos de propriedade são bem definidos (a um custo desprezível) e que a informação está disponível (também a um custo baixo). Em linhas gerais, o autor argumenta que a teoria neoclássica carece de uma melhor compreensão da \textbf{coordenação e cooperação} dos agentes econômicos. Além disso, enfatiza a importância dos \textbf{custos de transação e das instituições}.

\subsection{I}

Por mais que os economistas --- de modo geral --- tenho demorado para levar em consideração a importância das instituições, já vem de um tradição que explora os problemas de coordenação e o fazem por meio do arcabouço da \textbf{teoria dos jogos}. Em seguida, \autor pontua que a cooperação não é sustentável nas configurações mais próximas da realidade, ou seja, quando os jogos não são repetitivos; quando a informação não é completa e; quando existe um grande número de jogadores. Adiante, discute alguns avanços da literatura e coloca:

\begin{quotation}
	[U]nder what conditions can voluntary cooperation exist without the Hobbesian solu­tion of the imposition of a coercive state to create cooperative solutions? [...] We do not observe political anarchy in high-income countries. On the other hand the coercive power of the state has been employed throughout most of history in ways that have been inimicable to economic growth (North, 1981, Chapter 3).
	But it is difficult to sustain complex exchange without a third party to
	enforce agreements.
\end{quotation}
Em seguida, \autor explicita aquilo que considera o centro da análise das comunidades, convenções e da cooperação: qual o mínimo que é preciso saber sobre os demais agentes de modo a formar noções de seu comportamento e, com isso, ser capaz de interagir com eles?

\subsection{II}

O autor abre a seção afirmando que a competição elimina os problemas da informação completa e assimétrica. No entanto, para isso é preciso supor configurações institucionais e informacionais muito estringentes. Além disso, a teoria-padrão não apenas pressupõe que os agentes possuem objetivos bem definidos, mas que sabem as escolhas corretas para obtê-los. Outra limitação surge na presença de elevados custos de transação em que supõe-se que as instituições são desenvolvidas para gerar resultados eficientes e que independem do desempenho econômico.

Em linhas gerais, \autor afirma que nenhumas dessas condições extremas são observadas uma vez que os agentes econômicos atuam com informações incompletas a partir de modelos subjetivos e potencialmente errados cuja informação não é suficiente para corrigí-los. Além disso, instituições são criadas para servir aos interesses daqueles que possuem maior poder de barganha, ou seja, não são desenvolvidas necessariamente para serem eficientes. Em um mundo em que os custos de transação são desprezíveis, prossegue, este poder de barganha não é tão relevante, mas como este não é o caso, o poder de barganha afeta as instituições e, por conseguinte, o desempenho econômico:

\begin{quotation}
	If economies realize the gains from trade by creating relatively efficient
 institutions, it is because under certain circumstance5 the private objec­tives of those with the bargaining strength to alter institutions produce institutional solutions that turn out to be or evolve into socially efficient ones. The subjective models of the actors, the effectiveness of the institu­tions at reducing transaction costs, and the degree to which the institu­tions are malleable and respond to changing preferences and relative prices determine those circumstances.
\end{quotation}

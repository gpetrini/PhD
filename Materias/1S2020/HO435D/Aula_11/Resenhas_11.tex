\documentclass[11pt,lineno]{../style}



\newcommand{\autor}{Denzau e North (1994) }

\title{
\large{Nova Economia Institucional}\vspace{2pt}\\
\Huge{\autor - \textit{Shared Mental Models: Ideologies and Institutions}}
}
\date{19 de Junho de 2020}

\author[$\ast$]{Gabriel Petrini}

\affil[$\ast$]{PhD Student at Unicamp.}

\keywords{
	Keyword1\\ 
	Keyword2\\
	Keyword3
}

\runningtitle{Resenha} % For use in the footer 

%% For the footnote.
\runningauthor{Petrini}

\begin{abstract}

\end{abstract}

\begin{document}

\maketitle
\marginmark
\thispagestyle{firststyle}
% Please add here a significance statement to explain the relevance of your work

\section{Introdução}

Os autores abrem o artigo pontuando que ideologias e ideias importam, assim como as formas que são transmitidas entre os agentes uma vez que permitem lidar com a \textbf{incerteza}. Argumentam que pessoas com contextos culturais e experiências similares irão compartilhar modelos mentais razoavelmente convergentes enquanto agentes com experiências de aprendizado distintas terão teorias (ideologias) para interpretar o mundo diferentes. Neste artigo, \autor defendem que é preciso compreender as relações entre os modelos mentais, as ideologias que envolvem tais constructos e as instituições para entender o \textbf{processo decisório sob incerteza}. 

Adiante, definem cada um desses conceitos. Por \textit{ideologias}, os autores entendem como conjunto dos modelos mentais que os grupos de indivíduos desenvolvem para entender o mundo e como este deveria ser estruturado. \textit{Instituições}, são restrições formais e informais elaboradas para guiar as relações interpessoais. Por fim, \textit{modelos mentais} são representações internas a cada sistema cognitivo dos indivíduos e afirmam que estes modelos mentais são compartilhados intersubjetivamente.

Com estes conceitos em mãos, afirmam que as características sociais a serem modeladas nesse artigo são as necessidades de comunicação que permitem o desenvolvimento de uma experiência de aprendizado individual culturalmente posicionada que o antecedem de modo que não parte de uma \textit{tabula rasa}. Afirmam que o entendimento de como tais modelos evoluem ao longo do tempo e a relação entre eles é o principal objeto a ser analisado pelas ciências sociais de modo a substituir a caixa preta da racionalidade substantiva.

\section{O agente tomador de decisões frente à incerteza}

Os autores retomam a evolução da racionalidade substantiva na teoria econômica indo de Marshall a Von Neumann e Friedman. Afirmam que mesmo partindo da conjectura do ``\textit{as if}'', a hipótese da racionalidade substantiva ainda possui um paradigma uma vez que tal teoria tem um poder explicativo bastante baixo (mesmo ao nível de mercado).

\subsection{Condições para a racionalidade substantiva}

Nesta seção, \autor pontuam que a hipótese da racionalidade substantiva tem alta capacidade explicativa em experimentos do tipo \textit{Double Auctions} (DA). No entanto, argumentam que mudanças institucionais destes leilões explicam melhor o aumento da eficiência do que as demais mudanças.

\subsection{Atributos individuais de escolha}

Nesta seção, os autores discutem as características do ambiente de escolhas que determinam o sucesso do modelo de racionalidade substantiva.

\begin{description}
	\item[Complexidade] Quais são os modelos mentais compartilhados necessários para que uma escolha relevante seja tomada dadas as preferências individuais e os recursos disponíveis? De modo geral, a escolha é feita por meio da analogia de um cenário já conhecido, ou seja, parte de uma complexidade familiar para avaliar uma complexidade inédita;
	\item[Motivação] Se a escolha em questão envolve decisões relevantes para o indivíduo, o processo de aprendizado é muito mais rápido. Além disso, quando reportam evidências que o processo de aprendizado é irreversível;
	\item[Informação (qualidade e frequência)] Quão boa deve ser uma informação para que seja possível corrigir um modelo ruim?
\end{description}
A partir dessa categorização, os autores discutem o processo decisório em outros contextos.

\subsection{Decisão fácil ---  mercados competitivos}

\begin{description}
	\item[Complexidade] Ambientes menos complexos requerem modelos mentais menos elaborados. As instituições por si só reduzem a complexidade necessária aos modelos mentais;
	\item[Informação] Boa informação é crucial para aprimorar os modelos mentais;
	\item[Motivação] O \textit{feedback} é direto nos mercados privados, mas são mais problemáticos para bens não-privados;
\end{description}

\section{Incerteza forte e problemas mais complexos}

Os autores definem a incerteza forte (ou Knightiana) e afirmam que todas as pessoas iniciam a vida em situações desse tipo. Argumentam também que boa parte das decisões são pouco frequentes, sem alguma experiência prévia ou informação sobre os potenciais resultados. Nessas condições, afirma, a racionalidade substantiva não é um modelo suficientemente descritivo. Depois de exemplificarem um caso de decisão difícil, discutem processos de decisão coletiva por estarem sujeitos a incerteza forte. Em seguida, apresentam o conceito de complexidade limitada como um limite superior a capacidade explicativa da racionalidade substantiva. Nas situações que passam desse limiar, é necessário adotar procedimentos que diferem da dedução racional. Afirmam que para aprender por \textbf{indução}, os agentes precisam de modelos mentais para compreender as implicações das decisões tomadas, assim como identificar ações úteis para lidar com os resultados dessas ações. Modelar tais decisões requer modelar o tomador de decisões enquanto elaborador de modelos mentais que representam o mundo e aprendem de modo a aprimorar suas escolhas.

\section{Escolha e incerteza forte}

Apresentam a proposta de Heiner (1983) em que há um argumento complementar ao dos autores sobre escolha sob incerteza. Argumenta que existe um \textit{gap} entre a capacidade do agente e a dificuldade do problema a ser resolvido (C-D) de modo que tal agente elabora regras para restringir a flexibilidade de suas decisões nestes contextos. Em linhas gerais, tal norma (instituição) aprimora a capacidade de compreender o ambiente e de se comunicar. Em seguida, descrevem alguns experimentos na área de psicologia.

\section{Aprendendo e compartilhando modelos mentais}

Para compreender como os agentes lidam com a complexidade, é preciso entender como aprendem. Primeiramente, consiste em desenvolver uma estrutura que faça sentido dados os sinais recebidos. Tais estruturas consistem em categorias que formam os modelos mentais para explicar e compreender o ambiente dados alguns objetivos relevantes. Tanto esta categorização quanto os modelos mentais evoluem  e refletem o \textit{feedback} de novas experiências de modo que é sempre redefinido, aprimorada e inclui o contato com as ideias dos demais. Afirmam que a mente parece reordenar os modelos mentais em formas mais abstratas para estarem disponíveis para processar informações. A capacidade de generalizar, argumentar do mais particular para o mais geral e usar analogias são partes do processo de \textbf{redescrição}.

Ao nível individual, a \textbf{redescrição representacional} é o reconhecimento de categorias e conceitos na forma de um processo de aprendizado distinto a partir da atualização de parâmetros que ocorrem no processo de aprendizado normal. Uma vez que um conjunto de categorias e conceitos são adquiridos, o período de aprendizado normal é longo em relação a essas mudanças de perspectivas que acompanham a redescrição representacional e o resultado é um equilíbrio pontual.
Em seguida, destacam que o mundo é bastante complexo para que um indivíduo aprenda diretamente como tudo funciona de modo que as estruturas dos modelos mentais são derivados das experiências de cada indivíduo. Heranças culturais são formas de reduzir a divergência dos modelos mentais que as pessoas desenvolveram para constituir meios de transferir percepções intergeracionais unificadoras. 

Em seguida, os autores relacionam modelos mentais com ideologias em que são mapeadas as ações com os resultados dentre os quais o indivíduo poderia ter escolhido. Tal mapeamento fornece uma visão normativa para identificar aspectos cruciais da realidade para atingir certos objetivos. Adiante, usa o exemplo das relações de trabalho na visão Marxista.

\subsection{Aprendizado sob incerteza forte com modelos mentais compartilhados}

Nesta seção, os autores discutem o processo de aprendizado na presença de modelos mentais compartilhados (SMM). Afirmam que tal processo é lento e pode ser acelerado por meio de formas indiretas de aprendizado por meio de modelos artificiais criados pelos demais. Afirmam que SMM são associados com o modelo de aprendizado Bayesiano em que supõe que as dimensões internas dos modelos mentais usados para representar o mundo são corretas.

\subsection{Modelos mentais em um modelo de comunicação simples}

\autor afirmam que uma melhor comunicação geram uma convergência mais rápida dos SMM individuais. No entanto, afirmam que nem sempre o agente está plenamente consciente dos fatores que influenciam sua decisão, podendo advir do conhecimento tácito. Além disso, o processo de decodificação é quase sempre incompleto. O receptor da informação deve transformar tais canais de comunicação em padrões neurais, mas é afetado por padrões pre-existentes em sua mente. Em linhas gerais, a recepção da mensagem é fortemente influenciada pelas categorias e crenças que o ouvinte tinha sobre o mundo. Se um SSM está disponível, os conceitos dos modelos de diversas pessoas são mais similares.
Adiante, afirmam que os modelos mentais são compartilhados por meio da \textbf{comunicação} que, por sua vez, permite a criação de instituições e ideologias em um processo co-evolucionário. Reforçam ainda que as ideologias são relevantes para explicar o desempenho econômico pelos ganhos com a troca, produção e, principalmente, coordenação.


\section{Modelos mentais e instituições}

Os autores pontuam que os modelos mentais são particulares de cada indivíduo enquanto ideologias e instituições são criadas para ordenar as interações. Somado a isso, afirmam que o processo de redescrição representacional ajudam a explicar o processo de evolução da cognição humana.

\section{Dinâmica dos modelos mentais e das instituições}

A característica de \textit{path-dependence} do processo de desenvolvimento institucional decorre das forma que a cognição e as instituições evoluem no tempo.

\subsection{Mudanças nos modelos mentais enquanto um equilíbrio pontual}

Iniciam a seção descrevendo o processo de aprendizado Bayesiano em que crenças pré-estabelecidas são transicionadas em crenças posteriores como um processo de mudança gradual. As ``surpresas'' dos modelos Bayesianos podem ser descritas como redescrição representacional e envolvem trajetórias que podem ser descritas como equilíbrio pontual que requer avanços de pesquisas futuras. Adiante, os autores estabelecem um paralelo da proposta do artigo com o conceito de Ciência Normal de Kuhn a partir do exemplo de ideólogos puristas contrapostos com outras interpretações desta mesma ideologia.

\section{Conclusão}

Concluem o artigo pontuando que esta é uma proposta preliminar para analisar as implicações de como as pessoas tentam ordenar e estruturar o ambiente e se comunicar entre si. Retomam a noção de que as ideias e ideologias são relevantes e a compreensão de como essas ideias mudam e são comunicadas é central para desenvolver uma teoria que explique o desempenho das sociedades no tempo. Argumentam que a performance das economias é uma consequência de estruturas de incentivo decorrentes do arranjo institucional que, por sua vez, são funções dos modelos mentais e das ideologias dos agentes. Além disso, o processo de aprendizado criam um processo de \textit{path-dependence} nas ideias e ideologia e, portanto, nas instituições.

\begin{redbox}{Dúvidas e comentários}
	Os questionamentos abaixo surgiram de uma comparação das duas obras do North discutidas ao longo da disciplina.
	
	É possível dizer que North (1990) estava mais preocupado com o entendimento da divergência entre os países enquanto \autor estão mais interessados no processo de convergência dos modelos mentais? Também é possível dizer que o individualismo metodológico de North (modelos mentais) se soma a uma negação de um atomismo (modelos mentais \textit{compartilhados})? Pode-se dizer que os modelos mentais são espontâneos enquanto ideologias e instituições são criadas?
\end{redbox}

\end{document}

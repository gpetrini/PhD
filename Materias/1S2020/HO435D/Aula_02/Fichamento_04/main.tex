\documentclass[9pt,twocolumn,twoside,lineno]{style}

\articletype{NEI} % article type

\title{Coase  in Ménard (2000, Capítulo 1): \textit{The new institutional economics}}
\date{1º Semestre de  2020}

\author[$\ddagger$]{Gabriel Petrini}

\affil[$\ddagger$]{Doutorando no instituto de Economia da Unicamp}

\keywords{Keyword \\ Keyword2 \\ Keyword3 \\ ...}

\runningtitle{Fichamento} % For use in the footer 

%% For the footnote.
\runningauthor{Petrini}

\begin{abstract}
\end{abstract}

\begin{document}

\maketitle\articletypemark
\marginmark
\thispagestyle{firststyle}

% Please add here a significance statement to explain the relevance of your work
%\afterpage{
%	\begin{sigstatement}
%		\sffamily
%		\mdfdefinestyle{stylesigstyle}{linewidth=0.7pt,
%			backgroundcolor=styleblueback,linecolor=stylebluetext,
%			fontcolor=stylebluetext,innertopmargin=6pt,innerrightmargin=6pt,
%			innerbottommargin=6pt,innerleftmargin=6pt}
%		{%	
%			\begin{mdframed}[style=stylesigstyle]%
%				\section*{5 Seconds Synthesis}%
%				\lipsum[1-3]
%		\end{mdframed}}
%\end{sigstatement}
%}

% If your first paragraph (i.e. with the \dropcap) contains a list environment (quote, quotation, theorem, definition, enumerate, itemize...), the line after the list may have some extra indentation. If this is the case, add \parshape=0 to the end of the list environment.
	
Coase pontua que o termo ``Nova Economia Institucional'' (NEI) cunhada por Williamson possui uma preocupação distinta da ``Velha'' Economia Institucional\footnote{Pontua ainda que quando se refere ao \textit{mainstream}, está tratando especificamente da teoria \textbf{microeconômica} predominante.}. Em seguida, argumenta que há um distanciamento grande entre o objeto de estudo dos economistas e a realidade concreta e que tal afastamento está associado do que se entende pelo \textbf{objeto} de estudo. Desta discussão, destaca-se a seguinte passagem (p.~4):

\begin{quote}
	\textit{What this comes down to is that economists think of themselves as having a box of tools but no subject matter.}
\end{quote}
No entanto, Coase argumenta que o \textbf{entendimento do funcionamento do sistema econômico} (para ele, o objeto de estudo) é bastante relevante.

Em seguida, o autor usa Adam Smith como exemplo em que a divisão do trabalho permite o aumento da produtividade da economia. No entanto, afirma Coase, o aumento da especialização só é possível se existir trocas junto de custos de transação reduzidos. Estes custos de transação, por sua vez, dependem da institucionalidade de um país. Sendo assim, são as instituições que determinam o desempenho de uma economia e isso que dá à NEI um papel de destaque entre os economistas.


Por fim, argumenta que por mais que tenha enfatizado a existência da firma em seu estudo seminal, Coase afirma que não se pode restringir a estudo de uma única firma uma vez que o objeto de estudo do economista é uma estrutura complexa e inter-relacionada. Adiante, afirma que as mudanças no pensamento econômico vindas com a NEI não surgirão com uma alternativa abrupta e uníssona ao \textit{mainstream}, mas sim, de forma fragmentada e a partir de diferentes abordagens.

\end{document}
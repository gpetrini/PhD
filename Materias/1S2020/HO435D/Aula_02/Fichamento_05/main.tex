\documentclass[9pt,twocolumn,twoside,lineno]{style}

\articletype{NEI} % article type

\title{North in Ménard (2000, capítulo 2): \textit{Understanding institutions}}
\date{1º Semestre de 2020}

\author[$\ddagger$]{Gabriel Petrini}

\affil[$\ddagger$]{Doutorando no instituto de Economia da Unicamp}

\keywords{Keyword \\ Keyword2 \\ Keyword3 \\ ...}

\runningtitle{Fichamento} % For use in the footer 

%% For the footnote.
\runningauthor{Petrini}

\begin{abstract}
\end{abstract}

\begin{document}

\maketitle
\articletypemark
\marginmark
\thispagestyle{firststyle}

% Please add here a significance statement to explain the relevance of your work
%\afterpage{
%	\begin{sigstatement}
%		\sffamily
%		\mdfdefinestyle{stylesigstyle}{linewidth=0.7pt,
%			backgroundcolor=styleblueback,linecolor=stylebluetext,
%			fontcolor=stylebluetext,innertopmargin=6pt,innerrightmargin=6pt,
%			innerbottommargin=6pt,innerleftmargin=6pt}
%		{%	
%			\begin{mdframed}[style=stylesigstyle]%
%				\section*{5 Seconds Synthesis}%
%				\lipsum[1-3]
%		\end{mdframed}}
%\end{sigstatement}
%}

% If your first paragraph (i.e. with the \dropcap) contains a list environment (quote, quotation, theorem, definition, enumerate, itemize...), the line after the list may have some extra indentation. If this is the case, add \parshape=0 to the end of the list environment.
	

D. North inicia o capítulo destacando que somente quando as instituições são consideradas, é possível compreender o objeto de estudo da economia. Dito isso, avança em direção para a discussão das estruturas de uma sociedade que, segundo o autor, é uma junção de regras, normas, convenções, comportamentos e crenças dispostos de forma complexa. Tendo esse conceito em mente, a ideia de uma economia plenamente aos moldes do \textit{laissez-faire} não poderia existir uma vez que não existe um mercado eficiente que é insensível aos agentes econômicos (e o governo é um deles). Em outras palavras, tal eficiência dos mercados só é obtida por meio da organização de instituições formais e informais. Além disso, o que torna um mercado de capitais eficientes hoje não é necessariamente o mesmo fator que o manterá eficiente no futuro ou que tais modificações para alcançar a eficiência se deem de forma automática.
	
Sendo assim, um ponto caro para a teoria econômica partindo da perspectiva da NEI é compreender como essas estruturas evoluem no tempo. Para isso, North destaca alguns desafios:

\begin{enumerate}
	\item Compreender como as escolhas são tomadas no âmbito da economia política;
	\item Necessidade de se considerar as \textbf{instituições informais}
	\begin{itemize}
		\item Só se tem controle sobre as instituições \textbf{formais}
	\end{itemize}
	\item Execução dessas instituições
	\item Necessidade de se entender como as instituições \textbf{formais} mudam.
\end{enumerate}
Em seguida, North pontua que o objetivo não é substituir a teoria neoclássica --- que em sua leitura, apresenta uma teoria razoável de preços e quantidades ---, mas sim, torná-la aderente e aplicável para ``pessoas reais'' (\textit{human beings}).

Ao discutir como as instituições formais mudam, North contrapõe um mundo dinâmico frente às teorias estáticas. Além disso, argumenta que não basta partir da teoria convencional e adicionar uma dimensão institucional, mas sim incluir uma compreensão de como (e quais) sociedades evoluíram, quais os meios de entender como as instituições formais e informais mudam, como elas interagem com o conhecimento adquirido dessas sociedades. Adiante, afirma que existem várias formas de ir nessa direção, mas destaca que as mudanças institucionais refletem as \textbf{crenças} da sociedade e que isso requer que entendamos como as pessoas aprendem, o que e como aprendem e etc. Tais crenças, por sua vez, são traduzidas em instituições e estas instituições determinam como a economia evolui ao longo do tempo.


\end{document}
\section*{Williamson: Prólogo}

Williamson pontua que aquilo que entende por custos de transação é, por definição, um conceito multidisciplinar. Como consequência desta abordagem, as instituições atuam com o objetivo de minimizar tais custos. Em linhas gerais, este custo ocorre quando um bem ou serviço é transacionado entre configurações tecnológicas e organizacionais distintas e podem ser vistos como um paralelo ao atrito/fricção da operação do sistema econômico. Além disso, Williamson argumenta que os elementos constituintes da NEI se formaram de forma decentralizada.

\subsection*{Antecedentes (1930 em diante)}

Nesta seção, o autor pontua as influências da economia, direito e organizações e podem ser sintetizados da seguinte maneira:

\begin{description}
	\item[Economia:] {\color{white}{bla}}
	\begin{description}
		\item[Knight] Oportunismo faz parte do comportamento humano e não pode ser negligenciado pelo estudos das organizações econômicas
		\item[Commons] A transação é o principal elemento de análise da organização
		\item[Commons e Barnard] Uma organização econômica tem na harmonização das relações de troca o seu principal propósito
	\end{description}
	\item[Direito:] {\color{white}{bla}}
	\begin{description}
		\item[Llewellyn] O estudo do contrato é o equivalente no direito do estudo das organizações econômicas
	\end{description}
	\item[Organizações:] {\color{white}{bla}}
	\item[Coase] O estudo das formas de organização não pode ser desassociado dos custos de transação
\end{description}

\subsection*{Próximos 30 anos}

Seguindo a mesma estrutura da seção anterior

\begin{description}
	\item[Economia:] {\color{white}{bla}}
	\begin{description}
		\item[Hayek] Relevância do conhecimento idiossincrático e destaque para a importância da capacidade de adaptação das instituições
		\item[Coase] Importância da disponibilidade, distribuição e transmissão da informação entre os agentes
		\item[Arrow] A estrutura hierárquica de uma organização é uma variável de decisão interna
	\end{description}
	\item[Direito:] {\color{white}{bla}}
	\begin{description}
		\item[Macaulay] A disputas e desentendimentos contratuais são resolvidos principalmente por meio de acordos privados e menos por meio da juridicialização do problema (recorrer aos tribunais)
	\end{description}
	\item[Organizações:] {\color{white}{bla}}
	\item[Chandler] A forma de organização tem consequências importantes sobre a performance
	\item[Michael Polanyi] A firma não pode ser reduzida em termos tecnológicos
\end{description}


\subsection*{Panorama}

Nesta seção, Williamson apresenta os temas tratados ao longo do livro além de definir a \textbf{Transformação Fundamental}:

\begin{quotation}
	WhatI refer to as the `Fundamental Transformation' --- where by a large-numbers condition at the outset	(ex ante competition) is transformed into a small-numbers condition duringcontract execution and at contract renewal intervals (ex post competition)
\end{quotation}


\begin{sigstatement}
\sffamily
\mdfdefinestyle{stylesigstyle}{linewidth=0.7pt,
backgroundcolor=styleblueback,linecolor=stylebluetext,
fontcolor=stylebluetext,innertopmargin=6pt,innerrightmargin=6pt,
innerbottommargin=6pt,innerleftmargin=6pt}
{%	
\begin{mdframed}[style=stylesigstyle]%
	\section*{Dúvida}%
	Poderia dar um exemplo da ``Transformação fundamental''?
\end{mdframed}}
\end{sigstatement}
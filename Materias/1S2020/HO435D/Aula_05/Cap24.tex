\section*{Benham \& Benham: Measuring the costs of exchange}

\subsection*{Introdução}

Os autores iniciam o capítulo pontuando que os custos de transação são o principal conjunto de preços da economia, mas pouco se sabe sobre sua mensuração e variação.
Em seguida, definem custos de transação como \textbf{custos de oportunidade} que os agentes econômicos incorrem quando recorrem à uma forma de troca para obter um ativo específico dada uma configuração institucional.

\subsection*{Elementos dos custos de transação}

Os autores ressaltam que existem poucos estudos que estimam os custos de transação e que mesmos estes são pouco comparáveis. Dentre os motivos, pontuam a ausência de consenso na definição do termo. Partindo da contribuição de Furubotn e Reichter, elencam duas variantes dos custos de transação: (i) fixos e (ii) variáveis. Os primeiros estão associados à investimentos específicos em um dado arranjo institucional enquanto os segundos dependem do número ou do volume das transações.
Além disso, afirmam que os custos de produção e de transação são definidos \textbf{conjuntamente} e, portanto, existem dificuldades em separá-los. Também pontuam que se os custos de transação são muito elevados, muitas das transações associadas deixariam de ocorrer. Argumentam que a lei do preço único não se aplica também uma vez que indivíduos de uma mesma sociedade podem ter custos de transação distintos.

\subsection*{Custos de troca}

Os autores propõem analisar um subconjunto de custos de transação: custos de troca. Definem este conceito como o custo de oportunidade de todos os recursos (tempo, dinheiro e bens) de um indivíduo $i$ usar uma troca $j$ para obter um bem $k$ em um arranjo institucional $m$ ($C_{ijkm}$). Sendo assim, os custos de troca são a soma dos custos de produção e de transações específicas incorridas por um agente. Uma vez que não é possível decompor os custos diretamente em seu equivalente em produção e transação, irão adotar uma \textbf{análise comparativa}. Em outros termos, esta proposta enfatiza os custos de oportunidade que um indivíduo incorre ao estabelecer uma troca específica em uma dada especificidade institucional.
Para tanto, os autores precisam padronizar a metodologia adotada. Dentre as dificuldades, afirmam que os preços de mercado não refletem bem os custos de oportunidade. Sendo assim, optam por analisar os bens intermediários.

\subsection*{Estudos de casos}

Nesta seção, os autores apresentam alguns estudos de caso, são eles:

\begin{description}
	\item[Telefonia] Os custos de troca associados ao setor de telefonia determinam o tamanho das redes de comunicação, tamanho do mercado e grau de especialização. O estudo foca no custos de obtenção de um telefone comercial em que investigam os custos de transferência de propriedade.
	\item[Novo negócio] São investigados os custos associados a criação de um novo negócio em que foram calculados os custos em dias para conseguir os requerimentos necessários para tal. Relatam casos em que o tempo para conseguir iniciar um novo negócio é consideravelmente maior na ausência de contatos influentes.
	\item[Fronteiras nacionais] Argumenta que as transações transfronteiriças são um indicador do tamanho do mercado de uma economia. Para tanto, calculam o tempo de espera para escoar os produtos no porto.
\end{description}
Os estudos ilustrados anteriormente indicam que a \textbf{variação} dos custos de troca são enormes e que os custos de oportunidade associados à obtenção de um bem são bastante distintos dos preços de mercado.

\subsection*{Pesquisas futuras}

Os autores argumentam que o retorno desta agenda de pesquisa será maior quanto maior os esforços coletivos entre diferentes países. Além disso, os casos apresentados ilustram que esta metodologia é factível enquanto a variedade dos resultados indicam a importância de se estudar tais custos. Em seguida, afirmam algo semelhante às vantagens comparativas em que pontuam que se os custos de troca de um país forem consideravelmente maiores em um país do que em outro, é esperado que o país que apresente maiores custos não produza o bem em questão. Por outro lado, se os custos de oportunidade são menores, as transações podem em distâncias maiores e por períodos mais longos. Encerram afirmando que estar diferenças dependem do grau de especialização e do desempenho da economia.
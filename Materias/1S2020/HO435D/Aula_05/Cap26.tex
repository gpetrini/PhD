\section*{Brousseau e Fares: \textit{Incomplete contracts and governance structures: are incomplete contract theory and new institutional economics substitutes or complements?}}

\subsection*{Introdução}

Ao longo deste capítulo, os autores investigam as diferenças e compatibilidades da Teoria dos Contratos Incompletos (TCI) com a Nova Economia Institucional (NEI). Dado que esta última corrente tem sido analisada no decorrer da disciplina, será descrita em menor detalhe. Dito isso, cabe mencionar que a TCI se propõe a explicar os mecanismos de coordenação verticais dentro do aparato neoclássico. Além disso, as hipóteses racionais e cognitivas da TCI se diferem das da NEI uma vez que a primeira supõe racionalidade limitada apenas na instância que avalia e \textbf{executa os contratos}. Outra diferença é que a TCI analisa tipos de contratos que podem ser implementados em uma dada estrutura institucional.

\subsection*{Teoria dos contratos incompletos: análise dos impactos das instituições sobre os contratos}

\subsubsection*{Fundadores da TCI}

Os autores apresentam o modelo Grossman e Hart (G\&H) em que uma integração vertical mitiga (mas não suprime) o problema de espera e sub-investimento decorrente da incompletude dos contratos. Este modelo possui duas hipóteses principais: investimentos específicos não são-contratáveis dados os elevados custos de elaboração dos contratos; as variáveis relevantes para a relação contratual não são verificáveis por terceiro e, assim, não contratáveis. Em seguida, os autores explicitam o modelo formalmente comparando o resultado ótimo com o sub-ótimo em que há sub-investimento. Tal resultado decorre dos problemas de espera uma vez que as partes temem (\textit{ex ante}) uma desapropriação \textit{ex post}. Com a integração vertical, uma das partes investe em seu nível ótimo e, assim, captura todo o excedente \textit{ex post}.
Críticas ao modelo apontavam para a tensão entre as hipóteses de racionalidade substantiva entre as partes envolvidas junto da racionalidade limitada  para justificar os custos de elaboração contratual (Tirole 1999). Como resposta, surgiram modelos que parte da não-verificabilidade de todas as informações de uma das partes.

\subsubsection*{TCI e os aspectos de inverificabilidade}

Um aprimoramento do modelo G\&H é o desenvolvido por Hart e Moore (H\&M) em que os autores  defende que os contratos são incompletos por: (i) incapacidade do avaliador (juiz) verificar o estado da natureza; (ii) incapacidade de ambas as partes evitarem renegociações \textit{ex post}. Em linhas gerais, há sub-investimento por conta do juiz não ser capaz de executar níveis ótimos de transação. Sendo assim, o problema da inverificabilidade pela escolha de uma classe particular de contratos. Em resumo, a principal hipótese recai sobre o nível de verificabilidade do juiz. 
Em seguida, os autores comparam o modelo H\&M com os de Aghion (ADR) para indicar  que os resultados são sensíveis ao grau de verificabilidade do juiz. Em particular, os resultados do modelo são sensíveis à alocação do poder de barganha entre as partes e garantias do \textit{status quo} que incentivam a investir. Caso o poder de barganha seja todo alocado para uma das partes, o problema de espera pode ser resolvido.

\subsubsection*{Crítica de Tirole e limitações analíticas da TCI}

Tirole rejeita que a incompletude contratual seja endogeneizada pela hipótese de não-verificabilidade. Em particular, rejeita a ideia de que a incompletude dos contratos por conta da hipótese de informações observáveis, mas não verificáveis uma vez que as partes sempre poderiam completar os contratos ao implementar um mecanismo de revelação que faça com que esta informação se torne verificável ao juiz.
No entanto, alguns autores argumentam que este mecanismo de Tirole tem limitações por requerer a implementação de penalidades elevadas junto de compromissos críveis. Em seguida, os autores pontuam as questões de \textbf{inconsistência lógica} da TCI e que a hipótese de não-verificabilidade é \textit{ad hoc} porque a estrutura institucional (juiz) que executa os contratos continua exógena. Isso decorre porque o juiz não é a única parte envolvida no processo contratual que possui racionalidade limitada.
Diferentemente da TCI, a NEI endogeniza as instituições. Além disso, as imperfeições de cada mecanismos de coordenação na NEI são reduzias quando realizados conjuntamente. Em outras palavras, a diversidade dos mecanismos de governança garantem que os agentes desempenhem coordenações a um custo menor.

\subsection*{Incompletude contratual e institucional na NEI}

\subsubsection*{Causas da incompletude: Racionalidade limitada e incerteza fundamental}

Em linhas gerais, a hipótese de racionalidade limitada na NEI não se restringe aos juízes e se estende para todos os agentes. Sendo assim, ao assumir que as tomadas de decisão levam tempo e são custosas, os agentes podem errar e estão sujeitos à assimetrias de informação. A outra razão da incompletude dos contratos é a existência de incerteza fundamental e, assim, não são capazes de estabelecer contratos contingenciais que encobrem todas as possibilidades futuras eficientemente.

\subsubsection*{Natureza intrínseca das estruturas de governança: Autoridade e execução}

Invés de elaborar contratos que lidem com todas as contingências \textit{ex ante}, elaboram mecanismos de decisão \textbf{\textit{ex post}} que indicam comportamentos necessários para os contratantes assegurarem a coordenação de forma eficiente e garantir a execução dos compromissos assumidos. Esses mecanismos de decisão se baseiam no reconhecimento de uma autoridade pelos contratantes e pelo princípio de subordinação. A divisão destas funções entre as partes é diferente em cada estrutura de governança. Tais mecanismos, no entanto, são insuficientes para garantir a coordenação. Em outras palavras, regras e ordens devem ser executadas para garantir a credibilidade dos compromissos contratuais. Ou ainda,  a incompletude dos contratos  implementa um mecanismo de autoridade. Como consequência, além das limitações do sistema legal, a incompletude dos contratos \textbf{\textit{per se}} gera questões de execução e, por conta disso, os agentes econômicos devem implementar mecanismos de auto-execução nas estruturas de governança que elaboram. Esta auto-execução depende da implementação de três classes de mecanismos:

\begin{itemize}
	\item Esquema de incentivo e coerção
	\item Mecanismos de supervisão
	\item Mecanismos de arbitragem para resolução de conflitos
\end{itemize}
Em resumo, estruturas de governança articulam mecanismos de atuação que as partes devem tomar para obterem coordenação de forma eficaz e as classes mencionadas anteriormente para assegurar a auto-execução de arranjos contratuais.

\subsubsection*{Articulação entre estruturas de governança coletivas e entre indivíduos}

Estruturas de governança intra-individuais (EGI) são elaboradas pelos agentes para completar as obrigações contratuais \textit{ex ante} e assegurar a auto-execução. Tais EGIs são desempenhadas em um nível bidirecional entre indivíduos e instituições coletivas. Em resumo, ambas são formas complementares de coordenação em que os agentes ajustam as vantagens de cada uma para reduzir os custos de transação. Quando uma EGI é desenhada, o comportamento dos agentes econômicos é restringindo pelos mecanismos institucionais \textit{ex ante}.

A NEI enfatiza que as governanças coletivas são executadas por instituições completas e incompletas uma vez que são elaboradas para desempenhar uma interação econômica de governança de forma não-intencional; são elaboradas por agentes que possuem racionalidade limitada; e a partir de um denominador comum dado pelos conjuntos de transação. Como consequência, algumas estruturas de governança devem ser desenvolvidas ao nível inter-individual na tentativa de completar a incompletude das governanças coletivas.

\subsubsection*{Endogeinização de sistemas de governança complexos}

Uma vez que o arranjo institucional é imperfeito e incompleto, os agentes precisam implementar EGI para obter coordenação. No entanto, também podem participar da elaboração de estruturas de governança coletivas. Sendo assim, as EGIs podem moldar a matriz institucional e aprimorá-la. Estes mecanismos de decisão completam a incompletude das regras e garantem que são executadas. Sendo assim, instituições são combinações de regras e instituições organizacionais. Os autores também pontuam que as instituições privadas devem ser comparadas com as públicas. Argumentam que quando os agentes precisam resolver um problema de coordenação, deve escolher reduzir custos de transação privados quando coordenam por meio de um mecanismo coletivo. Uma vez que os agentes não podem impactar as instituições públicas diretamente, devem criar voluntariamente uma estrutura coletiva privada especializada de modo a obter formas mais eficientes de coordenação coletiva. O trecho a seguir resume esta discussão (p.~ 416):

\begin{quote}
	In sum, economic agents have the possibility of assigning the various governance tasks to various entities according to the respective nature of the tasks and the ability of the entities. This enables them to play on the complementarities among public institutions, private institutions and inter-individual governance structuresto try to obtain the lowest possible transaction costs. NIE is thus able to explain the co-genesis and the joint properties of interindividual (contractual) governance structures and institutions (collective governance structures)
\end{quote}


\subsection*{Conclusão}

A incompletude dos contratos na TCI decorre da racionalidade limitada da personificação da estrutura institucional (juiz) e que os diferentes níveis de racionalidade deste juiz determinam os mecanismos de implementação das negociações. Na NEI, a incompletude dos contratos é resultado da racionalidade limitada de todos agentes econômicos. Além disso, as instituições não são exógenas e são elaboradas de forma imperfeita de modo que não conseguem zeras os custos de transação. Os agentes tentam escolher estruturas de governança complementares para reduzir tais custos. Sendo assim, a principal questão da NEI é comparar os custos relativos gerados pela combinação de vários mecanismos de coordenação. Em resumo, concluem que a TCI e a NEI são mais complementares do que excludentes uma vez que tratam de temas distintos. A TCI pode ser criticada por sua inconsistência lógica ou pelas hipóteses \textit{ad hoc}. A NEI, por outro lado, possui grandes dificuldades metodológicas apesar de possuir uma consistência interna elevada. Segue uma tabela que compara ambas abordagens.

\begin{figure}[h]
	\centering
	\caption{Resumo da comparação entre TCI e NEI}
	\label{fig:screenshot001}
	\includegraphics[width=\linewidth]{screenshot001}
\end{figure}

\section*{Estabilidade e mudança na história econômica}

\autor abre o capítulo retomando que as instituições são criadas para reduzir incerteza nas interações humanas (trocas). Isso somado a tecnologia vigente, determinam os custos de produção (transformação e transação) e a lucratividade e factibilidade da participação da atividade econômica. Além disso, pontua que as instituições conectam o passado ao futuro de modo que o movimento histórico é majoritariamente incremental em que a atividade econômica é uma fração desta trajetória. O argumento é que as organizações econômicas, políticas, militares, etc surgem das oportunidades providas pelo arranjo institucional que, por sua vez, se altera incrementalmente.

\subsection*{I}

O autor inicia esta seção partindo do exemplo de relações de trocas simples em que o desenvolvimento de comércio de distâncias mais longas foi possível em função de alterações na estrutura econômica. Em seguida, afirma, é necessário um maior grau de especialização associado ao aumento do mercado. No último estágio, associado às economias ocidentais modernas, verifica-se um aumento da especialização, organizações com larga escala em todos os setores.

\subsection*{II}

Nesta seção, o autor pontua que se difere de Rostow uma vez que pretende avaliar as condições (instituições) necessárias associadas aos custos de produção para que haja um aumento na especialização e divisão do trabalho. Argumenta que na medida que o mercado aumenta, as trocas implicam maiores custos de transação uma vez que maiores são os recursos destinados à mensuração e execução das transações. Além disso, afirma que na medida que o volume das trocas aumenta, o problema da cooperação se torna cada vez maior. Em paralelo, a garantia dos direitos de propriedade requer organizações políticas e jurídicas para garantir os contratos alongo do tempo e espaço.
No entanto, argumenta que nada garante essa sequência de eventos e que o caso Europeu é um exemplo. Em seguida, afirma que a análise de sistemas mais primitivos de troca tem um objetivo duplo: (i) contrastar as forças que produziram estabilidade organizacional e institucional e; (ii) investigar as implicações sobre a dinâmica econômica.


\subsection*{III}

Nesta seção, o autor apresenta alguns exemplos de trocar ocorridas e ainda existentes no norte da África (suq) em que é caracterizada pela impessoalidade entre as partes. Em resumo, caracteriza este sistema de trocas pelos: (i) elevados custos de mensuração; (ii) tentativa contínua de clientalização; (iii) barganha intensiva em cada dimensão (vantagens decorrentes de ter mais informações que seus adversários). \autor questiona o porquê da existência de elementos ineficientes nesta forma de negociação.


\subsection*{IV}

O autor argumenta que tais organizações tribais --- e as habilidades e conhecimentos associados --- não induziam modificações produtivas do arranjo institucional. Em todos esses casos, continua, a fonte da mudança institucional é externa. Em seguida, analisa quais as inovações que reduziram os custos de transação. Estão associadas ao: (i) aumento da mobilidade de capital; (ii) redução dos custos de informação; (iii) possibilidade de \textit{spread} do risco. Dentre as inovações que aumentaram a mobilidade de capital, pontua técnicas que permitiam se esquivar da Lei de usura, métodos de descontos, negociação de títulos, controles sobre trocas de longas distâncias e transformação da incerteza em risco atuarial. Afirma também que essas inovações e mudanças institucionais estão associadas a duas forças: (i) economias de escala decorrente do aumento do volume das transações e (ii) desenvolvimento e aprimoramento de mecanismos que permitem a execução contratual. Dentre os determinantes das estruturas de execução contratual destaca a relação entre organizações mercantis e o Estado. Em especial, pontua que o Estado é um agente nesse movimento uma vez que permitiu o desenvolvimento de mercados de capitais e de instituições financeiras via negociação da dívida pública. Mais precisamente afirma: ``\textit{The
	shackling of arbitrary behavior of rulers and che development of imper­sonal rules that successfully bound both the state and voluntary organiza­
	tions were a key part of this institutional transformation.}''


\section{V}

O autor encerra o capítulo comparando as formas primitivas com às ocidentais modernas. Afirma que os agentes nas primeiras não induziram modificações no arranjo institucional em direção ao aumento da produtividade. Nas segundas, nota-se uma mudança incremental induzido pelos ganhos privados e aumentos de produtividade por meio de mudanças organizacionais e institucionais. Em seguida, enfatiza a importância de se conectar as mudanças na Europa ocidental com o estoque de conhecimento dos agentes e suas interações com a estrutura política e econômica. Encerra pontuando que os países baixos e a Inglaterra foram os portadores da mudanças institucional.
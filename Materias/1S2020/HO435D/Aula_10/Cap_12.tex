\section*{Instituições, teoria e performance econômica}

\autor pontua que as instituições determinam a trajetória de longo prazo das economias e que é necessário elaborar um modelo de mudança institucional.

\subsection*{I}

Nesta seção, faz algumas críticas aos modelos neoclássico, sobretudo em relação às hipóteses de racionalidade limitada e ao papel passivo das instituições e que estas duas hipóteses se complementam. No entanto, se as instituições são relevantes, a racionalidade procedural é uma forma mais consistente de lidar com a racionalidade dos agentes. Em seguida, destaca que pelas instituições serem imperfeitas, fornecem incentivos imperfeitos aos agentes. Adiante, discute quais seriam as condições necessárias para que os mercados políticos incorressem em custos de transação nulos e afirma que a \textbf{democracia} é o que mais se aproxima dessas condições. Além disso, uma vez que são os mercados políticos que propõem e executam as regras econômicas, é esperado que as atribuições dos direitos de propriedade sejam pouco eficientes (ou raramente eficientes).

\subsection*{II}

O autor abre a seção resumindo as implicações das instituições.

\begin{itemize}
	\item Os modelos (econômicos e políticos) são específicos de um certo conjunto de particularidades institucionais. O reconhecimento de tais restrições é fundamental para o aprimoramento da teoria e das políticas públicas. Mais uma vez, pontua que se as organizações centram esforços em atividades pouco produtivas é porque as instituições incentivaram tais atividades.
	\item O reconhecimento da relevância das instituições fará com que os modelos comportamentais sejam questionados de modo que sejam levados em consideração aspectos procedurais em relação ao processamento de informação. Argumenta que as hipóteses de racionalidade substantiva e dos mercados eficientes ofuscou as implicações da incompletude de informações e complexidade do ambiente, bem como da percepção subjetiva dos agentes.
	\item Ideias e ideologias são relevantes enquanto as instituições determinam o grau desta relevância. Ideias e ideologias moldam os constructos mentais que os agentes usam para interpretar a realidade e tomar decisões. Em particular, as instituições alteram os preços a ser pago pelas ideias e convicções.
	\item Economia e política são necessariamente conectados. O conjunto de configurações institucionais determinam as relações entre ambos os mercados
\end{itemize}


\subsection*{III}

\autor argumenta que a inclusão das instituições no escopo neoclássico requer modificações nesta teoria enquanto a elaboração de um modelo que explique a mudança econômica requer a construção de um arcabouço teórico novo em que o conceito de \textbf{path-dependence} é fundamental. Em particular, o constructo teórico pretendido incorpora informação incompleta, rendimentos crescentes e subjetividades à teoria padrão. Dito isso, exemplifica este arcabouço por meio do exemplo das diferenças entre América Espanhola e Britânica.

\begin{description}
	\item[Contexto:] Pontua como os preços relativos associados à mudanças nas tecnologias militarem são diferentes entre Inglaterra e Espanha. Também destaca as formas de resolução da tensão entre centralização e decentralização dos poderes.
	\item[Arcabouço institucional] Argumenta que o parlamento britânico permitiu o desenvolvimento de um governo representativo e assegurou melhor os direitos de propriedade e o sistema jurídico
	\item[Implicações organizacionais] O parlamento britânico criou o Banco Central da Inglaterra, assim como um sistema fiscal mais vinculado aos impostos, permitindo tanto uma solvência fiscal maior como o desenvolvimento de um mercado de capitais mais robusto.
	\item[\textit{Path dependence}] O autor destaca que a saída da crise fiscal do século XVII foi diferente na Inglaterra se comparado à Espanha e que tal diferença está associado tanto ao arranjo institucional quanto ao \textit{path-dependence}
	\item[Consequências subsequentes] O autor descreve como esses elementos foram fundamentais para explicar as diferenças entre as colônias britânicas e espanholas nas Américas. Resumidamente, afirma que o desempenho dos EUA é caracterizado por um federalismo político, rigor contábil, estrutura básica dos direitos de propriedade. Em conjunto, essas características permitiram o desenvolvimento de um mercado de capitais e crescimento econômico. As colônias espanholas, por outro lado, perpetuaram a centralização e burocratização política herdadas de suas metrópoles.
\end{description}

\autor encerra o capítulo com a seguinte passagem:

\begin{quotation}
	The divergent paths established by England and Spain in the New
	World have not converged despite the mediating factors of common ideological influences. In the former, an institutional framework has evolved
	that permits the complex impersonal exchange necessary to political sta­bility and to capture the potential economic gains of modem technology.
	In the latter, pecsonalistic relationships are still the key to much of the
	political and economic exchange. They arc a consequence of an evolving 	institutional framework that produces neither political stabilicy nor con­sistent realization of the potential of modern technology.
\end{quotation}

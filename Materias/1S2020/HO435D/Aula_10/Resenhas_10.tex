\documentclass[11pt,lineno]{../style}



\newcommand{\autor}{North (1990) }

\title{
\large{Nova Economia Institucional}\vspace{2pt}\\
\Huge{\autor - \textit{Institutions, Institutional change and Economic Performance}}
}
\date{12 de Junho de 2020}

\author[$\ast$]{Gabriel Petrini}

\affil[$\ast$]{PhD Student at Unicamp.}

\keywords{
	Keyword1\\ 
	Keyword2\\
	Keyword3
}

\runningtitle{Resenha} % For use in the footer 

%% For the footnote.
\runningauthor{Petrini}

\begin{abstract}

\end{abstract}

\begin{document}

\maketitle
\marginmark
\thispagestyle{firststyle}
% Please add here a significance statement to explain the relevance of your work

\section*{Instituições, teoria e performance econômica}

\autor pontua que as instituições determinam a trajetória de longo prazo das economias e que é necessário elaborar um modelo de mudança institucional.

\subsection*{I}

Nesta seção, faz algumas críticas aos modelos neoclássico, sobretudo em relação às hipóteses de racionalidade limitada e ao papel passivo das instituições e que estas duas hipóteses se complementam. No entanto, se as instituições são relevantes, a racionalidade procedural é uma forma mais consistente de lidar com a racionalidade dos agentes. Em seguida, destaca que pelas instituições serem imperfeitas, fornecem incentivos imperfeitos aos agentes. Adiante, discute quais seriam as condições necessárias para que os mercados políticos incorressem em custos de transação nulos e afirma que a \textbf{democracia} é o que mais se aproxima dessas condições. Além disso, uma vez que são os mercados políticos que propõem e executam as regras econômicas, é esperado que as atribuições dos direitos de propriedade sejam pouco eficientes (ou raramente eficientes).

\subsection*{II}

O autor abre a seção resumindo as implicações das instituições.

\begin{itemize}
	\item Os modelos (econômicos e políticos) são específicos de um certo conjunto de particularidades institucionais. O reconhecimento de tais restrições é fundamental para o aprimoramento da teoria e das políticas públicas. Mais uma vez, pontua que se as organizações centram esforços em atividades pouco produtivas é porque as instituições incentivaram tais atividades.
	\item O reconhecimento da relevância das instituições fará com que os modelos comportamentais sejam questionados de modo que sejam levados em consideração aspectos procedurais em relação ao processamento de informação. Argumenta que as hipóteses de racionalidade substantiva e dos mercados eficientes ofuscou as implicações da incompletude de informações e complexidade do ambiente, bem como da percepção subjetiva dos agentes.
	\item Ideias e ideologias são relevantes enquanto as instituições determinam o grau desta relevância. Ideias e ideologias moldam os constructos mentais que os agentes usam para interpretar a realidade e tomar decisões. Em particular, as instituições alteram os preços a ser pago pelas ideias e convicções.
	\item Economia e política são necessariamente conectados. O conjunto de configurações institucionais determinam as relações entre ambos os mercados
\end{itemize}


\subsection*{III}

\autor argumenta que a inclusão das instituições no escopo neoclássico requer modificações nesta teoria enquanto a elaboração de um modelo que explique a mudança econômica requer a construção de um arcabouço teórico novo em que o conceito de \textbf{path-dependence} é fundamental. Em particular, o constructo teórico pretendido incorpora informação incompleta, rendimentos crescentes e subjetividades à teoria padrão. Dito isso, exemplifica este arcabouço por meio do exemplo das diferenças entre América Espanhola e Britânica.

\begin{description}
	\item[Contexto:] Pontua como os preços relativos associados à mudanças nas tecnologias militarem são diferentes entre Inglaterra e Espanha. Também destaca as formas de resolução da tensão entre centralização e decentralização dos poderes.
	\item[Arcabouço institucional] Argumenta que o parlamento britânico permitiu o desenvolvimento de um governo representativo e assegurou melhor os direitos de propriedade e o sistema jurídico
	\item[Implicações organizacionais] O parlamento britânico criou o Banco Central da Inglaterra, assim como um sistema fiscal mais vinculado aos impostos, permitindo tanto uma solvência fiscal maior como o desenvolvimento de um mercado de capitais mais robusto.
	\item[\textit{Path dependence}] O autor destaca que a saída da crise fiscal do século XVII foi diferente na Inglaterra se comparado à Espanha e que tal diferença está associado tanto ao arranjo institucional quanto ao \textit{path-dependence}
	\item[Consequências subsequentes] O autor descreve como esses elementos foram fundamentais para explicar as diferenças entre as colônias britânicas e espanholas nas Américas. Resumidamente, afirma que o desempenho dos EUA é caracterizado por um federalismo político, rigor contábil, estrutura básica dos direitos de propriedade. Em conjunto, essas características permitiram o desenvolvimento de um mercado de capitais e crescimento econômico. As colônias espanholas, por outro lado, perpetuaram a centralização e burocratização política herdadas de suas metrópoles.
\end{description}

\autor encerra o capítulo com a seguinte passagem:

\begin{quotation}
	The divergent paths established by England and Spain in the New
	World have not converged despite the mediating factors of common ideological influences. In the former, an institutional framework has evolved
	that permits the complex impersonal exchange necessary to political sta­bility and to capture the potential economic gains of modem technology.
	In the latter, pecsonalistic relationships are still the key to much of the
	political and economic exchange. They arc a consequence of an evolving 	institutional framework that produces neither political stabilicy nor con­sistent realization of the potential of modern technology.
\end{quotation}

\section*{Estabilidade e mudança na história econômica}

\autor abre o capítulo retomando que as instituições são criadas para reduzir incerteza nas interações humanas (trocas). Isso somado a tecnologia vigente, determinam os custos de produção (transformação e transação) e a lucratividade e factibilidade da participação da atividade econômica. Além disso, pontua que as instituições conectam o passado ao futuro de modo que o movimento histórico é majoritariamente incremental em que a atividade econômica é uma fração desta trajetória. O argumento é que as organizações econômicas, políticas, militares, etc surgem das oportunidades providas pelo arranjo institucional que, por sua vez, se altera incrementalmente.

\subsection*{I}

O autor inicia esta seção partindo do exemplo de relações de trocas simples em que o desenvolvimento de comércio de distâncias mais longas foi possível em função de alterações na estrutura econômica. Em seguida, afirma, é necessário um maior grau de especialização associado ao aumento do mercado. No último estágio, associado às economias ocidentais modernas, verifica-se um aumento da especialização, organizações com larga escala em todos os setores.

\subsection*{II}

Nesta seção, o autor pontua que se difere de Rostow uma vez que pretende avaliar as condições (instituições) necessárias associadas aos custos de produção para que haja um aumento na especialização e divisão do trabalho. Argumenta que na medida que o mercado aumenta, as trocas implicam maiores custos de transação uma vez que maiores são os recursos destinados à mensuração e execução das transações. Além disso, afirma que na medida que o volume das trocas aumenta, o problema da cooperação se torna cada vez maior. Em paralelo, a garantia dos direitos de propriedade requer organizações políticas e jurídicas para garantir os contratos alongo do tempo e espaço.
No entanto, argumenta que nada garante essa sequência de eventos e que o caso Europeu é um exemplo. Em seguida, afirma que a análise de sistemas mais primitivos de troca tem um objetivo duplo: (i) contrastar as forças que produziram estabilidade organizacional e institucional e; (ii) investigar as implicações sobre a dinâmica econômica.


\subsection*{III}

Nesta seção, o autor apresenta alguns exemplos de trocar ocorridas e ainda existentes no norte da África (suq) em que é caracterizada pela impessoalidade entre as partes. Em resumo, caracteriza este sistema de trocas pelos: (i) elevados custos de mensuração; (ii) tentativa contínua de clientalização; (iii) barganha intensiva em cada dimensão (vantagens decorrentes de ter mais informações que seus adversários). \autor questiona o porquê da existência de elementos ineficientes nesta forma de negociação.


\subsection*{IV}

O autor argumenta que tais organizações tribais --- e as habilidades e conhecimentos associados --- não induziam modificações produtivas do arranjo institucional. Em todos esses casos, continua, a fonte da mudança institucional é externa. Em seguida, analisa quais as inovações que reduziram os custos de transação. Estão associadas ao: (i) aumento da mobilidade de capital; (ii) redução dos custos de informação; (iii) possibilidade de \textit{spread} do risco. Dentre as inovações que aumentaram a mobilidade de capital, pontua técnicas que permitiam se esquivar da Lei de usura, métodos de descontos, negociação de títulos, controles sobre trocas de longas distâncias e transformação da incerteza em risco atuarial. Afirma também que essas inovações e mudanças institucionais estão associadas a duas forças: (i) economias de escala decorrente do aumento do volume das transações e (ii) desenvolvimento e aprimoramento de mecanismos que permitem a execução contratual. Dentre os determinantes das estruturas de execução contratual destaca a relação entre organizações mercantis e o Estado. Em especial, pontua que o Estado é um agente nesse movimento uma vez que permitiu o desenvolvimento de mercados de capitais e de instituições financeiras via negociação da dívida pública. Mais precisamente afirma: ``\textit{The
	shackling of arbitrary behavior of rulers and che development of imper­sonal rules that successfully bound both the state and voluntary organiza­
	tions were a key part of this institutional transformation.}''


\section{V}

O autor encerra o capítulo comparando as formas primitivas com às ocidentais modernas. Afirma que os agentes nas primeiras não induziram modificações no arranjo institucional em direção ao aumento da produtividade. Nas segundas, nota-se uma mudança incremental induzido pelos ganhos privados e aumentos de produtividade por meio de mudanças organizacionais e institucionais. Em seguida, enfatiza a importância de se conectar as mudanças na Europa ocidental com o estoque de conhecimento dos agentes e suas interações com a estrutura política e econômica. Encerra pontuando que os países baixos e a Inglaterra foram os portadores da mudanças institucional.
\section*{Incorporando a análise institucional a história econômica: prospecções e desafios}


\subsection*{I}

\autor abre o capítulo pontuando algumas crítica à teoria neoclássica em que destaca a ausência de fricções (custos de transação) e irrelevância das instituições. Apesar disso, afirma que as contribuições da teoria padrão à alocação e determinação de preços são válidas. Em seguida, menciona as contribuições de Marx em que destaca a interação entre mudança tecnológica e institucional. No entanto, afirma que o caráter utopista da teoria marxista implica uma direção positiva à mudança institucional (formal e informal) e que este não é necessariamente o caso. Adiante, faz menção a autores precliométricos por darem ênfase à tecnologia. A proposta de \autor, por outro lado, enfatiza a tentativa dos agentes solucionarem problemas de \textbf{coordenação}.


\subsection*{II}

Nesta seção, o autor discute os determinantes da produtividade em que destaca a necessidade de uma melhor compreensão destas questões. Argumenta que o aumento no estoque de capital e de conhecimento são essenciais para a melhora do bem-estar das sociedades. Menciona alguns modelos neoclássicos de crescimento e discute temas relacionados à teoria da dependência, centro-periferia, imperialismo e etc. Dentre as críticas a esse último conjunto de teorias, pontua a ausência de uma explicação de que o arranjo institucional existente produz desenvolvimentos desiguais. Tanto a teoria neoclássica quanto as teorias da ``exploração'' partem de agentes maximizadores de riqueza e são motivados pelos incentivos da estrutura institucional existente. A diferença entre eles, afirma, é que o arranjo institucional implícito nos primeiros geram mercados competitivos e crescimento enquanto nos segundos este arranjo permite explorar as demais economias.


\subsection*{III}

O autor inicia esta seção pontuando que os \textbf{incentivos} são os principais determinantes do desempenho de longo prazo das economias e que tais incentivos são implícitos de formas distintas em diferentes teorias. Argumenta que estes incentivos mudaram consideravelmente ao longo do tempo. Para tanto, ilustra a partir da história econômica americana em que destaca uma herança britânica não só histórica, mas também política, intelectual e institucional. Em linhas gerais, destaca a eficiência adaptativa da matriz institucional americana que gerou um ambiente político e econômico que incentivava as organizações a centrarem esforços em atividades produtivas, desenvolvimento de habilidades e conhecimentos. Destaca também as mudanças nos modelos mentais compartilhados entre os agentes e organizações por serem capazes de influenciar as consequências e os resultados da matriz institucional. Outro elemento que chama atenção é o retorno crescente das instituições e relação com o \textit{path dependence}.

\subsection*{IV}

Uma vez que as instituições determinam o desempenho econômico de longo prazo, o autor se pergunta quais são os determinantes de \textbf{instituições eficientes}. As restrições informais são resultado de transmissão de valores culturais, extensões das regras formais e aparentam ter uma influência mais persistente sobre a estrutura institucional. Pontua ainda que tais tradições são reforçadas por ideologias e, portanto, é necessário compreender o que faz tais convicções se alterarem. Em resumo, destaca que mudanças nos \textbf{preços relativos} alteram normas e ideologias e reduzem os custos de informação. Associado a isso, destaca que o ator político estão em uma posição de iniciar mudanças mais radicais e usa a Revolução Gloriosa como exemplo deste ponto uma vez que permitiu o desenvolvimento de mercados de capitais. Em seguida, destaca a possibilidade de tensão entre instituições formais e informais em que mudanças radicais nas primeiras podem ser acompanhadas de rigidezes nas últimas de modo a gestar uma instabilidade política de longo prazo. Dito isso, encerra o livro com a seguinte passagem:

\begin{quotation}
	One gets efficient institutions by a polity that has built-in incentives to
	create and enforce efficient property rights. But it is hard - maybe impos­sible --- to model such a polity with wealth-maximizing actors uncon­strained by other considerations [...] The state then becomes nothing more than a machine to redistribute wealth	and income [...] We need to know much more about culturally derived norms of behavior and how they interact with formal rules to get better
	answers to such issues. 
\end{quotation}


	
\begin{redbox}{Dúvidas e comentários}
	
Seria	\autor 	um etapista histórico?\footnote{Sem juízo de valor.} Digo isso tendo em mente os exemplos do capítulo 13. Sendo mais justo, o autor inclusive comenta que se distancia de Rostow e que pretende analisar as condições necessárias que permitem um maior grau de especialização e de divisão do trabalho. No entanto, a partir da leitura parece ter um caminho linear: Para que ocorra uma maior especialização e divisão do trabalho, é preciso assegurar direitos de propriedade, monitoramento e execução dos contratos que, por sua vez, requer um sistema jurídico e político estabelecido em decorrência de uma maior divisão do trabalho, maiores custos de transação, $\ldots$

A interação entre instituições formais, informais e organizações parece diminuir esse tom etapista de North, mas ainda sim parece presente.
\end{redbox}

\end{document}

\section*{Incorporando a análise institucional a história econômica: prospecções e desafios}


\subsection*{I}

\autor abre o capítulo pontuando algumas crítica à teoria neoclássica em que destaca a ausência de fricções (custos de transação) e irrelevância das instituições. Apesar disso, afirma que as contribuições da teoria padrão à alocação e determinação de preços são válidas. Em seguida, menciona as contribuições de Marx em que destaca a interação entre mudança tecnológica e institucional. No entanto, afirma que o caráter utopista da teoria marxista implica uma direção positiva à mudança institucional (formal e informal) e que este não é necessariamente o caso. Adiante, faz menção a autores precliométricos por darem ênfase à tecnologia. A proposta de \autor, por outro lado, enfatiza a tentativa dos agentes solucionarem problemas de \textbf{coordenação}.


\subsection*{II}

Nesta seção, o autor discute os determinantes da produtividade em que destaca a necessidade de uma melhor compreensão destas questões. Argumenta que o aumento no estoque de capital e de conhecimento são essenciais para a melhora do bem-estar das sociedades. Menciona alguns modelos neoclássicos de crescimento e discute temas relacionados à teoria da dependência, centro-periferia, imperialismo e etc. Dentre as críticas a esse último conjunto de teorias, pontua a ausência de uma explicação de que o arranjo institucional existente produz desenvolvimentos desiguais. Tanto a teoria neoclássica quanto as teorias da ``exploração'' partem de agentes maximizadores de riqueza e são motivados pelos incentivos da estrutura institucional existente. A diferença entre eles, afirma, é que o arranjo institucional implícito nos primeiros geram mercados competitivos e crescimento enquanto nos segundos este arranjo permite explorar as demais economias.


\subsection*{III}

O autor inicia esta seção pontuando que os \textbf{incentivos} são os principais determinantes do desempenho de longo prazo das economias e que tais incentivos são implícitos de formas distintas em diferentes teorias. Argumenta que estes incentivos mudaram consideravelmente ao longo do tempo. Para tanto, ilustra a partir da história econômica americana em que destaca uma herança britânica não só histórica, mas também política, intelectual e institucional. Em linhas gerais, destaca a eficiência adaptativa da matriz institucional americana que gerou um ambiente político e econômico que incentivava as organizações a centrarem esforços em atividades produtivas, desenvolvimento de habilidades e conhecimentos. Destaca também as mudanças nos modelos mentais compartilhados entre os agentes e organizações por serem capazes de influenciar as consequências e os resultados da matriz institucional. Outro elemento que chama atenção é o retorno crescente das instituições e relação com o \textit{path dependence}.

\subsection*{IV}

Uma vez que as instituições determinam o desempenho econômico de longo prazo, o autor se pergunta quais são os determinantes de \textbf{instituições eficientes}. As restrições informais são resultado de transmissão de valores culturais, extensões das regras formais e aparentam ter uma influência mais persistente sobre a estrutura institucional. Pontua ainda que tais tradições são reforçadas por ideologias e, portanto, é necessário compreender o que faz tais convicções se alterarem. Em resumo, destaca que mudanças nos \textbf{preços relativos} alteram normas e ideologias e reduzem os custos de informação. Associado a isso, destaca que o ator político estão em uma posição de iniciar mudanças mais radicais e usa a Revolução Gloriosa como exemplo deste ponto uma vez que permitiu o desenvolvimento de mercados de capitais. Em seguida, destaca a possibilidade de tensão entre instituições formais e informais em que mudanças radicais nas primeiras podem ser acompanhadas de rigidezes nas últimas de modo a gestar uma instabilidade política de longo prazo. Dito isso, encerra o livro com a seguinte passagem:

\begin{quotation}
	One gets efficient institutions by a polity that has built-in incentives to
	create and enforce efficient property rights. But it is hard - maybe impos­sible --- to model such a polity with wealth-maximizing actors uncon­strained by other considerations [...] The state then becomes nothing more than a machine to redistribute wealth	and income [...] We need to know much more about culturally derived norms of behavior and how they interact with formal rules to get better
	answers to such issues. 
\end{quotation}

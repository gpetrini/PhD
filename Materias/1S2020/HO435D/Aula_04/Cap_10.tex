\section*{Capítulo 10: A organização do trabalho}

\newcommand{\autor}{Williamson}

O autor abre o capítulo pontuando que a teoria neoclássica só passa a analisar a organização do trabalho quando se trata de uma estrutura de mercado oligopolista. A NEI, por outro lado, argumenta que a \textbf{estrutura de governança} que organiza o trabalho é relevante.

\subsection*{Questão principal}

Tal como nos demais capítulos, \autor argumenta uma teoria do contrato é aplicável a todos tipos de transação, o que inclui negociações no mercado de trabalho.

\subsection*{Uma abordagem abstrata}

A economia dos custos de transação estabelece que as estruturas de governança devem se adequar aos atributos de suas transação para que tenham efeitos de reduzir os custos de transação. O mesmo se aplica para os sindicatos e demais formas de organização laboral.

\subsubsection{Dimensões}

\begin{description}
	\item[Governança] \autor argumenta que as estruturas de governança devem ser elaboradas para dar conta do grau de especificidade do ativo (trabalho) e que as formas de negociação dependem da frequência que ocorrem. Quanto mais incerte e mais contínua é uma transação, mais terá que se adaptar.
	\item[Mensuração] \autor discute formas de medir os custos de transação e as dificuldades decorrentes da imensurabilidade da contribuição individual de cada trabalhador ao produto final. Além disso, pontua a organicidade entre os trabalhadores e as subsequentes dificuldades de substituição decorrentes dela.
	\item[Combinação provisõria] Nesta seção, \autor propõe uma forma de avaliar as estruturas de governança de acordo com a especificidade do ativo ($k_0$ e $k_1$) e separabilidade ($S_0$, $S_1$). As combinações são:
	\begin{description}
		\item[Mercado Spot ($k_0$, $S_0$)] Nem empregado ou empregador se beneficiam com a manutenção da associação
		\item[Equipe primitiva ($k_0$, $S_1$)] Por mais que os trabalhadores sejam pouco especificados, a mensuração da contribuição individual é dificultada por conta 
		\item[Mercado obrigatório ($k_1$, $S_0$)] Diz respeito à atividades específicas de cada firma cuja atividade individual é de fácil mensuração. Tanto firma quanto empregador se beneficiam da continuidade desta transação. Recomenda-se salvaguardas e penalidades pecuniárias para desestimular demissões arbitrárias
		\item[Equipe conectada ($k_1$, $S_1$)] Autoexplicativo
	\end{description}
\end{description}

\subsubsection*{Desapropriação por Trabalhadores}

\begin{description}
	\item[Ativos a serem desapropriados] Trabalhadores irão disputar as quase-rendas geradas por sua especificidade encorporada em seu capital humano. As quase-rendas potencial que podem se apropriar é determinada por seus ativos individuais
	\item[Contratos de trabalho] Resumidamente, pontua que tais contratos costumam ser mais flexíveis do que aqueles entre firmas e, portanto, existem vantagens em relação ao uso do mercado
	\item[Governança] As governanças precisam ser flexíveis a ponto de evitar riscos de desapropriação
\end{description}

\subsection*{Organização laboral}

\subsubsection*{Negociações privadas}

Nesta subseção, \autor retoma o esquema apresentado no capítulo 1 em que o tipo de transação está associado com a especificidade dos ativos. Se opta-se por uma negociação pouco específica, a estrutura de governança será simples, caso contrário, devem ser feitas escolhas de salvaguardas.

\subsubsection*{Faces de uma organização laboral}

\begin{description}
	\item[Monopólio] \autor analisa os tipos de sindicados com objetivos de aumentar salários via controle da oferta de trabalho: classe, artesanais e industriais
	\item[Eficiência] Os sindicatos podem ter propósitos de eficiência uma vez que permitem funções de agência, propor estruturas de governança, ser uma fonte de informação sobre as necessidades e preferências dos trabalhadores e auxiliar a avaliação de ofertas de trabalho. Em seguida, argumenta que o incentivo da organização laboral aumenta com a especificidade dos trabalhadores e o mesmo pode ser dito sobre forma da estrutura de governança interna. Além disso, pontua que a criação de uma unidade de governança para avaliar abusos é de interesse de ambas as partes.
	\item[Voz] Resumidamente, sindicatos são instituições políticas que representam tanto as aspirações quanto os interesses políticos de seus membros.
\end{description}

\subsubsection*{Poder}

\autor pontua a necessidade de se analisar os contratos em sua totalidade e não de forma individual e pontual. Além disso, destaca que um erro comum na literatura é avaliar o jogo de formas a partir dos ganhos de uma negociação, ou seja, de um confronto individual.

\begin{description}
	\item[Risco não diversificado] O autor pontua que o grau de não diversificação aumenta com o grau de especialização. Os trabalhadores podem escolher entre trabalhos de propósitos gerais ou específicos.
	\item[Reputação] Os custos de se encerrar uma relação contratual são dispensáveis somente em uma relação pouco específica. Ao longo do processo de contratação, os candidatos são analisados de acordo com sua reputação.
\end{description}


\subsection*{Características problemáticas da organização laboral}

\begin{description}
	\item[Poder de monopólio] Organizações laborais possibilitam os trabalhadores tenham melhores condições de barganha. Firmas com que possuem mais ativos não-humanos são mais resistentes à organização sindical.
	\item[Oligarquia] A liderança dos sindicatos, assim como a liderança de outras grandes organizações, está frequentemente em posição de se consolidar e / ou buscar interesses.
	\item[Heterogeneidade] Destaca a dificuldade de se chegar em um acordo na medida que aumenta a heterogeneidade da força de trabalho
\end{description}

\subsection*{Conclusões}


\begin{itemize}
	\item Estruturas de governança especializadas são dispensáveis em transações de trabalho pouco específicas. Caso surjam, as uniões laborais surgirão tardiamente
	\item Mercado de trabalho especializado e sem salvaguardas está sujeito à expropriações e não é estável
	\item Mercados especializados e com salvaguardas são aqueles que organizações coletivas são mutuamente acordadas uma vez que ambas as partes se beneficiam dela. As estruturas de governança possuem maior grau de elaboração
\end{itemize}
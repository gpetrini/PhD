\documentclass[9pt,twocolumn,twoside,lineno]{style}

\articletype{NEI} % article type

\title{Langlois (1996, Capítulo 1): The New Institutional Economics: an Introductory Essay}
\date{13 de março de 2020}

\author[$\ddagger$]{Gabriel Petrini}

\affil[$\ddagger$]{Doutorando no instituto de Economia da Unicamp}

\keywords{Keyword \\ Keyword2 \\ Keyword3 \\ ...}

\runningtitle{Fichamento} % For use in the footer 

%% For the footnote.
\runningauthor{Petrini}

\begin{abstract}
\end{abstract}

\begin{document}

\maketitle
\articletypemark
\marginmark
\thispagestyle{firststyle}

% Please add here a significance statement to explain the relevance of your work
%\afterpage{
%	\begin{sigstatement}
%		\sffamily
%		\mdfdefinestyle{stylesigstyle}{linewidth=0.7pt,
%			backgroundcolor=styleblueback,linecolor=stylebluetext,
%			fontcolor=stylebluetext,innertopmargin=6pt,innerrightmargin=6pt,
%			innerbottommargin=6pt,innerleftmargin=6pt}
%		{%	
%			\begin{mdframed}[style=stylesigstyle]%
%				\section*{5 Seconds Synthesis}%
%				\lipsum[1-3]
%		\end{mdframed}}
%\end{sigstatement}
%}

% If your first paragraph (i.e. with the \dropcap) contains a list environment (quote, quotation, theorem, definition, enumerate, itemize...), the line after the list may have some extra indentation. If this is the case, add \parshape=0 to the end of the list environment.
	
\section{Introdução}

Nesta seção, o autor destaca que o objetivo deste capítulo é unir os temas presentes no livro no paradigma da Nova Economia Institucional (NEI), em particular, dando ênfase em como os autores compreendem alguns ``fenômenos econômicos''\footnote{Pontuam também que não tem por objetivo criticar a ortodoxia.}.
	
\section{Velho e novo institucionalismo}

Discute a adequabilidade do termo ``institucionalismo'' ao incorporar um conjunto distinto de ideias. Além disso, destaca a dificuldade de se comparar este institucionalismo que está surgindo com a tradição que o precedeu. Este ``velho'' institucionalismo, em particular, era pouco estruturado mas, no entanto, possui semelhanças com a NEI. Dentre elas, destaca-se a necessidade de se incluir contribuições de outras ciências sociais e que os fenômenos econômicos não devem ser analisados de forma estática, mas como um \textbf{processo histórico} e ``evolucionário''.

Dito isso, Langlois apresenta a crítica de \underline{Veblen} ao marginalismo que, em linhas gerais, discorda do tratamento da \textbf{natureza humana} desses autores por partir de uma psicologia hedonista ultrapassada e subsequente criação do \textit{homo economicus}. Além disso, Veblen associa o tratamento marginalista à abordagem Newtonian em que o agente econômico se torna \textbf{passivo} invés de ser um agente da mudança. Como contraponto, destaca que o comportamento do agente econômico está sujeito às convenções e instituições sociais. No entanto, argumenta Langlois, o método utilizado por Veblen e pelos velhos institucionalistas não era consistente com a própria crítica feita aos marginalistas.

Partindo do método materialista, Veblen tenta eliminar as motivações econômicas do comportamento humano. Exite, portanto, uma tensão nunca resolvida entre sua retórica humanista e suas hipóteses comportamentais, comprometendo a alternativa evolucionária ao marginalismo. Além disso, afirma que a crítica de Veblen ao marginalismo não se aplica à Menger que, por sua vez, estava interessado nas instituições e na ausência de equilíbrio. Em linhas gerais, a proposta de Menger não era de ignorar as instituições, mas sim, argumentar que são um fenômeno social \textbf{por si só} e precisariam de uma explicação teórica adequada. Sendo assim, Menger e não os velhos institucionalistas é um dos pais da NEI. A passagem seguinte resume o que foi discutido (p.~5):

\begin{quote}
	\textit{The problem with the Historical
	School and many of the early Institutionalists is that they wanted an
	economics with institutions but without theory; the problem with many
	neoclassicists is that they want economic theory without institutions;
	\textbf{what we should really want is both institutions and theory}}.
\end{quote}

\section{Temas emgergentes}

Segue uma lista dos principais temas:

\begin{itemize}
	\item O agente econômico é racional, mas não no sentido de maximizador;
	\item O fenômeno econômico é resultado de um processo de aprendizado dos agentes econômicos;
	\item A coordenação da atividade econômica não se resume a intermediação pelos preços, mas é resultado de um conjunto de instituições sociais que, por sua vez, são relevantes para a investigação econômica.
\end{itemize}

\section{Racionalidade}

Resumidamente, Langlois apresenta a controvérsia em torno do conceito de racionalidade marginalista. Um contraponto é o de Williamson que parte do conceito de racionalidade de Simon (orgânica) que, por sua vez é compatível com a abordagem evolucionária.

\section{O papel dual das instituições}

\subsection{Competição como um processo}

Langlois argumenta que o processo de \textbf{competição} é um bom exemplo de como as instituições são relevantes. A teoria da competição, por sua vez, parte dos \textbf{direitos de propriedade} como uma alternativa a competição marginalista. A abordagem da NEI, --- diferentemente da abordagem neoclássica em que a competição está associada à noção de equilíbrio (consequência da consistência lógica das relações matemáticas) --- trata a competição enquanto um processo que se passa sequencialmente no \textbf{tempo histórico}. Em outras palavras, enquanto a teoria marginalista dá ênfase à alocação de recursos via o caráter disciplinador/restritivo da competição,\footnote{Dito de outro modo, a competição nos termos marginalistas elimina a discricionariedade do agente econômico. Sendo assim, quanto mais imperfeita a competição, maior o grau de discricionariedade dos agentes.} a abordagem NEI entende a competição como um \textbf{processo dinâmico}, associado tanto à inovação e à mudança quanto à livre entrada de capitais. Uma consequência desta abordagem dinâmica da competição é a maior atenção dada a estruturas de mercado oligopolistas em que ações tomadas se assemelham com ``práticas monopolistas''.

Dito isso, Langlois argumenta que não são duas formas distintas de se \textbf{entender} a competição e \textbf{não duas formas de competição}. Uma delas é a competição enquanto estado de coisas (``\textit{state of affairs}'', concentração por exemplo), outro é enquanto um \textbf{processo}. Adiante, argumenta que uma abordagem \textit{à la} NEI dá mais ênfase a \textbf{questões normativas} do que a políticas associadas à estrutura de mercado\footnote{Uma das razões é que a estrutura de mercado se torna exógena ao partir de uma abordagem do tipo Estrutura-Conduta-Desempenho.}.

Em seguida, o autor discute a \textbf{hipótese Schumpeteriana} de que as firmas maiores inovam mais do que as menores. Argumenta que a inovação permite que as firmas se tornem maiores, e não o inverso e, portanto, as firmas maiores parecem mais inovativas. Adiante, discute os mecanismos de seleção tal como proposto por \underline{Nelson \& Winter} (e Hayek!) e afirma que esta é abordagem relevante ao se tratar da importância das instituições por evidenciar as implicações de diferentes arranjos institucionais. Além disso, argumenta que tal \textbf{abordagem comparativa} contrasta com a marginalista uma vez que esta última deriva conclusões normativas dos teoremas de bem-estar de equilíbrio geral. Mais especificamente, a abordagem institucional-comparativa é mais incompatível com questões associadas à barreiras à entrada. Adiante, argumenta que os \textbf{direitos de propriedade} são uma forma de analisar a competição em termos institucionais, uma vez que as \textbf{barreiras legais à entrada} influenciam a competição.

\subsection{Evolução das instituições sociais}

No início da seção, Langlois retoma a discussão anterior em que afirma que a estrutura de mercado é resultado da trajetória associada aos direitos de propriedade. Além disso, argumenta que o arranjo institucional não é completamente exógeno, mas \textbf{emergem} de um processo social. Dito isso, direciona a discussão para a contribuição de \underline{Coase}, em que a \textbf{distribuição dos direitos de propriedade} é relevante na presença de custos de \textbf{transação}.
 
A seguir, o autor destaca dois grandes grupos da NEI. O primeiro enfatiza que as instituições são \textbf{instâncias} de contratos ``\textit{market-like}'' entre indivíduos e o segundo que trata as instituições como \textbf{alternativas} desses contratos. Adiante, Langlois faz uma discussão sobre a definição de instituições em que destaca a categorização de Andrew Schotter que, diga-se de passagem, é baseada na teoria dos jogos e é reproduzida abaixo (grifos adicionados):

\begin{quote}
	``A social institution is a \textbf{regularity in social behavior} that is agreed to by all members of society, specifies behavioring specific recurrent situations,
	and is either \textbf{self-policed or policed by some external authority}
\end{quote}
Em seguida, faz uma distinção entre normas sociais e normas intra-firma em que, seguindo \underline{Hayek}, o primeiro tipo é abstrato enquanto o segundo são mais objetivos, mas ambos são instituições. Apesar das diferenças, o que permite enquadrar ambos os tipos de normas enquanto instituições é que são \textbf{regularidades comportamentais} compreensíveis em termos de normas/regras.

Dito isso, o autor discute a abordagem de \textbf{custos de transação}, destacando dois problemas:
\begin{description}
	\item[Desequilíbrio] Associado com o \textit{trade-off} entre \textbf{flexibilidade} e \textbf{eficiência} sob concorrência dinâmica (Schumpeteriana). Destaca que boa parte dos trabalhos pressupõe que a \textbf{eficiência} e não a flexibilidade é uma pré-condição para a sobrevivência da organização;
	\item[\textit{Path-Dependency}] Relacionado com o ponto anterior, afirma que uma questão relevante não é se uma instituição é eficiente agora, mas se é eficiente ao longo do processo evolucionário, ou seja, identificar o processo que selecionam as estruturas de governança.
\end{description}
De todo modo, argumenta que estes problemas surgem ao se interpretar a análise comparativa dos custos de transação enquanto \textbf{explicação para a origem} das instituições. Em linhas gerais, a análise institucional-comparativa de Williamson e a abordagem orgânica de Menger são complementares apesar de suas diferenças.
	
\end{document}
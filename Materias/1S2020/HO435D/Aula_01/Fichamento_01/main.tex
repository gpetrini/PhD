\documentclass[9pt,twocolumn,twoside,lineno]{style}

\articletype{NEI} % article type

\title{Furquim de Azevedo (1997): Capítulo 1 Antecedentes}
\date{13 de março de 2020}

\author[$\ddagger$]{Gabriel Petrini}

\affil[$\ddagger$]{Doutorando no instituto de Economia da Unicamp}

\keywords{Nova Economia Institucional \\ Coase \\ Custos de transação}

\runningtitle{Fichamento} % For use in the footer 

%% For the footnote.
\runningauthor{Petrini}

\begin{abstract}

\end{abstract}

\begin{document}

\maketitle
\articletypemark
\marginmark
\thispagestyle{firststyle}

% Please add here a significance statement to explain the relevance of your work
%\afterpage{
	\begin{sigstatement}
		\sffamily
		\mdfdefinestyle{stylesigstyle}{linewidth=0.7pt,
			backgroundcolor=styleblueback,linecolor=stylebluetext,
			fontcolor=stylebluetext,innertopmargin=6pt,innerrightmargin=6pt,
			innerbottommargin=6pt,innerleftmargin=6pt}
		{%	
			\begin{mdframed}[style=stylesigstyle]%
				\section*{5 Seconds Synthesis}%
	Ao longo deste capítulo, Furquim apresenta alguns expoentes da Nova Economia Institucional (NEI), bem como alguns de seus conceitos, contribuições e direções. Em linhas gerais, destaca-se a contribuição de Coase, o conceito de custos de transação, a importância das organizações e o questionamento da hipótese comportamental da ``racionalidade ilimitada'' utilizada pela ortodoxia na época. Outros elementos centrais a NEI são os direitos de propriedade e a especificidade dos ativos.
		\end{mdframed}}
\end{sigstatement}
%}

	
\section{Coase e os anos 30: o redirecionamento do enfoque}

O autor inicia o capítulo destacando os principais contribuições para a construção do que se entende por Nova Economia Institucional (NEI). Dentre eles, pontua a importância de \underline{Commons} por enfatizar a \textbf{transação} como unidade de análise como contraponto da firma enquanto unidade indivisível. Dito isso, os princípios básicos da transação são:

\begin{itemize}
	\item Conflito;
	\item Mutualidade e;
	\item Ordem
\end{itemize}	
Em seguida, segue para a contribuição de \underline{Knight} ao distinguir \textbf{risco de incerteza} e, principalmente, por destacar a \textbf{redução do desperdício} como motivo para se compreender as organizações econômicas. Dito isso, segue para \underline{Hayek} por enfatizar as \textbf{adaptações} no ambiente econômico. Por fim, pontua que apesar desses autores serem importantes, destaca que foi \underline{Coase} quem deu a contribuição mais importante para a formação da NEI por explicar a \textbf{gênese da firma}\footnote{O autor destaca que antes, a firma era vista como uma instância que transforma insumos em produtos apenas.}.
É a partir da contribuição de Coase que a firma passou a ser entendida como um espaço para a \textbf{coordenação} dos agentes econômicos \textbf{alternativo} ao mercado.

Feito este panorama, o autor afirma que Coase direcionou sua análise para duas formas abstratas de coordenação: firma e o mercado. Em linhas gerais, ambas possuem a função de \textbf{coordenar} a atividade econômica. Em seguida, apresenta uma primeira aproximação ao conceito de \textbf{custos de transação}. Resumidamente, pode ser compreendido como custos associados a utilização de determinada forma de organização e seriam de dois tipos:
\begin{itemize}
	\item Custos de coleta de informação e;
	\item Custos de negociação e de estabelecimento de um contrato
\end{itemize}
Adiante, o autor pontua alguns esforços na direção de tornar a NEI uma ciência tal como entendido por K. Poper, ou seja, tentativa de torná-la \textbf{falseável}.
Feito isso, argumenta que o principal legado de Coase é o enriquecimento da visão da firma que passa a ser entendida como um ``\textbf{complexo de contratos} regendo transações internas''.

\section{Economia da Informação: as bases vindas da ortodoxia}

Nessa seção, o autor discute alguns caminhos da ortodoxia e enfatiza o questionamento da \textbf{informação perfeita} e alguns avanços como:
\begin{itemize}
	\item Assimetria de informação;
	\item Teoria dos contratos;
	\item Teoria do Agente-Principal
\end{itemize}
assim como alguns conceitos dada a \textbf{divergência de interesses}, como:
	\begin{description}
		\item[\textit{Moral Hazard}] comportamento pós-contratual da parte que possui uma informação privada e pode dela tirar proveito em prejuízo a(s) sua(s) contraparte(s). Uma condição necessária para que se verifique \textit{moral hazard} é, portanto, a assimetria de informações.
		\item[Seleção adversa] Associado à adesão (ou não) a uma determinada transação e não mais uma relação pós-contratual. \textbf{Solução:} Sinalização.
		\item[Oportunismo] Introdução de um comportamento aético e seus respectivos custos.
	\end{description}

\section{Arrow e a Economia das Organizações}

Nessa seção, Furquim pontua a contribuição de \underline{Arrow} no entendimento das organizações enquanto formas de se obter benefícios --- \textit{dada uma falha no sistema de preços} --- advindos de uma \textbf{ação coletiva}. Além disso, este autor destaca que o mercado é mais sensível à assimetria de informações do que as organizações. Em seguida, afirma que dada a \textbf{incerteza}, o sistema de preços \textit{por si só} se torna complexo ao ponto de ser inviabilizado. Tal constatação permitiu o questionamento do conceito de racionalidade ilimitada tão usual na ortodoxia. Dito isso, conclui que as organizações podem alterar os custos de transação implícitos se utilização enquanto \textbf{instrimentos de coordenação}.


\section{Simon: redefinindo o agente econômico}

Ao longo desta seção, Furquim destaca a importância de Simon ao questionar o conceito de racionalidade ilimitada da ortodoxia e cujas contribuições para a NEI são:
\begin{itemize}
	\item Racionalidade limitada;
	\item Seleção de formas organizacionais e;
	\item Análise estrutural discreta.
\end{itemize}
Neste ponto, vale destacar que uma das implicações da racionalidade limitada é a \textbf{incompletude dos contratos} uma vez que é incapaz de antecipar todos os resultados e adversidades possíveis. Em seguida, Furquim pontua que Coase argumenta que as instituições --- notadamente a firma --- existem para economizar os custos de transação e que somente as formas organizacionais mais eficientes sobrevivem.

Dito isso, Furquim passa a descrever os processos de seleção de formas complexas introduzidas por Simon, são elas:
\begin{itemize}
	\item Seleção por tentativa e erro
	\item Seleção por experiência prévia\footnote{Esta forma depende do grau de similaridade com as situações passadas.}
\end{itemize}
Em resumo, os processos de seleção são \textbf{ativos}, contraponto ao processo darwinista. Por fim, destaca a ênfase dada por Simon para análises \textbf{qualitativas} enquanto centro de estudo (e não a teoria dos preços).

\section{Alchian \& Demsetz: a economia dos direitos de propriedade}

Nesta seção, é destacada a importância de se incluir os direitos de propriedade. Em linhas gerais, tal inclusão está associada com o entendimento de que uma transação é resultado de uma troca de direitos e estão relacionados à ocorrência de \textbf{externalidades}, sejam elas positivas ou negativas. Dada a relavância da contribuição de Alchian \& Demsetz, é analisado mais pormenorizadamente em quatro passos:

\begin{description}
	\item[\textit{team production}] Em uma firma, é possível captar os ganhos gerados pela organização cooperativa. Resumidamente, o todo é maior que a soma das partes.
	\item[``Indivisibilidade das contribuições''] Dada a existência da \textit{tem production}, torna-se mais difícil identificar qual a contribuição individual no todo. Sendo assim, na ausência de mecanismos de controle, a produção cooperativa tem o \textbf{estímulo à preguiça} como contrapartida;
	\item[Necessidade de mecanismos disciplinadores] Decorrência do passo anterior, a produção cooperativa necessita de mecanismos que discipline o comportamento dos agentes e;
	\item[Organização econômica e direitos de propriedade] A garantia do \textbf{incentivo à supervisão} decorre da distribuição dos ganhos extras advindos da produção ao supervisor.
\end{description}

\section{Williamson,Klein et alii: dimensionalizando as transações, o papel da especificidade de ativos}

Partindo da contribuição de \underline{Commons}, \underline{Williamson} procurou \textbf{atribuir dimensões} objetivas e observáveis às transações. Sendo assim, a contribuição de Coase poderia ser testada e, consequentemente, permitiria deduzir o nível de custos de transação, bem como a forma organizacional eficiente. Para tanto, Williamson introduz o conceito de \textbf{especificidade dos ativos} cuja definição é reproduzida abaixo (p.~50):

\begin{quote}
 Se uma determinada transação implica investimentos que lhe são
 específicos --- não podendo ser utilizados de forma alternativa
 sem uma perda considerável ---, a parte que arcou com esses investimentos fica em uma posição especialmente sujeita a alguma ação
 oportunista das demais partes.
\end{quote}
Em linhas gerais, a especificidade dos ativos está associada com os custos de se abrir mão dessa transação, assim como os riscos de se preservá-la. Em seguida, argumenta Furquim, os autores relacionam a especificidade dos ativos com a seleção da \textbf{forma organizacional} para gerir essa transação. Como consequência desse conceito, a proposição de Coase poderia ser testada.

Adiante, Furquim apresenta a contribuição de Klein et alli ao explicitarem outras dimensões das transações que são relevantes para a determinação da forma organizacional:
\begin{description}
	\item[Incerteza] Decorrência da impossibilidade de se estabelecer um sistema completo de contratos e, assim, se esquivar das possibilidades de ações oportunistas (de ambas as partes). Os custos associados a essas ações são, portanto, custos de transação;
	\item[Expectativas de crescimento da demanda] Dizem respeito aos custos da ação oportunista e menos à sua probabilidade de ocorrência.
\end{description}

\end{document}
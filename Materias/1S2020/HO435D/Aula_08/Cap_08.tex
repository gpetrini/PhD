\section*{Instituições e custos de transação e de transformação (Produção?)}

O autor abre o capítulo realçando que a definição de direitos de propriedade e execução de acordos requer recursos e que instituições, bem como o estado da tecnologia determinam os custos de \textbf{transação}. Desse modo, instituições desempenham um papel fundamental na determinação dos custos de \textbf{produção}. Em seguida, retoma que uma hierarquia de regras (tomada em conjunto) determina a estrutura formal de direitos em uma transação. Além disso, pontua a incompletude dos contratos junto da importância de instituições informais --- reputação, padrões de conduta aceitos, convenções, etc --- nos acordos.

\subsection*{I}

Nesta seção, \autor investiga uma única troca à luz dos elementos elencados anteriormente, avaliando a transferência de propriedade residencial nos EUA\footnote{Dentre as particularidades deste mercado, destaca tanto a hierarquia de regras formais que delegam poder aos Estados quanto as instituições do mercado de hipotecas.}. Em linhas gerais, as instituições definem o quão custosa será esta transação uma vez que são necessários recursos para mensurar tanto os atributos legais quanto físicos do bem a ser transacionado; custos de policiar e executar o acordo e; incerteza associada ao grau de imperfeição das etapas anteriores. Contraponto tal proposta ao paradigma neoclássico, o autor pontua que esta tradição não só considera que a informação é perfeita como a garantia dos direitos de propriedade. Em seguida, o autor pontua que as instituições também podem \textbf{aumentar} os custos de transação. Os países que tem um melhor desempenho econômico são aqueles cujas instituições, em média, reduzem os custos de transação.

\subsection*{II}

Retoma que as instituições afetam tanto os custos de transação quanto os de transformação. A influência das instituições neste último tipo de custo decorre dos seus efeitos sobre a tecnologia adotada. Tal como discutido anteriormente, as instituições podem atuar tanto promovendo atividade pró-produtividade quanto inibindo-as. Para ilustrar a importância da definição dos direitos de propriedade, compara as economias avançadas com às do terceiro mundo. Por um lado, o arranjo institucional aumenta os custos de transação, por outro, direitos de propriedade de propriedade inibem tecnologias intensivas em capital fixo. Como consequência, firmas são menores onde os direitos de propriedades são mal estabelecidos.No que diz respeito às instituições e ao mercado de trabalho, afirma:

\begin{quote}
	The unique feature of labor markets is that institutions are
	devised to take into account that the quantity and quality of output are
	influenced by the attitude of the productive factor - hence morale build­ing is a subsitute at the margin for investing in more monitoring.
\end{quote}

\subsection*{III}

Nesta seção, o autor ressalta as instituições políticas dada sua relevância em definir as instituições formais. Como discutido em outros capítulos, organizações surgem para extrair as oportunidades criadas pelas instituições e este é o caso das firmas. Em sociedades em que as instituições formas e os direitos de propriedade não são bem estabelecidos, o tamanho das firmas tenderá a ser menor e as firmas maiores existirão sob a tutela do Estado.

\subsection*{IV}

Nesta seção, o autor pontua as implicações do arcabouço teórico apresentado até então:

\begin{itemize}
	\item As restrições institucionais que definem as oportunidades a serem exploradas pelos agentes são de natureza formal e informal. As instituições formais serão alteradas somente quando o interesse daqueles que possuem elevado poder de barganha for atingido com a mudança. Além disso, esta interação entre instituições formais e informais permite que ocorram mudanças incrementais;
	\item As instituições (formais e informais) refletem os custos de mensuração e execução. Quanto maiores estes custos, maior a importância das instituições informais para delinear a transação. Verticalização é uma das formas de reduzir tais custos. Na medida em que as restrições informais dominem as formas de troca, elas geralmente adotam a forma de criar formas de contornar a probabilidade de deserção pela outra parte.
	\item Os custos de transação são a dimensão mais observável do arranjo institucional, no entanto a mensuração dos custos decorrentes de uma instituição em particular é de difícil precisão;
	\item Arranjo institucional é fundamental para o desempenho econômico. No entanto, algumas instituições podem aumentar os custos de transação.
\end{itemize}
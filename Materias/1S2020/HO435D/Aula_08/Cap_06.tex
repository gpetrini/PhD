\section*{Restrições formais}

\autor abre o capítulo afirmando que as sociedades se tornam mais complexas na medida que se tornam mais especializadas e que tal movimento está associado com um maior grau de \textbf{formalização das restrições}. Em seguida, pontua que tais restrições complementam e até mesmo melhoram a eficácia das instituições informais uma vez que reduzem os custos de soluções para relações mais complexas. Além disso, as regas formais também podem modificar, revisar e substituir as informais.

\subsection*{I}

As instituições formais incluem regras políticas, jurídicas, econômicas e contratos. Em linhas gerais, as regras políticas definem a estrutura hierárquica do governo, bem como a estrutura de decisão. As regras econômicas, por sua vez, definem os \textbf{direitos de propriedade} e a possibilidade de alienar um ativo ou um recurso. Por fim, os contratos contém as provisões específicas de um acordo de uma troca particular. Dado o poder de barganha inicial, as regras têm a função de facilitar trocas, sejam elas políticas ou econômicas. Associado a isso, a estrutura de direitos define as \textbf{oportunidades} de maximização da riqueza dos agentes que podem ser realizadas por meio da formação de outras trocas políticas ou econômicas\footnote{O autor pontua que a função das regras é viabilizar \textbf{algumas} trocas, não necessariamente todas elas.}. O autor também ressalta que tal linha argumentativa não implica eficiência do arranjo institucional uma vez que são derivadas em parte do interesse pessoal (e não social) das partes que as definem. A seguir, destaca que as regras são geralmente criadas com os custos de conformidade, ou seja, devem ser criados métodos para verificar se uma regra foi violada ou não. Encerra a seção argumentando que a tecnologia e os preços relativos alteram os ganhos associados a criação de tais regras.

\subsection*{II}

O autor inicia a seção pontuando que regras políticas e econômicas se co-determinam, ou seja, a relação de causalidade é bidirecional. Em equilíbrio, uma estrutura de direitos de propriedade será consistente com um conjunto particular de regras políticas. Dito isso, o autor irá analisar o sistema político. Observando o Estado ao longo do tempo, argumenta que é mais complexo na medida que é criado o conceito de corpo representativo que reflete os interesses de grupos e sua barganha com o \textit{ruler}. Rumando para as democracias representativas, pontua que o grau de complexidade aumenta dado o desenvolvimento de múltiplos grupos de interesse e por uma estrutura institucional mais elaborada para permitir trocas entre tais grupos de interesse. Dito isso, discute que as condições para a garantia de uma auto-execução (\textit{self-enforcement}) das relações entre os agentes se torna mais ineficaz e:

\begin{quote}
	Hence political institutions constitute ex ante agreements about coopera­tion among politicians. They reduce uncertainty by creating a stable
	structure of exchange. The result is a complicated system of committee
	structure, consisting of both formal rules and informal methods of orga­nization.
\end{quote}
Adiante, afirma que as condições para que os mercados políticos sejam eficientes são tão escassas quanto as dos mercados econômicos. Em outras palavras, o regime democrático não deve ser equiparado com mercados econômicos competitivos.

\subsection*{III}

\autor propõe uma primeira aproximação aos direitos de propriedade por meio do cálculo de custo benefício de desviar ou de garantir tais direitos comparando alternativas sob o \textit{status quo}. Argumenta que mudanças nos preços relativos/escassez induzem a criação de direitos de propriedade quando vale a pena incorrer em tais custos invés de desviar. Em seguida, pontua o problema da ineficiência dos direitos de propriedade em um trabalho anterior e argumenta que o presente livre amplia tal ponto. Em linhas gerais, o autor afirma que a eficiência dos mercados políticos são fundamentais para compreender esta questão:

\begin{quote}
	If political
	transaction costs are low and the political actors have accurate models to
	guide them, then efficient property rights will result. But the high transac­tion costs of political markets and subjective perceptions of the actors
	more often have resulted in property rights that do not induce economic
	growth, and the consequent organizations may have no incentive to create
	more productive economic rules. At issue is not only the incremental
	character of institutional change, but also the problem of devising institu­tions that can provide credible comminnent so that more efficient bar­
	gains can be struck.
\end{quote}

\subsection*{IV}

Nesta seção, o autor discute \textbf{contratos} e afirma que tal instituição formal além de ser uma base para se analisar as diferentes formas de organização como também lançam luz sobre as formas que as partes realizam por meio de elaboração de estruturas mais complexas. Em outras palavras, os contratos refletem  diferentes formas de se facilitar trocas (seja por meio de firmas, \textit{franchising} até verticalização). O autor encerra o capítulo pontuando que uma análise centrada somente nas regras formais leva a uma noção errada das relações entre restrições formais e desempenho econômico.
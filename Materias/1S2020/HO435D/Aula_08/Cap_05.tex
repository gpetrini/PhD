\section*{Restrições informais}

\autor abre o capítulo pontuando que todas as sociedades impõem restrições para estruturar as relações com os demais e que tais restrições diminuem os custos das interações humanas --- se comparado com sociedades sem tais restrições. Destaca ainda que a relevância das instituições informais pode ser observada ao comparar sociedades com restrições formais semelhantes mas que possuem resultados distintos. Em seguida, discute a \textbf{origem} das instituições informais em que destaca a transmissão informal de informações e, em especial, a cultura\footnote{
	Realça que a cultura provê uma forma de codificar e interpretação por meio da linguagem.
}. Dito isso, explicita que este capítulo trata da importância das instituições informais no processo de mudança e que tal importância se deve a sua continuidade.

\subsection*{I}

Nesta seção, o autor inicia as discussões a partir de interações sociais em que não há nenhuma instituição formal, ou seja, analisa as condições da preservação da ordem em sociedades sem Estado. Da revisão de literatura, pontua que as transações em sociedades tribais não são triviais e que tais relações dependem de uma rede de relações sociais ampla que, por sua vez, proporciona o desenvolvimento de instituições informais e, assim, estabilidade societária. A ordem nessas sociedades é resultado desta rede de relações em que as pessoas tem um conhecimento tanto dos demais quanto das formas de punição e que este último é a forma de evitar comportamento desviantes. A discussão da literatura pode ser resumida como:

\begin{quote}
	Posner's essay emphasizes the importance of kinship ties as the central
	insurance, protection, and law enforcement mechanisms of primitive so­
	cieties. Bates' study of Kenya ( I 9 8 9 ) equally focuses on the changing
	pattern of kinship ties in the context of political/economic conditions as
	the key to understanding the evolving institutional constraints of a society
	in rapid transition from a tribal society ro a market economy.	
\end{quote}

\subsection*{II}

O autor inicia a seção seguinte pontuando que as instituições informais também desempenham um papel relevante nas sociedades modernas. Em seguida, descreve tais instituições como: (i) Extensões/modificações das regras formais; (ii) normas socialmente sancionadas e; (iii) normas de conduta aplicadas internamente. Prossegue dizendo que as duas primeiras características podem ser modeladas a partir do paradigma da otimização de riqueza enquanto não é o caso para a última.

\subsection*{III}

Nesta seção, \autor discute a persistência das instituições informais e conclui que esta associada a elaboração de convenções para resolver problemas de coordenação que, como visto, é uma das questões mais relevantes para o autor. Destaca ainda que na ausência de tais restrições, as relações de troca não só podem não ocorrer como também ser obstaculizadas dado o grau de informação assimétrica e a alocação de recursos para sua mensuração. Adiante, pontua que tais questões não são de fácil tratamento a partir da maximização da riqueza e que não há um arcabouço teórico da sociologia capaz de lidar com \textit{payoffs} negativos de um comportamento não-maximizador.

\subsection*{IV}

Por fim, esta seção reúne as discussões das anteriores. \autor argumenta que o processo de obtenção de informação não é o único que explica a existência de instituições, mas ajuda a compreender como restrições informais são relevantes para determinar o conjunto de decisões dos agentes. No curto prazo, afirma, a cultura define a forma que os indivíduos processam e utilizam a informação e a forma que as instituições informais são especificadas. Além disso, pontua que as convenções e normas são específicas de cada cultura. Em seguida, pontua que o arcabouço dos custos de transação também permite explorar as instituições informais, mas uma vez que tais restrições não são observáveis, tal proposta aufere apenas evidências \textbf{indiretas} das mudanças nas instituições informais. \autor propõe que os códigos de conduta podem ser avaliados empiricamente pela examinação das mudanças marginais dos custos de expressar suas \textbf{convicções}. Apesar do pouco conhecimento sobre as condições que um indivíduo está disposto a seguir suas convicções, destaca que existem evidências de que tal função é negativamente inclinada e que este preço é baixo em muitas configurações institucionais. Além disso, destaca que a agenda de pesquisa que parte do processamento cultural de informações é incrementalmente relevante para compreender como as instituições evoluem e é uma fonte de \textit{path dependence}. Igualmente importante, afirma, é que as restrições informais associadas à cultura não mudam imediatamente em reação às mudanças das instituições formais. A tensão entre ambas produzem implicações importantes para entender como as economias mudam.
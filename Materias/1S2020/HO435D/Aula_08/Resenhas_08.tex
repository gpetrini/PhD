\documentclass[11pt,lineno]{../style}



\newcommand{\autor}{North (1990) }

\title{
\large{Nova Economia Institucional}\vspace{2pt}\\
\Huge{\autor - \textit{Institutions, Institutional change and Economic Performance}}
}
\date{29 de Maio de 2020}

\author[$\ast$]{Gabriel Petrini}

\affil[$\ast$]{PhD Student at Unicamp.}

\keywords{
	Keyword1\\ 
	Keyword2\\
	Keyword3
}

\runningtitle{Resenha} % For use in the footer 

%% For the footnote.
\runningauthor{Petrini}

\begin{abstract}

\end{abstract}

\begin{document}

\maketitle
\marginmark
\thispagestyle{firststyle}
% Please add here a significance statement to explain the relevance of your work

\section*{Restrições informais}

\autor abre o capítulo pontuando que todas as sociedades impõem restrições para estruturar as relações com os demais e que tais restrições diminuem os custos das interações humanas --- se comparado com sociedades sem tais restrições. Destaca ainda que a relevância das instituições informais pode ser observada ao comparar sociedades com restrições formais semelhantes mas que possuem resultados distintos. Em seguida, discute a \textbf{origem} das instituições informais em que destaca a transmissão informal de informações e, em especial, a cultura\footnote{
	Realça que a cultura provê uma forma de codificar e interpretação por meio da linguagem.
}. Dito isso, explicita que este capítulo trata da importância das instituições informais no processo de mudança e que tal importância se deve a sua continuidade.

\subsection*{I}

Nesta seção, o autor inicia as discussões a partir de interações sociais em que não há nenhuma instituição formal, ou seja, analisa as condições da preservação da ordem em sociedades sem Estado. Da revisão de literatura, pontua que as transações em sociedades tribais não são triviais e que tais relações dependem de uma rede de relações sociais ampla que, por sua vez, proporciona o desenvolvimento de instituições informais e, assim, estabilidade societária. A ordem nessas sociedades é resultado desta rede de relações em que as pessoas tem um conhecimento tanto dos demais quanto das formas de punição e que este último é a forma de evitar comportamento desviantes. A discussão da literatura pode ser resumida como:

\begin{quote}
	Posner's essay emphasizes the importance of kinship ties as the central
	insurance, protection, and law enforcement mechanisms of primitive so­
	cieties. Bates' study of Kenya ( I 9 8 9 ) equally focuses on the changing
	pattern of kinship ties in the context of political/economic conditions as
	the key to understanding the evolving institutional constraints of a society
	in rapid transition from a tribal society ro a market economy.	
\end{quote}

\subsection*{II}

O autor inicia a seção seguinte pontuando que as instituições informais também desempenham um papel relevante nas sociedades modernas. Em seguida, descreve tais instituições como: (i) Extensões/modificações das regras formais; (ii) normas socialmente sancionadas e; (iii) normas de conduta aplicadas internamente. Prossegue dizendo que as duas primeiras características podem ser modeladas a partir do paradigma da otimização de riqueza enquanto não é o caso para a última.

\subsection*{III}

Nesta seção, \autor discute a persistência das instituições informais e conclui que esta associada a elaboração de convenções para resolver problemas de coordenação que, como visto, é uma das questões mais relevantes para o autor. Destaca ainda que na ausência de tais restrições, as relações de troca não só podem não ocorrer como também ser obstaculizadas dado o grau de informação assimétrica e a alocação de recursos para sua mensuração. Adiante, pontua que tais questões não são de fácil tratamento a partir da maximização da riqueza e que não há um arcabouço teórico da sociologia capaz de lidar com \textit{payoffs} negativos de um comportamento não-maximizador.

\subsection*{IV}

Por fim, esta seção reúne as discussões das anteriores. \autor argumenta que o processo de obtenção de informação não é o único que explica a existência de instituições, mas ajuda a compreender como restrições informais são relevantes para determinar o conjunto de decisões dos agentes. No curto prazo, afirma, a cultura define a forma que os indivíduos processam e utilizam a informação e a forma que as instituições informais são especificadas. Além disso, pontua que as convenções e normas são específicas de cada cultura. Em seguida, pontua que o arcabouço dos custos de transação também permite explorar as instituições informais, mas uma vez que tais restrições não são observáveis, tal proposta aufere apenas evidências \textbf{indiretas} das mudanças nas instituições informais. \autor propõe que os códigos de conduta podem ser avaliados empiricamente pela examinação das mudanças marginais dos custos de expressar suas \textbf{convicções}. Apesar do pouco conhecimento sobre as condições que um indivíduo está disposto a seguir suas convicções, destaca que existem evidências de que tal função é negativamente inclinada e que este preço é baixo em muitas configurações institucionais. Além disso, destaca que a agenda de pesquisa que parte do processamento cultural de informações é incrementalmente relevante para compreender como as instituições evoluem e é uma fonte de \textit{path dependence}. Igualmente importante, afirma, é que as restrições informais associadas à cultura não mudam imediatamente em reação às mudanças das instituições formais. A tensão entre ambas produzem implicações importantes para entender como as economias mudam.
\section*{Restrições formais}

\autor abre o capítulo afirmando que as sociedades se tornam mais complexas na medida que se tornam mais especializadas e que tal movimento está associado com um maior grau de \textbf{formalização das restrições}. Em seguida, pontua que tais restrições complementam e até mesmo melhoram a eficácia das instituições informais uma vez que reduzem os custos de soluções para relações mais complexas. Além disso, as regas formais também podem modificar, revisar e substituir as informais.

\subsection*{I}

As instituições formais incluem regras políticas, jurídicas, econômicas e contratos. Em linhas gerais, as regras políticas definem a estrutura hierárquica do governo, bem como a estrutura de decisão. As regras econômicas, por sua vez, definem os \textbf{direitos de propriedade} e a possibilidade de alienar um ativo ou um recurso. Por fim, os contratos contém as provisões específicas de um acordo de uma troca particular. Dado o poder de barganha inicial, as regras têm a função de facilitar trocas, sejam elas políticas ou econômicas. Associado a isso, a estrutura de direitos define as \textbf{oportunidades} de maximização da riqueza dos agentes que podem ser realizadas por meio da formação de outras trocas políticas ou econômicas\footnote{O autor pontua que a função das regras é viabilizar \textbf{algumas} trocas, não necessariamente todas elas.}. O autor também ressalta que tal linha argumentativa não implica eficiência do arranjo institucional uma vez que são derivadas em parte do interesse pessoal (e não social) das partes que as definem. A seguir, destaca que as regras são geralmente criadas com os custos de conformidade, ou seja, devem ser criados métodos para verificar se uma regra foi violada ou não. Encerra a seção argumentando que a tecnologia e os preços relativos alteram os ganhos associados a criação de tais regras.

\subsection*{II}

O autor inicia a seção pontuando que regras políticas e econômicas se co-determinam, ou seja, a relação de causalidade é bidirecional. Em equilíbrio, uma estrutura de direitos de propriedade será consistente com um conjunto particular de regras políticas. Dito isso, o autor irá analisar o sistema político. Observando o Estado ao longo do tempo, argumenta que é mais complexo na medida que é criado o conceito de corpo representativo que reflete os interesses de grupos e sua barganha com o \textit{ruler}. Rumando para as democracias representativas, pontua que o grau de complexidade aumenta dado o desenvolvimento de múltiplos grupos de interesse e por uma estrutura institucional mais elaborada para permitir trocas entre tais grupos de interesse. Dito isso, discute que as condições para a garantia de uma auto-execução (\textit{self-enforcement}) das relações entre os agentes se torna mais ineficaz e:

\begin{quote}
	Hence political institutions constitute ex ante agreements about coopera­tion among politicians. They reduce uncertainty by creating a stable
	structure of exchange. The result is a complicated system of committee
	structure, consisting of both formal rules and informal methods of orga­nization.
\end{quote}
Adiante, afirma que as condições para que os mercados políticos sejam eficientes são tão escassas quanto as dos mercados econômicos. Em outras palavras, o regime democrático não deve ser equiparado com mercados econômicos competitivos.

\subsection*{III}

\autor propõe uma primeira aproximação aos direitos de propriedade por meio do cálculo de custo benefício de desviar ou de garantir tais direitos comparando alternativas sob o \textit{status quo}. Argumenta que mudanças nos preços relativos/escassez induzem a criação de direitos de propriedade quando vale a pena incorrer em tais custos invés de desviar. Em seguida, pontua o problema da ineficiência dos direitos de propriedade em um trabalho anterior e argumenta que o presente livre amplia tal ponto. Em linhas gerais, o autor afirma que a eficiência dos mercados políticos são fundamentais para compreender esta questão:

\begin{quote}
	If political
	transaction costs are low and the political actors have accurate models to
	guide them, then efficient property rights will result. But the high transac­tion costs of political markets and subjective perceptions of the actors
	more often have resulted in property rights that do not induce economic
	growth, and the consequent organizations may have no incentive to create
	more productive economic rules. At issue is not only the incremental
	character of institutional change, but also the problem of devising institu­tions that can provide credible comminnent so that more efficient bar­
	gains can be struck.
\end{quote}

\subsection*{IV}

Nesta seção, o autor discute \textbf{contratos} e afirma que tal instituição formal além de ser uma base para se analisar as diferentes formas de organização como também lançam luz sobre as formas que as partes realizam por meio de elaboração de estruturas mais complexas. Em outras palavras, os contratos refletem  diferentes formas de se facilitar trocas (seja por meio de firmas, \textit{franchising} até verticalização). O autor encerra o capítulo pontuando que uma análise centrada somente nas regras formais leva a uma noção errada das relações entre restrições formais e desempenho econômico.
\section*{Execução}

O autor abre o capítulo pontuando o porquê da execução contratual é imperfeita. O primeiro motivo decorre dos custos associados à mensuração do contrato e o outro resulta da função utilidade de quem executa o contrato influencia o resultado. Dito isso, nesse capítulo irá analisar os problemas decorrente da transferência de direitos de propriedade. Para que a transação ocorra é preciso que os custos de transação de conformidade (\textit{compliance}) valham a pena para os agentes. Dito isso, afirma que incapacidade de algumas sociedades desenvolverem mecanismos de execução de contratos eficientes e a baixo é uma das principais explicações para a estagnação dos países do terceiro mundo.

\subsection*{I}

\autor abre a seção se perguntando as condições necessárias para que um contrato seja auto-executado. Afirma que no paradigma da otimização, basta que os benefícios dos contratos superem seus custos. Em uma sociedade de trocas impessoais, afirma, bens e serviços (ou o desempenho de um agente) são caracterizados por muitos atributos; a transação ocorre ao longo do tempo e; não há relações repetidas. Dito isso, pontua que dentre as questões fundamentais para o desenvolvimento corresponde ao dilema das transações impessoais sem a execução de terceiros.

\subsection*{II}

nesta seção, \autor retoma algumas questões discutidas em capítulos anteriores referente às especificidades da validez da teoria dos jogos. Contrapõe estas condições específicas com um mundo de trocas impessoais em que muitos agentes se relacionam e adquirem pouca informação sobre eles. Além disso, pontua que na presença de \textbf{informação incompleta}, a cooperação não é sustentável ao menos que forem criadas instituições para prover informações suficientes às partes para policiar desvios. No entanto, para que uma instituição assegure a cooperação é preciso tanto que exista um mecanismo de comunicação e policiar a defecção quanto prover incentivos aos indivíduos para que exerção a punição quando necessário. Em linhas gerais, para que uma transação ocorra é necessário elaborar um arranjo institucional que permita aprimorar a execução e a mensuração enquanto o custo de transação resultante faz com que o custo de troca supere o nível da teoria neoclássica. Além disso, quanto mais complexo o sistema de troca no tempo e espaço, mas custosas e complexas serão as instituições para viabilizá-las.

\subsection*{III}

Nesta seção, o autor tenta delinear o que entende por um mecanismo de execução por terceiros que, a princípio, deveria ser uma parte neutra com a capacidade de medir os atributos do contrato e executar acordos de modo que a parte prejudicada deve ser recompensada ao custo que inviabilize a violação do contrato. Somado a isso, pontua que a execução é custosa e o mesmo vale para a investigar se a relação contratual foi rompida ou não. Descrito este conceito, \autor compara as economias desenvolvidas com às do terceiro mundo. Nas economias desenvolvidas, argumenta, o sistema jurídico é bem definido e bem especializado. Nas economias do terceiro mundo, por sua vez, o sistema jurídico é incerto não apenas pela ambiguidade da doutrina legal, mas pela incerteza em relação ao comportamento do agente.

Em seguida, afirma que a execução por terceiros requer o desenvolvimento de um Estado com força coercitiva capaz de monitorar os direitos de propriedade e executar contratos eficazmente. No entanto, pontua que não é claro como criar tal aparato. Outra dificuldade decorre da possibilidade dessa coerção no mercado político para defender seus próprios interesses às custas da sociedade. Dito isso, o autor encerra o capítulo com a seguinte pergunta: como criar restrições auto-executadas? Diz que parte da resposta decorre da criação de um sistema de execução eficaz e pelos restrições morais no comportamento e ambas levam tempo para se consolidar.
\section*{Instituições e custos de transação e de transformação (Produção?)}

O autor abre o capítulo realçando que a definição de direitos de propriedade e execução de acordos requer recursos e que instituições, bem como o estado da tecnologia determinam os custos de \textbf{transação}. Desse modo, instituições desempenham um papel fundamental na determinação dos custos de \textbf{produção}. Em seguida, retoma que uma hierarquia de regras (tomada em conjunto) determina a estrutura formal de direitos em uma transação. Além disso, pontua a incompletude dos contratos junto da importância de instituições informais --- reputação, padrões de conduta aceitos, convenções, etc --- nos acordos.

\subsection*{I}

Nesta seção, \autor investiga uma única troca à luz dos elementos elencados anteriormente, avaliando a transferência de propriedade residencial nos EUA\footnote{Dentre as particularidades deste mercado, destaca tanto a hierarquia de regras formais que delegam poder aos Estados quanto as instituições do mercado de hipotecas.}. Em linhas gerais, as instituições definem o quão custosa será esta transação uma vez que são necessários recursos para mensurar tanto os atributos legais quanto físicos do bem a ser transacionado; custos de policiar e executar o acordo e; incerteza associada ao grau de imperfeição das etapas anteriores. Contraponto tal proposta ao paradigma neoclássico, o autor pontua que esta tradição não só considera que a informação é perfeita como a garantia dos direitos de propriedade. Em seguida, o autor pontua que as instituições também podem \textbf{aumentar} os custos de transação. Os países que tem um melhor desempenho econômico são aqueles cujas instituições, em média, reduzem os custos de transação.

\subsection*{II}

Retoma que as instituições afetam tanto os custos de transação quanto os de transformação. A influência das instituições neste último tipo de custo decorre dos seus efeitos sobre a tecnologia adotada. Tal como discutido anteriormente, as instituições podem atuar tanto promovendo atividade pró-produtividade quanto inibindo-as. Para ilustrar a importância da definição dos direitos de propriedade, compara as economias avançadas com às do terceiro mundo. Por um lado, o arranjo institucional aumenta os custos de transação, por outro, direitos de propriedade de propriedade inibem tecnologias intensivas em capital fixo. Como consequência, firmas são menores onde os direitos de propriedades são mal estabelecidos.No que diz respeito às instituições e ao mercado de trabalho, afirma:

\begin{quote}
	The unique feature of labor markets is that institutions are
	devised to take into account that the quantity and quality of output are
	influenced by the attitude of the productive factor - hence morale build­ing is a subsitute at the margin for investing in more monitoring.
\end{quote}

\subsection*{III}

Nesta seção, o autor ressalta as instituições políticas dada sua relevância em definir as instituições formais. Como discutido em outros capítulos, organizações surgem para extrair as oportunidades criadas pelas instituições e este é o caso das firmas. Em sociedades em que as instituições formas e os direitos de propriedade não são bem estabelecidos, o tamanho das firmas tenderá a ser menor e as firmas maiores existirão sob a tutela do Estado.

\subsection*{IV}

Nesta seção, o autor pontua as implicações do arcabouço teórico apresentado até então:

\begin{itemize}
	\item As restrições institucionais que definem as oportunidades a serem exploradas pelos agentes são de natureza formal e informal. As instituições formais serão alteradas somente quando o interesse daqueles que possuem elevado poder de barganha for atingido com a mudança. Além disso, esta interação entre instituições formais e informais permite que ocorram mudanças incrementais;
	\item As instituições (formais e informais) refletem os custos de mensuração e execução. Quanto maiores estes custos, maior a importância das instituições informais para delinear a transação. Verticalização é uma das formas de reduzir tais custos. Na medida em que as restrições informais dominem as formas de troca, elas geralmente adotam a forma de criar formas de contornar a probabilidade de deserção pela outra parte.
	\item Os custos de transação são a dimensão mais observável do arranjo institucional, no entanto a mensuração dos custos decorrentes de uma instituição em particular é de difícil precisão;
	\item Arranjo institucional é fundamental para o desempenho econômico. No entanto, algumas instituições podem aumentar os custos de transação.
\end{itemize}

	
\begin{redbox}{Dúvidas e comentários}
	
	\autor 	pontua que sociedades mais complexas e especializadas (impessoais) requerem arranjos institucionais mais complexos por consequência para estruturar as relações entre os agentes (``jogadores''). Somado a isso, também ressalta a importância de mecanismos execução dos contratos e acordos entre as partes e que tais mecanismos não estão tão bem implementados nos países do ``Terceiro Mundo'' quanto nos países desenvolvidos. Em outras palavras, os países do Terceiro Mundo não tiveram um desempenho (econômico ou não) tão bom quanto os países desenvolvidos por conta de arranjos institucionais (formais) capazes de aprimorar as relações de troca. No entanto, o argumento também não pode ser inverso, ou seja, tais países não desenvolveram arranjos institucionais mais complexos porque não tiveram um desempenho (ao longo do tempo) tão bom quanto os países desenvolvidos? Dito de outro modo, a relação entre instituições e desempenho (\textit{economic change}) é unidirecional para North? Quão compatível (ou não) é a teoria de North com a de Chang\footnote{Me refiro ao ``Chutando a escada'' que só conheço de orelhada.}?
	
\end{redbox}

\end{document}

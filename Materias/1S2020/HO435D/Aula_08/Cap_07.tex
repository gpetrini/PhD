\section*{Execução}

O autor abre o capítulo pontuando o porquê da execução contratual é imperfeita. O primeiro motivo decorre dos custos associados à mensuração do contrato e o outro resulta da função utilidade de quem executa o contrato influencia o resultado. Dito isso, nesse capítulo irá analisar os problemas decorrente da transferência de direitos de propriedade. Para que a transação ocorra é preciso que os custos de transação de conformidade (\textit{compliance}) valham a pena para os agentes. Dito isso, afirma que incapacidade de algumas sociedades desenvolverem mecanismos de execução de contratos eficientes e a baixo é uma das principais explicações para a estagnação dos países do terceiro mundo.

\subsection*{I}

\autor abre a seção se perguntando as condições necessárias para que um contrato seja auto-executado. Afirma que no paradigma da otimização, basta que os benefícios dos contratos superem seus custos. Em uma sociedade de trocas impessoais, afirma, bens e serviços (ou o desempenho de um agente) são caracterizados por muitos atributos; a transação ocorre ao longo do tempo e; não há relações repetidas. Dito isso, pontua que dentre as questões fundamentais para o desenvolvimento corresponde ao dilema das transações impessoais sem a execução de terceiros.

\subsection*{II}

nesta seção, \autor retoma algumas questões discutidas em capítulos anteriores referente às especificidades da validez da teoria dos jogos. Contrapõe estas condições específicas com um mundo de trocas impessoais em que muitos agentes se relacionam e adquirem pouca informação sobre eles. Além disso, pontua que na presença de \textbf{informação incompleta}, a cooperação não é sustentável ao menos que forem criadas instituições para prover informações suficientes às partes para policiar desvios. No entanto, para que uma instituição assegure a cooperação é preciso tanto que exista um mecanismo de comunicação e policiar a defecção quanto prover incentivos aos indivíduos para que exerção a punição quando necessário. Em linhas gerais, para que uma transação ocorra é necessário elaborar um arranjo institucional que permita aprimorar a execução e a mensuração enquanto o custo de transação resultante faz com que o custo de troca supere o nível da teoria neoclássica. Além disso, quanto mais complexo o sistema de troca no tempo e espaço, mas custosas e complexas serão as instituições para viabilizá-las.

\subsection*{III}

Nesta seção, o autor tenta delinear o que entende por um mecanismo de execução por terceiros que, a princípio, deveria ser uma parte neutra com a capacidade de medir os atributos do contrato e executar acordos de modo que a parte prejudicada deve ser recompensada ao custo que inviabilize a violação do contrato. Somado a isso, pontua que a execução é custosa e o mesmo vale para a investigar se a relação contratual foi rompida ou não. Descrito este conceito, \autor compara as economias desenvolvidas com às do terceiro mundo. Nas economias desenvolvidas, argumenta, o sistema jurídico é bem definido e bem especializado. Nas economias do terceiro mundo, por sua vez, o sistema jurídico é incerto não apenas pela ambiguidade da doutrina legal, mas pela incerteza em relação ao comportamento do agente.

Em seguida, afirma que a execução por terceiros requer o desenvolvimento de um Estado com força coercitiva capaz de monitorar os direitos de propriedade e executar contratos eficazmente. No entanto, pontua que não é claro como criar tal aparato. Outra dificuldade decorre da possibilidade dessa coerção no mercado político para defender seus próprios interesses às custas da sociedade. Dito isso, o autor encerra o capítulo com a seguinte pergunta: como criar restrições auto-executadas? Diz que parte da resposta decorre da criação de um sistema de execução eficaz e pelos restrições morais no comportamento e ambas levam tempo para se consolidar.
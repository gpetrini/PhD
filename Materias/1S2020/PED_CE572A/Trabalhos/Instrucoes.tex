\documentclass[11pt]{beamer}
\usepackage[utf8]{inputenc}
\usepackage[T1]{fontenc}
\usepackage[portuguese]{babel}
\usepackage{amsmath}
\usepackage{amsfonts}
\usepackage{amssymb}
\usepackage{graphicx}
\usepackage{booktabs}
\usetheme{Dresden}
\usecolortheme{beaver}

\begin{document}
	\author{Mariano Laplane \and Gabriel Petrini}
	\title{Instruções do trabalho}
	\subtitle{CE 572}
	%\logo{}
	\institute{Instituto de Economia}
	%\date{}
	%\subject{}
	%\setbeamercovered{transparent}
	%\setbeamertemplate{navigation symbols}{}
	\begin{frame}[plain]
	\maketitle
\end{frame}

\begin{frame}
\frametitle{Objetivo do trabalho}

Analisar os modelos de crescimento --- sejam eles ortodoxos ou heterodoxos --- vistos ao longo da disciplina à luz de eventos recentes. Em particular, cada grupo deve descrever os efeitos (em nível e em taxa) sobre algumas variáveis dado o modelo a ser analisado.

\begin{alertblock}{Eventos recentes}
	Os grupos podem escolher vários ``eventos recentes'' a serem investigados. Seguem  algumas \textbf{sugestões}:
	
	\begin{itemize}
		\item ``Coronacrise''
		\item Desigualdade
		\item Reformas (previdência, teto de gastos, etc)
		\item $\ldots$
	\end{itemize}
\end{alertblock}
\end{frame}

\begin{frame}
\frametitle{Questões a serem respondidas}

Dado o problema a ser analisado, os grupos devem explicitar quais os impactos destes ``eventos recentes'' sobre algumas variáveis macroeconômicas.
Espera-se que os grupos evidenciem os efeitos de mudanças no processo tecnológico, distribuição de renda,  taxa de poupança/investimento (e de outras variáveis de interesse) sobre PIB, PIB \textit{per capita}, taxa de crescimento do produto, demanda agregada, entre outros. Em particular, os grupos devem indicar se esses efeitos são persistentes ou temporários.


\begin{alert}{Dica}
	
	Dizer que ``o modelo $M$ não explica os efeitos da variável $X$ sobre $Y$'' também é uma resposta : )
\end{alert}


\end{frame}


\begin{frame}
\frametitle{Estrutura Sugerida}

Segue uma estrutura \textbf{sugerida} de trabalho. Grupos que optarem por estruturas distintas não serão prejudicados.

\begin{itemize}
	\item Resumo 
	\item Introdução
	\item Apresentação do modelo a ser estudado
	\item Evidências empíricas e/ou Extensões e/ou Críticas do modelo
	\item Modelo à luz de eventos recentes
	\item Conclusão
\end{itemize}
\end{frame}


\begin{frame}{Resumo, Introdução e Conclusão}

Seguem algumas dicas e sugestões. Vale lembrar que os grupos não serão prejudicados se não seguirem estas sugestões.

\begin{alert}{Resumo}

Devem conter os principais elementos do trabalho; palavras-chave; \textit{abstract} é opcional. \textbf{Limite:} 20 linhas
\end{alert}

\begin{alert}{Introdução}

Contextualizar o modelo a ser analisado, bem como o tema particular a ser estudado (``evento recente''). Explicitar a estrutura do trabalho.
\end{alert}

\begin{alert}{Conclusão}
	
Retoma as discussões realizadas e, preferencialmente, os efeitos sobre algumas variáveis (slides seguintes). É um bom espaço para apontar as limitações do modelo e temas futuros a serem estudados (agenda de pesquisa futura).
\end{alert}

\end{frame}


\begin{frame}
\frametitle{Apresentação do modelo}
Apresentar o modelo a ser estudado. 

\textbf{Exemplo de apresentação:} o modelo $X$ parte da hipótese de que todo consumo é \underline{induzido} e que a distribuição de renda é \underline{exógena} de modo que o consumo agregado pode ser especificado da seguinte forma:

$$
C = W
$$
em que $C$ é o consumo agregado e $W$ é a massa de salários

\begin{alertblock}{Dica}

Parte dos efeitos a serem estudados podem ser respondidas a partir da exposição do modelo : )
\end{alertblock}
\end{frame}

\begin{frame}
\frametitle{Evidências, Extensões e Críticas}

Esta seção é bastante aberta em que os grupos podem indicar alguns avanços da literatura. Não é necessário uma exposição exaustiva de cada contribuição, mas espera-se que os grupos resenhem um conjunto considerável de trabalhos. \textbf{Dica:} alguns dos efeitos não respondidos anteriormente podem ser encontrados aqui.

\begin{alertblock}{Exemplo}
	
	\textbf{Fulano \textit{et al.}} (XXXX) encontram evidências de que a variável $Z$ tem efeitos positivos sobre $Y$ a partir de um modelo (método) $M$ no período $P$ para os casos $C$
	
	\textbf{Sicrano (XXXX)} estende o modelo $M$ por meio da inclusão da variável $Y$ para explicar $Z$
	
	\textbf{Beltrano (XXXX)} critica o modelo $M$ por não dar a devida atenção à $Z$.
\end{alertblock}

\end{frame}

\begin{frame}
\frametitle{Modelo à luz de eventos recentes}

Nesta seção, os grupos devem contextualizar o problema específico a ser estudado o porquê. Em particular, devem explicitar as implicações sobre algumas variáveis macroeconômicas e como o modelo estudado responde estas questões

\begin{alertblock}{Exemplo}
	
O evento $E$ tem causado efeitos sobre as variáveis $X, Y \text{ e } Z$ e tais mudanças são importantes por $A, B \text{ e } C$. A luz do modelo $M$, $X$ tem efeitos sobre a demanda agregada e sobre a taxa de crescimento, mas não é possível concluir quais são os impactos diretos sobre a taxa de crescimento do PIB per capita.
\end{alertblock}
\end{frame}


\begin{frame}
\frametitle{Formatação e instruções de entrega}
\begin{itemize}
	\item \textbf{Formato:} PDF
	\item \textbf{Onde:} Moodle
	\item \textbf{Limite:} 20 páginas (não há mínimo)
	\item \textbf{Referências bibliográficas}: Devem seguir as normas da ABNT
	\item \textbf{Margens e espaçamento:} Não especificado
	\item \textbf{Prazo:} A definir
\end{itemize}
\end{frame}

\begin{frame}
\frametitle{Sorteio dos modelos}

\begin{table}[h]
	\caption{Sorteio dos modelos}
	\begin{tabular}{lrl}
\toprule
{} &  Número de integrantes &                         Temas \\
\midrule
Grupo 1 &                      4 &                         Lucas \\
Grupo 2 &                      4 &                        Harrod \\
Grupo 3 &                      4 &                     Thirlwall \\
Grupo 4 &                      4 &  Supermultiplicador sraffiano \\
Grupo 5 &                      3 &      Multiplicador/Acelerador \\
Grupo 6 &                      4 &                            AK \\
Grupo 7 &                      2 &                         Solow \\
Grupo 8 &                      4 &                    Kaleckiano \\
Grupo 9 &                      4 &                         Romer \\
\bottomrule
\end{tabular}

\end{table}

\end{frame}

\end{document}
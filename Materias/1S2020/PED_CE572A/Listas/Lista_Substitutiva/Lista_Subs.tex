\documentclass[12pt,a4paper]{article}
\usepackage[
left = 3cm, right = 2cm,
top = 3cm, bottom = 2cm
]{geometry}
\usepackage[utf8]{inputenc}
\usepackage[brazilian]{babel}
\usepackage{amsmath}
\usepackage{amsfonts}
\usepackage{amssymb}
\usepackage[T1]{fontenc}
\usepackage{mathptmx}
\usepackage{graphicx}
\usepackage{physics}
\usepackage{siunitx}
\newcounter{prob}
\newcounter{subprob}
\renewcommand{\thesubprob}{\alph{subprob}}

\newcommand{\problem}{\setcounter{subprob}{0} \stepcounter{prob} \par \medskip \noindent \textbf{Questão~\theprob \ }}

\newcommand{\answer}{\par \medskip \noindent \textit{Resposta \ }}

\newcommand{\finalanswer}[1]{
	\begin{center} 
    	{\renewcommand{\arraystretch}{1.5}
		\renewcommand{\tabcolsep}{0.2cm} 
    	\begin{tabular}{|c|} 
    		\hline 
        	$ \displaystyle #1 $  \\ 
        	\hline 
    	\end{tabular}} 
   	\end{center}}

\newcommand{\subproblem}{\stepcounter{subprob} \par \smallskip \noindent \quad \textit{(\thesubprob) \ }}

\newcommand{\subanswer}{\par \smallskip \noindent \quad \textit{Resposta \ }}

\newcommand{\option}{\item[$\square$]}
\newcommand{\thisone}{\item[$\blacksquare$]}

\newenvironment{subitemize}{\begin{itemize}}{\end{itemize}}

\begin{document}

%%--CABEÇALHO--%%
	\begin{center}
    {\huge Lista de Exercícios Substitutiva \par}
    {\LARGE Macroeconomia III \par}
    {CE 572 \par}
    {1º Semestre de 2020}
	\end{center}

\section*{Teorias Keynesianas do crescimento}

\subsection*{Modelo de Harrod}

\problem Em que medida Harrod estende a Teoria Geral de Keynes para uma economia dinâmica?

\problem Partindo do modelo de Harrod em que a taxa de crescimento efetiva é maior que a garantida, descreva os encadeamentos dinâmicos. (1 ponto)

\noindent\textbf{Questão Bônus:} O que é necessário para que o modelo de Harrod se torne estável? (1 ponto)

\subsection*{Modelos acelerador/multiplicador}

\problem Apresente e discuta o modelo do acelerador rígido. (1 ponto)

\problem Discuta e apresente os demais modelos do tipo acelerador. Em sua resposta, destaque as diferenças em relação ao acelerador rígido (1 ponto)

\subsection*{Supermultiplicador sraffiano}

\problem Enuncie e defina o supermultiplicador sraffiano. Como ele se distingue do multiplicador keynesiano convencional? (1 ponto)

\problem Como se dá o princípio acelerador nesse modelo? Em que medida está associado com o princípio da demanda efetiva? (1 ponto)

\section*{Flutuações cíclicas}

\problem Quais as semelhanças e as diferenças entre a teoria do Kalecki do ciclo econômico e a dos modelos de acelerador-multiplicador? (1 ponto)


\problem Qual a importância e o significado econômico do coeficiente $d$ e do princípio do risco crescente para Kalecki? (1 ponto)

\section*{Crescimento com restrição no Balanço de pagamentos}

\problem Apresente e discuta a lei de Thirlwall. Qual seu significado econômico? (1 ponto)

\problem Qual a importância das elasticidades-renda da exportação/importação para este modelo? (1 ponto)

\noindent\textbf{Questão Bônus:} Relacione o modelo de Thirlwall com o supermultiplicador sraffiano (1 ponto)

\end{document}
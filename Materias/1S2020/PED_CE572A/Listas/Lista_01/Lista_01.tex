\documentclass[12pt,a4paper]{article}
\usepackage[
left = 3cm, right = 2cm,
top = 3cm, bottom = 2cm
]{geometry}
\usepackage[utf8]{inputenc}
\usepackage[brazilian]{babel}
\usepackage{amsmath}
\usepackage{amsfonts}
\usepackage{amssymb}
\usepackage[T1]{fontenc}
\usepackage{mathptmx}
\usepackage{graphicx}
\usepackage{physics}
\usepackage{siunitx}
\input{commands}

\begin{document}

%%--CABEÇALHO--%%
	\begin{center}
    {\huge Lista de Exercícios 01 \par}
    {\LARGE Macroeconomia III \par}
    {CE 572 \par}
    {1º Semestre de 2020}
	\end{center}

\section*{Capítulo 11}

\problem Qual a ideia do modelo de Solow? Tenha em mente as principais conclusões do modelo para responder a esta pergunta.

\problem Defina a ideia de \textit{steady state} (estado estacionário) para Solow.

\problem Quais as hipóteses básicas do modelo de Solow?

\problem Explique por que razão, no modelo de Solow, sem crescimento populacional e
sem progresso técnico, há um limite ao produto agregado e ao nível de renda por
trabalhador, para uma dada taxa de poupança. Descreva o impacto de um aumento na
taxa de poupança, explicando por que razão gera uma aceleração temporária do
crescimento e possibilita um nível de produto por trabalhador mais elevado no \textit{steady
state}, sem contudo determinar um processo de crescimento sustentado dessa relação.

\problem Dado um modelo de Solow com as seguintes especificações:


$$y = k^{1/2}$$
com
\begin{itemize}
	\item $s = 0,2$
	\item $\delta = 0,05$ 
	\item $n=0$
\end{itemize}
em que $y$ corresponde à produção per capita, $k$ ao capital per capita, $s$ é a taxa de poupança, $\delta$ é a taxa de depreciação e $n$ é a taxa de crescimento populacional, pergunta-se: qual será o nível de produção per capita no estado estacionário?


\problem Considere o modelo de crescimento de Solow com função de produção dada por ${Y = K^{\frac{1}{2}}\cdot L^{\frac{1}{2}}}$, sendo $Y =$ produto, $K =$ estoque de capital, $L =$ número de trabalhadores. Nessa economia, a população cresce a uma taxa constante igual a 5\%, a taxa de depreciação do estoque de capital é de 5\%, e a taxa de poupança é de 20\%. Calcule o valor do \underline{salário real} no estado de crescimento equilibrado.

\textbf{Dica:} Salário real é calculado de forma semelhante dos manuais de microeconomia.

\newpage\section*{Capítulo 12}

\problem Explique as características do \textit{steady state} na ausência de progresso técnico mas com crescimento da população. Qual a relação entre a taxa de crescimento da renda e a taxa de crescimento da população? Descreva o que ocorre no caso de um aumento da taxa de crescimento da população.

\problem Defina “crescimento endógeno” e compare esta visão com o modelo de crescimento de Solow.

\problem O quê os modelos de crescimento endógeno incluem que, até o modelo de Solow, não havia sido considerado?

\problem \paragraph{(ANPEC 2004, Ex. 14)} Considere uma economia cuja função de produção é dada por Y = $\sqrt{K}\sqrt{NA}$, em que $Y$, $K$, $N$ e $A$ representam, respectivamente, o produto, o estoque de capital, o número de trabalhadores e o estado da tecnologia. Por sua vez, a taxa de poupança é igual a 20\%, a taxa de depreciação é igual a 5\%, a taxa de crescimento do número de trabalhadores é igual a 2,5\% e a taxa de progresso tecnológico é igual a 2,5\%. Calcule valor do \underline{capital por trabalhador efetivo} no estado estacionário. 



\end{document}
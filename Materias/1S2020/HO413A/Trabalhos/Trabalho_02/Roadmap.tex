% Created 2020-06-25 qui 15:48
% Intended LaTeX compiler: pdflatex
\documentclass[11pt]{article}
\usepackage[utf8]{inputenc}
\usepackage[T1]{fontenc}
\usepackage{graphicx}
\usepackage{grffile}
\usepackage{longtable}
\usepackage{wrapfig}
\usepackage{rotating}
\usepackage[normalem]{ulem}
\usepackage{amsmath}
\usepackage{textcomp}
\usepackage{amssymb}
\usepackage{capt-of}
\usepackage{hyperref}
\author{Gabriel Petrini da Silveira}
\date{\today}
\title{}
\hypersetup{
 pdfauthor={Gabriel Petrini da Silveira},
 pdftitle={},
 pdfkeywords={},
 pdfsubject={},
 pdfcreator={Emacs 26.3 (Org mode 9.3.7)}, 
 pdflang={English}}
\begin{document}

\tableofcontents

\section{Sugestões de mudança}
\label{sec:org1977768}

\begin{itemize}
\item Cortar ações
\item Cortar administradores
\item Reformular investimento (com crédito)
\item Rever consistência
\end{itemize}

\begin{center}
\begin{tabular}{lll}
Número equação & Original & Modifcaido\\
(1) & \(\frac{I_t}{p\cdot K_{t-1}} = \alpha (u_{t-1} - u^d) + \beta r_{t-1} \epsilon h_{t-1}\) & \(\frac{I_t}{p\cdot K_{t-1}} = \beta_0  + \beta_1 (u_{t-1})\)\\
(2) & h\textsubscript{t} & -\\
(3) & \(r_t\) & \(r_t\)\\
(4) & \(Q_{fc}\) & \(Q_{fc}\)\\
(5) & \(Q_{fc}\) & \(Q_{fc}\)\\
(6) & \(u_t\) & \(u_t\)\\
(7) & \(I_t = \frac{I_t}{p\cdot K_{t-1}}\cdot p K_{t-1}\) & -\\
(8) &  & \\
(9) &  & \\
(10) &  & \\
(11) &  & \\
(12) & \(A_t = (1-\Theta) (\Pi p Q_t - i B_{f_{t-1}})\) & \(A_t = \Pi p Q_t - B_{f_{t-1}}\)\\
(13) & \(\Delta B = \varpi (I - A_t)\) & \(\Delta B_t = I - A_t\)\\
(14) & \(\Delta E_t\) & -\\
(15)  - (20) & Managers & -\\
(21) - (33) & cte & cte\\
(34) & denominador: \(p (1 - (1-s_\psi - \eta)\psi - (1-s_\Theta\Pi) - \frac{\eta \overline w}{\xi})\) & \(\Theta =0\)\\
(35) & \(P_e^*\) & -\\
\end{tabular}
\end{center}


\subsubsection{Valores iniciais}
\label{sec:org3be1370}

\begin{center}
\begin{tabular}{lrl}
Parâmetros & Baseline & Modificado\\
\hline
\(i\) & 0.035 & -\\
\(price_t\) & 3.0 & -\\
\(\gamma\) & 1.0 & -\\
\(\delta\) & 0.025 & -\\
\(\mu\) & 0.1 & -\\
\(\xi\) & 1.0 & -\\
\(\overline \omega\) & 0.2 & -\\
\(s_\psi\) & 0.6 & -\\
\(\sigma_\psi\) & 0.1 & -\\
\(\lambda\) & 0.2 & 0\\
\(N\) & 5000 & -\\
\(minsal\) & 0.1 & -\\
\(gmin\) & -0.25 & -0.025\\
\(gmax\) & 0.25 & 0.025\\
\(g_{basic}\) & 0.0 & 0.0005\\
\(\alpha\) & 0.005 & -\\
\(i_0\) & 0.02 & -\\
\(u_d\) & 0.8 & -\\
\(\gamma_u\) & 0.002 & -\\
\(\eta\) & 0.3 & -\\
\end{tabular}
\end{center}


\begin{center}
\begin{tabular}{lrl}
Valores defasados & Baseline & Modificado\\
\hline
\(A_t\) & 0.0 & 0.0\\
\(I_t\) & 20.0 & -\\
\(u_t\) & 0.7 & -\\
\(K_t\) & 10000 & -\\
\(h_{t-1}\) &  & \\
\(Cw_t\) & 0.1 & -\\
\(D_t\) & 0.0 & -\\
state\textsubscript{b}\textsubscript{it} & rand & rand\\
\(w_i\) & rand & rand\\
\(Ww_i\) & 0.0 & 0.0\\
\(B\) & 0.0 & -\\
\(D_totam\) & soma & -\\
basic & - & 50\\
\end{tabular}
\end{center}


\begin{center}
\begin{tabular}{lrl}
Variável & Valor & Derivados\\
\hline
\(EL\) & 4000 & -\\
\(N\) & 5000 & -\\
\(K\) & 10000 & -\\
\(Q\) & 3800 & -\\
\(I_t\) & 1800 & -\\
\(\gamma\) & 1.0 & -\\
\(\xi\) & 1.0 & -\\
\hline
Unrate & - & 0.8\\
\(u\) & - & 0.76\\
\(Cw_T\) & - & 2000\\
\(h\) & - & 0.47368421\\
\(Q_fc\) & - & 5000.\\
\end{tabular}
\end{center}

\section{Próximos passos}
\label{sec:org4aa4552}

\subsection{{\bfseries\sffamily TODO} Fim versão mais simples [0/4] [0\%]}
\label{sec:org0e1e256}
\begin{itemize}
\item[{$\square$}] Definir o outro gasto autônomo (diferente da renda básica) e estrutura (Jordão)
\item[{$\square$}] Se gasto do governo, definir arrecadação (Gabriel)
\item[{$\square$}] Atribuir parâmetros para cada agente (tornar mais ABM)
\item[{$\square$}] Escrever apresentação dos slides (João Paulo)
\end{itemize}

\subsection{{\bfseries\sffamily TODO} Aprimoramento 01 [0/3] [0\%]}
\label{sec:org4823717}
\begin{itemize}
\item[{$\square$}] Poupança dos trabalhadores -> Acúmulo de riqueza gerando renda (depósito bancário)
\item[{$\square$}] Distribuição dos salários a partir de uma distribuição de Pareto
\item[{$\square$}] Salário em função do salário mínimo (substituindo choques)
\end{itemize}

\subsection{{\bfseries\sffamily TODO} Aprimoramento 02 [0/4] [0\%]}
\label{sec:orgdf75752}
\begin{itemize}
\item[{$\square$}] Restrição de crédito
\item[{$\square$}] Reintrodução dos administradores
\item[{$\square$}] Reintrodução das ações
\item[{$\square$}] Reintrodução de inflação
\end{itemize}

\subsection{{\bfseries\sffamily TODO} Aprimoramento 03  [0\%]}
\label{sec:org2ae60fa}
\begin{itemize}
\item[{$\square$}] Mudar função investimento para função Kaleckiana
\end{itemize}
\section{Análise}
\label{sec:org773bf87}

\begin{table}[htbp]
\label{tab:orgb3624bf}
\centering
\begin{tabular}{llllllllll}
Cenário & \(\sigma_{\sigma_\psi}\) & \(\sigma_{s_\psi}\) & \(\eta\) & Inflação & \(g\) & \(u\) & Num of Borrowers & Gini & wbarocc\\
\hline
Baseline &  &  &  &  &  &  &  &  & \\
Sim\textsubscript{1} &  &  &  &  &  &  &  &  & \\
\end{tabular}
\end{table}


\begin{table}[htbp]
\label{tab:org48bf6a1}
\centering
\begin{tabular}{lllllllllll}
Cenário & min\textsubscript{sal} & p\textsubscript{minsal} & c & \(i\) & Inflação & \(g\) & \(u\) & Num of Borrowers & Gini & wbarocc\\
\hline
Baseline &  &  &  &  &  &  &  &  &  & \\
Sim\textsubscript{1} &  &  &  &  &  &  &  &  &  & \\
\end{tabular}
\end{table}
\end{document}
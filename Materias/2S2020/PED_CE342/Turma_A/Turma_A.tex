% Created 2020-11-21 sáb 15:06
% Intended LaTeX compiler: pdflatex
\documentclass[11pt]{article}
\usepackage[utf8]{inputenc}
\usepackage{lmodern}
\usepackage[T1]{fontenc}
\usepackage[top=3cm, bottom=2cm, left=3cm, right=2cm]{geometry}
\usepackage{graphicx}
\usepackage{longtable}
\usepackage{float}
\usepackage{wrapfig}
\usepackage{rotating}
\usepackage[normalem]{ulem}
\usepackage{amsmath}
\usepackage{textcomp}
\usepackage{marvosym}
\usepackage{wasysym}
\usepackage{amssymb}
\usepackage{amsmath}
\usepackage[theorems, skins]{tcolorbox}
\usepackage[style=abnt,noslsn,extrayear,uniquename=init,giveninits,justify,sccite,
scbib,repeattitles,doi=false,isbn=false,url=false,maxcitenames=2,
natbib=true,backend=biber]{biblatex}
\usepackage{url}
\usepackage[cache=false]{minted}
\usepackage[linktocpage,pdfstartview=FitH,colorlinks,
linkcolor=blue,anchorcolor=blue,
citecolor=blue,filecolor=blue,menucolor=blue,urlcolor=blue]{hyperref}
\usepackage{attachfile}
\usepackage{setspace}
\usepackage{tikz}
\usepackage{longtable, pdflscape, booktabs}
\author{Gabriel Petrini (PED)}
\date{2020/2S}
\title{Notas CE 342/A}
\begin{document}

\maketitle

\section{Monetaristas}
\label{sec:orgeea7c61}

| None |

\subsection{Notas}
\label{sec:org05426a4}
\begin{center}
\includegraphics[width=.9\linewidth]{./figs/monetaristas.png}
\end{center}

\subsection{Notas e presença}
\label{sec:org9eab56d}

\begin{center}
\includegraphics[width=.9\linewidth]{./figs/monetaristas_presenca.png}
\end{center}

\section{Novos Clássicos}
\label{sec:org449586e}

\begin{itemize}
\item[{$\square$}] Identificar respostas das questões (forms)
\item[{$\square$}] Computar notas
\item[{$\square$}] Preencher presenhas na tabela da Ana Rosa
\end{itemize}
| None |
\subsection{Notas}
\label{sec:org099ddc9}
\begin{center}
\includegraphics[width=.9\linewidth]{./figs/novosclassicos.png}
\end{center}
\subsection{Notas e Presença}
\label{sec:org84a4d9f}
\begin{center}
\includegraphics[width=.9\linewidth]{./figs/novosclassicos_presenca.png}
\end{center}
\subsection{Verificação de plágio por amostragem}
\label{sec:orgc31d6c1}
\subsubsection{Sorteio}
\label{sec:orgb26ac01}

\begin{minted}[frame=lines,fontsize=\scriptsize,linenos]{python}
pre_selecionados = ["238414"]
alunos = df.shape[0]
sample = 0.3
amostra = round(alunos*sample)
bad_df = df.index.isin(pre_selecionados)
elegivies = df.loc[~bad_df].query('`Resenha Novos Clássicos` > 0')["Resenha Novos Clássicos"].index.tolist()


resultado = np.sort(np.random.choice(
    a = elegivies,
    size = amostra,
    replace = False # Sem repetição
))
resultado = pd.DataFrame(resultado)
resultado.columns = ["RA Sorteados"]
resultado.index = [i + 1 for i in resultado.index]
print(resultado)
resultado
\end{minted}

\begin{verbatim}
    RA Sorteados
1              1
2              2
3              8
4             21
5             22
6             23
7             26
8             28
9             34
10            36
11            37
12            38
13            42
\end{verbatim}

\subsubsection{Resultado Turnitin}
\label{sec:org39ee502}

\begin{table}[htbp]
\caption{\label{TurnitinNvC}Resumo relatório de origilidadade}
\centering
\begin{tabular}{rl}
\hline
RA & Grau de semelhança\\
\hline
156242 & 15\%\\
187323 & 20\%\\
212900 & 1\%\\
219613 & 0\%\\
219907 & 8\%\\
222615 & 0\%\\
231302 & 2\%\\
232795 & 11\%\\
238414 & 2\%\\
239052 & 8\%\\
240409 & 0\%\\
244321 & 1\%\\
244379 & 1\%\\
245459 & 53\%\\
\hline
\end{tabular}
\end{table}

\begin{enumerate}
\item Gráfico
\label{sec:org8217568}
\begin{center}
\includegraphics[width=.9\linewidth]{./figs/turintin_NvC_fig.png}
\end{center}
\item Atualização das notas
\label{sec:orga106bf0}
\end{enumerate}
\subsection{Questões}
\label{sec:orged33750}

\begin{verbatim}
None
\end{verbatim}

\subsubsection{Importando pacotes e funções}
\label{sec:org5260dea}

\begin{center}
\includegraphics[width=.9\linewidth]{./figs/similarity_forms_1.png}
\end{center}


| None |


\section{Lista de presença e notas}
\label{sec:org5fb35ce}
| None |


\section{Seminários}
\label{sec:org62d0449}
\subsection{Sorteio}
\label{sec:orgff787a1}

| None |


\subsubsection{{\bfseries\sffamily TODO} Selecionar Atas e criar map para substituir}
\label{sec:org3cc9cc2}
\end{document}

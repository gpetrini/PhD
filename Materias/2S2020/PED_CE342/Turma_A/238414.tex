% Created 2020-10-19 seg 18:49
% Intended LaTeX compiler: pdflatex
\documentclass[11pt]{article}
\usepackage[utf8]{inputenc}
\usepackage{lmodern}
\usepackage[T1]{fontenc}
\usepackage[top=3cm, bottom=2cm, left=3cm, right=2cm]{geometry}
\usepackage{graphicx}
\usepackage{longtable}
\usepackage{float}
\usepackage{wrapfig}
\usepackage{rotating}
\usepackage[normalem]{ulem}
\usepackage{amsmath}
\usepackage{textcomp}
\usepackage{marvosym}
\usepackage{wasysym}
\usepackage{amssymb}
\usepackage{amsmath}
\usepackage[theorems, skins]{tcolorbox}
\usepackage[style=abnt,noslsn,extrayear,uniquename=init,giveninits,justify,sccite,
scbib,repeattitles,doi=false,isbn=false,url=false,maxcitenames=2,
natbib=true,backend=biber]{biblatex}
\usepackage{url}
\usepackage[cache=false]{minted}
\usepackage[linktocpage,pdfstartview=FitH,colorlinks,
linkcolor=blue,anchorcolor=blue,
citecolor=blue,filecolor=blue,menucolor=blue,urlcolor=blue]{hyperref}
\usepackage{attachfile}
\usepackage{setspace}
\usepackage{tikz}
\author{Gabriel Petrini (PED)}
\date{\today}
\title{238414 - Juliana Florentina Fernandes Leão}
\begin{document}

\maketitle


\section*{Notas}
\label{sec:org349c8eb}

\begin{center}
\begin{tabular}{lllll}
Tarefa & Conceito & Desenvolvimento & Clareza & Nota\\
\hline
 &  &  &  & \\
Resenha 1 & Atingido totalmente & Atingido totalmente & Atingido satisfatoriamente & Atingido satisfatoriamente\\
\end{tabular}
\end{center}

\section*{Resenha 1: Monetaristas}
\label{sec:orge3ac468}
\subsection*{Comentários individuais}
\label{sec:orgb912840}


No parágrafo de abertura, corretamente coloca o monetarismo enquanto contraponto ao keynesianismo ortodoxo. Adiante, corretamente pontua a TQM enquanto uma teoria da demanda por moeda. Poderia apenas ter enfatizado que tais hipóteses levam à conclusão de que a demanda por moeda é estável. A seguir, corretamente contextualiza a curva de Phillips aceleracionista. No entanto, até o momento não descreve o porquê de ser aceleracionista. Também esta correta a descrição de que a curva de Phillips é vertical no longo prazo. Há, porém, uma imprecisão a respeito da ilusão monetária. Esta diz respeito da interpretação de que aumentos nominais enquanto reais. Além disso,  não há menção à taxa natural de desemprego, fundamental para compreender o porquê da curva de Phillips ser vertical no longo prazo. Por fim, apesar de completa, sua resenha não apresenta explicitamente o porquê da necessidade de regras monetárias por mais que tenha exposto os elementos para chegar a essa conclusão.
\end{document}
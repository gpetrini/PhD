% Intended LaTeX compiler: pdflatex
\documentclass[presentation]{beamer}
  \usepackage{csquotes}
\usepackage[brazilian, ]{babel}
\usetheme{default}
\author{PED: Gabriel Petrini}
\date{04 de Janeiro de 2021}
\title{Panorama Macroeconômico e a relação com a Inflação}
\begin{document}

\maketitle
\begin{frame}{Outline}
\tableofcontents
\end{frame}




\section{Introdução}
\label{sec:org09677a8}

\begin{frame}[label={sec:orgfa047db}]{Estrutura de tópicos e convenções}
Cada parte da aula terá a seguinte estrutura

\begin{itemize}
\item Apresentação dos movimentos gerais
\item Apresentação dos dados
\end{itemize}

\alert{Convenções (exceto quando especificado o contrário):}
\begin{itemize}
\item As taxas de crescimento são todas em relação ao mesmo período do ano anterior
\item As séries são desazonalizadas e em termos reais
\end{itemize}
\end{frame}

\section{O cenário internacional, o setor externo e a herança maldita}
\label{sec:orgab5bd38}

\begin{frame}[label={sec:orgc896728}]{Panorama Internacional}
\begin{itemize}
\item Crise Mexicana (1994/5)
\item Brasileira (1998/9)
\item Argentina (2001/2002)
\end{itemize}

Situações desiguais no momento de sua inserção na globalização financeira condicionaram as opções de política

\begin{itemize}
\item México na etapa inicial de processo de estabilização monetária e já havia adotado algumas reformas liberalizantes
\item Argentina e Brasil enfrentavam contexto de estabilidade macroeconômica
\end{itemize}
\end{frame}


\begin{frame}[label={sec:org86baf7b}]{Crescimento e desequilíbrios globais}
Após fase conturbada de 1997/2002, a economia global consolida um arrajno dinâmico e desequiblibrado entre 2002/3 a 2007/8

\alert{Engrenagem comercial com 3 elos}
\begin{itemize}
\item Crescimento finance-led nos EUA
\begin{itemize}
\item Déficit comercial elevado
\end{itemize}
\item Estratégia trade-account nos países asiáticos
\begin{itemize}
\item Superávit comercial chines com os EUA
\end{itemize}
\item Impactos nas comodities
\end{itemize}

\alert{Superávits em conta corrente nos países emeergentes}
\begin{itemize}
\item Fluxos líquidos oficiais de Capital
\item Diversificação internacional entre os ricos
\item Fluxos brutos de capital
\item Financiamento do déficit americano, conundrun nas taxas de juros
\end{itemize}
\end{frame}

\begin{frame}[label={sec:orge8297c7}]{A Herança Maldita}
\begin{itemize}
\item Regime de política econômica
\item Instabilidade macroeconômica
\begin{itemize}
\item Baixo ritmo de crescimento
\end{itemize}
\item Dívida pública
\begin{itemize}
\item Relação dívida/PIB em expansão
\item Elevação do gasto com juros
\item Concentrada em títulos selic e indexados ao câmbio
\end{itemize}
\item Inflação
\end{itemize}
\end{frame}


\section{Governo Lula I}
\label{sec:orgc6f0ca3}

\begin{frame}[label={sec:orgdeb0797}]{Rumos Gerais}
\begin{itemize}
\item Retomada do crescimento
\item Melhora da situação das contas externas
\begin{itemize}
\item Desendividamento externo e crescimento das reservas
\item Diminuição da dívida líquida com um aumento menos acentuado da dívida bruta
\end{itemize}
\end{itemize}
\end{frame}

\begin{frame}[label={sec:org8242e3c}]{Fatores determinantes}
\begin{itemize}
\item Impulsos externos favoráveis
\begin{itemize}
\item Melhora no setor externo pelo lado comercial (commodities) e financiero
\begin{itemize}
\item Ajudam retomada em 2004, mas não puxam o crescimento
\end{itemize}
\end{itemize}
\item Motores do crescimento (expansão do mercado interno)
\begin{itemize}
\item Distribuição de renda e cŕedito
\begin{itemize}
\item Aumento do saldo total de crédito
\end{itemize}
\item Investimento induzido
\end{itemize}
\item Investimento público a partir de 2007 com o PAC
\end{itemize}
\end{frame}

\begin{frame}[label={sec:org07d9303}]{Outros resultados:}
\begin{itemize}
\item Queda da taxa de desemprego aberta
\item Valorização real do salário mínimo
\item Expansão do gasto federal total
\item Formação Bruta de Capital Fixo cresce bastante pós-07 em taxa
\begin{itemize}
\item Apesar do nível Baixo
\item Maior que a taxa de crescimento do consumo
\item Não é um crescimento puxado pelo consumo, mas o consumo puxa o Investimento
\end{itemize}
\item Aumento do consumo das famílias, mas menor que o investimento
\end{itemize}
\end{frame}


\begin{frame}[label={sec:orgce951aa}]{Política Econômica}
De um lado, um experimento desenvolvimentista junto de uma política macroeconômica conservadora e oscilante:
\begin{itemize}
\item Início bastante ortodoxo
\begin{itemize}
\item Apreciação cambial
\item Grandes superávitis primários
\item Taxa de juros elevadas
\end{itemize}
\end{itemize}

\alert{Contexto de transição complexa:} desconfiança dos credores e pressões financeiras
\end{frame}

\section{Governo Lula: Segunda Fase}
\label{sec:orgf9e2a08}

\begin{frame}[label={sec:orgee045da}]{Traços gerais}
\begin{itemize}
\item Retomada do Estado como elemento condutor do Crescimento
\item Conjunto de medidas a favor do Crescimento
\item Desenho da política fiscal no centro de proposta do desenvolvimento
\item Remontagem da capacidade de atuação dos atores públicos
\item Não desmonta aparato regulatório do modelo anterior
\item Investimento de apoio às atividades privadas
\end{itemize}
\end{frame}

\begin{frame}[label={sec:org0158148}]{Política Cambial}
Pouca mudança ao longo do tempo
\begin{itemize}
\item Valorização com reflexo na Inflação
\begin{itemize}
\item Aumenta poder de compra da população
\item Compras de divisas não evitou valorização
\begin{itemize}
\item Desdolarização
\end{itemize}
\end{itemize}
\end{itemize}
\end{frame}

\begin{frame}[label={sec:org225a311}]{Política Monetária e Fiscal}
Manutenção do regime de metas para a Inflação em seus moldes institucionais
\begin{itemize}
\item Altos níveis de juros reais
\end{itemize}
\begin{itemize}
\item Independência operacional do Banco Central
\item Lei de responsabilidade fiscal sem mudanças
\end{itemize}
\end{frame}

\begin{frame}[label={sec:orgfd05a94}]{Descontinuidades em Relação ao Lula I}
\begin{itemize}
\item Conjugação de políticas de incentivo à renda e ao mercado interno
\item Defesa da expansão da demanda como fator de impulso ao crescimento
\begin{itemize}
\item Programa de salário mínimo e gastos sociais
\item Expansão do crédito
\item Incentivo ao mercado interno
\item Apoio ao setor industrial
\end{itemize}
\item Ações desenvolvimentistas (conceito de Dutra Fonseca)
\begin{itemize}
\item Estado (Planejamento)
\item Nacionalismo (nação como unidade de referência)
\item Indústria
\end{itemize}
\end{itemize}
\end{frame}

\section{Dificuldades ao fim do governo Lula e rumo ao Dilma I}
\label{sec:org93d1b6e}

\begin{frame}[label={sec:orga106616}]{Dificuldades}
\begin{itemize}
\item Limites do crescimento com expansão da demanda de consumo via crédito e políticas sociais
\item Retomada da taxa de investimento, mas nível baixo
\begin{itemize}
\item Dificuldade de retomada do investimento público
\item Crise mundial e investimento privado
\end{itemize}
\item Estrutura produtiva
\end{itemize}
\end{frame}
\end{document}
% Created 2020-12-14 seg 18:00
% Intended LaTeX compiler: pdflatex
\documentclass[presentation]{beamer}
  \usepackage{csquotes, caption}
\usepackage[brazilian, ]{babel}
\usepackage[style=abnt,noslsn,extrayear,uniquename=init,giveninits,justify,sccite, scbib,repeattitles,doi=false,isbn=false,url=false,maxcitenames=2, natbib=true,backend=biber]{biblatex}
\addbibresource{refs.bib}
\addbibresource{/HDD/Org/all_my_refs.bib}
\AtBeginSection[]{\begin{frame}<beamer>\frametitle{Tópicos}\tableofcontents[currentsection]\end{frame}}
\usetheme{default}
\author{PEDs: Gabriel Petrini e Marília Cubero}
\date{04 de Janeiro de 2021}
\title{Panorama Macroeconômico e a relação com a Inflação}
\begin{document}

\maketitle
\begin{frame}{Outline}
\tableofcontents
\end{frame}

\section{Introdução}
\label{sec:org80161f6}

\begin{frame}[label={sec:org54e2f06}]{Estrutura das aulas}
A aula terá a seguinte estrutura
\begin{itemize}
\item \alert{Teórico:} Conjuntura pré-Regime de Metas e revisão teórica
\begin{itemize}
\item Diagrama analítico
\item Estrutura das atas do COPOM
\item Exemplificando com uma ata
\end{itemize}
\item \alert{Macroeconômico:} Apresentação dos dados e dos movimentos gerais
\begin{itemize}
\item Parte I: 2002-2011
\item Parte II: 2011-2020
\end{itemize}
\end{itemize}
\end{frame}


\section{Preâmbulo: Cenário internacional adverso, Implementação do RMI e a herança maldita}
\label{sec:org8467b32}

\begin{frame}[label={sec:orgd694309}]{A Herança Maldita \cite{belluzzoDepoisQuedaEconomia2002}}
\begin{itemize}
\item Desmonte ossatura Estado Desenvolvimentista
\item Crise Russa, Mexicana, Argentina \(\Rightarrow\) \emph{flight to quality}
\item Instabilidade macroeconômica \(\Rightarrow\) \emph{stop and go}
\item \(\Uparrow\) Dívida/PIB
\item Choque inflacionário \(\Rightarrow\) Câmbio e energia
\end{itemize}

\alert{Herança Maldita:}

\begin{itemize}
\item Instabilidade macroeconômica \(\Leftrightarrow \Uparrow r, \Downarrow e\)
\item Desemprego elevado
\item Crises bancárias e do Balanço de Pagamentos
\end{itemize}

\begin{block}{1999: Implementação do RMI}
\end{block}
\end{frame}

\section{RMI: Esquema analítico}
\label{sec:org1731fb2}

\begin{frame}[label={sec:org46351e0}]{Diagrama}
\begin{figure}[htb]
\centering
\caption{Representação do Modelo do regime de Metas para inflação} 
\includegraphics[width = 0.9\textwidth]{./figs/RMI.png}
\label{fig:ibcbr}
\caption*{\textbf{Fonte:} Elaboração própria}
\end{figure}
\end{frame}
\begin{frame}[label={sec:orgd37f975}]{Estrutura das Atas do Copom}
\begin{itemize}
\item Evolução recente da economia \(\Rightarrow\) \(h_{t-m}, h_{t}, (\pi_{t} - \pi^{\star}), \varepsilon^{D}, \varepsilon^{S}\)
\item Avaliação prospectiva das tendências de inflação \(\Rightarrow\) \(\pi^{e}\)
\item Implementação da política monetária \(\Rightarrow\) \(i_{t}, (\pi_{t} - \pi^{\star}), \pi^{e}_{t+n}\)
\item Inflação \(\Rightarrow\) Componentes de \(\pi\)
\item Atividade econômica \(\Rightarrow\) crédito, produção, NUCI, etc
\item Expectativas e sondagens \(\Rightarrow\) \(\pi^{e}\)
\item Mercado de trabalho \(\Rightarrow\) redimentos, taxa de desemprego
\item Crédito e inadimplência
\item Ambiente externo \(\Rightarrow\) \(\Delta e, \Delta p^{F}, \Delta i^{F}, \Delta x, \varepsilon^{F}\)
\item Comércio exterior e reservas internacionais
\begin{itemize}
\item Componentes do BP, fluxo de capitais e etc
\end{itemize}
\item Mercado monetário e operações de mercado aberto
\begin{itemize}
\item \alert{Memo:} Quantidade de moeda endógena
\item \(i_{t}\), spread bancário, oferta de títulos públicos (LTN, LFT, etc)
\end{itemize}
\end{itemize}
\end{frame}

\begin{frame}[label={sec:org3ae8d63}]{Exemplo: \href{https://www.bcb.gov.br/publicacoes/atascopom/28102020}{Ata Nº234}}
\begin{itemize}
\item \alert{Setor externo:} \(\Downarrow\) retomada de alguns setores (\(\varepsilon^{F}\))
\item Moderação da volatilidade dos ativos financeiros \(\Rightarrow\) favorece economias emergentes
\item \alert{Mercado doméstico:} Recuperação desigual
\begin{itemize}
\item \(\Uparrow\) incerteza sobre a retomada \(\Leftrightarrow \Downarrow Y_{\text{Emerg.}}\)
\end{itemize}
\item \(\Uparrow\) projeção da inflação (\(\pi^{e}\))
\begin{itemize}
\item \(\Uparrow p_{\text{Alimentos}}, p_{\text{Bens ind.}} \Leftrightarrow\) depreciação do Real e das \emph{commodities} (\(\Delta e, \Delta p^{F}\))
\item Perspectiva de normalização de preços com aumento extraordinário \(\Rightarrow \Downarrow\) produção e \(\Uparrow\) demanda (\(\varepsilon^{S}, \varepsilon^{D}\))
\end{itemize}
\item Ociosidade no setor de serviços (\(h_{t}_{\text{Serv}}\))
\item \(Y_{\text{Emerg.}}\) e adiamento de reformas \(\Rightarrow\) \(\Uparrow\) prêmio de risco (\(\Delta x\))
\item \(\Downarrow \Downarrow\) taxas de juros \(\Rightarrow\) instabilidade \(p_{\text{ativos}}\)
\begin{itemize}
\item Proximidade à ZLB (\(i_{t} \to 0\))
\end{itemize}
\item Projeções de inflação abaixo da meta (\(\pi_{t} < \pi^{\star}\))
\begin{itemize}
\item Expectativas de longo prazo ancoradas (\(\pi^{e} \to \pi^{\star}\))
\end{itemize}
\end{itemize}
\end{frame}

\begin{frame}[label={sec:orgec93b95}]{Exemplo: \href{https://www.bcb.gov.br/publicacoes/atascopom/28102020}{Ata Nº234 03/Nov/2020}}
\begin{figure}[htb]
\centering
\caption{Representação de Análise de uma ata do COPOM}
\includegraphics[width = 0.9\textwidth]{./figs/RMI_234.png}
\label{fig:ata234}
\caption*{\textbf{Fonte:} Elaboração própria}
\end{figure}
\end{frame}
\section{Governos Lula e suas fases}
\label{sec:org909d258}

\begin{frame}[label={sec:orgc04ef81}]{Crescimento e desequilíbrios globais \cite{carneiroSupremaciaDosMercados2006}}
Após fase conturbada de 1997/2002, a economia global consolida um arrajno dinâmico e desequiblibrado entre 2002/3 a 2007/8

\alert{Engrenagem comercial com 3 elos}
\begin{itemize}
\item Crescimento finance-led nos EUA
\begin{itemize}
\item Déficit comercial elevado
\end{itemize}
\item Estratégia trade-account nos países asiáticos
\begin{itemize}
\item Superávit comercial chinês com os EUA
\end{itemize}
\item Impactos nas commodities
\end{itemize}
\end{frame}

\begin{frame}[label={sec:orga11276d}]{Primeira Fase (2002-2006)}
Experimento desenvolvimentista junto de uma política macroeconômica conservadora:
\begin{itemize}
\item Continuidade do tripé
\item Visão teórica da \alert{política fiscal:} contração fiscal expansionista
\item \alert{Estratégia de crescimento:} visão liberal predominante
\begin{itemize}
\item reformas microeconômicas
\item regras estáveis de gestão
\item ampliar ajuste fiscal
\end{itemize}
\item Apreciação cambial
\item Taxa de juros elevadas
\end{itemize}

\alert{Contexto de transição complexa:} desconfiança dos credores e pressões financeiras
\end{frame}

\begin{frame}[label={sec:orge730534}]{Segunda fase (2006-2010)}
\begin{itemize}
\item Retomada do Estado como elemento condutor do Crescimento
\item Desenho da política fiscal no centro de proposta do desenvolvimento
\item Remontagem da capacidade de atuação dos atores públicos
\item Não desmonta aparato regulatório do modelo anterior
\item Investimento de apoio às atividades privadas
\end{itemize}
\end{frame}

\begin{frame}[label={sec:org133531b}]{Continuidades}
\begin{block}{Política cambial}
Pouco mudou ao longo do tempo

\begin{itemize}
\item Valorização com reflexo na inflação
\item Compras de divisas não evitou valorização
\item Impacto sobre o setor industrial
\end{itemize}
\end{block}

\begin{block}{Política monetária}
Rígido regime de metas de inflação

\begin{itemize}
\item altos níveis de juros reais
\item discussão sobre independência do Banco Central
\item conflito com meta de taxa de juros "desenvolvimentista"
\end{itemize}
\end{block}

\begin{block}{Política fiscal}
Não foi alterado o regime fiscal definido na era FHC

\begin{itemize}
\item Lei de Responsabilidade Fiscal sem mudanças
\item Não alterou mercado de dívida pública
\end{itemize}
\end{block}
\end{frame}

\begin{frame}[label={sec:org9306c02}]{Descontinuidades}
Conjugação de políticas de incentivo à renda e ao mercado interno

\begin{itemize}
\item Defesa da expansão da demanda como fator de impulso ao crescimento
\begin{itemize}
\item Política deliberada de inserção social
\item Expansão do crédito
\item \alert{Programa de Salário Mínimo}
\end{itemize}
\item Ações desenvolvimentistas
\begin{itemize}
\item Políticas de incentivo ao investimento: PAC e PDP
\item Gasto público como estratégia para elevar o crescimento
\item Política de fortalecimento dos Bancos Públicos e das empresas estatais
\end{itemize}
\end{itemize}
\end{frame}

\begin{frame}[label={sec:orgfe059b6}]{Alguns resultados}
\begin{itemize}
\item Queda da taxa de desemprego aberta
\item Valorização real do salário mínimo
\item Expansão do gasto federal total
\item Taxa crescimento da FBCF cresce pós-07
\item Aumento do consumo das famílias, mas menor que o investimento
\end{itemize}
\end{frame}

\begin{frame}[label={sec:org083bb91}]{Fatores determinantes}
\begin{itemize}
\item Impulsos externos favoráveis
\begin{itemize}
\item Melhora no setor externo pelo lado comercial (commodities) e financiero
\begin{itemize}
\item Ajudam retomada em 2004, mas não puxam o crescimento
\end{itemize}
\end{itemize}
\item Motores do crescimento (expansão do mercado interno)
\begin{itemize}
\item Distribuição de renda e cŕedito
\begin{itemize}
\item Aumento do saldo total de crédito
\end{itemize}
\item Investimento induzido
\end{itemize}
\item Investimento público a partir de 2007 com o PAC
\end{itemize}
\end{frame}

\section{Dados 2002-2011}
\label{sec:org01f38a5}

\begin{frame}[label={sec:org19b0086}]{PIB puxado pelo mercado doméstico}
\begin{figure}[htb]
\centering
\caption{Decomp. tx de crescimento do produto - Domésticos e externos} 
\includegraphics[width = 0.95\textwidth]{./figs/PIB_Decomp_I.png}
\label{fig:cycles}
\caption*{\textbf{Fonte:} BCB}
\end{figure}
\end{frame}


\begin{frame}[label={sec:org63eee7a}]{Crescimento e o investimento induzido}
\begin{figure}[htb]
\centering
\caption{Taxa de crescimento do produto - decomposição total} 
\includegraphics[width = 0.95\textwidth]{./figs/PIB_Decomp_Total_I.png}
\label{fig:PIB_Decomp_Total}
\caption*{\textbf{Fonte:} BCB}
\end{figure}
\end{frame}




\begin{frame}[label={sec:orgae58a3b}]{Investimento público em recuperação \cite{orair_investimento_2016}}
\begin{table}[htbp]
\caption{Taxa de crescimento do investimentos públicos (1994-2015)}
\centering
\begin{tabular}{rrrrr}
\hline
Ano & Gov. Central & Gov. Geral & Setor Público & PIB\\
\hline
1994-1998 & -5.1 & -2.7 & -0.9 & 2.6\\
1998-2002 & -1.2 & -2.0 & -1.9 & 2.3\\
2002-2006 & -0.6 & 0.6 & 0.4 & 3.5\\
2006-2010 & 25.4 & 13.5 & 17.0 & 4.6\\
2010-2014 & -0.4 & -0.1 & -0.1 & 2.2\\
2014-2015 & -6.2 & -4.0 & -5.2 & 0.3\\
\hline
\end{tabular}
\end{table}
\end{frame}



\begin{frame}[label={sec:orgb083449}]{Emprego}
\begin{figure}[htb]
\centering
\caption{Índice do Emprego Formal} 
\includegraphics[width = 0.9\textwidth]{./figs/EmpregoFormal_I.png}
\label{fig:EmpFormal}
\caption*{\textbf{Fonte:} MTb}
\end{figure}
\end{frame}


\begin{frame}[label={sec:orgc4b353c}]{Taxa de câmbio nominal}
\begin{figure}[htb]
\centering
\caption{ Índice da taxa de câmbio efetiva nominal\\Jun/1994=100 } 
\includegraphics[width = 0.95\textwidth]{./figs/CambioNominal_I.png}
\label{fig:cambio}
\caption*{\textbf{Fonte:} BCB-DSTAT}
\end{figure}
\end{frame}


\begin{frame}[label={sec:org1c7fa67}]{Taxa de juros selic}
\begin{figure}[htb]
\centering
\caption{Taxa de juros selic a.a. (efetivo x meta)\\Anualizada base 252} 
\includegraphics[width = 0.95\textwidth]{./figs/Selic_I.png}
\label{fig:Selic}
\caption*{\textbf{Fonte:} Copom e BCB-Demab}
\end{figure}
\end{frame}


\begin{frame}[label={sec:org5b20894}]{Finalmente, inflação (ops, IPCA)}
\begin{figure}[htb]
\centering
\caption{IPCA e Metas para Inflação} 
\includegraphics[width = 0.975\textwidth]{./figs/IPCA_I.png}
\label{fig:IPCA}
\caption*{\textbf{Fonte:} BCB}
\end{figure}
\end{frame}
\begin{frame}[label={sec:orge1e97ec}]{Composição do IPCA \cite{bcb_2019_Atualizacoes}}
\begin{table}[htbp]
\caption{IPCA: estruturas de ponderação – janeiro de 2018}
\centering
\begin{tabular}{lrrr}
\hline
Grupo & 2008-2009 & 2017-2018 & \(\Delta\)\\
\hline
Alimentação e bebidas & 24.58 & 18.99 & -5.59\\
Habitação & 15.72 & 15.16 & -0.56\\
Artigos de residência & 3.98 & 4.02 & 0.04\\
Vestuário & 5.96 & 4.8 & -1.16\\
Transporte & 18.28 & 20.84 & 2.55\\
Saúde e cuidados pessoais & 12.04 & 13.46 & 1.41\\
Despesas pessoais & 10.96 & 10.60 & -0.36\\
Educação & 4.83 & 5.95 & 1.13\\
Comunicação & 3.65 & 6.19 & 2.54\\
\hline
\end{tabular}
\end{table}
\end{frame}
\begin{frame}[label={sec:org592814f}]{Preços livres, monitorados, serviços}
\begin{figure}[htb]
\centering
\caption{IPCA e seus componentes: preços livres, monitorados e serviços} 
\includegraphics[width = 0.975\textwidth]{./figs/Livres_Administrados_I.png}
\label{fig:livres_adm}
\caption*{\textbf{Fonte:} BCB}
\end{figure}
\end{frame}


\section{Governo(s?) Dilma}
\label{sec:org63be10d}

\begin{frame}[label={sec:org504d2c9}]{Os três motores \cite{serrano_demanda_2015}}
\begin{figure}[htb]
\centering
\caption{Os três motores do crescimento} 
\includegraphics[width = 0.75\textwidth]{./figs/Tenores.png}
\label{fig:tenores}
\caption*{\textbf{Fonte:} Elaboração própria}
\end{figure}

\alert{Principal mudança:} do incentivo à demanda agregada ao incetivo ao investimento privado.
\end{frame}

\begin{frame}[label={sec:org687622e}]{Primeira fase (2011-2014)}
\alert{Estado} na estratégia de crescimento econômico
\begin{itemize}
\item Políticas macroeconômicas \(\Rightarrow \Uparrow\) investimento privado
\item Reconhecimento da \alert{indústria} como determinante do crescimento
\end{itemize}

\begin{block}{Mudança no contexto doméstico e internacional}
\begin{itemize}
\item \alert{Doméstico:} Aceleração da inflação, expansão do crédito \(\Rightarrow\) medidas macroprudenciais
\item \alert{Internacional:} Cenário externo permissivo (miniciclo das \emph{commodities})
\end{itemize}
\end{block}
\end{frame}


\begin{frame}[label={sec:orgbd2455e}]{Dificuldades em aberto \cite{mello_2017_industrialismo}}
\begin{itemize}
\item Limites do crescimento com expansão da demanda de consumo via crédito e políticas sociais
\item Retomada da taxa de investimento, mas nível baixo
\begin{itemize}
\item Dificuldade de retomada do investimento público
\item Crise mundial e investimento privado
\end{itemize}
\item Estrutura produtiva
\end{itemize}
\end{frame}

\begin{frame}[label={sec:org689bd99}]{Guinadas de política econômica}
\begin{block}{Política fiscal}
Isenções fiscais \(\Rightarrow\) incentivar investimento privado
\end{block}

\begin{block}{Política monetária}
Redução na taxa de juros \(\Rightarrow \Downarrow\) juros de longo prazo
\end{block}

\begin{block}{Política cambial}
Intervenção no mercado de derivativos \(\Rightarrow\) \alert{Desvalorização} cambial \(\Rightarrow \Uparrow\) Indústria
\end{block}

\begin{block}{Rupturas}
\begin{itemize}
\item Redução da arrecadação \(\Rightarrow \Uparrow\) dívida/PIB
\item Câmbio valorizado \(\Leftrightarrow\) Juros elevados \(\nRightarrow\) Controle inflacionário
\end{itemize}
\end{block}
\end{frame}

\begin{frame}[label={sec:org3912a52}]{Medidas macroprudenciais}
\alert{Medidas macroprudenciais:} Redução do crescimento do crédito.

\begin{itemize}
\item Redução da taxa de crescimento da renda disponível real
\item Aumento dos depósitos compulsórios
\item Aumento do capital mínimo exigido dos bancos para empréstimos ao consumidor de prazos mais longos
\item Aumento do percentual mínimo de pagamento de cartões de crédito
\end{itemize}

\begin{block}{Implicações}
\begin{itemize}
\item Aumento do spread do crédito ao consumo final
\item Diminuição dos prazos
\item Elimina \alert{boom} de consumo
\item Aumento da inadimplência
\end{itemize}
\end{block}
\end{frame}

\begin{frame}[label={sec:org4cc5260}]{Fatores internacionais, políticos e institucionais}
\begin{block}{Internacionais \(\Rightarrow\) deterioração do BP e da indústria}
\begin{itemize}
\item Crise do Euro
\item Redução do ritmo de crescimento da China
\item Fim do miniciclo de \emph{commodities} e câmbio desvalorizado
\end{itemize}
\end{block}
\begin{block}{Políticos  e institucionais \(\Rightarrow\) Incerteza eleitoral}
\begin{itemize}
\item Manifestações de 2013
\item Avanço da Operação Lava Jato
\item Cutucando onças com varas curtas \cite{singer_cutucando_2015}
\end{itemize}
\end{block}
\end{frame}

\begin{frame}[label={sec:orgf33f8f1}]{Desaceleração rudimentar I}
\begin{table}[htbp]
\caption{Comparação das taxas de crescimento}
\centering
\begin{tabular}{lll}
\hline
 & 2004-2010 & 2011-2014\\
\hline
PIB & 4.4\% & 2.1\%\\
Produção industrial & 3.6\% & -0.9\%\\
Taxa de desemprego & 9.0\% & 5.4\%\\
Crédito para habitação & 21,5\% & 4,6\%\\
Hipotecas & 20,1\% & 29,3\%\\
Salário real (emp. formal) & 2,9\% & 2,9\%\\
Renda disp. (Famílias) & 5,3\% & 1,2\%\\
\hline
\end{tabular}
\end{table}
\end{frame}


\begin{frame}[label={sec:org8ef045c}]{Desaceleração rudimentar II}
\begin{table}[htbp]
\caption{Política fiscal}
\centering
\begin{tabular}{lll}
\hline
 & 2004-2010 & 2011-2014\\
\hline
Superávit primário/PIB & 3,2\% & 1,7\%\\
Receitas do setor público & 7,2\% & 1,2\%\\
Transf. públicas para as famílias & 5,6\% & 4,9\%\\
Invest. Emp. Estatais (Federal) & 16,3\% & -2,7\%\\
Investimento Adm, Pública & 14,0\% & -1,0\%\\
\hline
\end{tabular}
\end{table}

\alert{Principal mudança:} do industrialismo à austeridade
\end{frame}

\begin{frame}[label={sec:org599ca6c}]{Segunda fase (2015-?)}
Choques recessivos \(\Rightarrow\) enfrentar os "desequilíbrios" da economia brasileira:

\begin{itemize}
\item \alert{Fiscal:} Cortes dos gastos do governo
\item \alert{Cambial:} Desvalorização acentuada (50\%)
\item \alert{Preços administrados:} realinhamento (choque) das tarifas de eletricidade e combustível
\item \alert{Monetário:} Elevação da taxa de juros (14,25\%)
\end{itemize}
\end{frame}

\begin{frame}[label={sec:org8366b0b}]{\href{https://www.causaoperaria.org.br/brasil-o-golpe-a-opera-do-fim-do-mundo-artista-retrata-o-golpe-de-estado-no-pais/}{Prêambulo para a ópera do fim do mundo}}
\begin{figure}[htb]
\centering
\caption{Brasil, O Golpe: A Ópera do fim do mundo} 
\includegraphics[width = 0.9\textwidth]{./figs/opera.png}
\caption*{\textbf{Fonte:} Jornal GGN}
\end{figure}
\end{frame}

\section{Governos (?) Temer e Início Bolsonaro}
\label{sec:org8da5895}
\begin{frame}[label={sec:orge530860}]{Continuidades: Reformas}
Desde o Governo Temer (2016-2019) até o Governo Bolsonaro (2019-) houve uma mudança profunda na condução da política econômica:
\begin{itemize}
\item Políticas ortodoxas
\item Reformas liberalizantes
\begin{itemize}
\item \alert{Trabalhista (2016):} Flexibilização no mercado de trabalho
\item \alert{Previdência (2019):} Justificativa de reduzir o déficit da previdência
\item \alert{Tributária (?):} A caminho
\item \alert{Privatizações:} Já iniciadas pelas subsidiárias e avançando
\item \alert{Abertura comercial:} \(\Downarrow\) Tarifas alfandegária \(\Rightarrow \Uparrow\) competitividade via mercado
\end{itemize}
\end{itemize}
\end{frame}

\begin{frame}[label={sec:org8b2bc2d}]{Continuidades: Políticas macroeconômicas}
\begin{block}{Política fiscal}
\begin{itemize}
\item Implementação do teto de gastos \(\Rightarrow\) contracionista \(\Leftrightarrow\) contração fiscal expansionista
\item Liberalização de recursos (FGTS e PIS/PASEP) \(\Rightarrow\) estimular demanda agregada
\end{itemize}
\end{block}
\begin{block}{Política monetária}
\begin{itemize}
\item Política monetária conservadora \(\Rightarrow\) ancorar expectativas inflacionárias
\end{itemize}
\end{block}
\begin{block}{Política cambial}
Mais flexível \(\Rightarrow\) conversibilidade do real
\end{block}
\end{frame}

\section{Corona-crise}
\label{sec:org82716c3}

\begin{frame}[label={sec:org0797522}]{Crise sanitária \(\Rightarrow\) crise econômica? \href{https://www.economia.unicamp.br/covid19/o-impacto-economico-da-pandemia-do-covid-19-e-a-contracao-do-pib-no-primeiro-trimestre-de-2020-nao-e-culpa-da-politica-de-saude-publica}{Nota Cecon Nº14}}
\begin{itemize}
\item A contração do PIB tem forte relação com a pandemia do Covid-19, mas não com as políticas de saúde pública
\item Tendência de desaceleração do IBC-Br desde o último trimestre de 2019.
\begin{itemize}
\item Contágio econômico: Desaceleração \(\Rightarrow\) contração
\end{itemize}
\item Desaceleração econômica é relativamente mais rápida que a difusão da pandemia e anterior à implementação das primeiras políticas de isolamento social.
\end{itemize}

\begin{block}{Resumo}
É a \alert{pandemia} que deprime a economia, e não as políticas de saúde pública capazes de controlá-la.

Crise econômica \(\Leftrightarrow\) Pandemia

Crise econômcia \(\nLeftrightarrow\) Isolamento social \(\Rightarrow \Downarrow\) Pandemia
\end{block}
\end{frame}

\begin{frame}[label={sec:org5846345}]{Políticas econômicas}
\begin{block}{Medidas de proteção da renda}
\begin{itemize}
\item Auxílio emergencial (R\$600,00 \(\approx\) US\$ 120,00)
\item Liberação de liquidez no sistema financeiro internacional
\item Atenuação de requerimento de capital
\item Políticas de crédito para o setor empresarial
\end{itemize}
\end{block}
\end{frame}
\section{Dados 2011-2020}
\label{sec:orgd4e65ae}



\begin{frame}[label={sec:org0e0d5be}]{PIB puxado pelo mercado doméstico}
\begin{figure}[htb]
\centering
\caption{Decomposição da taxa de crescimento do produto - Domésticos e externos} 
\includegraphics[width = 0.9\textwidth]{./figs/PIB_Decomp.png}
\label{fig:cycles}
\caption*{\textbf{Fonte:} BCB}
\end{figure}
\end{frame}


\begin{frame}[label={sec:org73f98b4}]{Crescimento e o investimento induzido}
\begin{figure}[htb]
\centering
\caption{Taxa de crescimento do produto - decomposição total} 
\includegraphics[width = 0.9\textwidth]{./figs/PIB_Decomp_Total.png}
\label{fig:PIB_Decomp_Total}
\caption*{\textbf{Fonte:} BCB}
\end{figure}
\end{frame}



\begin{frame}[label={sec:org5cc11ec}]{Emprego}
\begin{figure}[htb]
\centering
\caption{Índice do Emprego Formal} 
\includegraphics[width = 0.9\textwidth]{./figs/EmpregoFormal.png}
\label{fig:EmpFormal}
\caption*{\textbf{Fonte:} MTb}
\end{figure}
\end{frame}





\begin{frame}[label={sec:org19a79e9}]{Taxa de câmbio nominal}
\begin{figure}[htb]
\centering
\caption{ Índice da taxa de câmbio efetiva nominal\\Jun/1994=100 } 
\includegraphics[width = 0.9\textwidth]{./figs/CambioNominal.png}
\label{fig:cambio}
\caption*{\textbf{Fonte:} BCB-DSTAT}
\end{figure}
\end{frame}


\begin{frame}[label={sec:orge714c95}]{Taxa de juros selic}
\begin{figure}[htb]
\centering
\caption{Taxa de juros selic a.a. (efetivo x meta)\\Anualizada base 252} 
\includegraphics[width = 0.9\textwidth]{./figs/Selic.png}
\label{fig:Selic}
\caption*{\textbf{Fonte:} Copom e BCB-Demab}
\end{figure}
\end{frame}


\begin{frame}[label={sec:orga611b61}]{Finalmente, inflação (ops, IPCA)}
\begin{figure}[htb]
\centering
\caption{IPCA e Metas para Inflação} 
\includegraphics[width = 0.9\textwidth]{./figs/IPCA.png}
\label{fig:IPCA}
\caption*{\textbf{Fonte:} BCB}
\end{figure}
\end{frame}
\begin{frame}[label={sec:org756aede}]{Preços livres, monitorados, serviços}
\begin{figure}[htb]
\centering
\caption{IPCA e seus componentes: preços livres, monitorados e serviços} 
\includegraphics[width = 0.95\textwidth]{./figs/Livres_Administrados.png}
\label{fig:livres_adm}
\caption*{\textbf{Fonte:} BCB}
\end{figure}
\end{frame}

\begin{frame}[label={sec:orge7370c4}]{Consumo de energia não-residencial (dados diários)}
\begin{figure}[htb]
\centering
\includegraphics[width = 0.95\textwidth]{./figs/Energia.png}
\label{fig:cycles}
\caption*{\textbf{Fonte:} \textcite{bastos_2020_impacto}}
\end{figure}
\end{frame}

\begin{frame}[label={sec:org9789174}]{Indicadores de antecedentes: IBC-Br}
\begin{figure}[htb]
\centering
\includegraphics[width = 0.95\textwidth]{./figs/IBCBr_corona.png}
\label{fig:cycles}
\caption*{\textbf{Fonte:} \textcite{bastos_2020_impacto}}
\end{figure}
\end{frame}
\section{Referências}
\label{sec:org2413a03}

\begin{frame}[label={sec:org89fadde}]{Referências}

\printbibliography
\end{frame}
\end{document}

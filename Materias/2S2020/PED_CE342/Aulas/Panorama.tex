% Created 2020-10-09 sex 19:39
% Intended LaTeX compiler: pdflatex
\documentclass[presentation]{beamer}
  \usepackage{csquotes, caption}
\usepackage[brazilian, ]{babel}
\usetheme{default}
\author{PED: Gabriel Petrini}
\date{04 de Janeiro de 2021}
\title{Panorama Macroeconômico e a relação com a Inflação}
\begin{document}

\maketitle
\begin{frame}{Outline}
\tableofcontents
\end{frame}


\section{Introdução}
\label{sec:orgfee4e1f}

\begin{frame}[label={sec:orgf53fa8f}]{Estrutura de tópicos e convenções}
A aula terá a seguinte estrutura

\begin{itemize}
\item Apresentação dos movimentos gerais
\item Apresentação dos dados
\end{itemize}

\alert{Convenções (exceto quando especificado o contrário):}
\begin{itemize}
\item As taxas de crescimento são todas em relação ao mesmo período do ano anterior
\item As séries são desazonalizadas e em termos reais
\end{itemize}
\end{frame}

\section{O cenário internacional, o setor externo e a herança maldita}
\label{sec:org12c75d3}

\begin{frame}[label={sec:org51cb1ae}]{Panorama Internacional}
\begin{itemize}
\item Crise Mexicana (1994/5)
\item Brasileira (1998/9)
\item Argentina (2001/2002)
\end{itemize}
\end{frame}

\begin{frame}[label={sec:orgee6c1ab}]{Crescimento e desequilíbrios globais}
Após fase conturbada de 1997/2002, a economia global consolida um arrajno dinâmico e desequiblibrado entre 2002/3 a 2007/8

\alert{Engrenagem comercial com 3 elos}
\begin{itemize}
\item Crescimento finance-led nos EUA
\begin{itemize}
\item Déficit comercial elevado
\end{itemize}
\item Estratégia trade-account nos países asiáticos
\begin{itemize}
\item Superávit comercial chinês com os EUA
\end{itemize}
\item Impactos nas commodities
\end{itemize}

\alert{Superávits em conta corrente nos países emeergentes}
\begin{itemize}
\item Fluxos líquidos oficiais de Capital
\item Financiamento do déficit americano, conundrun nas taxas de juros
\end{itemize}
\end{frame}

\begin{frame}[label={sec:org01f57a0}]{A Herança Maldita}
\begin{itemize}
\item Instabilidade macroeconômica
\begin{itemize}
\item Baixo ritmo de crescimento
\end{itemize}
\item Dívida pública
\begin{itemize}
\item Relação dívida/PIB em expansão
\item Elevação do gasto com juros
\item Concentrada em títulos selic e indexados ao câmbio
\end{itemize}
\item Inflação
\end{itemize}
\end{frame}


\section{Governos Lula e suas fases}
\label{sec:orgf2f86e9}

\begin{frame}[label={sec:org89020f5}]{Primeira Fase}
Experimento desenvolvimentista junto de uma política macroeconômica conservadora:
\begin{itemize}
\item Continuidade do tripé
\item Visão teórica da \alert{política fiscal:} contração fiscal expansionista
\item \alert{Estratégia de crescimento:} visão liberal predominante
\begin{itemize}
\item reformas microeconômicas
\item regras estáveis de gestão
\item ampliar ajuste fiscal
\end{itemize}
\item Apreciação cambial
\item Taxa de juros elevadas
\end{itemize}

\alert{Contexto de transição complexa:} desconfiança dos credores e pressões financeiras
\end{frame}

\begin{frame}[label={sec:org0a99b08}]{Segunda fase}
\begin{itemize}
\item Retomada do Estado como elemento condutor do Crescimento
\item Desenho da política fiscal no centro de proposta do desenvolvimento
\item Remontagem da capacidade de atuação dos atores públicos
\item Não desmonta aparato regulatório do modelo anterior
\item Investimento de apoio às atividades privadas
\end{itemize}
\end{frame}


\begin{frame}[label={sec:org4744f6a}]{Fatores determinantes}
\begin{itemize}
\item Impulsos externos favoráveis
\begin{itemize}
\item Melhora no setor externo pelo lado comercial (commodities) e financiero
\begin{itemize}
\item Ajudam retomada em 2004, mas não puxam o crescimento
\end{itemize}
\end{itemize}
\item Motores do crescimento (expansão do mercado interno)
\begin{itemize}
\item Distribuição de renda e cŕedito
\begin{itemize}
\item Aumento do saldo total de crédito
\end{itemize}
\item Investimento induzido
\end{itemize}
\item Investimento público a partir de 2007 com o PAC
\end{itemize}
\end{frame}

\begin{frame}[label={sec:orgb7fece7}]{Alguns resultados}
\begin{itemize}
\item Queda da taxa de desemprego aberta
\item Valorização real do salário mínimo
\item Expansão do gasto federal total
\item Formação Bruta de Capital Fixo cresce bastante pós-07 em taxa
\begin{itemize}
\item Apesar do nível Baixo
\item Maior que a taxa de crescimento do consumo
\item Não é um crescimento puxado pelo consumo, mas o consumo puxa o Investimento
\end{itemize}
\item Aumento do consumo das famílias, mas menor que o investimento
\end{itemize}
\end{frame}




\section{(Des)Continuidades e dificuldades}
\label{sec:org53956e8}

\begin{frame}[label={sec:org151f19d}]{Continuidades}
\begin{block}{Política cambial}
Pouco mudou ao longo do tempo

\begin{itemize}
\item Valorização com reflexo na inflação
\item Compras de divisas não evitou valorização
\item Impacto sobre o setor industrial
\end{itemize}
\end{block}

\begin{block}{Política monetária}
Rígido regime de metas de inflação

\begin{itemize}
\item altos níveis de juros reais
\item discussão sobre independência do Banco Central
\item conflito com meta de taxa de juros "desenvolvimentista"
\end{itemize}
\end{block}

\begin{block}{Política fiscal}
Não foi alterado o regime fiscal definido na era FHC

\begin{itemize}
\item Lei de Responsabilidade Fiscal sem mudanças
\item Não alterou mercado de dívida pública
\end{itemize}
\end{block}
\end{frame}


\begin{frame}[label={sec:org9e2855b}]{Descontinuidades}
Conjugação de políticas de incentivo à renda e ao mercado interno

\begin{itemize}
\item Defesa da expansão da demanda como fator de impulso ao crescimento
\begin{itemize}
\item Política deliberada de inserção social
\item Expansão do crédito
\item \alert{Programa de Salário Mínimo}
\end{itemize}
\item Ações desenvolvimentistas
\begin{itemize}
\item Políticas de incentivo ao investimento: PAC e PDP
\item Gasto público como estratégia para elevar o crescimento
\item Política de fortalecimento dos Bancos Públicos e das empresas estatais
\end{itemize}
\end{itemize}
\end{frame}


\begin{frame}[label={sec:org8ad6c51}]{Dificuldades}
\begin{itemize}
\item Limites do crescimento com expansão da demanda de consumo via crédito e políticas sociais
\item Retomada da taxa de investimento, mas nível baixo
\begin{itemize}
\item Dificuldade de retomada do investimento público
\item Crise mundial e investimento privado
\end{itemize}
\item Estrutura produtiva
\end{itemize}
\end{frame}

\section{Governo(s?) Dilma}
\label{sec:orga47a162}

\begin{frame}[label={sec:orgd9166f1}]{Os três tenores}
\end{frame}

\begin{frame}[label={sec:orgec4495d}]{Medidas macroprudenciais}
\alert{Medidas macroprudenciais:*} Redução do crescimento do crédito.

\begin{itemize}
\item Redução da taxa de crescimento da renda disponível real
\item Aumento dos depósitos compulsórios
\item Aumento do capital mínimo exigido dos bancos para empréstimos ao consumidor de prazos mais longos
\item Aumento do percentual mínimo de pagamento de cartões de crédito
\end{itemize}

\alert{\alert{Implicações:}}

\begin{itemize}
\item Aumento do spread do crédito ao consumo final
\item Diminuição dos prazos
\item Elimina \alert{boom} de consumo
\item Aumento da inadimplência
\end{itemize}
\end{frame}


\begin{frame}[label={sec:org3764534}]{Desaceleração rudimentar I}
\begin{center}
\begin{tabular}{lll}
\hline
 & 2004-2010 & 2011-2014\\
PIB & 4.4\% & 2.1\%\\
Produção industrial & 3.6\% & -0.9\%\\
Taxa de desemprego & 9.0\% & 5.4\%\\
\hline
\end{tabular}
\end{center}

\alert{Principal mudança:} do incentivo à demanda agregada ao incetivo ao investimento privado.
\end{frame}

\begin{frame}[label={sec:org516dd4d}]{Desaceleração rudimentar II}
\begin{table}[htbp]
\caption{Consumo das famílias}
\centering
\begin{tabular}{lll}
\hline
 & 2004-2010 & 2011-2014\\
Crédito para habitação & 21,5\% & 4,6\%\\
Hipotecas & 20,1\% & 29,3\%\\
Salário real (emprego formal) & 2,9\% & 2,9\%\\
Renda disponível real das famílias & 5,3\% & 1,2\%\\
\hline
\end{tabular}
\end{table}
\end{frame}

\begin{frame}[label={sec:orgaa70703}]{Desaceleração rudimentar III}
\begin{table}[htbp]
\caption{Política fiscal}
\centering
\begin{tabular}{lll}
 & 2004-2010 & 2011-2014\\
Superávit primário/PIB & 3,2\% & 1,7\%\\
Receitas do setor público & 7,2\% & 1,2\%\\
Transferências públicas para as famílias & 5,6\% & 4,9\%\\
Investimento das Empresas Estatais (Federal) & 16,3\% & -2,7\%\\
Investimento Adm, Pública & 14,0\% & -1,0\%\\
\end{tabular}
\end{table}
\end{frame}


\begin{frame}[label={sec:orga4b1767}]{Prêambulo para a ópera do fim do mundo}
\href{./opera.jpg}{Brasil, o Golpe: A Ópera do fim do mundo - Flávio Tavares}
\end{frame}

\section{Idade das trevas}
\label{sec:org347f1a9}

\section{Dados}
\label{sec:orgfbb3ce7}


\begin{frame}[label={sec:orge7410d9}]{Índice EMBI Brasil (Fim de período)}
\end{frame}



\begin{frame}[label={sec:orgb4a5093}]{O ciclo das commodities}
\begin{figure}[htb]
\centering
\caption{Índice de Commodities - Brasil\\Média móvel 12 meses} 
\includegraphics[width = .9\textwidth]{./figs/Commodities.png}
\caption*{\textbf{Fonte:} BCB-Depec}
\end{figure}
\end{frame}


\begin{frame}[label={sec:org623e7bd}]{O ciclo de liquidez internacional para países emergentes}
\alert{Dado:} IIF (?)
\end{frame}

\begin{frame}[label={sec:org5f2c293}]{Evolução das reservas internacionais líquidas}
\end{frame}

\begin{frame}[label={sec:org163e8a5}]{PIB puxado pelo mercado doméstico}
\alert{Dado:} Decomposição do crescimento (doméstico e exportações líquidas)
\end{frame}

\begin{frame}[label={sec:orgf92839b}]{Crescimento e o investimento induzido}
\alert{Dado:} Decomposição do crescimento (Todos os componentes da demanda)
\end{frame}


\begin{frame}[label={sec:org43c3a62}]{Crédito total}
\end{frame}

\begin{frame}[label={sec:org48eb068}]{Emprego}
\alert{Dado:} EVOLUÇÃO DO NÚMERO TOTAL DE EMPREGADOS COM VÍNCULO FORMAL  DE EMPREGO - MTE/RAAIS
\end{frame}


\begin{frame}[label={sec:org3b3d345}]{Evolução do salário mínimo real}
\end{frame}

\begin{frame}[label={sec:org77de168}]{Consumo das famílias}
\end{frame}

\begin{frame}[label={sec:orgd1e9208}]{Investimento público em recuperação}
\alert{Dado:} Taxa de investimento público (\%PIB)
\end{frame}

\begin{frame}[label={sec:orgbae9273}]{A evolução favorável da dívida pública}
\alert{Dado:} Evolução da dívida bruta e líquida (\%PIB)
\end{frame}

\begin{frame}[label={sec:org1724963}]{Superávit primário}
\end{frame}

\begin{frame}[label={sec:org9dcc1d6}]{Relação dívida externa líquida}
\end{frame}


\begin{frame}[label={sec:org1d5647c}]{Taxa de câmbio nominal}
\end{frame}

\begin{frame}[label={sec:org65502a9}]{Taxa de juros selic}
\end{frame}

\begin{frame}[label={sec:org52259c3}]{Preços livres vs Administrados}
\end{frame}

\begin{frame}[label={sec:orgc676766}]{Finalmente, inflação (ops, IPCA)}
\alert{Dado:} IPCA, centro da meta e limites
\end{frame}
\end{document}
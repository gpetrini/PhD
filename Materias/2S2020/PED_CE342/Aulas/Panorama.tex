% Created 2020-12-08 ter 19:41
% Intended LaTeX compiler: pdflatex
\documentclass[presentation]{beamer}
  \usepackage{csquotes, caption}
\usepackage[brazilian, ]{babel}
\usepackage[style=abnt,noslsn,extrayear,uniquename=init,giveninits,justify,sccite, scbib,repeattitles,doi=false,isbn=false,url=false,maxcitenames=2, natbib=true,backend=biber]{biblatex}
\addbibresource{refs.bib}
\addbibresource{/HDD/Org/all_my_refs.bib}
\usetheme{default}
\author{PED: Gabriel Petrini}
\date{04 de Janeiro de 2021}
\title{Panorama Macroeconômico e a relação com a Inflação}
\begin{document}

\maketitle
\begin{frame}{Outline}
\tableofcontents
\end{frame}

\section{Introdução}
\label{sec:orgba5f31e}

\begin{frame}[label={sec:org7c3b14f}]{Estrutura das aulas}
A aula terá a seguinte estrutura
\begin{itemize}
\item \alert{Teórico:} Conjuntura pré-Regime de Metas e revisão teórica
\begin{itemize}
\item Diagrama analítico
\item Estrutura das atas do COPOM
\item Exemplificando com uma ata
\end{itemize}
\item \alert{Macroeconômico:} Apresentação dos dados e dos movimentos gerais
\begin{itemize}
\item Parte I: 2002-2011
\item Parte II: 2011-2020
\end{itemize}
\end{itemize}
\end{frame}


\section{Preâmbulo: Cenário internacional adverso, Implementação do RMI e a herança maldita}
\label{sec:orgd34d33f}

\begin{frame}[label={sec:orgf7f3202}]{A Herança Maldita \cite{belluzzoDepoisQuedaEconomia2002}}
\begin{itemize}
\item Crise Russa, Mexicana, Argentina \(\Rightarrow\) \emph{flight to quality}
\item Instabilidade macroeconômica \(\Rightarrow\) \emph{stop and go}
\item Dívida pública
\begin{itemize}
\item Relação dívida/PIB em expansão
\item Elevação do gasto com juros
\item Concentrada em títulos selic e indexados ao câmbio
\end{itemize}
\item Choque inflacionário \(\Rightarrow\) Câmbio e energia
\end{itemize}

\begin{block}{1999: Implementação do RMI}
\end{block}
\end{frame}

\section{RMI: Esquema analítico}
\label{sec:orgeb518e1}

\begin{frame}[label={sec:org772e10c}]{Diagrama}
\begin{figure}[htb]
\centering
\caption{Representação do Modelo do regime de Metas para inflação} 
\includegraphics[width = 0.9\textwidth]{./figs/RMI.png}
\label{fig:ibcbr}
\caption*{\textbf{Fonte:} Elaboração própria}
\end{figure}
\end{frame}
\begin{frame}[label={sec:org926d2d4}]{Estrutura das Atas do Copom}
\begin{itemize}
\item Evolução recente da economia \(\Rightarrow\) \(h_{t-m}, h_{t}, (\pi_{t} - \pi^{\star}), \varepsilon^{D}, \varepsilon^{S}\)
\item Avaliação prospectiva das tendências de inflação \(\Rightarrow\) \(\pi^{e}\)
\item Implementação da política monetária \(\Rightarrow\) \(i_{t}, (\pi_{t} - \pi^{\star}), \pi^{e}_{t+n}\)
\item Inflação \(\Rightarrow\) Componentes de \(\pi\)
\item Atividade econômica \(\Rightarrow\) crédito, produção, NUCI, etc
\item Expectativas e sondagens \(\Rightarrow\) \(\pi^{e}\)
\item Mercado de trabalho \(\Rightarrow\) redimentos, taxa de desemprego
\item Crédito e inadimplência
\item Ambiente externo \(\Rightarrow\) \(\Delta e, \Delta p^{F}, \Delta i^{F}, \Delta x, \varepsilon^{F}\)
\item Comércio exterior e reservas internacionais
\begin{itemize}
\item Componentes do BP, fluxo de capitais e etc
\end{itemize}
\item Mercado monetário e operações de mercado aberto
\begin{itemize}
\item \alert{Memo:} Quantidade de moeda endógena
\item \(i_{t}\), spread bancário, oferta de títulos públicos (LTN, LFT, etc)
\end{itemize}
\end{itemize}
\end{frame}

\begin{frame}[label={sec:orgfe34296}]{Exemplo: Ata}
\end{frame}
\section{Governos Lula e suas fases}
\label{sec:org5e26f66}

\begin{frame}[label={sec:org1afcf8f}]{Crescimento e desequilíbrios globais \cite{carneiroSupremaciaDosMercados2006}}
Após fase conturbada de 1997/2002, a economia global consolida um arrajno dinâmico e desequiblibrado entre 2002/3 a 2007/8

\alert{Engrenagem comercial com 3 elos}
\begin{itemize}
\item Crescimento finance-led nos EUA
\begin{itemize}
\item Déficit comercial elevado
\end{itemize}
\item Estratégia trade-account nos países asiáticos
\begin{itemize}
\item Superávit comercial chinês com os EUA
\end{itemize}
\item Impactos nas commodities
\end{itemize}
\end{frame}

\begin{frame}[label={sec:org5528e8f}]{Primeira Fase}
Experimento desenvolvimentista junto de uma política macroeconômica conservadora:
\begin{itemize}
\item Continuidade do tripé
\item Visão teórica da \alert{política fiscal:} contração fiscal expansionista
\item \alert{Estratégia de crescimento:} visão liberal predominante
\begin{itemize}
\item reformas microeconômicas
\item regras estáveis de gestão
\item ampliar ajuste fiscal
\end{itemize}
\item Apreciação cambial
\item Taxa de juros elevadas
\end{itemize}

\alert{Contexto de transição complexa:} desconfiança dos credores e pressões financeiras
\end{frame}

\begin{frame}[label={sec:org5c45f1e}]{Segunda fase}
\begin{itemize}
\item Retomada do Estado como elemento condutor do Crescimento
\item Desenho da política fiscal no centro de proposta do desenvolvimento
\item Remontagem da capacidade de atuação dos atores públicos
\item Não desmonta aparato regulatório do modelo anterior
\item Investimento de apoio às atividades privadas
\end{itemize}
\end{frame}


\begin{frame}[label={sec:org18bf49b}]{Dificuldades em aberto \cite{melloIndustrialismoAusteridadePolitica}}
\begin{itemize}
\item Limites do crescimento com expansão da demanda de consumo via crédito e políticas sociais
\item Retomada da taxa de investimento, mas nível baixo
\begin{itemize}
\item Dificuldade de retomada do investimento público
\item Crise mundial e investimento privado
\end{itemize}
\item Estrutura produtiva
\end{itemize}
\end{frame}

\section{Dados 2002-2011}
\label{sec:org8b69459}
\begin{frame}[label={sec:org4af0acd}]{Continuidades}
\begin{block}{Política cambial}
Pouco mudou ao longo do tempo

\begin{itemize}
\item Valorização com reflexo na inflação
\item Compras de divisas não evitou valorização
\item Impacto sobre o setor industrial
\end{itemize}
\end{block}

\begin{block}{Política monetária}
Rígido regime de metas de inflação

\begin{itemize}
\item altos níveis de juros reais
\item discussão sobre independência do Banco Central
\item conflito com meta de taxa de juros "desenvolvimentista"
\end{itemize}
\end{block}

\begin{block}{Política fiscal}
Não foi alterado o regime fiscal definido na era FHC

\begin{itemize}
\item Lei de Responsabilidade Fiscal sem mudanças
\item Não alterou mercado de dívida pública
\end{itemize}
\end{block}
\end{frame}


\begin{frame}[label={sec:orge9a33e0}]{Descontinuidades}
Conjugação de políticas de incentivo à renda e ao mercado interno

\begin{itemize}
\item Defesa da expansão da demanda como fator de impulso ao crescimento
\begin{itemize}
\item Política deliberada de inserção social
\item Expansão do crédito
\item \alert{Programa de Salário Mínimo}
\end{itemize}
\item Ações desenvolvimentistas
\begin{itemize}
\item Políticas de incentivo ao investimento: PAC e PDP
\item Gasto público como estratégia para elevar o crescimento
\item Política de fortalecimento dos Bancos Públicos e das empresas estatais
\end{itemize}
\end{itemize}
\end{frame}


\begin{frame}[label={sec:orgd4788de}]{Alguns resultados}
\begin{itemize}
\item Queda da taxa de desemprego aberta
\item Valorização real do salário mínimo
\item Expansão do gasto federal total
\item Taxa crescimento da FBCF cresce pós-07
\item Aumento do consumo das famílias, mas menor que o investimento
\end{itemize}
\end{frame}



\begin{frame}[label={sec:org3b56542}]{Fatores determinantes}
\begin{itemize}
\item Impulsos externos favoráveis
\begin{itemize}
\item Melhora no setor externo pelo lado comercial (commodities) e financiero
\begin{itemize}
\item Ajudam retomada em 2004, mas não puxam o crescimento
\end{itemize}
\end{itemize}
\item Motores do crescimento (expansão do mercado interno)
\begin{itemize}
\item Distribuição de renda e cŕedito
\begin{itemize}
\item Aumento do saldo total de crédito
\end{itemize}
\item Investimento induzido
\end{itemize}
\item Investimento público a partir de 2007 com o PAC
\end{itemize}
\end{frame}


\begin{frame}[label={sec:orgdc61fbf}]{PIB puxado pelo mercado doméstico}
\begin{figure}[htb]
\centering
\caption{Decomposição da taxa de crescimento do produto - Domésticos e externos} 
\includegraphics[width = 0.9\textwidth]{./figs/PIB_Decomp_I.png}
\label{fig:cycles}
\caption*{\textbf{Fonte:} BCB}
\end{figure}
\end{frame}


\begin{frame}[label={sec:org5e7cc8e}]{Crescimento e o investimento induzido}
\begin{figure}[htb]
\centering
\caption{Taxa de crescimento do produto - decomposição total} 
\includegraphics[width = 0.9\textwidth]{./figs/PIB_Decomp_Total_I.png}
\label{fig:PIB_Decomp_Total}
\caption*{\textbf{Fonte:} BCB}
\end{figure}
\end{frame}




\begin{frame}[label={sec:org30f1f94}]{Investimento público em recuperação \cite{orair_investimento_2016}}
\begin{table}[htbp]
\caption{Taxa de crescimento do investimentos públicos (1994-2015)}
\centering
\begin{tabular}{rrrrr}
\hline
Ano & Gov. Central & Gov. Geral & Setor Público & PIB\\
\hline
1994-1998 & -5.1 & -2.7 & -0.9 & 2.6\\
1998-2002 & -1.2 & -2.0 & -1.9 & 2.3\\
2002-2006 & -0.6 & 0.6 & 0.4 & 3.5\\
2006-2010 & 25.4 & 13.5 & 17.0 & 4.6\\
2010-2014 & -0.4 & -0.1 & -0.1 & 2.2\\
2014-2015 & -6.2 & -4.0 & -5.2 & 0.3\\
\hline
\end{tabular}
\end{table}
\end{frame}



\begin{frame}[label={sec:org22998d6}]{Emprego}
\begin{figure}[htb]
\centering
\caption{Índice do Emprego Formal} 
\includegraphics[width = 0.9\textwidth]{./figs/EmpregoFormal_I.png}
\label{fig:EmpFormal}
\caption*{\textbf{Fonte:} MTb}
\end{figure}
\end{frame}


\begin{frame}[label={sec:org636c4ad}]{Taxa de câmbio nominal}
\begin{figure}[htb]
\centering
\caption{ Índice da taxa de câmbio efetiva nominal\\Jun/1994=100 } 
\includegraphics[width = 0.9\textwidth]{./figs/CambioNominal_I.png}
\label{fig:cambio}
\caption*{\textbf{Fonte:} BCB-DSTAT}
\end{figure}
\end{frame}


\begin{frame}[label={sec:orgaca690e}]{Taxa de juros selic}
\begin{figure}[htb]
\centering
\caption{Taxa de juros selic a.a. (efetivo x meta)\\Anualizada base 252} 
\includegraphics[width = 0.9\textwidth]{./figs/Selic_I.png}
\label{fig:Selic}
\caption*{\textbf{Fonte:} Copom e BCB-Demab}
\end{figure}
\end{frame}


\begin{frame}[label={sec:org24e2c88}]{Finalmente, inflação (ops, IPCA)}
\begin{figure}[htb]
\centering
\caption{IPCA e Metas para Inflação} 
\includegraphics[width = 0.9\textwidth]{./figs/IPCA_I.png}
\label{fig:IPCA}
\caption*{\textbf{Fonte:} BCB}
\end{figure}
\end{frame}
\begin{frame}[label={sec:org34db6da}]{Composição do IPCA \cite{bcb_2019_Atualizacoes}}
\begin{table}[htbp]
\caption{IPCA: estruturas de ponderação – janeiro de 2018}
\centering
\begin{tabular}{lrrr}
\hline
Grupo & POF 2008-2009 & POF 2017-2018 & Diferença (p.p.)\\
\hline
Alimentação e bebidas & 24.58 & 18.99 & -5.59\\
Habitação & 15.72 & 15.16 & -0.56\\
Artigos de residência & 3.98 & 4.02 & 0.04\\
Vestuário & 5.96 & 4.8 & -1.16\\
Transporte & 18.28 & 20.84 & 2.55\\
Saúde e cuidados pessoais & 12.04 & 13.46 & 1.41\\
Despesas pessoais & 10.96 & 10.60 & -0.36\\
Educação & 4.83 & 5.95 & 1.13\\
Comunicação & 3.65 & 6.19 & 2.54\\
\hline
\end{tabular}
\end{table}
\end{frame}
\begin{frame}[label={sec:org0b5621e}]{Preços livres, monitorados, serviços}
\begin{figure}[htb]
\centering
\caption{IPCA e seus componentes: preços livres, monitorados e serviços} 
\includegraphics[width = 0.65\textwidth]{./figs/Livres_Administrados_I.png}
\label{fig:livres_adm}
\caption*{\textbf{Fonte:} BCB}
\end{figure}
\end{frame}


\section{Governo(s?) Dilma}
\label{sec:orgb533564}

\begin{frame}[label={sec:org8278899}]{Os três "motores" \cite{serrano_demanda_2015}}
\begin{figure}[htb]
\centering
\caption{Os três motores do crescimento} 
\includegraphics[width = 0.9\textwidth]{./figs/Tenores.png}
\label{fig:tenores}
\caption*{\textbf{Fonte:} Elaboração própria}
\end{figure}

\alert{Principal mudança:} do incentivo à demanda agregada ao incetivo ao investimento privado.
\end{frame}

\begin{frame}[label={sec:orga8bc420}]{Medidas macroprudenciais}
\alert{Medidas macroprudenciais:} Redução do crescimento do crédito.

\begin{itemize}
\item Redução da taxa de crescimento da renda disponível real
\item Aumento dos depósitos compulsórios
\item Aumento do capital mínimo exigido dos bancos para empréstimos ao consumidor de prazos mais longos
\item Aumento do percentual mínimo de pagamento de cartões de crédito
\end{itemize}

\begin{block}{Implicações}
\begin{itemize}
\item Aumento do spread do crédito ao consumo final
\item Diminuição dos prazos
\item Elimina \alert{boom} de consumo
\item Aumento da inadimplência
\end{itemize}
\end{block}
\end{frame}


\begin{frame}[label={sec:orga3cb664}]{Desaceleração rudimentar I}
\begin{table}[htbp]
\caption{Comparação das taxas de crescimento}
\centering
\begin{tabular}{lll}
\hline
 & 2004-2010 & 2011-2014\\
PIB & 4.4\% & 2.1\%\\
Produção industrial & 3.6\% & -0.9\%\\
Taxa de desemprego & 9.0\% & 5.4\%\\
\hline
\end{tabular}
\end{table}
\end{frame}


\begin{frame}[label={sec:org36ec198}]{Desaceleração rudimentar II}
\begin{table}[htbp]
\caption{Consumo das famílias}
\centering
\begin{tabular}{lll}
\hline
 & 2004-2010 & 2011-2014\\
Crédito para habitação & 21,5\% & 4,6\%\\
Hipotecas & 20,1\% & 29,3\%\\
Salário real (emp, formal) & 2,9\% & 2,9\%\\
Renda disponível das famílias & 5,3\% & 1,2\%\\
\hline
\end{tabular}
\end{table}
\end{frame}

\begin{frame}[label={sec:org0d3102e}]{Desaceleração rudimentar III}
\begin{table}[htbp]
\caption{Política fiscal}
\centering
\begin{tabular}{lll}
\hline
 & 2004-2010 & 2011-2014\\
Superávit primário/PIB & 3,2\% & 1,7\%\\
Receitas do setor público & 7,2\% & 1,2\%\\
Transf. públicas para as famílias & 5,6\% & 4,9\%\\
Invest. Emp. Estatais (Federal) & 16,3\% & -2,7\%\\
Investimento Adm, Pública & 14,0\% & -1,0\%\\
\hline
\end{tabular}
\end{table}
\end{frame}


\begin{frame}[label={sec:org634d435}]{\href{https://www.causaoperaria.org.br/brasil-o-golpe-a-opera-do-fim-do-mundo-artista-retrata-o-golpe-de-estado-no-pais/}{Prêambulo para a ópera do fim do mundo}}
\begin{figure}[htb]
\centering
\caption{Brasil, O Golpe: A Ópera do fim do mundo} 
\includegraphics[width = 0.9\textwidth]{./figs/opera.png}
\caption*{\textbf{Fonte:} Jornal GGN}
\end{figure}
\end{frame}

\section{Governos (?) Temer e Início Bolsonaro}
\label{sec:org64e7e08}

\section{Corona-crise}
\label{sec:org8030c3b}
\section{Dados 2011-2020}
\label{sec:org59ed6e2}



\begin{frame}[label={sec:orgc0b5249}]{PIB puxado pelo mercado doméstico}
\begin{figure}[htb]
\centering
\caption{Decomposição da taxa de crescimento do produto - Domésticos e externos} 
\includegraphics[width = 0.9\textwidth]{./figs/PIB_Decomp.png}
\label{fig:cycles}
\caption*{\textbf{Fonte:} BCB}
\end{figure}
\end{frame}


\begin{frame}[label={sec:org02fee50}]{Crescimento e o investimento induzido}
\begin{figure}[htb]
\centering
\caption{Taxa de crescimento do produto - decomposição total} 
\includegraphics[width = 0.9\textwidth]{./figs/PIB_Decomp_Total.png}
\label{fig:PIB_Decomp_Total}
\caption*{\textbf{Fonte:} BCB}
\end{figure}
\end{frame}



\begin{frame}[label={sec:org747bf3b}]{Emprego}
\begin{figure}[htb]
\centering
\caption{Índice do Emprego Formal} 
\includegraphics[width = 0.9\textwidth]{./figs/EmpregoFormal.png}
\label{fig:EmpFormal}
\caption*{\textbf{Fonte:} MTb}
\end{figure}
\end{frame}





\begin{frame}[label={sec:org80f4cba}]{Taxa de câmbio nominal}
\begin{figure}[htb]
\centering
\caption{ Índice da taxa de câmbio efetiva nominal\\Jun/1994=100 } 
\includegraphics[width = 0.9\textwidth]{./figs/CambioNominal.png}
\label{fig:cambio}
\caption*{\textbf{Fonte:} BCB-DSTAT}
\end{figure}
\end{frame}


\begin{frame}[label={sec:org0fce1f8}]{Taxa de juros selic}
\begin{figure}[htb]
\centering
\caption{Taxa de juros selic a.a. (efetivo x meta)\\Anualizada base 252} 
\includegraphics[width = 0.9\textwidth]{./figs/Selic.png}
\label{fig:Selic}
\caption*{\textbf{Fonte:} Copom e BCB-Demab}
\end{figure}
\end{frame}


\begin{frame}[label={sec:org6b2ad99}]{Finalmente, inflação (ops, IPCA)}
\begin{figure}[htb]
\centering
\caption{IPCA e Metas para Inflação} 
\includegraphics[width = 0.9\textwidth]{./figs/IPCA.png}
\label{fig:IPCA}
\caption*{\textbf{Fonte:} BCB}
\end{figure}
\end{frame}
\begin{frame}[label={sec:org836ad54}]{Preços livres, monitorados, serviços}
\begin{figure}[htb]
\centering
\caption{IPCA e seus componentes: preços livres, monitorados e serviços} 
\includegraphics[width = 0.65\textwidth]{./figs/Livres_Administrados.png}
\label{fig:livres_adm}
\caption*{\textbf{Fonte:} BCB}
\end{figure}
\end{frame}



\section{Referências}
\label{sec:org97121ac}

\begin{frame}[label={sec:org912bf33}]{Referências}

\printbibliography
\end{frame}
\end{document}

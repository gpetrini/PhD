% Created 2020-12-09 qua 18:53
% Intended LaTeX compiler: pdflatex
\documentclass[11pt]{article}
\usepackage[utf8]{inputenc}
\usepackage{lmodern}
\usepackage[T1]{fontenc}
\usepackage[top=3cm, bottom=2cm, left=3cm, right=2cm]{geometry}
\usepackage{graphicx}
\usepackage{longtable}
\usepackage{float}
\usepackage{wrapfig}
\usepackage{rotating}
\usepackage[normalem]{ulem}
\usepackage{amsmath}
\usepackage{textcomp}
\usepackage{marvosym}
\usepackage{wasysym}
\usepackage{amssymb}
\usepackage{amsmath}
\usepackage[theorems, skins]{tcolorbox}
\usepackage[style=abnt,noslsn,extrayear,uniquename=init,giveninits,justify,sccite,
scbib,repeattitles,doi=false,isbn=false,url=false,maxcitenames=2,
natbib=true,backend=biber]{biblatex}
\usepackage{url}
\usepackage[cache=false]{minted}
\usepackage[linktocpage,pdfstartview=FitH,colorlinks,
linkcolor=blue,anchorcolor=blue,
citecolor=blue,filecolor=blue,menucolor=blue,urlcolor=blue]{hyperref}
\usepackage{attachfile}
\usepackage{setspace}
\usepackage{tikz}
\author{Gabriel Petrini}
\date{\today}
\title{Atas Compom}
\begin{document}

\maketitle
\tableofcontents

-\textbf{- mode: org -}-

\section*{\textit{<2002-02-21 qui> } \href{https://www.bcb.gov.br/htms/copom/not2002022068.asp?frame=1}{Ata Nº68}}
\label{sec:org78f48b3}
\subsection*{Atividade Econômica}
\label{sec:org4b7baa1}
\begin{itemize}
\item Recuperação econômica
\begin{itemize}
\item Indústria e comércio
\begin{itemize}
\item Destaque crescimento bens de consumo duráveisi
\item Concessionárias em queda
\end{itemize}
\item Queda dos estoques e mercado de trabalho reforçam (queda do desemprego)
\item Delineada em 2001
\item Recuo produção dos bens de capital e crescimento da importação (aumento absorção doméstica)
\end{itemize}
\item Maior nível de confiança (consumidores e empresários) \(\Rightarrow\) tendência do nível de atividade
\begin{itemize}
\item Melhora dos efeitos negativos do setor externo adverso
\end{itemize}
\item Fatores de sustentação demanda
\begin{itemize}
\item Crescimento da massa de rendimentos
\item Condições de crédito
\end{itemize}
\item Atendimento parcial da demanda reprimida de energia
\item Superávit transações externas (importante)
\begin{itemize}
\item Queda das exportações menor que das importações
\item Sinais de substituição de importações associada à \textbf{depreciação}
\end{itemize}
\item Aumento da produção \textbf{não} sinaliza pressão sobre o nível de utilização da capacidade produtiva
\begin{itemize}
\item Elevado grau de ociosidade
\item Maior fluxo de investimento externo \(\Rightarrow\) aumento de produtividade \(\Rightarrow\) expansão da demanda é sustentável
\end{itemize}
\end{itemize}
\subsection*{Ambiente externo}
\label{sec:org970a6d4}
\begin{itemize}
\item Recuperação da atividade nos EUA e manutenção da taxa de juros americana (interrompe flexibilização)
\begin{itemize}
\item Impacto positivo no comércio mundial e nos mercados mundiais de capitais
\item Diminuição dos efeitos negativos do \emph{flight to quality}
\end{itemize}
\item Espiral deflacionária no Japão
\begin{itemize}
\item Contrapõe a retomada dos outros países
\end{itemize}
\item Taxa de juros do Euro inalterada
\end{itemize}
\subsection*{Preços}
\label{sec:orgc1e9557}
\begin{itemize}
\item IPCA elevou (acumulado 7,62\%), mas em desaceleração
\begin{itemize}
\item Alta dos preços dos alimentos
\begin{itemize}
\item Condições atmosféricas
\end{itemize}
\item Preços administrados
\item Queda do preço da gasolina, mas aumento do gás (retirada subsídios)
\end{itemize}
\item IGP-DI similar ao anterior
\item Queda dos preços industriais
\item Efeitos sazonais do aumento dos gastos com educação
\item Espera-se \textbf{desaceleração} dada expectativa de redução dos preços dos alimentos
\end{itemize}
\subsection*{Mercado monetário e operações de mercado aberto}
\label{sec:orgaa2a312}
\begin{itemize}
\item Movimentos de pequena intensidade na curva de juros
\begin{itemize}
\item Manutenção da taxa de juros reverteu curva de juros negativa
\end{itemize}
\item Tentativa de desconcentração dos vencimentos dos títulos 
\begin{itemize}
\item Maior demanda por títulos de prazos mais curtos
\end{itemize}
\item Retomada do oferecimento de LFT cuja demanda foi reduzida
\item Aumento dívida pública doméstica
\end{itemize}
\subsection*{Avaliações prospectivas das tendências de inflação}
\label{sec:org92c0168}

\begin{itemize}
\item Elevação IPCA
\item Redução nos preços de gasolina, gás, bujão
\item Projeção de reajuste nas tarifas de energia elétrica
\item Elevação da inclinação da curva de juros
\item Estabilidade do prêmio de risco soberano
\item Expectativa de aumento dos preços administrados
\item Hipótese de cumprimento das metas de superávit primário (setor público consolidado)
\end{itemize}
\subsection*{Diretrizes de política monetária}
\label{sec:org65bbd08}
\textbf{Decisão:} Reduzir a taxa de juros
\begin{itemize}
\item Preços administrados não recuaram tanto quanto esperados
\item Pressão por recomposição das taxas de lucro e repasse cambial pode \textbf{diminuir}
\begin{itemize}
\item Expansão moderada do crédito
\item Aumento de investidores, da capacidade produtiva e fim do racionamento de energia \(\Rightarrow\) ampliação do potencial produtivo
\begin{itemize}
\item \textbf{Memo:} Capacidade ociosa
\end{itemize}
\item Expansão da demanda pode não pressionar preços
\end{itemize}
\item Expectativas de inflação estáveis
\item Cenário internacional menos adverso, mas ainda sim incerto
\item Política monetária poderia ser flexibilizada
\item Queda da inflação em direção à meta
\end{itemize}

\section*{\textit{<2002-10-30 qua> } \href{https://www.bcb.gov.br/publicacoes/atascopom/01102002}{Ata Nº76}}
\label{sec:orgee24b0c}

\subsection*{Diretrizes de política monetária}
\label{sec:org24d198d}
\textbf{Decisão:} aumentar taxa de juros em 3 p.p.
\begin{itemize}
\item Índices indicam aumento da inflação
\begin{itemize}
\item Depreciação cambial real
\item IPCA acima do valor esperado
\item Inflação de preços livres
\end{itemize}
\item Expectativas de mercado de aumento de inflação e o mesmo vale para a projeção do COPOM
\item Maior grau de incerteza dados os resultados da \textbf{eleição presidencial}
\item Projeção da inflação futura esta acima da meta
\begin{itemize}
\item Recomenda uma política monetária mais restrita mesmo que não relacionada com aumento da demanda
\item Depreciação cambial significativa
\end{itemize}
\item Reunião extraordinária
\end{itemize}

\section*{\textit{<2002-10-22 ter> } \href{https://www.bcb.gov.br/publicacoes/atascopom/22102002}{Ata Nº77}}
\label{sec:org866ccd2}
\subsection*{Atividade Econômica}
\label{sec:org67d2d7e}
\begin{itemize}
\item Estabilidade do nível de atividade junto de movimentos opostos
\begin{itemize}
\item Aumento de incertezas (eleição)
\item Efeitos negativos no mercado financeiro \(\Rightarrow\) efeitos negativos sobre consumo de bens de elevado valor agregado e investimento
\item Efeitos positivos sobre comércio varejista
\begin{itemize}
\item FGTS
\item Aumento do nível de emprego
\item Renda agrícola
\item Desepenho favorável da balança comercial
\item Aumento do faturamento
\begin{itemize}
\item Bens de consumo semi e não-durávies
\end{itemize}
\item Redução IPI
\item Expansão dos negócios a vista
\end{itemize}
\item Expansão do crédito \(\Rightarrow\) aumento varejo
\begin{itemize}
\item Aumento inadimplência
\end{itemize}
\end{itemize}
\item Incertezas futuras e volativlidade cambial
\begin{itemize}
\item Redução índice de confiança do consumidor
\end{itemize}
\item Queda inadimplência de crédito livre \(\Rightarrow\) maior seletividade dos bancos
\item Aumento da atividade industrial
\begin{itemize}
\item Produção extrativa mineral
\item Nível de produção constante
\item Aumento da produção dos bens intermediários
\item Recuo produção dos bens de capital e de consumo duráveis
\item Expansão das vendas industriais
\item Estabilidade do nível de utilização da capacidade e uso de estoques
\end{itemize}
\item Aumento da taxa de desemprego aberta
\item Superávit na conta de transações correntes, balança comercial e queda da remessas líquidas ao exterior
\end{itemize}
\subsection*{Ambiente externo}
\label{sec:orgd0ab7c0}
\begin{itemize}
\item Aversão ao risco e elevada volatilidade nos mercados globais
\item Expectativa de redução da recuperação econômica mundial
\begin{itemize}
\item Ataque terrorista na Indonésia e tensão entre EUA e Iraque
\item Dúvidas sobre capacidade de recuperação dos EUA
\item Ligeiro crescimento da produção industrial no Japão
\item Sinais de recuperação na zona do Euro
\end{itemize}
\end{itemize}
\subsection*{Preços}
\label{sec:org5e785cb}
\begin{itemize}
\item Aumento dos preços livres
\begin{itemize}
\item Taxa de câmbio deve ser a principal pressão inflacionária
\item Elevação IPCA
\begin{itemize}
\item Alimentação e bebidas
\item Variação cambial
\end{itemize}
\item IGP-DI eleva-se por conta do câmbio e dos efeitos intrassafra
\begin{itemize}
\item Elevação dos preços agrícolas e industriais
\end{itemize}
\end{itemize}
\end{itemize}
\subsection*{Mercado monetário e operações de mercado aberto}
\label{sec:org7234583}
\begin{itemize}
\item Tendência de queda dos juros em função da redução das incertezas quanto ao cenário político-econômico
\item Maior demanda por NTN-C
\item Objetivo de reduzir volatilidade dos preços das LFTs
\end{itemize}
\subsection*{Avaliações prospectivas das tendências de inflação}
\label{sec:org4ac4d20}
\begin{itemize}
\item Incerteza quanto as trajetórias dos preços dos derivados de petróleo
\item Projeção do aumento das tarifas de eletricidade
\item Aumento dos preços administrados dado repasse cambial
\item Trajetória decendente do \emph{spread bancário}
\end{itemize}


\subsubsection*{Resultados}
\label{sec:org86d2198}

\begin{itemize}
\item Aumento do IPCA mensal e acumulado
\item Hipótese de cumprimento da meta de superávit primário
\item Manutenção da taxa de juros e da taxa de câmbio
\end{itemize}
\subsection*{Diretrizes de política monetária}
\label{sec:org81e5f26}
\textbf{Decisão:} 


\textbf{Decisão:} Manter meta em 21\%
\begin{itemize}
\item Crescimento das vendas e estabilidade da produção industrial
\item Crescimento da massa de rendimento dos salários
\item Estabilidade no nível de emprego
\item Aumento  do consumo deve-se concentrar nos bens não-duráveis
\item Redução dos estoques, mas acima do nível desejado
\item Aumento dos custos decorrentes do repasse cambial
\item Redução do nível de utilização da capacidade
\item Ajuste na conta corrente
\begin{itemize}
\item Depreciação cambial
\item Crescimento lento
\item Melhora na conta corrente e bom desempenho da balança comercial
\end{itemize}
\item Crise de confiança reduziu crédito ao Brasil e aumento da percepção do risco soberano
\item Estabilização e recuperação do mercado financeiro
\begin{itemize}
\item Real se manteve estável
\item Maior demanda por títulos no mercado doméstico
\end{itemize}
\item Expectativas de aumento da inflação e maior dispersão das expectativas
\begin{itemize}
\item Copom também reavaliou para cima por conta da depreciação cambial
\begin{itemize}
\item Entendimento que se trata de uma inflação de custo
\end{itemize}
\item Aumento gasolina
\end{itemize}
\end{itemize}

\section*{\textit{<2003-06-19 qui> } \href{https://www.bcb.gov.br/publicacoes/atascopom/01062003}{Ata Nº85}}
\label{sec:org3763141}
\subsection*{Evolução recente da inflação}
\label{sec:orgbd06b63}
\begin{itemize}
\item Queda contínua da inflação
\begin{itemize}
\item Redução preços do atacado \(\Rightarrow\) apreciação cambial
\item Preços ao consumidores cairam, mas menos que os preços do atacado
\begin{itemize}
\item Preços agrícolas e industriais
\end{itemize}
\end{itemize}
\item Redução IPCA
\begin{itemize}
\item Reajuste das tarifas de energio foi o que mais contribuiu para aumento
\begin{itemize}
\item Aumento da tarifa de ônibus, água e esgoto
\end{itemize}
\item Atenção à safra de arroz
\item Queda tanto dos preços livres como nos monitorados
\begin{itemize}
\item Redução dos produtos não comercializáveis
\item Redução menos intensa dos bens comercializáveis
\item Redução do preço da gasolina e do álcool
\end{itemize}
\end{itemize}
\end{itemize}
\subsection*{Avaliação prospectiva das tendências da inflação}
\label{sec:orgbd20fc2}
\begin{itemize}
\item Elevação do preço da gasolina
\item Diminuição dos preços monitorados \(\Rightarrow\) apreciação cambial
\item Projeções de que a inflação futura estará acima da meta
\end{itemize}
\subsection*{Implementação da política monetária}
\label{sec:org6f6aac2}
\textbf{Decisão:} Reduzir meta em 26\% para garantir desaceleração recente da inflação
\begin{itemize}
\item Fatores do declínio da inflação
\begin{itemize}
\item Elevada depreciação passada explica inércia da inflação (persistência)
\item Apreciação cambial tem efeitos de redução da inflação
\item Atuação da política monetária para reverter expectativas de inflação e pressões inflacionárias via controle da demanda agregada
\item Normalização dos preços in natura
\end{itemize}
\item Destaque para a persistência inflacionária e subsequente recomposição das margens de lucro
\item Destaque para apreciação cambial
\begin{itemize}
\item Elevação da liquidez internacional
\item Queda do risco Brasil
\end{itemize}
\item Atividade econômica em desaceleração, mas não uniforme (diferença dos preços relativos)
\begin{itemize}
\item Expansão do setor agrícola e extração
\item Maior retração para o setor de bens duráveis 
\begin{itemize}
\item Dependente das condições de crédito
\end{itemize}
\end{itemize}
\item Convergência das expectativas de inflação para meta
\item Destaque para a persistência inflacionária
\end{itemize}
\subsection*{Atividade econômica}
\label{sec:org5f8113e}
\begin{itemize}
\item Queda generalizada da venda no varejo 
\begin{itemize}
\item Destaque para duráveis e semi-duráveis
\item Recuperação da confiança do consumidor
\end{itemize}
\item Fraco desempenho dos indicadores de investimento
\begin{itemize}
\item Adversidades da conjuntura econômica
\item Produção industrial estável
\end{itemize}
\item Expansão do setor extrativo e recuo na indústria de transformação
\item Redução do grau de utilização da capacidade instalada
\end{itemize}
\subsection*{Mercado de trabalho}
\label{sec:org4729886}
\begin{itemize}
\item Aumento do emprego formal
\item Estabilidade na taxa de ocupação junto do aumento do emprego informal
\end{itemize}
\subsection*{Crédito e inadimplência}
\label{sec:org3315c75}
\begin{itemize}
\item Estabilidade no saldo de operações de crédito
\begin{itemize}
\item Aumento no capital de giro
\end{itemize}
\item Estabilidade na taxa de inadimplência
\end{itemize}
\subsection*{Ambiente externo}
\label{sec:org6debbcb}
\begin{itemize}
\item Expectativa de recuperação econômica mundial ainda baixa
\item Lenta retomada da atividade econômica nos EUA
\item Recuo no Japão e Alemanha junto de tendência de deflação
\item Redução da taxa de juros pelo BCE
\end{itemize}
\subsection*{Setor Externo}
\label{sec:orga935421}
\begin{itemize}
\item Superávit balança comercial sobretudo por conta da melhora das exportações e ligeiro declínio das importações
\item Saldo positivo em transações correntes e superávit nas transações unilaterais
\begin{itemize}
\item Ingressos líquidos na conta financeira
\end{itemize}
\end{itemize}
\subsection*{Mercado monetário e operações de crédito aberto}
\label{sec:org239717d}
\begin{itemize}
\item Curva de juros mais negativa
\begin{itemize}
\item Divulgação dos indicadores de inflação
\item Apreciação cambial
\item Redução do risco país
\item Aprovação do projeto da reforma da previdência
\end{itemize}
\item Rolagem da dívida por meio de \emph{swaps}
\item Aumento do prazo médio de emissão da dívida
\item Aumento da dívida interna
\end{itemize}

\section*{\textit{<2006-04-17 seg> } \href{https://www.bcb.gov.br/publicacoes/atascopom/24042008}{Ata Nº 134}}
\label{sec:org80409ab}
\subsection*{Evolução recente da economia}
\label{sec:org69fd198}
\begin{itemize}
\item IPCA estável, mas inflação em aceleração em 12 meses
\begin{itemize}
\item Destaque para preços livres que aumentaram mais rápido do que os preços administrados
\begin{itemize}
\item Aceleração dos bens comercializáveis e não-comercializáveis (in natura e serviços especialmente)
\end{itemize}
\item Apreciação cambial
\item Sinais de divergência da meta
\end{itemize}
\item Recuo IGP-DI inicial seguido de aceleração
\begin{itemize}
\item Aceleração dos preços industriais desde 2007
\item Destaque para repasse a depender da expectativa futura sobre inflação
\end{itemize}
\item Redução da produção industrial geral, mas crescimento no ano \(\Rightarrow\) estabilidade macroeconômica
\begin{itemize}
\item Destaque para setor farmacêutico
\item Continuidade de expansão do setor industrial
\begin{itemize}
\item Expansão do crédito, emprego e renda
\item Impulsos fiscais
\item Recomposição de estoques
\end{itemize}
\item Avanço produção de bens de capital e de bens duráveis
\end{itemize}
\item Aumento na taxa de desemprego e da remuneração
\begin{itemize}
\item Expansão do emprego formal
\end{itemize}
\item Crescimento da demanda doméstica (sobretudo varejo)
\begin{itemize}
\item Sem sinais de redução
\item Móveis e eletrodomésticos e setores mais sensíveis à expansão da renda e do crédito
\end{itemize}
\item Manutenção da confiança do consumidor
\item Aumento do grau de utilização da capacidade instalada
\begin{itemize}
\item Crescimento da absorção dos bens de capital
\item Aumento de insumos da construção civil \(\Rightarrow\) ampliação da capacidade
\end{itemize}
\item Desempenho robusto da balança comercial
\begin{itemize}
\item Aumento do preço de commodities
\item Redução do saldo comercial \(\Rightarrow\) transações correntes mais deficitário
\end{itemize}
\item Desaceleração da economia norte americana em função da crise
\begin{itemize}
\item Mais avaliações negativas
\item O mesmo vale para Japão e Alemanha
\item Indicação de relaxamento monetário internacional adicional
\item Parece não ter afetado a economia doméstica
\end{itemize}
\item Elevação do preço e da volatilidade do petróleo
\begin{itemize}
\item Cenário de trabalho de preços de combustíveis inalterados
\end{itemize}
\end{itemize}
\subsection*{Avaliação prospectiva das tendências da inflação}
\label{sec:orgb26cbdf}
\begin{itemize}
\item Sem reajuste dos preço dos combustíveis
\item Manutenção da tarifa telefônica e dos preços administrados
\item Diminuição do \emph{spread} de juros
\item Manutenção da meta de superávit primário com possibilidade de redução
\begin{itemize}
\item Implementação do Programa Piloto de Investimentos (PPI)
\end{itemize}
\item Aumento das medianas de expectativas da inflação
\item Convergência da inflação para acima da meta
\end{itemize}
\subsection*{Implementação da política monetária}
\label{sec:orgc802dda}
\begin{itemize}
\item Aceleração da inflação em função da maior robustez da atividade doméstica
\begin{itemize}
\item Apesar do aumento da inflação e do investimento
\end{itemize}
\item Balanço de pagamentos não deve apresentar risco à inflação
\item Desaceleração dos mercados mundiais, mas riscos inflacionários a níveis globais \(\Rightarrow\) benigno
\begin{itemize}
\item Reduziria exportações líquidas (demanda)
\item Queda de preço de algumas commodities \(\Rightarrow\) menor inflação doméstica
\item Posibilidade de maior aversão ao risco e incerteza
\item Possibilidade de restrição de oferta setorial \(\Rightarrow\) repasse ao consumidor
\end{itemize}
\item Possibilidade de um maior impulsionamento em função do aumento da massa salarial e do crédito além das transferências governamentais
\item Destaque para redução do hiato do produto
\item Ênfase ao combate dos efeitos inicias da inflação e não os persistentes
\begin{itemize}
\item Sinais de excesso de aquecimento da demanda
\end{itemize}
\end{itemize}
\subsection*{Inflação}
\label{sec:org3a3a211}
\begin{itemize}
\item Aumento do IPCA e do IGP-DI
\begin{itemize}
\item Destaque para o preço dos alimentos, habitação e transportes
\item Desaceleração dos preços livres e aumento dos monitorados
\end{itemize}
\end{itemize}
\subsection*{Atividade econômica}
\label{sec:org81a2b94}
\begin{itemize}
\item Aumento das vendas no varejo e queda no comércio
\item Aumento do investimento, bens de capital e insumos
\begin{itemize}
\item Destaque à maquinas agrícolas e infraestrutura
\item Aumento da importação dos bens de capital
\end{itemize}
\item Expansão da atividade industrial
\begin{itemize}
\item Estabilidade no grau de utilização da capacidade instalada
\end{itemize}
\item Redução da produção física da indústria
\begin{itemize}
\item Destaque ao setor farmacêutico
\end{itemize}
\item Aumento produção de automóveis
\item Aumento produção de grãos
\end{itemize}
\subsection*{Expectativas e sondagens}
\label{sec:orgb241847}
\begin{itemize}
\item Aumento do índice de confiança do consumidor e estabilidade nas condições atuais
\item O mesmo vale para a Confiança da Indústria
\item Aumento do NUCI no setor de bens de capital em função da demanda doméstica e expectativas otimistas para o futuro
\end{itemize}
\subsection*{Mercado de trabalho}
\label{sec:org55b658a}
\begin{itemize}
\item Criação de postos de trabalhos formais
\begin{itemize}
\item Destaque para construção cívil e serviços
\item Redução da taxa de desemprego
\item Redução do emprego na indústria
\end{itemize}
\item Aumento do rendimento habitual real
\end{itemize}
\subsection*{Crédito e inadimplência}
\label{sec:orgf175e45}
\begin{itemize}
\item Aumento do saldo de empréstimos
\begin{itemize}
\item Atenção para recursos livres e direcionados para habitação
\end{itemize}
\item Elevação da taxa de juros média
\item Ampliação das operações de crédito
\item Redução da taxa de inadimplência
\end{itemize}
\subsection*{Ambiente externo}
\label{sec:org3ec212b}
\begin{itemize}
\item Percepção de declínio na taxa de expansão da economia norte-americana  e seus efeitos sobre economia mundial
\begin{itemize}
\item Temores nos mercados europeus
\end{itemize}
\item Inflação global em função de alimentos e energia
\item Elevação da inflação chinesa
\end{itemize}
\subsection*{Comércio exterior e reservas internacionais}
\label{sec:orge935117}
\begin{itemize}
\item Superávit na balança comercial maior que anteriormente
\item Aumento das reservas internacionais
\end{itemize}
\subsection*{Mercado monetário e operações de mercado aberto}
\label{sec:orgbb834dc}
\begin{itemize}
\item Elevação da curva de juros
\begin{itemize}
\item Destaque para piora do cenário externo
\end{itemize}
\item Realização de \emph{swap} cambial reverso
\item Operações compromissadas longas
\end{itemize}

\section*{\textit{<2009-01-20 ter> } \href{https://www.bcb.gov.br/publicacoes/atascopom/21012009}{Ata Nº140}}
\label{sec:orgd8ad855}
\subsection*{Evolução rencente da economia}
\label{sec:org7be22ce}
\begin{itemize}
\item Recuo inflação (IPCA)
\begin{itemize}
\item Tanto preços livres quanto administrados
\item Aceleração do preços dos bens comercializáveis e não-comercializáveis
\item Persistência da inflação de serviços
\end{itemize}
\item Convergência da inflação calculada pelos outros indicadores
\begin{itemize}
\item Variação negativa do IGP-DI
\item Destaque para depreciação cambial \(\Rightarrow\) preços agrícolas e industriais
\end{itemize}
\item Redução da produção industrial
\begin{itemize}
\item Destaque para problemas climáticos no Sul
\item Interrupção no ciclo de expansão
\begin{itemize}
\item Ajuste de estoque e redução da produção
\end{itemize}
\item Destaque para recuo no setor de bens de capital, mas lidera expansão acumulada
\begin{itemize}
\item Destaque para restrição de crédito
\end{itemize}
\end{itemize}
\item Indicadores ambíguos de mercado de trabalho
\begin{itemize}
\item Aumento na taxa de desemprego aberta
\item Aumento do Rendimento Mensal Médio
\item Perda de dinamismo na geral de empregos formais
\end{itemize}
\item Redução das vendas do varejo
\begin{itemize}
\item Destaque para queda acentuada do setor automotivo
\begin{itemize}
\item Restrição de crédito e redução nos indicadores de confiança
\end{itemize}
\item Transferências de renda auxiliaram para que a queda não fosse maior
\end{itemize}
\item Redução do grau de utilização da capacidade
\begin{itemize}
\item Redução das pressões de demanda sobre capacidade produtiva
\end{itemize}
\item Piora no saldo da balança comercial
\begin{itemize}
\item Reversão da trajetória de apreciação cambial
\item Queda dos preços dos bens exportados atua na direção oposta
\end{itemize}
\item Continuidade do estresse dos mercados financeiros
\begin{itemize}
\item Aumento da aversão ao risco
\item Contração da liquidez internacional \(\Rightarrow\) fluxos de capitais mais escassos \(\Rightarrow\) volatilidade das moedas emergentes
\end{itemize}
\item Tendências contracionistas a nível global
\begin{itemize}
\item Choque negativo dos termos de troca \(\Rightarrow\) \emph{commodities}
\item Deterioração da qualidade do crédito
\item Destaque para tendência de depreciação cambial e estímulos fiscais nas economias maduras
\end{itemize}
\item Elevada volatilidade do preço do petróleo
\begin{itemize}
\item Efeitos via cadeias produtivas e expectativas futuras de inflação
\end{itemize}
\end{itemize}
\subsection*{Avaliação prospectiva das tendências de inflação}
\label{sec:orgf78dab1}
\begin{itemize}
\item Sem inflação de combustíveis
\item Manutenção do reajuste do setor de telefonia e o mesmo vale para os preços administrados
\item Redução da mediana das expectativas de inflação
\item Conclusão: recuo da inflação
\end{itemize}
\subsection*{Implementação da política monetária}
\label{sec:org060117e}
\begin{itemize}
\item Arrefecimento da expansão econômica
\item Desaquecimento das demais economias
\item Aversão ao risco \(\Rightarrow\) ajustes no BP
\begin{itemize}
\item Afeta demanda por ativos brasileiros
\item Diminuição dos preços de \emph{commodities} e da demanda externa
\item Efeitos benignos sobre a inflação
\end{itemize}
\item Diminuição de riscos de pressão inflacionárias locais
\item Expectativas de inflação estão superiores à meta apesar do recuo
\item Crise internacional \(\Rightarrow\) efeitos negativos sobre confiança
\begin{itemize}
\item Expansão da economia depende mais ainda do crescimento da massa salarial e das transferências do governo
\end{itemize}
\item Queda do nível de atividade deve reduzir pressões inflacionárias
\begin{itemize}
\item Destaque para preço dos ativos brasileiros
\item Copom pontua necessidade de cautela para níveis de inflação compatíveis com a meta, mas há margem para flexibilização
\end{itemize}
\item Condições financeiras restritivas \(\Rightarrow\) contração da demanda \(\Rightarrow\) pressão desinflacionária
\item \textbf{Decisão:} Reduzir taxa de juros em 100 p.b. sem viés
\end{itemize}
\subsection*{Inflação}
\label{sec:org6d88167}
\begin{itemize}
\item Indicação de recuo da inflação medida pelo IPCA
\begin{itemize}
\item Desaceleração dos preços livres e dos monitorados
\end{itemize}
\end{itemize}
\subsection*{Atividade econômica}
\label{sec:org93dadb1}
\begin{itemize}
\item Retração das vendas no varejo
\item Retração do crédito e da inadimplência
\item Redução da produção de bens de capital e importação dessa categoria
\item Desaceleração do rítmo de produção industrial
\end{itemize}
\subsection*{Expectativas e sondagens}
\label{sec:org508709e}
\begin{itemize}
\item Redução do índice de confiança do consumidor e o mesmo vale para os indicadores de confiança dos empresários
\end{itemize}
\subsection*{Mercado de trabalho}
\label{sec:org5504b20}
\begin{itemize}
\item Redução do número de postos de trabalho
\end{itemize}
\subsection*{Crédito e inadimplência}
\label{sec:org82101d9}
\subsection*{Ambiente externo}
\label{sec:orgaa4c802}
\subsection*{Comércio exterior e reservas internacionais}
\label{sec:orgc555f18}
\subsection*{Mercado monetário e operações de mercado aberto}
\label{sec:orgd29569d}
\section*{\textit{<2011-01-09 dom> } \href{https://www.bcb.gov.br/publicacoes/atascopom/31082011}{Ata Nº161}}
\label{sec:org24de280}
\subsection*{Evolução rencente da economia}
\label{sec:org6d151a1}
\begin{itemize}
\item Aceleração da inflação medida pelo IPCA
\begin{itemize}
\item Reflete tanto preços livres quanto administrados
\begin{itemize}
\item Destaque aos bens comercializáveis e dos não-comercializáveis e queda do preço dos bens duráveis
\end{itemize}
\item Inflação de serviços elevada
\item Sugere persistência da alta dos preços
\end{itemize}
\item Outros índices de inflação sugerem trajetória similar
\item Tendência de moderação da taxa de crescimento medido pelo IBCBr
\item Aumento do índice de confiança dos serviços
\item Alta do índice de produção fabril
\begin{itemize}
\item Expansão do índice de produção industrial no acumulado 12 meses
\item Crescimento tanto no setor de bens de consumo duráveis quanto no setor de bens de capital, mas é o menor dentre as categorias de uso
\end{itemize}
\item Redução na taxa de desemprego e atingiu mínimo histórico
\item Crescimento do rendimento médio habitual real
\begin{itemize}
\item Destaque para sustentação no crescimento da demanda
\end{itemize}
\item Crescimento no volume de vendas no comércio
\item Grau de utilização da capacidade relativamente constante com tendência de queda na margem
\item Aumento do saldo da balança comercial
\begin{itemize}
\item Exportações cresceram mais que o crescimento das importações
\item Queda no déficit de transações correntes
\item Entrada de fluxos de capitais superiores à necessidade de financiamento externo
\end{itemize}
\item Período de elevada incerteza
\begin{itemize}
\item Destaque para maior risco para a estabilidade financeira global
\begin{itemize}
\item Exposição de bancos internacionais a dívidas soberanas
\end{itemize}
\item Aumento dos indicadores de aversão ao risco
\end{itemize}
\item Elevada volatilidade do preço do barril de petróleo \(\Rightarrow\) estabilidade da demanda global
\end{itemize}
\subsection*{Avaliação prospectiva das tendências de inflação}
\label{sec:orgebe394f}
\subsection*{Implementação da política monetária}
\label{sec:org390defa}
\subsection*{Inflação}
\label{sec:org97d03fc}
\subsection*{Atividade econômica}
\label{sec:org9690dae}
\subsection*{Expectativas e sondagens}
\label{sec:org06529c5}
\subsection*{Mercado de trabalho}
\label{sec:org2e48ff4}
\subsection*{Crédito e inadimplência}
\label{sec:org6733542}
\subsection*{Ambiente externo}
\label{sec:org38c76d7}
\subsection*{Comércio exterior e reservas internacionais}
\label{sec:orgcf65165}
\subsection*{Mercado monetário e operações de mercado aberto}
\label{sec:orgfac70bd}
\section*{\textit{<2020-11-03 ter> } \href{https://www.bcb.gov.br/publicacoes/atascopom/28102020}{Ata Nº234}}
\label{sec:orga5fc477}
\subsection*{Atualização da conjuntura econômica e do cenário básico do Copom}
\label{sec:org7f02a97}
\begin{itemize}
\item \textbf{Setor externo:} Desaceleração da retomada de alguns setores
\begin{itemize}
\item Evolução da COVID-19
\item Possibilidade de resução dos estímulos governamentais
\end{itemize}
\item Moderação da volatilidade dos ativos financeiros \(\Rightarrow\) favorece economias emergentes
\item \textbf{Mercado doméstico:} Recuperação desigual entre setores
\begin{itemize}
\item Renda emergencial
\item Setores mais afetados continuam com a mesma tendência
\item Aumento da incerteza sobre a retomada da economia \(\Rightarrow\) redução da renda emergencial
\end{itemize}
\item Indicadores da inflação compatíveis com a meta
\item Aumento da projeção da inflação
\begin{itemize}
\item Alta do preço dos alimentos e de bens industriais \(\Rightarrow\) depreciação do Real e das \emph{commodities}
\item Perspectiva de normalização de preços com aumento extraordinário \(\Rightarrow\) restrição na produção e aumento da demanda
\item Diganóstico de que este choque é temporário
\end{itemize}
\item Ociosidade no setor de serviços
\item Auxílios fiscais e adiamento de reformas \(\Rightarrow\) aumento do prêmio de risco
\item Maiores reduções nas taxas de juros \(\Rightarrow\) instabilidade nos preços dos ativos
\begin{itemize}
\item Proximidade à ZLB
\end{itemize}
\item Projeções de inflação abaixo da meta
\begin{itemize}
\item Expectativas de longo prazo ancoradas
\end{itemize}
\end{itemize}
\section*{Texto apresentação}
\label{sec:org711526d}

\subsection*{Panorama FHC e implementação do regime de metas}
\label{sec:org8bffe35}

A década de 90, diferentemente da década anterior, está inserida em um contexto de retomada dos fluxos voluntários de capital (alta da liquidez internacional) em que as reformas liberalizantes desempenharam um papel norteador.
Nesses moldes, caberia ao Estado promover a estabilidade macroeconômica bem como funcionamento dos mercados.
A dinamização da economia, por sua vez, estaria a cargo do setor privado junto dos ganhos (de produtividade e competitividade) decorrentes das aberturas comercial e financeira.
Como consequência, desmontou-se a ossatura do Estado desenvolvimentista.

Em termos da \textbf{estrutura produtiva}, a combinação de abertura comercial e moeda sobrevalorizada impuseram dificuldades à indústria nacional. 
Soma-se a isso o comportamento patrimonialista do IDE que não ampliou a produtividade/competitividade, aumentando o coeficiente importado (acima do exportado) assim como não foi direcionado para ampliar capacidade produtiva. 

No que diz respeito à \textbf{política monetária}, destaca-se seu alinhamento com a âncora cambial e subsequente controle inflacionário. 
Em meio a esses objetivos, a taxa de juros (elevadas) teve um papel duplo:
(i) atrair fluxos de capitais e;
(ii) conter a absorção doméstica.
Desse modo, a estabilização monetária foi estabelecida às custas da estabilidade macroeconômica \cite{belluzzoDepoisQuedaEconomia2002}.
Em paralelo, como resultado das queimas de reservas em 1997/8, a âncora cambial foi substituída pelo regime de câmbio flutuante enquanto, em 1999, aderiu-se ao regime de metas para a inflação de modo a estabelecer uma âncora nominal doméstica.

Nos governos FHC, portanto, manteve-se o binômio juros elevado-câmbio sobrevalorizado, implicando em uma dinâmica \emph{stop and go}, aumento do desemprego e crises bancárias e do balanço de pagamentos.
É em meio ao esse cenário adverso (``Herança Maldita'') que o governo Lula se inicia mas foi seguido de uma redução da restrição externa em decorrência da engrenagem comercial entre EUA-China-Periferia, elevando o ciclo de liquidez e promovendo melhora nos termos de troca (\emph{boom} de \emph{commodities}) \cite{carneiroDesenvolvimentoEmCrise2002}. 

\subsection*{Breve esquema do modelo de metas}
\label{sec:orgdabf650}

\subsection*{Continuísmo e descontinuidades dos governos Lula}
\label{sec:org7a146ca}

Apesar da melhora da adversidade do setor externo, tal governo é marcado, até 2006, por um \textbf{continuísmo}. 
A preocupação com a solvência da dívida pública bem como da adesão da contração fiscal expansionista é um dos traços desse conservadorismo. 
Do ponto de vista dos fluxos, verificou-se uma deterioração da composição dos gastos em que ampliou-se os encargos com juros em detrimento do investimento público para gerar superávits fiscais.
Já em termos dos estoques, observa-se uma melhora da dívida pública (principalmente externa) junto de uma piora de seus prazos e custos.
A melhora da dívida em moeda estrangeira, no entanto, não indica que a restrição externa foi superada, mas sim que reaparece sob novas formas.
Por fim, destaca-se que o governo também aprofunda a abertura financeira, extinguindo a conta CC5 e unificando o mercado de câmbio livre e flutuante \cite{pratesSECAOIVInsercao}.

Apesar deste continuísmo macroeconômico, o governo Lula apresentou uma ruptura não-trivial: centralidade da questão fiscal.
Esta ênfase da questão social é impulsionada com o progressivo abandono do conservadorismo fiscal.
A \textbf{partir de 2006} ---  e acentuado no pós-crise --- tem início uma estratégia mais clara que coloca no Estado o papel de
articular o desenvolvimento em que o planejamento de longo prazo é retomado (PAC). 
Assim, a despeito de um banco central ainda bastante ortodoxo e da política de valorização cambial, coube a política fiscal assumir para si o papel de provedora do crescimento. 
Assume esse papel, no entanto, sem que se mude o regime fiscal instaurado por FHC, mas há uma nova gestão fiscal e orçamentária.

Sendo assim, a contração fiscal expansionista deu lugar a uma política fiscal mais ativa sem implicar em uma deterioração da dinâmica da dívida pública. 
Dentre os fatores que permitiram tal guinada, destaca-se os descontos do investimento público na metas de superávit primário da LDO bem como aumento da arrecadação.
Em linhas gerais, o aumento das receitas reflete a centralidade da questão social mencionada anteriormente que foi além de transferências e gastos sociais e avançou em direção às mudanças no mercado de trabalho (formalismo e ganhos salariais) e ampliação do crédito.
Em conjunto, tais medidas promoveram --- dada a redução da restrição externa --- o crescimento baseado no mercado interno junto da melhora na distribuição de renda.
Essa combinação de crescimento com inclusão social --- levada adiante pelo governo Dilma --- é entendida como a base da estratégia social-desenvolvimentista no período \cite{biancarelliVelhaSenhoraEm2019}. 

Em relação a \textbf{estrutura produtiva}, nota-se o insucesso em promover uma política industrial e creditícia capaz de reverter algumas tendências estruturais.
Nesses termos, a restrição externa reaparece sob novas formas que não são captadas pelos indicadores de solvência: redução do conteúdo tecnológico das exportações e aumento do coeficiente importando.
Em outras palavras, o governo Lula se encerra com um \textbf{descompasso} entre uma estrutura de demanda modernizada que não foi acompanhada da modernização da estrutura de oferta \cite{melloIndustrialismoAusteridadePolitica}.
O governo Dilma herda esses desafios estruturais, buscando readequar o modelo de
crescimento em contexto internacional mais difícil.

\subsection*{Governo Dilma I}
\label{sec:org4a0703d}

Em meio ao diagnóstico de \textbf{sobreaquecimento} da economia, são adotadas medidas macroprudenciais e de redução de gasto.
Após o entendimento de que esse diagnóstico foi inadequado, o governo Dilma passa a
aprofundar a direção do desenvolvimentismo.
De modo a superar os entraves mencionados anteriormente, adota-se uma agenda industrialista.
No centro da análise estava a necessidade de melhorar a competitividade industrial (pela desvalorização cambial) e a rentabilidade do investimento privado por meio de redução de custos (política monetária e fiscal). Vale notar que esta reordenação de prioridades foi feita sem que, para isso, fosse necessário abandonar a centralidade da questão social.

Alinhado com a agenda industrialista e em meio a retomada (súbita) dos fluxos de capital, a política cambial esteve no centro.
De modo a evitar a valorização e volatilidade do câmbio para melhorar a competitividade industrial, foram feitas intervenções no mercado interbancário.
Como resultado, reduziu a especulação pela apreciação e desprendeu o real do ciclo de liquidez internacional.
Contudo, a partir de 2014, em face de pressões para desvalorização, tais medidas de regulação cambial foram retiradas e a política cambial passou a contar
basicamente com intervenções no mercado de derivativos por meio de swaps cambiais e anuncia programa de leilões de câmbio
\cite{cintraFinanciamentoContasExternas2015,melloIndustrialismoAusteridadePolitica}.

Além do rompimento da valorização cambial enquanto instrumento de controle inflacionário, a \textbf{política monetária} também foi redirecionada para a agenda industrialista (sem romper com o RMI).
Na ausência de um padrão de financiamento autônomo de longo prazo, os bancos públicos foram acionados para reduzir a taxa de juros e os spreads bancários para facilitar o
financiamento do desenvolvimento.
Com redução da Selic e política cambial ativa, o controle de preços contou com o
represamento de preços administrados (com finalidade também de reduzir custos).
Na tentativa de ampliar o horizonte temporal do financiamento público, alterou-se o manejo da dívida pública substituindo LFTs por títulos pré-fixados para, assim, estimular o mercado de longo prazo. 


Em paralelo, fez-se uso da política fiscal por meio de incentivos, desonerações e concessões para promover investimento privado (em infraestrutura).
Além disso, é importante destacar que, na ausência de um padrão de financiamento autônomo de longo prazo, o financiamento ficou a cargo do setor público (BNDES).
Dentre as consequências, verificou-se uma deterioração do saldo fiscal cujo principal condicionante foi a redução das \textbf{receitas}.
No que diz respeito aos gastos, destaca-se uma redução do investimento público em que ampliou-se a participação relativa dos gastos com pessoal.
Assim, reduziu-se o efeito expansionista da política fiscal dada a concentração de gastos com menores multiplicadores fiscais.
\subsection*{Conclusão e comparação geral}
\label{sec:orge4a7512}
A título de conclusão, resta uma comparação geral.
Os governos FHC e parte do governo Lula são caracterizados pelo prosseguimento da agenda liberalizante em que o papel do Estado foi minorado.
Apedas de algumas continuidades, o governo Lula promoveu um ineditismo histórico: conjugação de crescimento com inclusão social.
Tal resultado --- levado adiante pelo governo Dilma --- decorre tanto da questão social quando do abandono progressivo do conservadorismo fical, reposicionando o papel do Estado no desenvolvimento.
Em conjunto, tais medidas promoveram uma dinamização do mercado interno e modernização da demanda que não foi acompanhada igualmente pela estrutura produtiva.
De modo a superar tais desafios, adotou-se uma agenda industrialista que, por sua vez, não foi industrializante.
Tal insucesso deve ser analisado além da conjuntura e da política econômico, incluindo elementos estruturais e de economia política.
\end{document}

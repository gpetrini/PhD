% Intended LaTeX compiler: pdflatex
\documentclass[11pt]{article}
\usepackage[utf8]{inputenc}
\usepackage{lmodern}
\usepackage[T1]{fontenc}
\usepackage[top=3cm, bottom=2cm, left=3cm, right=2cm]{geometry}
\usepackage{graphicx}
\usepackage{longtable}
\usepackage{float}
\usepackage{wrapfig}
\usepackage{rotating}
\usepackage[normalem]{ulem}
\usepackage{amsmath}
\usepackage{textcomp}
\usepackage{marvosym}
\usepackage{wasysym}
\usepackage{amssymb}
\usepackage{amsmath}
\usepackage[theorems, skins]{tcolorbox}
\usepackage[style=abnt,noslsn,extrayear,uniquename=init,giveninits,justify,sccite,
scbib,repeattitles,doi=false,isbn=false,url=false,maxcitenames=2,
natbib=true,backend=biber]{biblatex}
\usepackage{url}
\usepackage[cache=false]{minted}
\usepackage[linktocpage,pdfstartview=FitH,colorlinks,
linkcolor=blue,anchorcolor=blue,
citecolor=blue,filecolor=blue,menucolor=blue,urlcolor=blue]{hyperref}
\usepackage{attachfile}
\usepackage{setspace}
\usepackage{tikz}
\author{PED: Gabriel Petrini}
\date{21 de Setembro de 2020}
\title{Resenha acadêmica}
\begin{document}

\maketitle
\setcounter{tocdepth}{1}
\tableofcontents


\section*{Introdução}
\label{sec:org8b334d5}

\subsection*{O que é uma resenha?}
\label{sec:orgbed96c1}

Reflexão sobre um texto de outro

\begin{itemize}
\item Propósito específico
\item Leitura ativa
\end{itemize}

\textbf{Importante:} Resenha \(\neq\) Resumo \(\neq\) Fichamento

\begin{NOTES}
\textbf{Resumo:} Condensação de um texto preservando a intenção do autor
\end{NOTES}


\subsection*{E uma resenha acadêmica?}
\label{sec:org6330160}

Existem vários tipos de resenha, vamos nos ater à \uline{resenha acadêmica}.

Posicionamento crítico do objeto analisado em que são apresentadas relações entre seus elementos mais importantes:

\begin{itemize}
\item Contextualização
\item Questões centrais
\item Como argumento é construído
\end{itemize}

\subsection*{Etapas para elaborar uma resenha}
\label{sec:orga2f5406}


\begin{figure}[htbp]
\centering
\includegraphics[width=.9\linewidth]{./esquema.png}
\caption{\label{fig:esquema}Um esquema óbvio, mas necessário}
\end{figure}


\section*{Como ler um texto acadêmico?}
\label{sec:org196c99a}

A leitura ativa é a parte mais importante no processo de uma resenha. A escrita é o resultado.
Segue uma \uline{sugestão} de como ler um texto acadêmico:

\texttt{FISH-5SS}

\subsection*{O que preciso saber?}
\label{sec:org8881daf}

FISH!

\begin{itemize}
\item \textbf{F} ield: Área do texto \(\Rightarrow\) Com quem os autores estão dialogando?
\item \textbf{I} mportance: Relevância \(\Rightarrow\) Por que este texto está sendo resenhado?
\item \textbf{S} upporting ideias/data: Embasamento teórico/Empírico \(\Rightarrow\) Quais as informações prévias para a compreensão do texto?
\item \textbf{H} ypothesis: Quais são as hipóteses centrais?
\end{itemize}


\begin{NOTES}
Este é um esqueleto que eu uso em alguns casos em que acho pertinente. Um dos motivos pelo qual gosto desta forma é a flexibilidade.
Obviamente é possível incluir outros elementos importantes como é o caso da \textbf{Metodologia} adotada

Lembrando mais uma vez que é apenas uma sugestão. Existem inúmeras alternativas ao FISH:

\begin{itemize}
\item Dar títulos aos parágrafos
\item Mapas mentais/fluxosgramas
\end{itemize}

\textbf{Importante:} Se não conseguiu detectar alguns desses elementos, pode ser o caso de uma leitura mal feita/passiva
\end{NOTES}

\subsection*{Como sei que sei?}
\label{sec:org88b38ac}

5SS!

\textbf{5 Seconds Syntesis:} Resumir a ideia ao mínimo possível

\begin{itemize}
\item Um parágrafo que sintetiza os elementos centrais do texto a ser resenhado
\begin{itemize}
\item Capacidade de síntese é uma das mais importantes
\end{itemize}
\end{itemize}

\begin{NOTES}
\texttt{Minha tese ficou longa porque não tive tempo de escrever pouco}

Carlos Lessa
\end{NOTES}

\section*{Como escrever uma resenha acadêmica?}
\label{sec:org424747e}

\subsection*{O conhecimento e o texto acadêmico}
\label{sec:orgd8167ec}

\textbf{Conhecimento acadêmico:} Saber algo é a capacidade de escrever um parágrafo coeso sobre isso


\textbf{Quanto preciso saber para escrever?:}

\begin{itemize}
\item Uma primeira aproximação é a partir da esquematização proposta pelo método FISH
\item Outra informação relevante é sobre o limite de páginas
\begin{itemize}
\item Isso irá \emph{delimitar} quantos parágrafos (ideias coesas) a resenha terá
\end{itemize}
\end{itemize}


\subsection*{Enfim, a resenha}
\label{sec:orge23b908}

Feita a leitura \uline{ativa} do texto, sistematizados seus principais elementos, basta organizar as ideias de forma coesa. 

\begin{NOTES}
Seguem algumas perguntas que podem guiar as resenhas desta disciplina:
\end{NOTES}

\begin{itemize}
\item Como o contexto se relaciona com as hipóteses adotadas? São razoáveis?
\item Em que medida estas ideias se diferenciam das anteriores? Existem elementos em comum?
\item Quais são as implicações e proposições macroeconômicas?
\item \textbf{Memo:} Resenha não é resumo. Não é necessário preservar a sequência e/ou intenção do autor
\end{itemize}

\begin{NOTES}
Qual a importância da visão predominante na época em que outras ideias foram propostas?
\begin{itemize}
\item Em que medida a expressividade na Síntese Neoclássica foi relevante para as escolas que a sucederam?
\end{itemize}
\end{NOTES}

\section*{O que esperamos das resenhas?}
\label{sec:orgc3a642a}

\subsection*{Instruções}
\label{sec:org2b6d315}

\begin{itemize}
\item Individuais
\item Forma de entrega: em discussão
\item Devem ter no máximo duas páginas
\end{itemize}

Serão devolvidas aos alunos

\subsection*{Elementos}
\label{sec:org45ef676}

As resenhas serão avaliadas de acordo com as respostas dadas às seguintes questões
ao longo do texto:

\begin{itemize}
\item Qual é o tema do texto? Qual o objetivo do autor?
\item Qual é a hipótese/argumento central do autor?
\item Quais os conceitos utilizados?
\item Como ele desenvolve e organiza os argumentos?
\item Qual a conclusão do texto?
\end{itemize}


A forma em que as ideias são apresentadas (clareza e organização do texto) também
será avaliada.

\subsection*{Cronograma}
\label{sec:org594369b}

\begin{center}
\begin{tabular}{ll}
Resenha & Data\\
\hline
Monetaristas & 19/10\\
Novos Clássicos & 09/11\\
Novos Keynesianos & 23/11\\
Modelo de Metas para Inflação & 30/11\\
\hline
\end{tabular}
\end{center}

\section*{Referências}
\label{sec:org020b90b}

\begin{itemize}
\item \href{https://posgraduando.com/fish-qtcr-5ss-leitura-artigos/}{Método FISH/5SS}
\item \href{https://www.youtube.com/watch?v=vECQOychZyY}{O que é um parágrafo?}
\end{itemize}

\subsection*{Obrigado}
\label{sec:orgcbd429b}

\section*{Como saber o que escrever?}
\label{sec:org416a854}


\textbf{Memo:} Saber algo é a capacidade de escrever um parágrafo coeso sobre isso

\begin{table}[htbp]
\caption{\href{https://www.youtube.com/watch?v=vECQOychZyY}{O que é um parágrafo?}}
\centering
\begin{tabular}{ll}
\hline
Qual a dificuldade do leitor ao se deparar com este conhecimento? & Tipo de parágrafo\\
\hline
"Acreditar" & Auxiliador\\
Compreender & Dissertativo\\
Concordar & Argumentativo\\
\hline
\end{tabular}
\end{table}
\end{document}
% Created 2020-10-05 seg 11:06
% Intended LaTeX compiler: pdflatex
\documentclass[11pt]{article}
\usepackage[utf8]{inputenc}
\usepackage{lmodern}
\usepackage[T1]{fontenc}
\usepackage[top=3cm, bottom=2cm, left=3cm, right=2cm]{geometry}
\usepackage{graphicx}
\usepackage{longtable}
\usepackage{float}
\usepackage{wrapfig}
\usepackage{rotating}
\usepackage[normalem]{ulem}
\usepackage{amsmath}
\usepackage{textcomp}
\usepackage{marvosym}
\usepackage{wasysym}
\usepackage{amssymb}
\usepackage{amsmath}
\usepackage[theorems, skins]{tcolorbox}
\usepackage[style=abnt,noslsn,extrayear,uniquename=init,giveninits,justify,sccite,
scbib,repeattitles,doi=false,isbn=false,url=false,maxcitenames=2,
natbib=true,backend=biber]{biblatex}
\usepackage{url}
\usepackage[cache=false]{minted}
\usepackage[linktocpage,pdfstartview=FitH,colorlinks,
linkcolor=blue,anchorcolor=blue,
citecolor=blue,filecolor=blue,menucolor=blue,urlcolor=blue]{hyperref}
\usepackage{attachfile}
\usepackage{setspace}
\usepackage{tikz}
\usepackage[portuguese, english]{babel}
\author{Ana Rosa Ribeiro Mendonça, Gabriel Petrini (PED)}
\date{28 de setembro  e 05 de outrobro de 2020}
\title{Contextualização da adoção do Regime de Metas para Inflação}
\begin{document}

\maketitle

\section*{Introdução: Debate acerca da condução da política monetária}
\label{sec:org7e08517}

\subsection*{Regras x Discrição}
\label{sec:orgd1eb54d}

\subsubsection*{Ativismo monetário (discricionarismo)}
\label{sec:org71c50ba}

\begin{itemize}
\item Liberdade para o uso da Política Monetária diante da \uline{conjuntura econômica}
\begin{itemize}
\item Maior raio de manobra
\end{itemize}
\item Não há compromissos com \uline{objetivos} preestabelecidos
\item Não há uma especificação dos instrumentos a serem utilizados
\end{itemize}

\subsubsection*{Regras}
\label{sec:org9080a4f}

\begin{itemize}
\item Condiciona Política Monetária a objetivos preestabelecidos
\item Não há alteração no formato da Política Monetária em função da conjuntura
\item Mecanismo de comprometimento ou de incentivo para se evitar o \textbf{viés inflacionário}
\begin{itemize}
\item Maior credibilidade e previsibilidade à política monetária
\end{itemize}
\end{itemize}

\subsection*{Regimes monetários}
\label{sec:org27e9307}

Baseados em regras para a política monetária amparada em determinada regra (âncora)

\begin{itemize}
\item \textbf{Âncora nominal} \(\Rightarrow\) adotar uma variável nominal
\item \textbf{Regime de metas cambiais} \(\Rightarrow\) taxa de câmbio
\begin{itemize}
\item Formas: Commodity (\emph{ex:} real) ou moeda externa (\emph{ex:} dólar)
\item \textbf{Brasil:} 1995-1999
\item Crises em economias periféricas
\begin{itemize}
\item \emph{Overshooting} não era suficiente para eliminar as especulações em moeda de economias periféricos. O mesmo vale para empréstimos \emph{standby} no Brasil
\end{itemize}
\end{itemize}
\item \textbf{Regime de metas monetárias:} Supõe relação estável entre estoque de moeda e nível de preços
\begin{itemize}
\item \textbf{Meta} \(\Rightarrow\) determinada quantia de moeda
\item \textbf{Banco Central} \(\Rightarrow\) controlar base monetária
\begin{itemize}
\item Os agregados monetários todos não são passíveis de controle do Banco Central, mas sim, gerida por agentes fora do sistema regulatório
\end{itemize}
\item Variação inadequada da quantidade de moeda \(\Rightarrow\) inflação
\end{itemize}
\item \textbf{Regime de metas para inflação} \(\Rightarrow\) taxa de inflação
\begin{itemize}
\item Maior transparência à Política Monetária e associada à pauta do Banco Central independente
\item \textbf{Banco Central} \(\Rightarrow\) Taxa de juros de curto prazo
\begin{itemize}
\item Inflação não esta sob o controle do Banco Central, mas sim os juros
\end{itemize}
\end{itemize}
\end{itemize}

Papel na estabilidade de preços:

\begin{itemize}
\item Coordenação de expectativas de inflação
\item Balizamento da formação de preços
\end{itemize}

\section*{Plano Real e suas fases}
\label{sec:org0787d82}

\subsection*{Ajuste Fiscal (preparação)}
\label{sec:org882f60c}

\begin{itemize}
\item \textbf{Diagnóstico:} equilíbrio fiscal fundamental para sucesso de política de estabilidade de preços
\begin{itemize}
\item Relação negativa entre déficit fiscal e inflação
\item Subavaliação da inflação na composição do orçamento e receitas indexadas
\end{itemize}
\item \textbf{PAI} (1993): Programa de Ação Imediata
\begin{itemize}
\item Desordem financeira e administrativa do setor público contribuíam para a inflação
\item Diminuição e maior eficiência dos gastos públicos
\item Criação do IPMF
\end{itemize}
\item \textbf{FSE} (1994):
\begin{itemize}
\item Reduzir rigidez dos gastos da união
\item Desvincular receitas de saída, em especial com gastos sociais
\end{itemize}
\end{itemize}

Medidas não foram suficientes para evitar aumento da dívida líquida do setor público

\subsection*{Reforma Monetária}
\label{sec:org27d68a4}

\textbf{Objetivo:} Eliminar  a inércia inflacionária (indexação)
\begin{itemize}
\item Assincronia de reajustes de preços e rendimentos
\item \textbf{Solução:} alinhas os preços relativos por meio de um indexador universal e contemporâneo (URV, unidade de conta)
\begin{itemize}
\item Cálculo da URV era feito a partir da ponderação de índices de preços (IGP-M, IPCA e IPC)
\item Não era o movimento cambial que definia o valor da URV, mas sim uma atuação do Banco Central para que a variação cambial fosse da mesma proporção da variação da inflação em termos de URV
\end{itemize}
\end{itemize}


\subsubsection*{Âncora Monetária e âncora cambial}
\label{sec:org4537881}
\begin{itemize}
\item Âncora Monetária
\label{sec:orgb955423}
\textbf{Duração:} Julho de 1994 a  janeiro de 1999

\begin{itemize}
\item Relação fixa entre câmbio e real
\item Processo de remonetização
\begin{itemize}
\item Não se verifica a estabilidade de circulação da moeda
\item Forte crescimento da Base Monetária
\end{itemize}
\item Inconsistência conceitual
\begin{itemize}
\item BCB continuou publicar a programação monetária
\item Flutuação da taxa de câmbio \(\Rightarrow\) Valorização forte do real
\begin{itemize}
\item Para redução da inflação (e tarifas de importação) dos bens transacionáveis
\begin{itemize}
\item Isso somado à pressão pela liberalização comercial e imposição da concorrência
\item Convergir a inflação doméstica à inflação externa
\end{itemize}
\end{itemize}
\end{itemize}
\end{itemize}
\item Âncora cambial
\label{sec:org8fa9f77}

\begin{itemize}
\item Flutuação cambial (Julho a setembro de 1994)
\begin{itemize}
\item Banda assimétrica anunciada
\item BCB não interveio
\item \textbf{Situação externa:} elevada liquidez
\item Diminuição do imposto de importação
\end{itemize}
\item Taxa de câmbio fixa (1US\$ = 0,84 R\$)
\begin{itemize}
\item BCB interveio no mercado de câmbio
\item Setembro de 1994 a fevereiro de 1995
\item Crise Mexicana \(\Rightarrow\) reversão na liquidez internacional
\begin{itemize}
\item Perda de reservas
\end{itemize}
\end{itemize}
\end{itemize}
\end{itemize}
\end{document}
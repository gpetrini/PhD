% Created 2020-09-28 seg 20:02
% Intended LaTeX compiler: pdflatex
\documentclass[11pt]{article}
\usepackage[utf8]{inputenc}
\usepackage{lmodern}
\usepackage[T1]{fontenc}
\usepackage[top=3cm, bottom=2cm, left=3cm, right=2cm]{geometry}
\usepackage{graphicx}
\usepackage{longtable}
\usepackage{float}
\usepackage{wrapfig}
\usepackage{rotating}
\usepackage[normalem]{ulem}
\usepackage{amsmath}
\usepackage{textcomp}
\usepackage{marvosym}
\usepackage{wasysym}
\usepackage{amssymb}
\usepackage{amsmath}
\usepackage[theorems, skins]{tcolorbox}
\usepackage[style=abnt,noslsn,extrayear,uniquename=init,giveninits,justify,sccite,
scbib,repeattitles,doi=false,isbn=false,url=false,maxcitenames=2,
natbib=true,backend=biber]{biblatex}
\usepackage{url}
\usepackage[cache=false]{minted}
\usepackage[linktocpage,pdfstartview=FitH,colorlinks,
linkcolor=blue,anchorcolor=blue,
citecolor=blue,filecolor=blue,menucolor=blue,urlcolor=blue]{hyperref}
\usepackage{attachfile}
\usepackage{setspace}
\usepackage{tikz}
\usepackage[portuguese, english]{babel}
\author{Ana Rosa Ribeiro Mendonça, Gabriel Petrini (PED)}
\date{28 de setembro de 2020}
\title{Contextualização da adoção do Regime de Metas para Inflação}
\begin{document}

\maketitle

\section*{Introdução: Debate acerca da condução da política monetária}
\label{sec:org8e0d4c6}

\subsection*{Regras x Discrição}
\label{sec:org764b7b6}

\subsubsection*{Ativismo monetário (discricionarismo)}
\label{sec:orga3d17dc}

\begin{itemize}
\item Liberdade para o uso da Política Monetária diante da \uline{conjuntura econômica}
\begin{itemize}
\item Maior raio de manobra
\end{itemize}
\item Não há compromissos com \uline{objetivos} preestabelecidos
\item Não há uma especificação dos instrumentos a serem utilizados
\end{itemize}

\subsubsection*{Regras}
\label{sec:org5a212d1}

\begin{itemize}
\item Condiciona Política Monetária a objetivos preestabelecidos
\item Não há alteração no formato da Política Monetária em função da conjuntura
\item Mecanismo de comprometimento ou de incentivo para se evitar o \textbf{viés inflacionário}
\begin{itemize}
\item Maior credibilidade e previsibilidade à política monetária
\end{itemize}
\end{itemize}

\subsection*{Regimes monetários}
\label{sec:orga611347}

Baseados em regras para a política monetária amparada em determinada regra (âncora)

\begin{itemize}
\item \textbf{Âncora nominal} \(\Rightarrow\) adotar uma variável nominal
\item \textbf{Regime de metas cambiais} \(\Rightarrow\) taxa de câmbio
\begin{itemize}
\item Formas: Commodity (\emph{ex:} real) ou moeda externa (\emph{ex:} dólar)
\item \textbf{Brasil:} 1995-1999
\item Crises em economias periféricas
\begin{itemize}
\item \emph{Overshooting} não era suficiente para eliminar as especulações em moeda de economias periféricos. O mesmo vale para empréstimos \emph{standby} no Brasil
\end{itemize}
\end{itemize}
\item \textbf{Regime de metas monetárias:} Supõe relação estável entre estoque de moeda e nível de preços
\begin{itemize}
\item \textbf{Meta} \(\Rightarrow\) determinada quantia de moeda
\item \textbf{Banco Central} \(\Rightarrow\) controlar base monetária
\item Variação inadequada da quantidade de moeda \(\Rightarrow\) inflação
\end{itemize}
\item \textbf{Regime de metas para inflação} \(\Rightarrow\) taxa de inflação
\end{itemize}

Papel na estabilidade de preços:

\begin{itemize}
\item Coordenação de expectativas de inflação
\item Balizamento da formação de preços
\end{itemize}
\end{document}
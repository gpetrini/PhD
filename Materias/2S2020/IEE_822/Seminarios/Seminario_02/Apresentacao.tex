% Created 2021-03-08 seg 18:46
% Intended LaTeX compiler: pdflatex
\documentclass[presentation]{beamer}
  \usepackage{caption, subcaption}
\usepackage[brazilian, ]{babel}
\usepackage[style=abnt,noslsn,extrayear,uniquename=init,giveninits,justify,sccite, scbib,repeattitles,doi=false,isbn=false,url=false,maxcitenames=2, natbib=true,backend=biber]{biblatex}
\addbibresource{/HDD/Org/all_my_refs.bib}
\AtBeginSection[]{\begin{frame}<beamer>\frametitle{Estrutura da apresentação}\tableofcontents[currentsection]\end{frame}}
\usetheme{default}
\author{Gabriel Petrini}
\date{06 de Janeiro de 2021}
\title{Panorama das regularidades empíricas 2 - \textcite{custodio_2013_Why}}
\begin{document}

\maketitle
\section{\fullcite{custodio_2013_Why}}
\label{sec:org893fb00}

\begin{frame}[label={sec:orgbf87cd4}]{Quem são os autores?}
\begin{figure}
\caption{Autores}
\begin{subfigure}{.3\linewidth}
\centering
\includegraphics[width=.95\textwidth]{./figs/custodio.jpeg}
\caption{Cláudia Custódio\\(Imperial College)}
\end{subfigure}%
\begin{subfigure}{.3\linewidth}
\centering
\includegraphics[width=\textwidth]{./figs/Ferreira.jpg}
\caption{Miguel A. Ferreira\\(Nova SBE)}
\end{subfigure}%\\[1ex]
\begin{subfigure}{.3\linewidth}
\centering
\includegraphics[width=.8\textwidth]{./figs/Laureano.jpeg}
\caption{Luís Laureano\\(IUL, Lisboa)}
\end{subfigure}
\end{figure}


Conjuntamente (dois primeiros principalmente) pesquisam sobre finanças corporativas.
Artigos publicados em:
\begin{itemize}
\item Jornal of Financial Economics (este)
\item Management Science
\item Society and Economics
\end{itemize}
\end{frame}

\begin{frame}[label={sec:org6eb3b19}]{TL;DR (\emph{a.k.a} nem li; nem lerei)}
\alert{Objetivo:} Compreender o porquê da redução da maturidade da dívida de \(\pm\) 13 mil firmas americanas não-financeiras (1976-2008)
\begin{itemize}
\item Características das firmas explica pouco
\item Assimetria de informação explica um pouco mais, mas insuficiente
\item Novas firmas listadas na bolsa e oferta de crédito (desregulamentações financeiras) são relevantes
\item Restrito aos EUA
\end{itemize}


\begin{block}{Relevância e questões abertas}
\begin{itemize}
\item \(\Uparrow\) estrutura de dívida  curto prazo \(\Rightarrow \Uparrow\) Exposição ao risco
\item Pode ter amplificado efeitos da GFC
\item Existe algum dialogo com a literatura da financeirização?
\end{itemize}
\end{block}
\end{frame}


\begin{frame}[label={sec:orgbfb61a4}]{Estrutura do artigo}
\begin{block}{Amostra e descrição dos dados}
\end{block}
\begin{block}{Redução da maturidade da dívida e características das firmas}
\end{block}
\begin{block}{A demanda por maturidade de dívida mudou?}
\end{block}
\begin{block}{Novas firmas listadas em bolsa}
\end{block}
\begin{block}{Emissão de dívida e fatores da oferta de crédito}
\end{block}
\end{frame}
\section{Amostra e descrição dos dados}
\label{sec:org9a953e9}
\begin{frame}[label={sec:org85fc5b3}]{Desenho amostral}
\end{frame}
\begin{frame}[label={sec:org4634a17}]{Variáveis a ser explicada}
\end{frame}
\section{Características das firmas}
\label{sec:org10ba415}
\section{Função demanda por maturidade de dívida}
\label{sec:orgb26e0a3}
\section{Novas firmas listadas em bolsa}
\label{sec:org43735f8}
\section{Fatores de oferta}
\label{sec:org12a3e39}
\section{Conclusão}
\label{sec:org044f298}
\begin{frame}[label={sec:orgc980383}]{Retomada e implicações}
\end{frame}
\end{document}

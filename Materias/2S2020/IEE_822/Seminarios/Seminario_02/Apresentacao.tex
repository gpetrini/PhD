% Created 2021-03-09 ter 11:37
% Intended LaTeX compiler: pdflatex
\documentclass[presentation]{beamer}
  \usepackage{caption, subcaption}
\usepackage[brazilian, ]{babel}
\usepackage[style=abnt,noslsn,extrayear,uniquename=init,giveninits,justify,sccite, scbib,repeattitles,doi=false,isbn=false,url=false,maxcitenames=2, natbib=true,backend=biber]{biblatex}
\addbibresource{/HDD/Org/all_my_refs.bib}
\usetheme{default}
\author{Gabriel Petrini}
\date{06 de Janeiro de 2021}
\title{Panorama das regularidades empíricas 2 - \textcite{custodio_2013_Why}}
\begin{document}

\maketitle
\section{\fullcite{custodio_2013_Why}}
\label{sec:orgad194aa}

\begin{frame}[label={sec:orgd54013d}]{Quem são os autores?}
\begin{figure}
\caption{Autores}
\begin{subfigure}{.3\linewidth}
\centering
\includegraphics[width=.95\textwidth]{./figs/custodio.jpeg}
\caption{Cláudia Custódio\\(Imperial College)}
\end{subfigure}%
\begin{subfigure}{.3\linewidth}
\centering
\includegraphics[width=\textwidth]{./figs/Ferreira.jpg}
\caption{Miguel A. Ferreira\\(Nova SBE)}
\end{subfigure}%\\[1ex]
\begin{subfigure}{.3\linewidth}
\centering
\includegraphics[width=.8\textwidth]{./figs/Laureano.jpeg}
\caption{Luís Laureano\\(IUL, Lisboa)}
\end{subfigure}
\end{figure}


Conjuntamente (dois primeiros principalmente) pesquisam sobre finanças corporativas.
Artigos publicados em:
\begin{itemize}
\item Jornal of Financial Economics (este)
\item Management Science
\item Society and Economics
\end{itemize}
\end{frame}

\begin{frame}[label={sec:org0e95a44}]{TL;DR (\emph{a.k.a} nem li; nem lerei)}
\alert{Objetivo:} Compreender o porquê da redução da maturidade da dívida de \(\pm\) 13 mil firmas americanas não-financeiras (1976-2008)
\begin{itemize}
\item Características das firmas explica pouco
\item Assimetria de informação explica um pouco mais, mas insuficiente
\item Novas firmas listadas na bolsa e oferta de crédito (desregulamentações financeiras) são relevantes
\item Restrito aos EUA
\end{itemize}


\begin{block}{Relevância e questões abertas}
\begin{itemize}
\item \(\Uparrow\) estrutura de dívida  curto prazo \(\Rightarrow \Uparrow\) Exposição ao risco
\item Pode ter amplificado efeitos da GFC
\item Existe algum dialogo com a literatura da financeirização?
\end{itemize}
\end{block}
\end{frame}


\begin{frame}[label={sec:org8de9697}]{Estrutura do artigo}
\begin{block}{Amostra e descrição dos dados}
\end{block}
\begin{block}{Redução da maturidade da dívida e características das firmas}
\end{block}
\begin{block}{A demanda por maturidade de dívida mudou?}
\end{block}
\begin{block}{Novas firmas listadas em bolsa}
\end{block}
\begin{block}{Emissão de dívida e fatores da oferta de crédito}
\end{block}
\end{frame}
\section{Amostra e descrição dos dados}
\label{sec:org9a5a1dd}
\begin{frame}[label={sec:org6b69b21}]{Variável a ser explicada}
\begin{figure}[htbp]
\caption{Dívida com maturidade com mais de 3 anos (\%)}
\centerline{\includegraphics[width=\textwidth]{figs/Tendencia.png}}
\end{figure}
\end{frame}


\begin{frame}[label={sec:orge63e05f}]{Firmas americanas não-financeiras (1976-2008, Compustat)}
\begin{center}
\begin{tabular}{lrr}
\hline
Variável & Média & Mediana\\
\hline
Maturidade da dívida &  & \\
\hline
+3 anos & 0.438 & 0.460\\
+5 anos & 0.280 & 0.179\\
\hline
Características das firmas &  & \\
\hline
Tamanho & 0.242 & 0.104\\
Oportunidade de investimento (proxy) & 1.847 & 1.306\\
Lucros anormais & -0.029 & 0.007\\
Alavancagem & 0.273 & 0.242\\
Assimetria de informação (proxy, P\&D) & 0.040 & 0.000\\
CAPEX & 0.074 & 0.050\\
Índice de gov. corporativas & 9.152 & 9.000\\
Nota de rating? (dummy) & 0.229 & 0.000\\
Paga dividendo? (dummy) & 0.370 & 0.000\\
Fluxo de caixa & 0.131 & 0.061\\
Idade & 14 & 9\\
\hline
\end{tabular}
\end{center}
\end{frame}

\section{Características das firmas}
\label{sec:org73d53cf}

\begin{frame}[label={sec:org7cf5288}]{Discussão teórica}
\begin{table}[htbp]
\caption{Expectativas direcionais sobre maturidade da dívida}
\centering
\begin{tabular}{lll}
\hline
Variável & Direção & Proxy\\
\hline
Tamanho da firma & + & Percentil NYSE\\
Oport. crescimento & - & Market-to-book\\
Acionistas Vs. gerentes & - & Ind. governanças corporativas\\
Custos de agência & \(\Leftrightarrow\) & Alavancagem e oport. invest.\\
Assimetria de info. & - & P\&D\\
Melhores projetos & - & Lucros anormais\\
Volatilidade dos ativos & - & -\\
Alavancagem & + & Comparação com a mediana\\
Pagamento de dividendo & + & Dummy\\
Fluxo de caixa & - & Comparação com a mediana\\
\hline
\end{tabular}
\end{table}


\begin{itemize}
\item Dívida de curto prazo minimiza custos de agência \(\Rightarrow\) Renegociação mais frequente
\begin{itemize}
\item Subinvestimento
\item Substituição de ativos
\end{itemize}
\end{itemize}
\end{frame}

\begin{frame}[label={sec:orgc66c9c2}]{Conclusões}
\begin{table}[htbp]
\caption{Alguns dos resultados para redução da maturidade da dívida}
\centering
\begin{tabular}{lll}
\hline
Variável & Direção & Procede?\\
\hline
Tamanho da firma & + & Pouco\\
Acionistas Vs gerentes & - & Não\\
Custos de agência & \(\Leftrightarrow\) & Não\\
Assimetria de info. & - & Sim\\
Pagamento de dividendo & + & Sim, mas insuficiente\\
Fluxo de caixa & - & Sim, mas insuficiente\\
\hline
\end{tabular}
\end{table}


\begin{itemize}
\item Mudanças na composição da industria parecem ser mais relevantes
\item Tendência negativa restrita aos EUA
\end{itemize}
\end{frame}

\section{Os fatores de demanda explicam?}
\label{sec:orgba1ed01}

\begin{frame}[label={sec:orge8b1671}]{Estratégia de estimação}
\begin{latex}
\begin{equation}
\text{\% Div. curto prazo} \sim \text{Carac. Firmas}
\end{equation}
\end{latex}

\begin{block}{Modelos alternativos}
\begin{itemize}
\item Características das firmas
\item Fatores macroeconômicos
\item Dummies para décadas
\item Também estimam Tobit para verificar robustez dos resultados
\item Abordagem Fama and Macbeth (1976-1979)
\end{itemize}
\end{block}
\end{frame}

\begin{frame}[label={sec:org64e1dcf}]{Resultados e discussão}
\begin{itemize}
\item Os coeficientes (exceto lucros anormais) apresentam os sinais esperados
\begin{itemize}
\item Hipótese da sinalização não é verificada
\end{itemize}
\item Dummies para as décadas sugerem que composição da indústria é relevante
\item Fatores macroeconômicos explicam um pouco mais do que as características das firmas
\end{itemize}


\begin{block}{Principal resultado até aqui}
Caracterísitcas de demanda (características das firmas) explicam muito pouco a redução da maturidade da dívida.
\end{block}
\end{frame}

\section{Novas firmas listadas em bolsa}
\label{sec:orgeb41bb6}
\begin{frame}[label={sec:org4fd6851}]{Listing vintage effect}
\begin{figure}[htbp]
\caption{Dívida com maturidade com mais de 3 anos (\%)}
\centerline{\includegraphics[width=\textwidth]{figs/Listing.png}}
\end{figure}
\end{frame}
\begin{frame}[label={sec:org2256689}]{Estratégia de estimação}
\begin{itemize}
\item Dividem as firmas a partir do ano que foram listadas na bolsa
\item Ao incluir essa dummy, a tendência passa a ser positiva
\end{itemize}

\begin{block}{Conclusões e Explicações}
O efeito das novas firmas que entram na bolsa é suficiente para explicar
\begin{itemize}
\item Dados não corroboram hipótese de que esta variável só captura o ciclo de vida de uma firma
\item Uma firma jovem listada em bolsa nos anos 80s-90s tem mais dívida de curto prazo relativamente
\item Desregulamentações financeiras \(\Rightarrow\) maior acesso a \emph{public equity} (?)
\item Associado a fatores de oferta ou demanda das firmas?
\end{itemize}
\end{block}
\end{frame}

\section{Fatores de oferta}
\label{sec:orgc318f4d}

\begin{frame}[label={sec:org79a9cf8}]{Mudança de ênfase: emissão}
\begin{center}
\(\Uparrow\) Emissão \(\Leftrightarrow\) Preferências reveladas \(\Leftrightarrow\) demanda positiva por dívida
\end{center}

\begin{itemize}
\item Elimina as características das firmas (demanda)
\item Firmas maiores não parece reduzir a maturidade da dívida
\begin{itemize}
\item Estoque de dívida permanece no balanço de firmas maiores e mais antigas
\end{itemize}
\item \(\Delta\) Composição da dívida "pública" (ações) \(\Rightarrow\) Redução da maturidade
\begin{itemize}
\item Mesmo não ocorre para dívida bancária ("privada")
\end{itemize}
\end{itemize}

\begin{block}{Resultado das estimações}
\begin{itemize}
\item Fatores de demanda não explicam este resultado
\item Não ocorre com dívida bancária
\item Fatores de oferta parecem relevantes
\end{itemize}
\end{block}
\end{frame}

\begin{frame}[label={sec:org14c6838},fragile]{Os fatores de oferta}
 \begin{center}
Estrutura K firmas \(\Leftarrow\) oferta de crédito \(\Leftarrow\) demanda do investidor
\end{center}

\begin{block}{Abordagens: oferta de crédito permanece a mesma?}
\begin{itemize}
\item Se fatores de oferta não são relevantes, controle (listada pre-1980) \texttt{==}  tratamento (pós-80) \(\Rightarrow\) Fatores de demanda
\begin{itemize}
\item Condições de oferta não são as mesmas \(\Rightarrow\) Oferta de crédito é relevante
\end{itemize}
\item Introdução de efeitos exógenos
\begin{description}
\item[{1986-93}] Investidores concentraram em firmas com notas especulativas \(\Rightarrow\) Restringe maturidade da dívida das firmas com nota de investimento
\item[{2006-8}] Impacto distinto sobre a maturidade da dívida (firmas sem rating com maturidade menor)
\end{description}
\end{itemize}
\end{block}

\begin{block}{Conclusões}
\begin{itemize}
\item Demanda do investimento é relevante
\item Restrito aos EUA \(\Leftrightarrow\) Mercado corporativo mais desenvolvido
\end{itemize}
\end{block}
\end{frame}

\section{Conclusão}
\label{sec:orgd4dbafa}
\begin{frame}[label={sec:org999dfa2}]{Retomada e implicações}
Redução da maturidade da dívida das firmas dos EUA:
\begin{itemize}
\item Concentrada nas firmas menores
\item Maior grau de assimetria de informação \(\Rightarrow\) menor maturidade
\item Conflitos de agência e sinalizações explicam pouco
\item Novas firmas listadas em bolsa \(\Rightarrow\) explicam um pouco mais
\item Características das firmas não explicam a totalidade
\item Oferta de crédito é relevante
\item Concentrada na dívida não-bancária ("pública")
\end{itemize}

\begin{block}{Implicações}
\begin{itemize}
\item Firmas estão mais expostas a choques de crédito e de liquidez
\item Tais mudanças podem ter exacerbado os efeitos da GFC
\end{itemize}
\end{block}

\begin{block}{Limitação?}
\begin{itemize}
\item Os metodos adotados são os mais adequados para variáveis qualitativas
\end{itemize}
\end{block}
\end{frame}
\end{document}

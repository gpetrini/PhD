% Created 2020-12-22 ter 18:19
% Intended LaTeX compiler: pdflatex
\documentclass[presentation]{beamer}
  \usepackage{caption}
\usepackage[brazilian, ]{babel}
\usepackage[style=abnt,noslsn,extrayear,uniquename=init,giveninits,justify,sccite, scbib,repeattitles,doi=false,isbn=false,url=false,maxcitenames=2, natbib=true,backend=biber]{biblatex}
\addbibresource{/HDD/Org/all_my_refs.bib}
\AtBeginSection[]{\begin{frame}<beamer>\frametitle{Artigos}\tableofcontents[currentsection]\end{frame}}
\usetheme{default}
\author{Gabriel Petrini}
\date{06 de Janeiro de 2021}
\title{Panorama das regularidades empíricas - \textcite{schularickCreditBoomsGone2012} e  \textcite{sharpe_2020_Why}}
\begin{document}

\maketitle
\section{\textcite{schularickCreditBoomsGone2012}: Booms de crédito acabam: Política Monetária, ciclos alavancados e crises financeiras (1870-2008)}
\label{sec:org9c2ccdb}

\begin{frame}[label={sec:orgb9f86eb}]{Quem são os autores?}
\begin{figure}[htb]
\centering
\caption{Allan Taylor (Harvard) + Moritz Schularick (Free University of Berlin)}
\includegraphics[width = 0.35\textwidth]{./figs/Schularick_Taylor.png}
\label{fig:autores01}
\caption*{\textbf{Fonte:} Suspeita}
\end{figure}

Conjuntamente contribuem no \emph{MacroFinance Lab} e recentemente pesquisam

\begin{itemize}
\item Macroeconomia financeira, bancária e instabilidade financeira
\item Finanças internacionais e história econômica
\end{itemize}
\end{frame}

\begin{frame}[label={sec:orgb5cdef0}]{TL;DR (\emph{a.k.a} nem li; nem lerei)}
Autores apresentam um painel longo (1870-2008) para 14 países da OECD

\begin{itemize}
\item Em constante atualização (Ver \href{http://www.macrohistory.net/data/}{Macro history data})
\item Duas eras do capitalismo financeiro
\end{itemize}

Estimam vários modelos/especificações (29!) para verificar os antecedentes das crises financeiras
\begin{itemize}
\item \alert{Pergunta:} Fim \emph{Boom} Crédito?
\end{itemize}

\begin{block}{Relevância e questões abertas}
\begin{itemize}
\item Se crises financeiras são raras, é necessário olhar para um panorama mais longo
\item Compreensão dos atencedentes das crises financeiras \(\Rightarrow\) Bancos centrais
\item Crises financeiras são mais prováveis quanto maior o aprofundamento financeiro?
\end{itemize}
\end{block}
\end{frame}

\begin{frame}[label={sec:org04172ac}]{Estrutura do artigo}
\begin{block}{Moeda, crédito e crises financeiras no longo prazo}
\end{block}

\begin{block}{Dados}
\end{block}

\begin{block}{Moeda e crédito em duas eras do capitalismo financeiro}
\end{block}

\begin{block}{Moeda, crédito e produto no pós-crise financeira}
\end{block}

\begin{block}{\emph{Boom} de crédito e crises financeiras}
\end{block}

\begin{block}{Testes de robustez}
\end{block}
\end{frame}
\begin{frame}[label={sec:orgd447fca}]{Moeda, Crédito e Crises financeiras no longo prazo: Duas eras}
\begin{block}{Duas Eras do Capitalismo Financeiro}
\begin{itemize}
\item \alert{\emph{Money View} (1870-1939):} Relação estável entre moeda e crédito
\item \alert{\emph{Credit View} (1945-\(\ldots\)):} Taxa de crescimento do Crédito muito maior que o estoque de moeda e que o PIB
\begin{itemize}
\item \uline{Pergunta:} Crédito para quê/quem? \cite{jordaGreatMortgagingHousing2016}
\end{itemize}
\end{itemize}
\end{block}
\begin{block}{O que mudou?}
\begin{itemize}
\item "Dinâmica das crises" bem diferente
\begin{itemize}
\item Maiores perdas em termos do produto
\item Sistema financeiro mais complexo
\end{itemize}
\item \emph{Big Govern.} e \emph{Big Bank} \(\Leftrightarrow\) Lições da Grande Depressão
\end{itemize}
\end{block}
\end{frame}
\begin{frame}[label={sec:org7677aea}]{Duas eras, uma imagem}
\begin{figure}[htb]
\centering
\caption{Agregados financeiros/M3} 
\includegraphics[width = 0.95\textwidth]{./figs/AgregadosM3.png}
\label{fig:agregados}
\caption*{\textbf{Fonte:} \textcite[p.~ 1035]{schularickCreditBoomsGone2012}}
\end{figure}
\end{frame}

\begin{frame}[label={sec:org7231322}]{Duas eras, algumas observações}
\begin{center}
\begin{tabular}{lll}
\hline
 & Fase I & Fase II\\
\hline
Preços & Deflação & Pressão inflacionária\\
Agregados monetários & Decrecimento & Expansão\\
\(\Delta\) Política Monetária & - & Resposta mais rápida e ativa\\
\(\Delta\) Institucionais & - & Instituições depositárias\\
\hline
\end{tabular}
\end{center}

\begin{itemize}
\item Fenômeno comum aos vários países da base de dados
\item Argumentam que Fase I pode ser explicada pela visão "monetária" de Friedman
\item Fase II seria explicada por outra teoria \(\Leftrightarrow\) \emph{Credit View} \(\Leftrightarrow\) \textcite{minsky_1977_Financial}
\end{itemize}
\end{frame}

\begin{frame}[label={sec:org37b40a6}]{Dados e estratégia das estimações}
Base de dados permite analizar variávies até então pouco comparáveis

\begin{itemize}
\item Empréstimos e ativos bancários
\item Preço de ativos? \cite{jordaGreatMortgagingHousing2016}
\end{itemize}


$$
Pr(\text{Crise Fin.}_{i,t}) = L^{N}(\text{Crédito}) + L(\text{Controles})
$$
em que
\begin{itemize}
\item \alert{Crédito:} Crédito bancário deflacionado pelo índice de preços
\item \alert{Controles:} Vários! (Mesmo!)
\end{itemize}

\begin{block}{Observação econométrica}
\begin{itemize}
\item Seria o caso de um painel longo? (\(T>>N\))
\end{itemize}
\end{block}
\end{frame}
\begin{frame}[label={sec:org9dcb224}]{Alguns dos resultados de algumas estimações}
\begin{center}
\begin{center}
\begin{tabular}{llllll}
\hline
Método & OLS & OLS & OLS & Logit & Logit\\
Efeitos fixos & - & País & País + Ano & - & País\\
\hline
\(\sum\) Coef. & \(0.425^{\star\star\star}\) & \(0.417^{\star\star\star}\) & \(0.443^{\star\star\star}\) & \(10.10^{\star\star\star}\) & \(9.697^{\star\star\star}\)\\
\(\sum L_{Ns} = 0\)\footnotemark & 0.001 & 0.002 & 0.001 & 0.000 & 0.00408\\
\(R^2\) & 0.016 & 0.023 & 0.290 & 0.0434 & 0.0659\\
\(F, \chi^{2}\)\textsuperscript{\ref{org5be5290}} & 0.001 & 0.045 & 0.000 & 0.000 & 0.00663\\
AUROC & \(0.673^{\star\star\star}\) & \(0.720^{\star\star\star}\) & \(0.952^{\star\star\star}\) & \(0.673^{\star\star\star}\) & \(0.717^{\star\star\star}\)\\
\hline
 &  &  &  &  & \\
\end{tabular}
\end{center}\footnotetext[1]{\label{org5be5290}p-valor}
\end{center}


\begin{block}{Conclusão}
Defasagem da taxa de crescimento de crédito (até 5 anos) \(\Rightarrow\) indicativo de crise financeira
\begin{itemize}
\item Lags conjuntamente estatisticamente significates a 1\%
\item Destaque para quando a segunda derivada é negativa
\end{itemize}
\end{block}
\end{frame}
\begin{frame}[label={sec:orgf9d79a9}]{Sobre as outras estimações e robustez dos resultados}
\begin{itemize}
\item Agregados monetários não são tão preditivos quanto crédito bancário
\begin{itemize}
\item Poderia ser preditivo na primeira era financeira\footnote{Crédito se ajusta bem em ambas as eras}
\end{itemize}
\item Foram introduzidos vários controles para evitar o viés de omissão \(\Rightarrow\) resultado muda pouco
\item Resultados inconclusivos se o crédito é utilizado para financiar investimento ou consumo
\begin{itemize}
\item \uline{Sugere-se} que crises financeiras são mais prováveis se Crédito \(\Rightarrow\) Investimento
\end{itemize}
\item \(\Uparrow\) Preço dos ativos (ações) \(\Rightarrow \Downarrow\) colateral \(\Rightarrow \Uparrow\) Instabilidade financeira?
\begin{itemize}
\item Resultados inconclusivos, mas mais preocupantes na medida que o setor financeiro cresce
\end{itemize}
\end{itemize}


\begin{block}{Resumo}
Crédito bancário continua sendo o principal antecedente das crises financeiras
\end{block}
\end{frame}

\section{\textcite{sharpe_2020_Why}: Por que o investimento (das firmas) é tão insensível à taxa de juros?}
\label{sec:org3e61b16}

\begin{frame}[label={sec:org0d2dfc7}]{Quem são os autores?}
\begin{figure}[htb]
\centering
\caption{Steve A. Sharpe (Stanford) + Gustavo A. Suarez (Harvard)}
\includegraphics[width = 0.35\textwidth]{./figs/Sharpe_Suarez.png}
\label{fig:autores01}
\caption*{\textbf{Fonte:} Suspeita}
\end{figure}

Economistas do FED (Whashington). Atualmente pesquisam:

\begin{itemize}
\item Política monetária e Mercados Financeiros
\item Mercados de crédito
\end{itemize}
\end{frame}

\begin{frame}[label={sec:org0e0013f}]{Estrutura do artigo}
\begin{block}{Embasamento teórico}
\begin{block}{Sensibilidade do investimento a taxa de juros nos modelos convencionais}
\end{block}
\begin{block}{Da teoria à empiria}
\end{block}
\end{block}
\begin{block}{Quetionário e principais descobertas}
\end{block}
\begin{block}{Características das firmas respondentes}
\end{block}
\begin{block}{Estimação da regressão}
\end{block}
\begin{block}{Outros fatores potenciais por trás da insensibilidade}
\end{block}
\end{frame}
\begin{frame}[label={sec:org4c84397}]{TL;DR (\emph{a.k.a} nem li; nem lerei)}
Questionário indica elevada \alert{insensibilidade} do investimento (\emph{ex ante}) à taxa de juros
\begin{itemize}
\item Teoria sugere o inverso
\item Por quê? Lucros retidos e rigidezes
\end{itemize}

$$
IRR_{k} = \frac{A_{k}}{c_{k}} \geq \text{Hurdle rate} \nRightarrow \text{Juros}
$$

\begin{itemize}
\item Resultados são bastante robustos e pouco sensíveis às características amostra
\begin{itemize}
\item Há uma assimetria em relação a um aumento/redução dos juros, mas o conclusão permanece
\item Simetria do "porque não"
\end{itemize}
\end{itemize}

\begin{block}{Relevância}
\begin{itemize}
\item Teorias do investimento, finanças corporativas e canais de transmissão da política monetária
\end{itemize}
\end{block}
\end{frame}
\begin{frame}[label={sec:org35cf161}]{Revisão da literatura teórica}
\begin{table}[htbp]
\caption{Mecanismo esperados dos juros ao investimento}
\centering
\begin{tabular}{ll}
\hline
Modelo (tipo de custo, \(C\)) & Juros \((r) \Leftrightarrow\) Investimento \((I)\)\\
\hline
Uso do capital & \(\Uparrow r \Rightarrow \Uparrow C \Rightarrow \Downarrow I\)\\
Ajustamento & \(I = f(q(t)) = \int_{t}^{{\infty}} \pi_{K}(s)\exp^{-r(s-t)}ds\)\\
Irreversibilidade e incerteza & Região de inação em torno de \(q(t)\)\\
Financeiros (ext.) & \(I = f\left(\frac{q(t)}{\text{C fin.}}\right) \Rightarrow \Delta\) Estru. K \(\Rightarrow \Downarrow\) Sens.\\
\hline
\end{tabular}
\end{table}

\begin{block}{Resumo}
Espera-se uma sensibilidade elevada entre taxa de juros e investimento; essa sensibilidade aumenta quanto maior a perspectiva de expansão da firma; maiores custos de financialmento externo; maior rigidez da estrutural de capital
\end{block}
\end{frame}
\begin{frame}[label={sec:org796c0bf}]{Da sensibilidade à rigidez}
\begin{figure}[htb]
\centering
\caption{Taxa mínima de retorno VS Taxa de juros de Longo Prazo} 
\includegraphics[width = 0.95\textwidth]{./figs/Hurdle.png}
\label{fig:cycles}
\caption*{\textbf{Fonte} \textcite[p.~6]{sharpe_2020_Why}}
\end{figure}
\end{frame}

\begin{frame}[label={sec:org4fd368d}]{Questionário e desenho amostral}
\begin{itemize}
\item 680 Firmas de indústrias não-financeiras
\item \alert{Controles:} Tamanho da firma, da indústria e regime proprietário
\end{itemize}


\begin{block}{Questão}
\begin{quote}
Quanto os planos de investimento se alterariam (em p.p.) dada uma queda/elevação da taxa de juros?
\end{quote}
\end{block}

\begin{block}{Respostas}
\begin{itemize}
\item Não se aplica
\item \(0.5 \ldots 3.0+\)
\item Nada
\begin{itemize}
\item Por quê?!
\end{itemize}
\end{itemize}
\end{block}
\end{frame}

\begin{frame}[label={sec:orgd079670}]{Características das firmas}
\begin{table}[htbp]
\caption{Características da amostra}
\centering
\begin{tabular}{lr}
\hline
Características das firmas & Porcentagem (sem subgrupos)\\
\hline
Sem planos de tomar empréstimos & 51\\
Preocupações com \(K\) de giro & 26\\
\hline
Incerteza & 32\\
Expectativa cresc. receita \(\geq 5\%\) & 58\\
Expectativa cresc. lucro \(\geq 5\%\) & 53\\
Renda \(\leq 100 Mi\) & 57\\
Privada & 78\\
\hline
Indústria & 32\\
Serviços & 18\\
Varejo & 14\\
Outros setores & 37\\
\hline
\end{tabular}
\end{table}
\end{frame}

\begin{frame}[label={sec:orgda691ea}]{Resultados}
\begin{table}[htbp]
\caption{Sensibilidade dos planos de investimento às taxas de juros}
\centering
\begin{tabular}{rll}
\hline
\(\Delta\) Investimento (p.p) & Queda dos juros & Aumento dos juros\\
\hline
0.5 & 3\% & 6\%\\
1 & 5\% & 10\%\\
2 & 8\% & 16\%\\
3 & 5\% & 11\%\\
3+ & 11\% & 20\%\\
\hline
\(\sum\) senvível & 32\% & 63\%\\
\hline
0 & 68\% & 37\%\\
Não se aplica & 139 & 146\\
\hline
\end{tabular}
\end{table}
\end{frame}

\begin{frame}[label={sec:orgbc3edf6}]{Por que não?!}
\begin{table}[htbp]
\caption{Razões para a insensibilidade da taxa de juros}
\centering
\begin{tabular}{lll}
\hline
Motivo & Aumento & Redução\\
\hline
Financiamento por fluxo de caixa & 32\% & 49\%\\
Juros já estão baixos/ \(\pi > r\) & 27\% & 11\%\\
Elevado endividamento & 4\% & 1\%\\
Restrição de crédito & 2\% & 2\%\\
\hline
\(\sum\) razões financeiras & 65\% & 63\%\\
\hline
Investimento depende da demanda/LP & 17\% & 17\%\\
Sem lucros adicionais & 10\% & 11\%\\
Elevada incerteza & 3\% & 1\%\\
Não intensiva em \(K\) /Outros & 5\% & 7\%\\
\hline
\(\sum\) razões não-financeiras & 35\% & 37\%\\
\hline
\end{tabular}
\end{table}
\end{frame}
\begin{frame}[label={sec:orgb6e7a2c}]{Regressão}
\begin{block}{Probit e a insensibilidade aos juros}
\begin{align*}
Pr(\text{Não reagir}) =& \alpha_{ind} + \beta_{1}\text{Sem plano para pedir empréstimos} + \\
& \beta_{2}\text{K de giro} + \beta_{3}\text{Balanço Patrimonial} + \beta_{4}\text{Incerteza} + \\
& \beta_{5}\text{Expec. Cresc.} + \beta_{6}\text{Tamanho} + \beta_{7}\text{Privada}
\end{align*}
\end{block}
\begin{block}{Tobit e o investimento induzido pelos juros}
$$
\text{Threshold rate increase} = max\{3.1, \delta X_{i} + u_{i}\}
$$
\end{block}
\begin{block}{Conclusão}
\begin{itemize}
\item \alert{Insensibilidade:} \(\Uparrow\) Fluxo de caixa \(\Leftrightarrow \Downarrow\) pedir empréstimos
\item \alert{Sensibilidade:} Consistente com o modelo \emph{probit}
\item \alert{Taxa de crescimento esperada:} Resultados ambíguos
\end{itemize}
\end{block}
\end{frame}
\begin{frame}[label={sec:org512e393}]{Outras explicações pontenciais}
\begin{itemize}
\item \alert{Prevalência de acordos de dívida:} Restringem investimento
\begin{itemize}
\item \uline{Contra-factual:} Insensibilidade é comum às firmas pequenas e grandes
\end{itemize}
\item \alert{Excesso de precaução no ano do questionário:} Crise europeia e expectativas de uma menor demanda
\begin{itemize}
\item Outras perguntas do questionário não sugerem isso
\end{itemize}
\item \alert{Mudanças nos juros pouco prováveis:} Difícil de imaginar tais cenários
\begin{itemize}
\item Agentes atribuem uma baixa probabilidade de reduções ainda maiores nos juros
\end{itemize}
\end{itemize}

\begin{block}{Alternativas}
\begin{itemize}
\item Importância da estrutura patrimonial das \alert{famílias} \(\Rightarrow\) Transmissão da PM
\item Investimento resindencial e Consumo de bens durávies \(\Rightarrow\) Sensíveis aos juros
\end{itemize}
\end{block}
\end{frame}

\section{Unindo os pontos?}
\label{sec:org274e241}

\begin{frame}[label={sec:org33c901b}]{\textcite{schularickCreditBoomsGone2012}: How I Learned to Stop Worrying and Love \textcite{minsky_1977_Financial}?}
Algumas aproximações com algumas das conclusões de \textcite{minsky_1977_Financial}, mas \(\ldots\)

\begin{itemize}
\item Pouca atenção à posição financeira e balanço patrimonial dos agentes
\item Não explica mecanismos pelos quais o crédito ajudaria a anteceder as crises
\item Preço dos ativos aparece como um controle e apenas retoma como proposição de política econômica
\end{itemize}

\begin{block}{Algumas questões}
\begin{itemize}
\item Sempre \emph{credit view}?
\item \emph{Shadow banking} intensificaria essas conclusões?
\item Como lidar com as mudanças de definições ao longo do tempo?
\item Por mais que identificam um fenômeno comum, o que explicaria a diferença de intensidade entre países?
\end{itemize}
\end{block}
\end{frame}

\begin{frame}[label={sec:orgcd7ef7c}]{\textcite{sharpe_2020_Why}: How I Learned to Stop Loving and Worry \textcite{minsky_1977_Financial}?}
\textcite{schularickCreditBoomsGone2012} reportam a importância do crédito para os ciclos financeiros. No entanto, \textcite{sharpe_2020_Why} sugerem que as firmas não reagem tanto quanto as teorias \emph{a la} \cite{minsky_1977_Financial} descrevem

\begin{itemize}
\item Se investimento (das firmas) é insensível aos juros, o que fica da HIF?
\item Existe alguma não-linearidade relevante que não foi considerada?
\begin{itemize}
\item \alert{Memo:} Assimetria de sensibilidade é acompanhada da simetria de "motivos"
\end{itemize}
\end{itemize}

\begin{block}{Outras Questões abertas}
\begin{itemize}
\item Tais resultados se restringem ao EUA/período analizado?
\item \textcite{minsky_1977_Financial} seria mais aplicável a outros gastos que não investimento produtivo?
\item Será que a metodologia utilizada é a mais adequada para dados qualitativos?
\end{itemize}
\end{block}
\end{frame}
\end{document}

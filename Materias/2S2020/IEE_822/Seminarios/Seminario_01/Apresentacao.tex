% Created 2020-12-21 seg 18:25
% Intended LaTeX compiler: pdflatex
\documentclass[presentation]{beamer}
  \usepackage{caption}
\usepackage[brazilian, ]{babel}
\usepackage[style=abnt,noslsn,extrayear,uniquename=init,giveninits,justify,sccite, scbib,repeattitles,doi=false,isbn=false,url=false,maxcitenames=2, natbib=true,backend=biber]{biblatex}
\addbibresource{/HDD/Org/all_my_refs.bib}
\AtBeginSection[]{\begin{frame}<beamer>\frametitle{Artigos}\tableofcontents[currentsection]\end{frame}}
\usetheme{default}
\author{Gabriel Petrini}
\date{06 de Janeiro de 2021}
\title{Panorama das regularidades empíricas - \textcite{schularickCreditBoomsGone2012} e  \textcite{sharpe_2020_Why}}
\begin{document}

\maketitle
\section{\textcite{schularickCreditBoomsGone2012}: Booms de crédito acabam: Política Monetária, ciclos alavancados e crises financeiras (1870-2008)}
\label{sec:org4780dc4}

\begin{frame}[label={sec:orgbe70d63}]{Quem são os autores}
\begin{figure}[htb]
\centering
\caption{Allan Taylor + Moritz Schularick} 
\includegraphics[width = 0.35\textwidth]{./figs/Schularick_Taylor.png}
\label{fig:autores01}
\caption*{\textbf{Fonte:} Suspeita}
\end{figure}
\end{frame}

\begin{frame}[label={sec:orgdc204c0}]{TL;DR (\emph{a.k.a} nem li; nem lerei)}
Autores apresentam um painel longo (1870-2008) para 14 países da OECD

\begin{itemize}
\item Em constante atualização (Ver \href{http://www.macrohistory.net/data/}{Macro history data})
\item Duas eras do capitalismo financeiro
\end{itemize}

Estimam vários modelos/especificações (29!) para verificar os antecedentes das crises financeiras
\begin{itemize}
\item \alert{Pergunta:} Fim \emph{Boom} Crédito?
\end{itemize}

\begin{block}{Relevância e questões abertas}
\begin{itemize}
\item Se crises financeiras são raras, é necessário olhar para um panorama mais longo
\item Compreensão dos atencedentes das crises financeiras \(\Rightarrow\) Bancos centrais
\item Crises financeiras são mais prováveis quanto maior o aprofundamento financeiro?
\end{itemize}
\end{block}
\end{frame}

\begin{frame}[label={sec:org08b47aa}]{Estrutura do artigo}
\begin{block}{Moeda, crédito e crises financeiras no longo prazo}
\end{block}

\begin{block}{Dados}
\end{block}

\begin{block}{Moeda e crédito em duas eras do capitalismo financeiro}
\end{block}

\begin{block}{Moeda, crédito e produto no pós-crise financeira}
\end{block}

\begin{block}{\emph{Boom} de crédito e crises financeiras}
\end{block}

\begin{block}{Testes de robustez}
\end{block}
\end{frame}
\begin{frame}[label={sec:orgdeea6ee}]{Moeda, Crédito e Crises financeiras no longo prazo: Duas eras}
\begin{block}{Duas Eras do Capitalismo Financeiro}
\begin{itemize}
\item \alert{\emph{Money View} (1870-1939):} Relação estável entre moeda e crédito
\item \alert{\emph{Credit View} (1945-\(\ldots\)):} Taxa de crescimento do Crédito muito maior que o estoque de moeda e que o PIB
\begin{itemize}
\item \uline{Pergunta:} Crédito para quê/quem? \cite{jordaGreatMortgagingHousing2016}
\end{itemize}
\end{itemize}
\end{block}
\begin{block}{O que mudou?}
\begin{itemize}
\item "Dinâmica das crises" bem diferente
\begin{itemize}
\item Maiores perdas em termos do produto
\item Sistema financeiro mais complexo
\end{itemize}
\item \emph{Big Govern.} e \emph{Big Bank} \(\Leftrightarrow\) Lições da Grande Depressão
\end{itemize}
\end{block}
\end{frame}
\begin{frame}[label={sec:org25dfaf0}]{Duas eras, uma imagem}
\begin{figure}[htb]
\centering
\caption{Agregados financeiros/M3} 
\includegraphics[width = 0.95\textwidth]{./figs/AgregadosM3.png}
\label{fig:agregados}
\caption*{\textbf{Fonte:} \textcite[p.~ 1035]{schularickCreditBoomsGone2012}}
\end{figure}
\end{frame}

\begin{frame}[label={sec:org19044e6}]{Duas eras, algumas observações}
\begin{center}
\begin{tabular}{lll}
\hline
 & Fase I & Fase II\\
\hline
Preços & Deflação & Pressão inflacionária\\
Agregados monetários & Decrecimento & Expansão\\
\(\Delta\) Política Monetária & - & Resposta mais rápida e ativa\\
\(\Delta\) Institucionais & - & Instituições depositárias\\
\hline
\end{tabular}
\end{center}

\begin{itemize}
\item Fenômeno comum aos vários países da base de dados
\item Argumentam que Fase I pode ser explicada pela visão "monetária" de Friedman
\item Fase II seria explicada por outra teoria \(\Leftrightarrow\) \emph{Credit View} \(\Leftrightarrow\) \textcite{minsky_1977_Financial}
\end{itemize}
\end{frame}

\begin{frame}[label={sec:org8571091}]{Dados e estratégia das estimações}
Base de dados permite analizar variávies até então pouco comparáveis

\begin{itemize}
\item Empréstimos e ativos bancários
\item Preço de ativos? \cite{jordaGreatMortgagingHousing2016}
\end{itemize}


$$
Pr(\text{Crise Fin.}_{i,t}) = L^{N}(\text{Crédito}) + L(\text{Controles})
$$
em que
\begin{itemize}
\item \alert{Crédito:} Crédito bancário deflacionado pelo índice de preços
\item \alert{Controles:} Vários! (Mesmo!)
\end{itemize}

\begin{block}{Observação econométrica}
\begin{itemize}
\item Seria o caso de um painel longo? (\(T>>N\))
\end{itemize}
\end{block}
\end{frame}
\begin{frame}[label={sec:orgbae1cf5}]{Alguns dos resultados de algumas estimações}
\begin{center}
\begin{center}
\begin{tabular}{llllll}
\hline
Método & OLS & OLS & OLS & Logit & Logit\\
Efeitos fixos & - & País & País + Ano & - & País\\
\hline
\(\sum\) Coef. & \(0.425^{\star\star\star}\) & \(0.417^{\star\star\star}\) & \(0.443^{\star\star\star}\) & \(10.10^{\star\star\star}\) & \(9.697^{\star\star\star}\)\\
\(\sum L_{Ns} = 0\)\footnotemark & 0.001 & 0.002 & 0.001 & 0.000 & 0.00408\\
\(R^2\) & 0.016 & 0.023 & 0.290 & 0.0434 & 0.0659\\
\(F, \chi^{2}\)\textsuperscript{\ref{org58f7ebf}} & 0.001 & 0.045 & 0.000 & 0.000 & 0.00663\\
AUROC & \(0.673^{\star\star\star}\) & \(0.720^{\star\star\star}\) & \(0.952^{\star\star\star}\) & \(0.673^{\star\star\star}\) & \(0.717^{\star\star\star}\)\\
\hline
 &  &  &  &  & \\
\end{tabular}
\end{center}\footnotetext[1]{\label{org58f7ebf}p-valor}
\end{center}


\begin{block}{Conclusão}
Defasagem da taxa de crescimento de crédito (até 5 anos) \(\Rightarrow\) indicativo de crise financeira
\begin{itemize}
\item Lags conjuntamente estatisticamente significates a 1\%
\item Destaque para quando a segunda derivada é negativa
\end{itemize}
\end{block}
\end{frame}
\begin{frame}[label={sec:orge3ab564}]{Sobre as outras estimações e robustez dos resultados}
\begin{itemize}
\item Agregados monetários não são tão preditivos quanto crédito bancário
\begin{itemize}
\item Poderia ser preditivo na primeira era financeira\footnote{Crédito se ajusta bem em ambas as eras}
\end{itemize}
\item Foram introduzidos vários controles para evitar o viés de omissão \(\Rightarrow\) resultado muda pouco
\item Resultados inconclusivos se o crédito é utilizado para financiar investimento ou consumo
\begin{itemize}
\item \uline{Sugere-se} que crises financeiras são mais prováveis se Crédito \(\Rightarrow\) Investimento
\end{itemize}
\item \(\Uparrow\) Preço dos ativos (ações) \(\Rightarrow \Downarrow\) colateral \(\Rightarrow \Uparrow\) Instabilidade financeira?
\begin{itemize}
\item Resultados inconclusivos, mas mais preocupantes na medida que o setor financeiro cresce
\end{itemize}
\end{itemize}


\begin{block}{Resumo}
Crédito bancário continua sendo o principal antecedente das crises financeiras
\end{block}
\end{frame}

\section{\textcite{sharpe_2020_Why}: Por que o investimento (das firmas) é tão insensível à taxa de juros?}
\label{sec:org22dcfe2}

\begin{frame}[label={sec:org2fa9063}]{Quem são os autores}
\end{frame}

\begin{frame}[label={sec:orgd41adf3}]{Dissecando o artigo}
\end{frame}

\begin{frame}[label={sec:org5546fb2}]{TL;DR (\emph{a.k.a} nem li; nem lerei)}
\emph{Surveys} empresariais indicam elevada insensibilidade do investimento (\emph{ex ante}) à taxa de juros
\begin{itemize}
\item Teoria sugere o inverso
\item Por quê? Lucros retidos e rigidezes
\end{itemize}

$$
IRR_{k} = \frac{A_{k}}{c_{k}} \geq \text{Hurdle rate} \nRightarrow \text{Juros}
$$

\begin{itemize}
\item Resultados são bastante robustos e pouco sensíveis à amostra
\begin{itemize}
\item Há uma assimetria em relação a um aumento/redução dos juros, mas o conclusão permanece
\end{itemize}
\end{itemize}

\begin{block}{Relevância}
\begin{itemize}
\item Teorias do investimento, finanças corporativas e canais de transmissão da política monetária
\end{itemize}
\end{block}
\end{frame}

\section{Unindo os pontos?}
\label{sec:org91ac82b}

\begin{frame}[label={sec:org1189f44}]{\textcite{schularickCreditBoomsGone2012}: How I Learned to Stop Worrying and Love \textcite{minsky_1977_Financial}?}
Algumas aproximações com algumas das conclusões de \textcite{minsky_1977_Financial}, mas \(\ldots\)

\begin{itemize}
\item Pouca atenção à posição financeira e balanço patrimonial dos agentes
\item Não explica mecanismos pelos quais o crédito ajudaria a anteceder as crises
\item Preço dos ativos aparece como um controle e apenas retoma como proposição de política econômica
\end{itemize}

\begin{block}{Algumas questões}
\begin{itemize}
\item Sempre \emph{credit view}?
\item \emph{Shadow banking} intensificaria essas conclusões?
\item Como lidar com as mudanças de definições ao longo do tempo?
\item Por mais que identificam um fenômeno comum, o que explicaria a diferença de intensidade entre países?
\end{itemize}
\end{block}
\end{frame}

\begin{frame}[label={sec:org6935dfe}]{\textcite{sharpe_2020_Why}: How I Learned to Stop Loving and Worry \textcite{minsky_1977_Financial}?}
\end{frame}
\end{document}
